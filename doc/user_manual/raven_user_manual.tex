%
% This is an example LaTeX file which uses the SANDreport class file.
% It shows how a SAND report should be formatted, what sections and
% elements it should contain, and how to use the SANDreport class.
% It uses the LaTeX article class, but not the strict option.
% ItINLreport uses .eps logos and files to show how pdflatex can be used
%
% Get the latest version of the class file and more at
%    http://www.cs.sandia.gov/~rolf/SANDreport
%
% This file and the SANDreport.cls file are based on information
% contained in "Guide to Preparing {SAND} Reports", Sand98-0730, edited
% by Tamara K. Locke, and the newer "Guide to Preparing SAND Reports and
% Other Communication Products", SAND2002-2068P.
% Please send corrections and suggestions for improvements to
% Rolf Riesen, Org. 9223, MS 1110, rolf@cs.sandia.gov
%
\documentclass[pdf,12pt]{INLreport}
% pslatex is really old (1994).  It attempts to merge the times and mathptm packages.
% My opinion is that it produces a really bad looking math font.  So why are we using it?
% If you just want to change the text font, you should just \usepackage{times}.
% \usepackage{pslatex}
\usepackage{times}
\usepackage{longtable}
\usepackage[FIGBOTCAP,normal,bf,tight]{subfigure}
\usepackage{amsmath}
\usepackage{amssymb}
\usepackage[labelfont=bf]{caption}
\usepackage{pifont}
\usepackage{enumerate}
\usepackage{listings}
\usepackage{fullpage}
\usepackage{xcolor}          % Using xcolor for more robust color specification
\usepackage{ifthen}          % For simple checking in newcommand blocks
\usepackage{textcomp}
%\usepackage{authblk}         % For making the author list look prettier
%\renewcommand\Authsep{,~\,}

% Custom colors
\definecolor{deepblue}{rgb}{0,0,0.5}
\definecolor{deepred}{rgb}{0.6,0,0}
\definecolor{deepgreen}{rgb}{0,0.5,0}
\definecolor{forestgreen}{RGB}{34,139,34}
\definecolor{orangered}{RGB}{239,134,64}
\definecolor{darkblue}{rgb}{0.0,0.0,0.6}
\definecolor{gray}{rgb}{0.4,0.4,0.4}

\lstset {
  basicstyle=\ttfamily,
  frame=single
}

\setcounter{secnumdepth}{5}
\lstdefinestyle{XML} {
    language=XML,
    extendedchars=true,
    breaklines=true,
    breakatwhitespace=true,
%    emph={name,dim,interactive,overwrite},
    emphstyle=\color{red},
    basicstyle=\ttfamily,
%    columns=fullflexible,
    commentstyle=\color{gray}\upshape,
    morestring=[b]",
    morecomment=[s]{<?}{?>},
    morecomment=[s][\color{forestgreen}]{<!--}{-->},
    keywordstyle=\color{cyan},
    stringstyle=\ttfamily\color{black},
    tagstyle=\color{darkblue}\bf\ttfamily,
    morekeywords={name,type},
%    morekeywords={name,attribute,source,variables,version,type,release,x,z,y,xlabel,ylabel,how,text,param1,param2,color,label},
}
\lstset{language=python,upquote=true}

\usepackage{titlesec}
\newcommand{\sectionbreak}{\clearpage}
\setcounter{secnumdepth}{4}

%\titleformat{\paragraph}
%{\normalfont\normalsize\bfseries}{\theparagraph}{1em}{}
%\titlespacing*{\paragraph}
%{0pt}{3.25ex plus 1ex minus .2ex}{1.5ex plus .2ex}

%%%%%%%% Begin comands definition to input python code into document
\usepackage[utf8]{inputenc}

% Default fixed font does not support bold face
\DeclareFixedFont{\ttb}{T1}{txtt}{bx}{n}{9} % for bold
\DeclareFixedFont{\ttm}{T1}{txtt}{m}{n}{9}  % for normal

\usepackage{listings}

% Python style for highlighting
\newcommand\pythonstyle{\lstset{
language=Python,
basicstyle=\ttm,
otherkeywords={self, none, return},             % Add keywords here
keywordstyle=\ttb\color{deepblue},
emph={MyClass,__init__},          % Custom highlighting
emphstyle=\ttb\color{deepred},    % Custom highlighting style
stringstyle=\color{deepgreen},
frame=tb,                         % Any extra options here
showstringspaces=false            %
}}


% Python environment
\lstnewenvironment{python}[1][]
{
\pythonstyle
\lstset{#1}
}
{}

% Python for external files
\newcommand\pythonexternal[2][]{{
\pythonstyle
\lstinputlisting[#1]{#2}}}

\lstnewenvironment{xml}
{}
{}

% Python for inline
\newcommand\pythoninline[1]{{\pythonstyle\lstinline!#1!}}

% Named Colors for the comments below (Attempted to match git symbol colors)
\definecolor{RScolor}{HTML}{8EB361}  % Sonat (adjusted for clarity)
\definecolor{DPMcolor}{HTML}{E28B8D} % Dan
\definecolor{JCcolor}{HTML}{82A8D9}  % Josh (adjusted for clarity)
\definecolor{AAcolor}{HTML}{8D7F44}  % Andrea
\definecolor{CRcolor}{HTML}{AC39CE}  % Cristian
\definecolor{RKcolor}{HTML}{3ECC8D}  % Bob (adjusted for clarity)
\definecolor{DMcolor}{HTML}{276605}  % Diego (adjusted for clarity)
\definecolor{PTcolor}{HTML}{990000}  % Paul

\def\DRAFT{} % Uncomment this if you want to see the notes people have been adding
% Comment command for developers (Should only be used under active development)
\ifdefined\DRAFT
  \newcommand{\nameLabeler}[3]{\textcolor{#2}{[[#1: #3]]}}
\else
  \newcommand{\nameLabeler}[3]{}
\fi
\newcommand{\alfoa}[1] {\nameLabeler{Andrea}{AAcolor}{#1}}
\newcommand{\cristr}[1] {\nameLabeler{Cristian}{CRcolor}{#1}}
\newcommand{\mandd}[1] {\nameLabeler{Diego}{DMcolor}{#1}}
\newcommand{\maljdan}[1] {\nameLabeler{Dan}{DPMcolor}{#1}}
\newcommand{\cogljj}[1] {\nameLabeler{Josh}{JCcolor}{#1}}
\newcommand{\bobk}[1] {\nameLabeler{Bob}{RKcolor}{#1}}
\newcommand{\senrs}[1] {\nameLabeler{Sonat}{RScolor}{#1}}
\newcommand{\talbpaul}[1] {\nameLabeler{Paul}{PTcolor}{#1}}
% Commands for making the LaTeX a bit more uniform and cleaner
\newcommand{\TODO}[1]    {\textcolor{red}{\textit{(#1)}}}
\newcommand{\xmlAttrRequired}[1] {\textcolor{red}{\textbf{\texttt{#1}}}}
\newcommand{\xmlAttr}[1] {\textcolor{cyan}{\textbf{\texttt{#1}}}}
\newcommand{\xmlNodeRequired}[1] {\textcolor{deepblue}{\textbf{\texttt{<#1>}}}}
\newcommand{\xmlNode}[1] {\textcolor{darkblue}{\textbf{\texttt{<#1>}}}}
\newcommand{\xmlString}[1] {\textcolor{black}{\textbf{\texttt{'#1'}}}}
\newcommand{\xmlDesc}[1] {\textbf{\textit{#1}}} % Maybe a misnomer, but I am
                                                % using this to detail the data
                                                % type and necessity of an XML
                                                % node or attribute,
                                                % xmlDesc = XML description
\newcommand{\default}[1]{~\\*\textit{Default: #1}}
\newcommand{\nb} {\textcolor{deepgreen}{\textbf{~Note:}}~}

%%%%%%%% End comands definition to input python code into document

%\usepackage[dvips,light,first,bottomafter]{draftcopy}
%\draftcopyName{Sample, contains no OUO}{70}
%\draftcopyName{Draft}{300}

% The bm package provides \bm for bold math fonts.  Apparently
% \boldsymbol, which I used to always use, is now considered
% obsolete.  Also, \boldsymbol doesn't even seem to work with
% the fonts used in this particular document...
\usepackage{bm}

% Define tensors to be in bold math font.
\newcommand{\tensor}[1]{{\bm{#1}}}

% Override the formatting used by \vec.  Instead of a little arrow
% over the letter, this creates a bold character.
\renewcommand{\vec}{\bm}

% Define unit vector notation.  If you don't override the
% behavior of \vec, you probably want to use the second one.
\newcommand{\unit}[1]{\hat{\bm{#1}}}
% \newcommand{\unit}[1]{\hat{#1}}

% Use this to refer to a single component of a unit vector.
\newcommand{\scalarunit}[1]{\hat{#1}}

% \toprule, \midrule, \bottomrule for tables
\usepackage{booktabs}

% \llbracket, \rrbracket
\usepackage{stmaryrd}

\usepackage{hyperref}
\hypersetup{
    colorlinks,
    citecolor=black,
    filecolor=black,
    linkcolor=black,
    urlcolor=black
}

\newcommand{\wiki}{\href{https://github.com/idaholab/raven/wiki}{RAVEN wiki}}

% Compress lists of citations like [33,34,35,36,37] to [33-37]
\usepackage{cite}

% If you want to relax some of the SAND98-0730 requirements, use the "relax"
% option. It adds spaces and boldface in the table of contents, and does not
% force the page layout sizes.
% e.g. \documentclass[relax,12pt]{SANDreport}
%
% You can also use the "strict" option, which applies even more of the
% SAND98-0730 guidelines. It gets rid of section numbers which are often
% useful; e.g. \documentclass[strict]{SANDreport}

% The INLreport class uses \flushbottom formatting by default (since
% it's intended to be two-sided document).  \flushbottom causes
% additional space to be inserted both before and after paragraphs so
% that no matter how much text is actually available, it fills up the
% page from top to bottom.  My feeling is that \raggedbottom looks much
% better, primarily because most people will view the report
% electronically and not in a two-sided printed format where some argue
% \raggedbottom looks worse.  If we really want to have the original
% behavior, we can comment out this line...
\raggedbottom
\setcounter{secnumdepth}{5} % show 5 levels of subsection
\setcounter{tocdepth}{5} % include 5 levels of subsection in table of contents

% ---------------------------------------------------------------------------- %
%
% Set the title, author, and date
%
\title{RAVEN User Manual}
%\author{%
%\begin{tabular}{c} Author 1 \\ University1 \\ Mail1 \\ \\
%Author 3 \\ University3 \\ Mail3 \end{tabular} \and
%\begin{tabular}{c} Author 2 \\ University2 \\ Mail2 \\ \\
%Author 4 \\ University4 \\ Mail4\\
%\end{tabular} }


\author{
\textbf{\textit{Project Manager:}}
 \\Cristian Rabiti\\
 \textbf{\textit{Principal Investigator and Technical Leader:}}
 \\Andrea Alfonsi\\
\textbf{\textit{Main Developers:}}
\\Andrea Alfonsi
\\Diego Mandelli
\\Joshua Cogliati
\\Congjian Wang
\\Paul W. Talbot
\\Daniel P. Maljovec
\\Robert Kinoshita\\
\textbf{\textit{Former Developers:}} \\Sonat Sen
\\Jun Chen\\
\textbf{\textit{Contributors:}}
\\Alessandro Bandini (Post-Processor)
\\Ivan Rinaldi (documentation)
\\Claudia Picoco (new external code interface)
\\James B. Tompkins (new external code interface)
\\Matteo Donorio (new external code interface)
\\Fabio Giannetti (new external code interface)
}
% \\James B. Tompkins}   Just people who actually ``developed'' a significant capability in the code should be placed here. Andrea
%\author{\textbf{\textit{Main Developers:}}  \\Andrea Alfonsi}
%\author{\\Joshua Cogliati}
%\author{\\Diego Mandelli}
%\author{\\Robert Kinoshita}
%\author{\\Sonat Sen}
%
%\author{Cristian Rabiti}
%\author{Andrea Alfonsi}
%\author{Joshua Cogliati}
%\author{Diego Mandelli}
%\author{Robert Kinoshita}
%\author{Sonat Sen}
%\affil{Idaho National Laboratory, Idaho Falls, ID 83402}
%\\\{cristian.rabiti, andrea.alfonsi, joshua.cogliati, diego.mandelli, robert.kinoshita, ramazan.sen\}@inl.gov}

% There is a "Printed" date on the title page of a SAND report, so
% the generic \date should [WorkingDir:]generally be empty.
\date{}


% ---------------------------------------------------------------------------- %
% Set some things we need for SAND reports. These are mandatory
%
\SANDnum{INL/EXT-15-34123}
\SANDprintDate{March 2017}
\SANDauthor{Cristian Rabiti, Andrea Alfonsi, Joshua Cogliati, Diego Mandelli,
Robert Kinoshita, Sonat Sen, Congjian Wang, Paul W. Talbot, Daniel P. Maljovec, Jun Chen}
\SANDreleaseType{Revision 6 draft}

% ---------------------------------------------------------------------------- %
% Include the markings required for your SAND report. The default is "Unlimited
% Release". You may have to edit the file included here, or create your own
% (see the examples provided).
%
% \include{MarkOUO} % Not needed for unlimted release reports

\def\component#1{\texttt{#1}}

% ---------------------------------------------------------------------------- %
\newcommand{\systemtau}{\tensor{\tau}_{\!\text{SUPG}}}

% Added by Sonat
\usepackage{placeins}
\usepackage{array}

\newcolumntype{L}[1]{>{\raggedright\let\newline\\\arraybackslash\hspace{0pt}}m{#1}}
\newcolumntype{C}[1]{>{\centering\let\newline\\\arraybackslash\hspace{0pt}}m{#1}}
\newcolumntype{R}[1]{>{\raggedleft\let\newline\\\arraybackslash\hspace{0pt}}m{#1}}

% end added by Sonat
% ---------------------------------------------------------------------------- %
%
% Start the document
%

\begin{document}
    \maketitle

    % ------------------------------------------------------------------------ %
    % An Abstract is required for SAND reports
    %
%    \begin{abstract}
%    \input abstract
%    \end{abstract}


    % ------------------------------------------------------------------------ %
    % An Acknowledgement section is optional but important, if someone made
    % contributions or helped beyond the normal part of a work assignment.
    % Use \section* since we don't want it in the table of context
    %
%    \clearpage
%    \section*{Acknowledgment}



%	The format of this report is based on information found
%	in~\cite{Sand98-0730}.


    % ------------------------------------------------------------------------ %
    % The table of contents and list of figures and tables
    % Comment out \listoffigures and \listoftables if there are no
    % figures or tables. Make sure this starts on an odd numbered page
    %
    \cleardoublepage		% TOC needs to start on an odd page
    \tableofcontents
    %\listoffigures
    %\listoftables


    % ---------------------------------------------------------------------- %
    % An optional preface or Foreword
%    \clearpage
%    \section*{Preface}
%    \addcontentsline{toc}{section}{Preface}
%	Although muggles usually have only limited experience with
%	magic, and many even dispute its existence, it is worthwhile
%	to be open minded and explore the possibilities.


    % ---------------------------------------------------------------------- %
    % An optional executive summary
    %\clearpage
    %\section*{Summary}
    %\addcontentsline{toc}{section}{Summary}
    %\input{Summary.tex}
%	Once a certain level of mistrust and skepticism has
%	been overcome, magic finds many uses in todays science



%	and engineering. In this report we explain some of the
%	fundamental spells and instruments of magic and wizardry. We
%	then conclude with a few examples on how they can be used
%	in daily activities at national Laboratories.


    % ---------------------------------------------------------------------- %
    % An optional glossary. We don't want it to be numbered
%    \clearpage
%    \section*{Nomenclature}
%    \addcontentsline{toc}{section}{Nomenclature}
%    \begin{description}
%          \item[alohomoral]
%           spell to open locked doors and containers
%          \item[leviosa]
%           spell to levitate objects
%    \item[remembrall]
%           device to alert you that you have forgotten something
%    \item[wand]
%           device to execute spells
%    \end{description}


    % ---------------------------------------------------------------------- %
    % This is where the body of the report begins; usually with an Introduction
    %
    \SANDmain		% Start the main part of the report

\input{introduction.tex}
\input{nomenclature.tex}
% definitions

% content
\section{Installation}
\subsection{Overview}
\label{sec:install overview}

The installation of the RAVEN code is a straightforward procedure;
depending on the usage purpose and machine architecture, the
installation process slightly differs.

In the following sections, the recommended installation procedure is outlined.  For alternatives, we encourage
checking the \wiki.  The machines on which
RAVEN is tested and developed, however, use the standard installation procedures outlined below.

The installation process will involve three steps:
\begin{itemize}
  \item Installing prerequisites, which depends on your operating system;
  \item Installing conda;
  \item Installing RAVEN.
\end{itemize}

Depending on your operating system (Windows in section \ref{sec:install windows}, MacOSX in section
\ref{sec:install mac}, Ubuntu Linux in section \ref{sec:install ubunutu}), follow the instructions for installing prerequisites, then continue with
installing conda (section \ref{sec:install conda}), and then installing RAVEN (section \ref{sec:clone raven}).


\subsection{Linux Ubuntu Installation}
\label{sec:install ubunutu}
The following instructions are for installing RAVEN on a Linux machine running Ubuntu 16.04 or greater.  Some
explanations of alternatives for other Linux distributions may be provided on the \wiki.

To install the prerequisite packages, the following terminal command should be executed (note this requires
administrative privileges):

\begin{lstlisting}[language=bash]
 sudo apt-get install libtool git python-dev swig g++
\end{lstlisting}

\paragraph{Optional LateX installation}
Optionally, if you want to be able to edit and rebuild the manuals, you can
install \TeX~Live and its related packages:
\begin{lstlisting}[language=bash]
  sudo apt-get install texlive-latex-base \
  texlive-extra-utils texlive-latex-extra texlive-math-extra
\end{lstlisting}

Once the above are installed, proceed with installing conda (see section \ref{sec:install conda}).

\subsection{Mac OSX Installation}
\label{sec:install mac}

When using an Apple Macintosh computer, software dependencies are met
by following steps:
\begin{itemize}
  \item Install the XCode command line tools from Apple,
  \item Install the XQuartz  X-Window system server,
\end{itemize}

\subsubsection{Installing XCode Command Line Tools}

The XCode command line tools package from Apple Computer provides the C++
compilers and git source code control tools needed to obtain and build RAVEN.
It is freely available from the Apple store. In order to obtain it the following command should be launched in an open terminal:
\begin{lstlisting}[language=bash]
 xcode-select --install
\end{lstlisting}

\subsubsection{Installing XQuartz}
XQuartz is an implementation of the X Server for the Mac OSX operating system.
XQuartz is freely available on the web and can be downloaded from the link
 \url{https://dl.bintray.com/xquartz/downloads/XQuartz-2.7.9.dmg}.
\\After downloaded, install the package.

With XCode and XQuartz installed, continue on to install conda (see section \ref{sec:install conda}).

\nb While \texttt{gcc} and
\texttt{git} are also required, they are installed by default in the OSX system.

\subsection{Microsoft Windows}
\label{sec:install windows}

The process of establishing the required environment for Windows is notably more involved than the other two
systems; however, it is straightforward.  First, RAVEN has the following prerequisites on Windows:

\begin{itemize}
    \item A system running a 64-bit version of Microsoft Windows. Installation and operation
        has been verified on Windows 7, 10, and Windows Server 2012 R2 Standard.
    \item At least 9 Gigabytes of available disk space:
    \begin{itemize}
        \item 0.5 GB for MSYS2, including supporting tools and git source code control
        \item 1.5 GB for Python language and supporting packages
        \item 1.5 GB for RAVEN and the MOOSE framework
        \item 5.0 GB for the Visual Studio compiler needed to build RAVEN
    \end{itemize}
\end{itemize}

\subsubsection{A Visual Guide}
Note: An illustrated version of this procedure may be found on the \wiki.

\subsubsection{MSYS2 environment}
Since RAVEN requires a UNIX-like shell to function, the freely-available software package called MSYS2 is
used to
provide this functionality.  More information about MSYS2 is available at
\url{https://sourceforge.net/p/msys2/wiki/MSYS2%20introduction/}.
\nb While there is also a 32-bit version of MSYS2 available, the RAVEN installation described here will not work with it.

\begin{enumerate}
    \item Obtain and run the latest basic 64-bit MSYS2 installer from \url{ https://msys2.github.io/} (As of this writing it is named
	msys2-x86\_64-20161025.exe and is approximately 67 Megabytes in size).
    \item The page with the download also contains installation instructions. Perform the steps described there up to
	step 6 to install a minimal MSYS2 system and bring it up to date. Make sure that you install to path
        C:\textbackslash{}msys64.  This installation will create shortcuts in the Windows start menu that may be used
        to start UNIX-Like shells:
		\begin{itemize}
	    		\item MSYS2 Shell
	    		\item MinGW-w64 Win32 Shell
	    		\item MinGW-w64 Win64 Shell
		\end{itemize}
        When working with RAVEN, it is recommended to use "MinGW-w64 Win64 Shell", although any of them should work.
    \item Use the MSYS2 package manager {\it pacman} to install a few tools that will be needed later.  Enter the following command in an MSYS shell window:

\begin{lstlisting}[language=bash]
USER@HOSTNAME MINGW64 ~
$ pacman -S git winpty make
\end{lstlisting}
	The package manager will then download and install those packages (and their dependencies) from the MSYS2
	repository.
\end{enumerate}

\subsubsection{Install Python Language and Package Support}
\begin{enumerate}
	\item Download the latest 64-bit installer for Windows Python 2.7 from
		\url{https://conda.io/miniconda.html} and install it.  \item The installer
		will ask whether Python should be installed for only the logged in user or
		for all users.  Either option will work for RAVEN.
	\item Locate and test the Python installation.   Open a Windows command prompt and enter the
		command "{\it where python}", which attempts to locate a the Python language interpreter
		in the current system path.  This looks like:

    \begin{lstlisting}[language=bash, basicstyle=\small]
    C:\Users\USERID> where python
    C:\Users\USERID\AppData\Local\Continuum\Miniconda2\python.exe
    \end{lstlisting}

	\item Setup MSYS2 to find Python.  MSYS2 has its own separate PATH which must also be adjusted
		so that Python and its associated tools may be found. This is done by converting the
		file system location of Python determined in the previous step to its MSYS2-compatible
		equivalent and using the result to setup MSYS2 so that it too can find it in the future.
		\newline \newline
		This is done by turning all backslashes ('\textbackslash') in the path to be converted to
		forward slashes ('/'), and changing the drive letter from its '\textless letter\textgreater:'
		form to '/ \textless letter\textgreater'. In addition, any spaces in the path must
		be escaped using a backslash ('\textbackslash') when converted.
		\newline \newline
		For example:

\begin{lstlisting}[language=bash]
C:\Users\USERID\AppData\Local\Continuum\Miniconda2
\end{lstlisting}
		becomes
\begin{lstlisting}[language=bash]
/c/Users/USERID/AppData/Local/Continuum/Miniconda2
\end{lstlisting}
		for MSYS2. Here is an example with spaces that need to be escaped:
\begin{lstlisting}[language=bash]
C:\Program Files\Common Files
\end{lstlisting}
		converted to MSYS2 form would become
\begin{lstlisting}[language=bash]
/c/Program\ Files/Common\ Files
\end{lstlisting}
		\medskip
		Three separate paths must be added to MSYS2 to enable all of the Python tools needed
		to be found.  These are:

		\smallskip
\begin{tabular}{| l | l |}
	\hline
	{\bf Path} & {\bf Purpose} \\\hline
	\textless Converted path from above\textgreater	& Python executable \\\hline
	\textless Converted path from above\textgreater /Scripts &	Conda (Needed to manage Python packages) \\\hline
	\textless Converted path from above\textgreater /Library/bin &	Swig (Needed to build RAVEN) \\\hline
\end{tabular}

		\medskip
		These paths are added using shell commands that append new entries to the existing PATH
		variable without overwriting it.  These commands take the following form:

\begin{lstlisting}[language=bash, basicstyle=\tiny]
export PATH=/c/Users/USERID/AppData/Local/Continuum/Miniconda2:$PATH
export PATH=/c/Users/USERID/AppData/Local/Continuum/Miniconda2/Scripts:$PATH
export PATH=/c/Users/USERID/AppData/Local/Continuum/Miniconda2/Library/bin:$PATH
\end{lstlisting}

		To configure these needed paths in MSYS2 so that they persist, file "~/.bashrc" will need
		to be edited.  This may be done either using an MSYS2-based editor such as {\it vim}
		(VI-iMproved, which is included in the installation) or a Windows-based editor like
		{\it Wordpad} (included with Windows).  Another excellent open source editor for
		Windows is {\it Notepad++} \url{https://notepad-plus-plus.org/}, which is also good
		for editing RAVEN input files.

	\item Test Python in MSYS2.  At this point open a new MSYS2 shell window and see if Python is
		now found in the PATH:
		\newline
		Note: Due to the way that Python interacts with the MSYS2 shell, when using Python by
		itself in MSYS2 the {\it winpty} utility is provided. (If Python is run without winpty,
		it may appear to sit there and do nothing. Pressing \textless Ctrl\textgreater -C will
		interrupt it.)


\begin{lstlisting}[language=bash, basicstyle=\tiny]
USER@HOSTNAME MINGW64 ~
$ winpty python
Python 2.7.13 |Anaconda 4.0.0 (64-bit)| (default, Dec 19 2016, 13:29:36) [MSC v.1500 64 bit (AMD64)] on win32
Type "help", "copyright", "credits" or "license" for more information.
Anaconda is brought to you by Continuum Analytics.
Please check out: http://continuum.io/thanks and https://anaconda.org
>>>
>>> quit()

\end{lstlisting}

	\item Install needed Python packages.  RAVEN requires several Python packages to function properly.
		Now the {\it conda} command will be used to download and install them in an automated manner. The
		following asks {\it conda} to obtain the specified versions of the listed packages, as well as all
		of their dependencies.
		\smallskip

    \begin{lstlisting}[language=bash]
    conda install numpy=1.11.0 h5py=2.6.0 scipy=0.17.1 \
      scikit-learn=0.17.1 matplotlib=1.5.1 python=2.7 \
      hdf5 swig pylint lxml
    \end{lstlisting}

\end{enumerate}

\subsubsection{Compiler Installation and Configuration}
\begin{enumerate}
	\item Download and install Visual Studio.  A C++ language compiler that supports C++11 features
		is needed to perform this step. Microsoft's Visual Studio Community Edition is free and
		available from \url{https://www.visualstudio.com/downloads/}.

		The current version (as of this writing) is 2017. The 2015 and 2017 versions have been
		successfully used to build RAVEN. Professional and Enterprise versions of these will
		also work. If one of these is already present on your system, it is not necessary to
		obtain another one. Note that because C++11 language features are required, the
		"Microsoft Visual C++ Compiler for Python 2.7" often used for building Python
		add-ons will {\bf not} work.

		After downloading and running the Visual Studio installer, it will ask what features
		to install. For building RAVEN, "Desktop development with C++" is needed at a minimum.
		Installation of other Visual Studio features should be fine.

	\item Let the build system know where to find the compiler.  When the build system attempts
		to search for an installed compiler, this process often fails with the error message
		"Unable to find vcvarsall.bat".  This happens because Python version 2.7 has not been
		updated to automatically locate modern Visual Studio installations. To solve this it
		is necessary to help the Python build system find the C++ compiler on the system.
		The easiest way to do this is create a Windows batch (.BAT) file that will redirect
		the build system to the information it needs. First, locate the file VCVARSALL.BAT
		file installed as part of Visual Studio on your system. This location will usually
		be something like the following:
		\smallskip

\begin{tabular}{| l | l |}
	\hline
	{\bf Visual Studio Version} & {\bf Directory containing VCVARSALL.BAT} \\\hline
	2015 & C:\textbackslash Program Files (x86)\textbackslash Microsoft Visual Studio 14.0\textbackslash VC \\\hline
	2017 & C:\textbackslash Program Files (x86)\textbackslash Microsoft Visual \\
               & Studio\textbackslash 2017\textbackslash Community\textbackslash VC\textbackslash
                Auxiliary\textbackslash Build \\\hline
\end{tabular}

		\medskip
		Once the target file has been located it is necessary to create a couple of directories
		and one file. The first directory created must be named "VC" and should be created
		somewhere outside of the RAVEN source tree (such as your MinGW home directory):

\begin{lstlisting}[language=bash]
USER@HOSTNAME MINGW64 ~
$ mkdir VC
\end{lstlisting}

		The next directory to be created must be inside the one just created. It is suggested
		to name it "target", because it is there that we will point the Python build system:

\begin{lstlisting}[language=bash]
USER@HOSTNAME MINGW64 ~
$ cd VC

USER@HOSTNAME MINGW64 ~/VC
$ mkdir target

USER@HOSTNAME MINGW64 ~/VC
$ ls
target

\end{lstlisting}

		The file to be created is named "VCVARSALL.BAT", and it must be written in the VC
		directory that was just made. The Python build system will be configured to find this
		file, which then redirects it to the actual file. Use a text editor (such as
		{\it vim} or {\it notepad} as described above) to create the file VCVARSALL.BAT:

\begin{lstlisting}[language=bash]
USER@HOSTNAME MINGW64 ~/VC
$ vim VCVARSALL.BAT
\end{lstlisting}

		or

\begin{lstlisting}[language=bash]
USER@HOSTNAME MINGW64 ~/VC
$ notepad VCVARSALL.BAT
\end{lstlisting}

		One line will need to be added to the new file VCVARSALL.BAT:

\begin{lstlisting}[language=bash, basicstyle=\tiny]
CALL "<Full Path To VCVARSALL.BAT File Installed by Visual Studio>" %1 %2 %3 %4 %5
\end{lstlisting}

		For example, in the case of Visual Studio 2017 Community installed in the default
		location this would be:

\begin{lstlisting}[language=bash, basicstyle=\tiny]
CALL "C:\Program Files (x86)\Microsoft Visual Studio\2017\Community\VC\Auxiliary\Build\VCVARSALL.BAT" %1 %2 %3 %4 %5
\end{lstlisting}

		Note the double quotes around the path and file name. These are necessary because
		there are spaces in some of the directory names that make up the full location of
		VCVARSALL.BAT.

		After creating the new {\it VCVARSALL.BAT} in the directory {\it VC}, one more thing
		needs to be done to inform the Python build system where this file just created is.
		During the build process, an {\it environment variable} "VS90COMNTOOLS" will be checked.
		The value of VS90COMNTOOLS will need to be set to the {\it target} directory just below
		the location of VCVARSALL.BAT file just created.

		For example, if VCVARSALL.BAT was created in directory VC under your MinGW home
		directory, the variable VS90COMNTOOLS should point to \textasciitilde /VC/target.

\begin{lstlisting}[language=bash]
USER@HOSTNAME MINGW64 ~
$ export VS90COMNTOOLS=~/VC/target

USER@HOSTNAME MINGW64 ~
$ echo $VS90COMNTOOLS
/home/user/VC/target
\end{lstlisting}

\end{enumerate}



Once the compiler installation and configuration is complete, you are prepared to install the RAVEN libraries
(see section \ref{sec:install conda}).




\subsection{Conda: Python Dependencies}
\label{sec:install conda}

The standard installation procedure for RAVEN includes using Miniconda (often simply referred to as
\emph{conda}) to install the Python libraries required to run RAVEN.  If conda cannot be made available on an
operating system, refer to the wiki (listed above) for alternatives.  To install miniconda, follow the
instructions for your operating system at \url{https://conda.io/miniconda.html}.

\nb Since RAVEN currently relies
on some Python 2.7-only libraries, make sure to install the 2.7 64-bit version of miniconda.

Once conda is installed, proceed to installing RAVEN itself (section \ref{sec:clone raven}).


\subsection{Installing RAVEN}
\label{sec:clone raven}

Once the RAVEN dependencies have been installed  and conda is present
(see section \ref{sec:install overview}), the rest of RAVEN can be installed.

The installation of RAVEN involves the following steps:
\begin{itemize}
  \item Obtain the source code,
  \item Install the prerequisite Python libraries using conda,
  \item Compile
\end{itemize}




\subsubsection{Obtaining RAVEN Source Code}
RAVEN is hosted publicly as a \texttt{Git} repo on \texttt{GitHub}
and can be viewed at \url{https://github.com/idaholab/raven/wiki}.
In the event that access to \texttt{GitHub} is impossible, contact the user list and other arrangements may be
possible.  In general, however, using the git repository assures the most consistent usage and update process.

To clone RAVEN, navigate in a terminal to the desired destination, for example \texttt{~/projects}.  Then run
the commands
\begin{lstlisting}[language=bash]
git clone https://github.com/idaholab/raven.git
cd raven
git submodule update --init
\end{lstlisting}
This will obtain RAVEN as well as other submodules that RAVEN uses.  In the future, whenever we declare a path
starting with \texttt{raven/}, we refer to the cloned directory.




\subsubsection{Installing Python Libraries}
RAVEN depends heavily on Python, and uses conda to maintain a separate environment to prevent conflicts with
other Python library installations.  This separate environment is called \texttt{raven\_libraries}.

In order to establish this environment, navigate to \texttt{raven}, then
\begin{lstlisting}[language=bash]
cd scripts
./establish\_conda\_env.sh --install
\end{lstlisting}
Assure that there are no errors in this process, then continue to compiling RAVEN.

\nb If \texttt{conda} is not installed in the default location, then the path to the conda definitions
needs to be provided, for example
\begin{lstlisting}[language=bash]
cd scripts
./establish\_conda\_env.sh --install --conda-defs /path/to/miniconda2/etc/profile.d/conda.sh
\end{lstlisting}
replacing \texttt{/path/to} with the install path for \texttt{conda}.




\subsubsection{Compiling RAVEN}
Once Python libraries are established and the source code present, navigate to \texttt{raven} and run
\begin{lstlisting}[language=bash]
./build_raven
\end{lstlisting}
This will compile several dependent libraries.  This step has the highest potential for revealing problems in
the operating system setup, particularly for Windows.  See troubleshooting on the \wiki for help sorting out
difficulties.




\subsubsection{Testing RAVEN}
\label{sec:testing raven}
To test the installation of RAVEN, navigate to \texttt{raven}, then run the command
\begin{lstlisting}[language=bash]
./run_tests -j2
\end{lstlisting}
where \texttt{-j2} signifies running with 2 processors.  If more processors are available, this can be
increased, but using all or more than all of the available processes can slow down the testing dramatically.
This command runs RAVEN's regression tests, analytic tests, and unit tests.  The number of tests changes
frequently as the code's needs change, and the time taken to run the tests depends strongly on the number of
processors and processor speed.

At the end of the tests, a number passed, skipped, and failing will be reported.  Having some skipped tests is
expected; RAVEN has many tests that apply only to particular configurations or codes that are not present on
all machines.  However, no tests should fail; if there are problems, consult the troubleshooting section on
the \wiki.


\subsubsection{Updating RAVEN}
RAVEN updates frequently, and new features are added while bugs are fixed on a regular basis.  To update
RAVEN, navigate to \texttt{raven}, then run the commands
\begin{lstlisting}[language=bash]
git pull
./scripts/establish_conda_env.sh --install
./build_raven
\end{lstlisting}


\subsubsection{In-use Testing}

At any time, tests can be checked by re-running the installation tests as
described in Section \ref{sec:testing raven}.


\section{Running RAVEN}
\label{HowToRun}

% I don't think this is mentioned earlier? Andrea answers :D It mentioned in the Introduction
%As already mentioned,
The RAVEN code is a blend of C++, C, and Python software. The entry point
resides on the Python side and is accessible via a command line interface.
%
After following the instructions in the previous Section, RAVEN is ready to be
used.
%
The \texttt{raven\_framework} script is in the raven folder.
%
To run RAVEN, open a terminal and use the following command (replace \texttt{<inputFileName.xml>} with your RAVEN input file):

\begin{lstlisting}[language=bash]
raven_framework <inputFileName.xml>
\end{lstlisting}

Using \texttt{raven\_framework} is the recommended way to run RAVEN.  In the event bypassing the typical
environment loading and checks is desired, it can also be run via
the \texttt{Driver.py} script using python, with the input file as argument.  However, this is not
recommended, as it will use whatever default versions of Python and other libraries are discovered, rather
than the matching libraries set up during installation.

\input{ravenStructure.tex}
\input{runInfo.tex}
\input{files.tex}
\input{variablegroups.tex}
\section{Distributions}
\label{sec:distributions}
\newcommand{\distname}[1]{\textbf{#1}}
\newcommand{\distattrib}[1]{\textit{#1}}

%%%%%%%%%%%%%%%%%%%%%%%%%%%%%%%%%%%%%%%%%%%%%%%%%%%%%%%%%%%%%%%%%%%%%%%%%%%%%%%%
% If you are confused by the input of this document, please make sure you see
% these defined commands first. There is no point writing the same thing over
% and over and over and over and over again, so these will help us reduce typos,
% by just editing a template sentence.
\newcommand{\nameDescription}{\xmlAttr{name},
  \xmlDesc{required string attribute}, user-defined name of this distribution.
  %
  \nb As with other objects, this identifier can be used to reference this
  specific entity from other input blocks in the XML.}
\newcommand{\specBlock}[2]{The specifications of this distribution must be
  defined within #1 \xmlNode{#2} XML block.}
\newcommand{\attrIntro}{This XML node accepts one attribute:}
\newcommand{\attrsIntro}{This XML node accepts the following attributes:}
\newcommand{\subnodeIntro}{This distribution can be initialized with the
  following child node:}
\newcommand{\subnodesIntro}{This distribution can be initialized with the
  following children nodes:}
%%%%%%%%%%%%%%%%%%%%%%%%%%%%%%%%%%%%%%%%%%%%%%%%%%%%%%%%%%%%%%%%%%%%%%%%%%%%%%%%

%\maljdan{Do we want to provide the equations of each distribution?}
%\alfoa{TBH I do not think so. It is not a theory manual... I put few equations in the ROM section just to explain better the meaning of some parameters.}

RAVEN provides support for several probability distributions.
%
Currently, the user can choose among several 1-dimensional distributions and
$N$-dimensional ones, either custom or multidimensional normal.

The user will specify the probability distributions, that need to be used during
the simulation, within the \xmlNode{Distributions} XML block:
\begin{lstlisting}[style=XML]
<Simulation>
   ...
  <Distributions>
    <!-- All the necessary distributions will be listed here -->
  </Distributions>
  ...
</Simulation>
\end{lstlisting}

In the next two sub-sections, the input requirements for all of the
distributions are reported.

%%%%%% 1-Dimensional Probability distributions
\subsection{1-Dimensional Probability Distributions}
\label{subsec:1dDist}
This sub-section is organized in two different parts: 1) continuous 1-D
distributions and 2) discrete 1-D distributions.
%
These two paragraphs cover all the requirements for using the different
distribution entities.
%
%%%%%% paragraph 1-Dimensional Continuous Distributions.
\subsubsection{1-Dimensional Continuous Distributions}
\label{subsubsec:1DContinuous}
In this paragraph all the 1-D distributions currently available in RAVEN are
reported.

Firstly, all the probability distributions functions in the code can be
truncated by using the following keywords:
\begin{lstlisting}[style=XML]
<Distributions>
   ...
   <aDistributionType>
      ...
      <lowerBound>aFloatValue</lowerBound>
      <upperBound>aFloatValue</upperBound>
      ...
   </aDistributionType>
</Distributions>
\end{lstlisting}
Each distribution has a pre-defined, default support (domain) based on its
definition, however these domains can be shifted/stretched using the appropriate
\xmlNode{low} and \xmlNode{high} parameters where applicable, and/or truncated
using the nodes in the example above, namely \xmlNode{lowerBound} and
\xmlNode{upperBound}.
For example, the Normal distribution domain is $[-\infty,+\infty]$, and thus
cannot be shifted or stretched, as it is already unbounded, but can be
truncated.
%
RAVEN currently provides support for 13 1-Dimensional distributions.
%
In the following paragraphs, all the input requirements are reported and
commented.

%%%%%% Beta
\paragraph{Beta Distribution}
\label{Beta}
The \distname{Beta} distribution is parameterized by two positive shape parameters, denoted by
$\alpha$ and $\beta$, that appear as exponents of the random variable. Its default support
(domain) is $x \in [0, 1]$.
%
The distribution domain can be changed, specifying new boundaries, to fit the user's needs.
%
The user can specify a \distname{Beta} distribution in two ways.  The standard
is to provide the parameters \xmlNode{low}, \xmlNode{high}, \xmlNode{alpha},
and \xmlNode{beta}.  Alternatively, to approximate a normal
distribution that falls to 0 at the endpoints, the user may provide
the parameters \xmlNode{low}, \xmlNode{high}, and \xmlNode{peakFactor}. The
peak factor is a value between 0 and 1 that determines the peakedness of the
distribution.  At 0 it is dome-like ($\alpha=\beta=4$) and at 1 it is very
strongly peaked around the mean ($\alpha=\beta=100$).  A reasonable approximation
to a Gaussian normal is a peak factor of 0.5.

\specBlock{a}{Beta}
%
\attrIntro
\vspace{-5mm}
\begin{itemize}
  \itemsep0em
  \item \nameDescription
\end{itemize}
\vspace{-5mm}
\subnodesIntro
\begin{itemize}
  \item Standard initialization:
    \begin{itemize}
    \item \xmlNode{alpha}, \xmlDesc{float, conditional required parameter}, first shape
     parameter.  If specified, \xmlNode{beta} must also be inputted and
     \xmlNode{peakFactor} can not be specified.
%
     \item \xmlNode{beta}, \xmlDesc{float, conditional required parameter}, second shape
      parameter.  If specified, \xmlNode{alpha} must also be inputted and
      \xmlNode{peakFactor} can not be specified.
%
       \item \xmlNode{low}, \xmlDesc{float, optional parameter}, lower domain
       boundary. \default{0.0}
 %
       \item \xmlNode{high}, \xmlDesc{float, optional parameter}, upper domain,
         boundary. \default{1.0}
      \end{itemize}
 %
     \item Alternative initialization:
     \begin{itemize}
        \item \xmlNode{peakFactor}, \xmlDesc{float, optional parameter}, alternative
         to specifying \xmlNode{alpha} and \xmlNode{beta}.  Acceptable values range from
        0 to 1.
%
       \item \xmlNode{low}, \xmlDesc{float, optional parameter}, lower domain
       boundary. \default{0.0}
 %
       \item \xmlNode{high}, \xmlDesc{float, optional parameter}, upper domain,
         boundary. \default{1.0}
      \end{itemize}
  %
\end{itemize}

Example:
\begin{lstlisting}[style=XML]
<Distributions>
  ...
  <Beta name='aUserDefinedName'>
     <low>aFloatValue</low>
     <high>aFloatValue</high>
     <alpha>aFloatValue</alpha>
     <beta>aFloatValue</beta>
  </Beta>
  <Beta name='aUserDefinedName2'>
     <low>aFloatValue</low>
     <high>aFloatValue</high>
     <peakFactor>aFloatValue</peakFactor>
  </Beta>
  ...
</Distributions>
\end{lstlisting}

%%%%%% Exponential
\paragraph{Exponential Distribution}
\label{Exponential}
The \distname{Exponential} distribution has a default support of
$x \in [0, +\infty)$.

\specBlock{an}{Exponential}
%
\attrIntro
\vspace{-5mm}
\begin{itemize}
  \itemsep0em
  \item \nameDescription
\end{itemize}
\vspace{-5mm}
\subnodeIntro
\begin{itemize}
  \item \xmlNode{lambda}, \xmlDesc{float, required parameter}, rate parameter.
  \item \xmlNode{low}, \xmlDesc{float, optional parameter}, lower domain
     boundary. \default{0.0}
  %
\end{itemize}

Example:
\begin{lstlisting}[style=XML]
<Distributions>
  ...
  <Exponential name='aUserDefinedName'>
    <lambda>aFloatValue</lambda>
    <low>aFloatValue</low>
  </Exponential>
  ...
</Distributions>
\end{lstlisting}

%%%%%% Gamma
\paragraph{Gamma Distribution}
\label{Gamma}
The \distname{Gamma} distribution is a two-parameter family of continuous
probability distributions.
%
The common exponential distribution and $\chi$-squared distribution are special
cases of the gamma distribution.
%
Its default support is $x \in [0,+\infty]$.

\specBlock{a}{Gamma}
%
\attrIntro
\vspace{-5mm}
\begin{itemize}
  \itemsep0em
  \item \nameDescription
\end{itemize}
\vspace{-5mm}
\subnodesIntro
\begin{itemize}
  \item \xmlNode{alpha}, \xmlDesc{float, required parameter}, shape parameter.
  %
  \item \xmlNode{beta}, \xmlDesc{float, optional parameter}, 1/scale or the
  inverse scale parameter. \default{1.0}
  \item \xmlNode{low}, \xmlDesc{float, optional parameter}, lower domain
  boundary. \default{0.0}
  %
\end{itemize}

Example:
\begin{lstlisting}[style=XML]
<Distributions>
  ...
  <Gamma name='aUserDefinedName'>
    <alpha>aFloatValue</alpha>
    <beta>aFloatValue</beta>
    <low>aFloatValue</low>
  </Gamma>
  ...
</Distributions>
\end{lstlisting}

%%%%%% Laplace
\paragraph{Laplace Distribution}
\label{Laplace}
The \distname{Laplace} distribution is a two-parameter continuous
probability distribution.  It is the distribution of the differences
between two independent random variables with identical exponential
distributions.
%
Its default support is $x \in (-\infty,+\infty)$.

\specBlock{a}{Laplace}
%
\attrIntro
\vspace{-5mm}
\begin{itemize}
  \itemsep0em
  \item \nameDescription
\end{itemize}
\vspace{-5mm}
\subnodesIntro
\begin{itemize}
\item \xmlNode{location}, \xmlDesc{float, required parameter},
  determines the location or shift of the distribution.
\item \xmlNode{scale}, \xmlDesc{float, required parameter}, must be
  greater than 0, and determines how spread out the distribution is.
\end{itemize}

Example:
\begin{lstlisting}[style=XML]
<Distributions>
  ...
  <Laplace name='aUserDefinedName'>
    <location>aFloatValue</location>
    <scale>aFloatValue</scale>
  </Laplace>
  ...
</Distributions>
\end{lstlisting}

%%%%%% Logistic
\paragraph{Logistic Distribution}
\label{Logistic}
The \distname{Logistic} distribution is similar to the
normal distribution with a CDF that is an instance of a logistic function ($Cdf(x) = \frac{1}{1+e^{-\frac{(x-location)}{scale})}}$).
%
It resembles the normal distribution in shape but has heavier tails (higher
kurtosis).
%
Its default support is $x \in [-\infty,+\infty]$.

\specBlock{a}{Logistic}
%
\attrIntro
\vspace{-5mm}
\begin{itemize}
  \itemsep0em
  \item \nameDescription
\end{itemize}
\vspace{-5mm}
\subnodesIntro
\begin{itemize}
  \item \xmlNode{location}, \xmlDesc{float, required parameter}, the
  distribution
  mean.
  %
  \item \xmlNode{scale}, \xmlDesc{float, required parameter}, scale parameter
  that
  is proportional to the standard deviation ($\sigma ^{2}=\frac{1}{3}\pi^{2}scale^{2} $).
  %
\end{itemize}

Example:
\begin{lstlisting}[style=XML]
<Distributions>
  ...
  <Logistic name='aUserDefinedName'>
    <location>aFloatValue</location>
    <scale>aFloatValue</scale>
  </Logistic>
  ...
</Distributions>
\end{lstlisting}

%%%%%% LogNormal
\paragraph{LogNormal Distribution}
\label{LogNormal}
The \distname{LogNormal} distribution is a distribution with the
logarithm of the random variable being normally distributed.
%
Its default support is $x \in [0, +\infty]$.

\specBlock{a}{LogNormal}
%
\attrIntro
\vspace{-5mm}
\begin{itemize}
  \itemsep0em
  \item \nameDescription
\end{itemize}
\vspace{-5mm}
\subnodesIntro
\begin{itemize}
  \item \xmlNode{mean}, \xmlDesc{float, required parameter}, the log of the distribution
  mean or expected value.
  %
  \item \xmlNode{sigma}, \xmlDesc{float, required parameter}, standard
  deviation.
  \item \xmlNode{low}, \xmlDesc{float, optional parameter}, lower domain
  boundary. \default{0.0}
  %
\end{itemize}
\nb The \xmlNode{mean} and \xmlNode{sigma} listed above are NOT the mean and standard deviation of the
distribution; they are the mean and standard deviation of the log of the distribution.  Using the following
notation:
\begin{itemize}
  \item $\mu_\ell$: the $\mu$ parameter of the lognormal distribution, which RAVEN expects in the
    \xmlNode{mean} node;
  \item $\sigma_\ell$: the $\sigma$ parameter of the lognormal distribution, which RAVEN expects in the
    \xmlNode{sigma} node;
  \item $M$: the user-desired mean value of the distribution;
  \item $S$: the user-desired standard deviation of the distribution;
\end{itemize}
a conversion is defined to translate from mean $M$ and standard deviation $S$ into the parameters RAVEN
expects:
\begin{equation}
  \mu_\ell = \log\left(\frac{M}{\sqrt{1+\frac{S^2}{M^2}}}\right),
\end{equation}
\begin{equation}
  \sigma_\ell = \sqrt{\log{1+\frac{S^2}{M^2} }}.
\end{equation}

Example:
\begin{lstlisting}[style=XML]
<Distributions>
  ...
  <LogNormal name='aUserDefinedName'>
    <mean>aFloatValue</mean>
    <sigma>aFloatValue</sigma>
    <low>aFloatValue</low>
  </LogNormal>
  ...
</Distributions>
\end{lstlisting}

%%%%%% LogUniform
\paragraph{LogUniform Distribution}
\label{LogNormal}
The \distname{LogNormal} distribution is a distribution associated to
a variable $y=h(x)=e^{x}$ where variable x is uniform distributed.
This distribution supports not only the case  $y=h(x)=e^{x}$ (natural case) but also
the case where $y=h(x)=10^{x}$ (decimal case).
%

Its default support is $x \in [h(lowerBound),h(upperBound)]$.

\specBlock{a}{LogUniform}
%
\attrIntro
\vspace{-5mm}
\begin{itemize}
  \itemsep0em
  \item \nameDescription
\end{itemize}
\vspace{-5mm}
\subnodesIntro
\begin{itemize}
  \item \xmlNode{lowerBound}, \xmlDesc{float, required parameter}, domain lower boundary.
  %
  \item \xmlNode{upperBound}, \xmlDesc{float, required parameter}, domain upper boundary.
  %
  \item \xmlNode{base}, \xmlDesc{string, required parameter}, case type (decimal or natural).
  %
\end{itemize}

Example:
\begin{lstlisting}[style=XML]
<Distributions>
  ...
  <LogUniform name="x_dist">
    <upperBound>1.0</upperBound>
    <lowerBound>3.0</lowerBound>
    <base>natural</base>
  </LogUniform>
  ...
</Distributions>
\end{lstlisting}

%%%%%% Normal
\paragraph{Normal Distribution}
\label{Normal}
The \distname{Normal} distribution is an extremely
useful continuous distribution.
%
Its utility is due to the central limit theorem, which states that, under mild
conditions, the mean of many random variables independently drawn from the same
distribution is distributed approximately normally, irrespective of the form of
the original distribution.
%
Its default support is $x \in [-\infty, +\infty]$.

\specBlock{a}{Normal}
%
\attrIntro
\vspace{-5mm}
\begin{itemize}
  \itemsep0em
  \item \nameDescription
\end{itemize}
\vspace{-5mm}
\subnodesIntro
\begin{itemize}
  \item \xmlNode{mean}, \xmlDesc{float, required parameter}, the distribution
  mean
  or expected value.
  %
  \item \xmlNode{sigma}, \xmlDesc{float, required parameter}, the standard
  deviation.
  %
\end{itemize}

Example:
\begin{lstlisting}[style=XML]
<Distributions>
  ...
  <Normal name='aUserDefinedName'>
    <mean>aFloatValue</mean>
    <sigma>aFloatValue</sigma>
  </Normal>
  ...
</Distributions>
\end{lstlisting}

%%%%%% Triangular
\paragraph{Triangular Distribution}
\label{Triangular}
The \distname{Triangular} distribution is a continuous distribution that has a
triangular shape for its PDF.
%
%It is often used where the distribution is only vaguely known.
%
Like the uniform distribution, upper and lower limits are ``known,'' but a
``best guess,'' of the mode or center point is also added.
%
It has been recommended as a ``proxy'' for the beta distribution.
%
Its default support is $x \in [min,max]$.

\specBlock{a}{Triangular}
%
\attrIntro
\vspace{-5mm}
\begin{itemize}
  \itemsep0em
  \item \nameDescription
\end{itemize}
\vspace{-5mm}
\subnodesIntro
\begin{itemize}
  \item \xmlNode{apex}, \xmlDesc{float, required parameter}, peak location
  %``best guess'', also called, peak factor.
  %
  \item \xmlNode{min}, \xmlDesc{float, required parameter}, domain lower
  boundary.
  %
  \item \xmlNode{max}, \xmlDesc{float, required parameter}, domain upper
  boundary.
  %
\end{itemize}

Example:
\begin{lstlisting}[style=XML]
<Distributions>
  ...
  <Triangular name='aUserDefinedName'>
    <apex>aFloatValue</apex>
    <min>aFloatValue</min>
    <max>aFloatValue</max>
  </Triangular>
  ...
</Distributions>
\end{lstlisting}

%%%%%% Uniform
\paragraph{Uniform Distribution}
\label{Uniform}
The \distname{Uniform} distribution is a continuous distribution with a
rectangular-shaped PDF.
%
It is often used where the distribution is only vaguely known, but upper and
lower limits are known.
%
Its default support is $x \in [lower,upper]$.

\specBlock{a}{Uniform}
%
\attrIntro
\vspace{-5mm}
\begin{itemize}
  \itemsep0em
  \item \nameDescription
\end{itemize}
\vspace{-5mm}
\subnodesIntro
\begin{itemize}
  \item \xmlNode{lowerBound}, \xmlDesc{float, required parameter}, domain lower
  boundary.
  %
  \item \xmlNode{upperBound}, \xmlDesc{float, required parameter}, domain upper
  boundary.
  %
\end{itemize}
\nb Since the Uniform distribution is a rectangular-shaped PDF, the truncation does not have any effect;
 this is the reason why the children nodes are the ones generally used for truncated distributions.
Example:
\begin{lstlisting}[style=XML]
<Distributions>
  ...
  <Uniform name='aUserDefinedName'>
    <lowerBound>aFloatValue</lowerBound>
    <upperBound>aFloatValue</upperBound>
  </Uniform>
  ...
</Distributions>
\end{lstlisting}

%%%%%% Weibull
\paragraph{Weibull Distribution}
\label{Weibull}
The \distname{Weibull} distribution is a continuous distribution that is often
used in the field of failure analysis; in particular, it can mimic distributions
where the failure rate varies over time.
%
If the failure rate is:
\vspace{-5mm}
\begin{itemize}
  \itemsep0em
  \item constant over time, then $k = 1$, suggests that items are failing from
  random events;
  \item decreases over time, then $k < 1$, suggesting ``infant mortality'';
  \item increases over time, then $k > 1$, suggesting ``wear out'' - more likely
  to fail as time goes by.
  %
\end{itemize}
\vspace{-5mm}
Its default support is $x \in [0, +\infty)$.

\specBlock{a}{Weibull}
%
\attrIntro
\vspace{-5mm}
\begin{itemize}
  \itemsep0em
  \item \nameDescription
\end{itemize}
\vspace{-5mm}
\subnodesIntro
\begin{itemize}
  \item \xmlNode{k}, \xmlDesc{float, required parameter}, shape parameter.
  %
  \item \xmlNode{lambda}, \xmlDesc{float, required parameter}, scale parameter.
  \item \xmlNode{low}, \xmlDesc{float, optional parameter}, lower domain
  boundary. \default{0.0}
  %
\end{itemize}

Example:
\begin{lstlisting}[style=XML]
<Distributions>
  ...
  <Weibull name='aUserDefinedName'>
    <lambda>aFloatValue</lambda>
    <k>aFloatValue</k>
    <low>aFloatValue</low>
  </Weibull>
  ...
</Distributions>
\end{lstlisting}

%%%%%% Custom1D
\paragraph{Custom1D Distribution}
\label{Custom1D}
The \distname{Custom1D} distribution is a custom continuous distribution that can be initialized from a dataObject
generated by RAVEN.
This distribution cannot be initialized from a dataObject directly but through a .csv file.
This file must contain the values of either cdf or pdf of the random variable sampled along the range of the desired 
random variable.
In the distribution block of the RAVEN input file, the user needs to specify which file (including its working directory)
needs to be used to initialize the distribution. In addition, the user is required to specify which type (cdf or pdf) or values
are contained in the file and also the IDs of both the random variable and cdf/pdf.
Thus the csv file contains a set of points that samples the function $pdf(x)$ or $cdf(x)$ for several values of the stochastic variable $x$. 
The user needs to specify which variable IDs correspond to $x$ and $pdf(x)$ (or $cdf(x)$).
The distribution create a fourth order spline interpolation from the provided input points.
%
Note that the support of this distribution is set between the minimum and maximum values of the random variable which are 
specified in the distribution input file.

Refer to the test example ($tests/framework/test_distributionCustom1D.xml$) for more clarification.

\specBlock{a}{Custom1D}
%
\attrIntro
\vspace{-5mm}
\begin{itemize}
  \itemsep0em
  \item \nameDescription
\end{itemize}
\vspace{-5mm}
\subnodesIntro
\begin{itemize}
  \item \xmlNode{dataFilename}, \xmlDesc{string, required parameter}, file name to be used to initialize the distribution.
  \item \xmlNode{workingDir}, \xmlDesc{string, optional parameter}, relative working directory that contains the input file.
  \item \xmlNode{functionType}, \xmlDesc{string, required parameter}, type of initialization values specified in the input file (pdf or cdf).
  \item \xmlNode{variableID}, \xmlDesc{string, required parameter}, ID of the variable contained in the input file.
  \item \xmlNode{functionID}, \xmlDesc{string, required parameter}, ID of the function associated to the variableID contained in the input file.
\end{itemize}

Example:
\begin{lstlisting}[style=XML]
<Distributions>
  ...
    <Custom1D name="pdf_custom">
      <dataFilename>PointSetFile2_dump.csv</dataFilename>
      <functionID>pdf_values</functionID>
      <variableID>x</variableID>
      <functionType>pdf</functionType>
      <workingDir>custom1D/</workingDir>
    </Custom1D>
    <Custom1D name="cdf_custom">
      <dataFilename>PointSetFile3_dump.csv</dataFilename>
      <functionID>cdf_values</functionID>
      <variableID>x</variableID>
      <functionType>cdf</functionType>
      <workingDir>custom1D/</workingDir>
    </Custom1D>
  ...
</Distributions>
\end{lstlisting}


The example above initializes two distributions from two .csv files. 
For example, the first distribution retrieves the pdf values, located in the column with label $pdf_values$, for several locations of the variable located in the column 
with label $x$ in the file $PointSetFile2_dump.csv$.


%%%%%% paragraph 1-Dimensional Discrete Distributions.
\subsubsection{1-Dimensional Discrete Distributions.}
\label{subsubsec:1DDiscrete}
RAVEN currently supports 3 discrete distributions.
%
In the following paragraphs, the input requirements are reported.

%%%%%% Bernoulli
\paragraph{Bernoulli Distribution}
\label{Bernoulli}
The \distname{Bernoulli} distribution is a discrete distribution of the outcome
of a single trial with only two results, 0 (failure) or 1 (success), with a
probability of success \distattrib{p}.
%
It is the simplest building block on which other discrete distributions of
sequences of independent Bernoulli trials can be based.
%
Basically, it is the binomial distribution (k = 1, \distattrib{p}) with only
one trial.
%
Its default support is $k \in {0, 1}$.

\specBlock{a}{Bernoulli}
%
\attrIntro
\vspace{-5mm}
\begin{itemize}
  \itemsep0em
  \item \nameDescription
\end{itemize}
\vspace{-5mm}
\subnodeIntro
\begin{itemize}
  \item \xmlNode{p}, \xmlDesc{float, required parameter}, probability of
  success.
  %
 \end{itemize}
Example:
\begin{lstlisting}[style=XML]
<Distributions>
  ...
  <Bernoulli name='aUserDefinedName'>
    <p>aFloatValue</p>
  </Bernoulli>
  ...
</Distributions>
\end{lstlisting}

%%%%%% Binomial
\paragraph{Binomial Distribution}
\label{Binomial}
The \distname{Binomial} distribution is the discrete probability distribution of
the number of successes in a sequence of \distattrib{n} independent yes/no
experiments, each of which yields success with probability \distattrib{p}.
%
Its default support is $k \in {0, 1, 2, ..., n}$.

\specBlock{a}{Binomial}
%
\attrIntro
\vspace{-5mm}
\begin{itemize}
  \itemsep0em
  \item \nameDescription
\end{itemize}
\vspace{-5mm}
\subnodesIntro
\begin{itemize}
  \item \xmlNode{p}, \xmlDesc{float, required parameter}, probability of
  success.
  %
  \item \xmlNode{n}, \xmlDesc{integer, required parameter}, number of
  experiments.
  %
\end{itemize}

Example:
\begin{lstlisting}[style=XML]
<Distributions>
  ...
  <Binomial name='aUserDefinedName'>
    <n>aIntegerValue</n>
    <p>aFloatValue</p>
  </Binomial>
  ...
</Distributions>
\end{lstlisting}

%%%%%% Geometric
\paragraph{Geometric Distribution}
\label{Geometric}
The \distname{Geometric} distribution is a one-parameter discrete probability distribution.
%
The distribution uses the probability $p$ that trial will be
successful.  The geometric distribution gives the probability of
observing $k$ trials before the first success.
%
Its support is $k \in {0, 1, 2, ..., n}$.

\specBlock{a}{Geometric}

\attrIntro
\vspace{-5mm}
\begin{itemize}
  \itemsep0em
\item \nameDescription
\end{itemize}
\vspace{-5mm}
\subnodesIntro
\begin{itemize}
\item \xmlNode{p}, \xmlDesc{float, required parameter}, the success fraction for the trials.
\end{itemize}

Example:
\begin{lstlisting}[style=XML]
<Distributions>
  ...
  <Geometric name='aUserDefinedName'>
    <p>aFloatValue</p>
  </Geometric>
  ...
</Distributions>
\end{lstlisting}

%%%%%% Poisson
\paragraph{Poisson Distribution}
\label{Poisson}
The \distname{Poisson} distribution is a discrete probability distribution that
expresses the probability of a given number of events occurring in a fixed
interval of time and/or space if these events occur with a known average rate
and independently of the time since the last event.
%
Its default support is $k \in {1, 2, 3, 4, ...}$.

\specBlock{a}{Poisson}
%
\attrIntro
\vspace{-5mm}
\begin{itemize}
  \itemsep0em
  \item \nameDescription
\end{itemize}
\vspace{-5mm}
\subnodeIntro
\begin{itemize}
  \item \xmlNode{mu}, \xmlDesc{float, required parameter}, mean rate of
  events/time.
  %
\end{itemize}

Example:
\begin{lstlisting}[style=XML]
<Distributions>
  ...
  <Poisson name='aUserDefinedName'>
    <mu>aFloatValue</mu>
  </Poisson>
  ...
</Distributions>
\end{lstlisting}


%%%%%% Categorical
\paragraph{Categorical Distribution}
\label{Categorical}
The \distname{Categorical} distribution is a discrete distribution that describes the result of a random variable that can have $K$ possible outcomes. 
The probability of each outcome is separately specified.
The possible outcomes must be only numerical values (either integer or float numbers). No string can be assigned to any outcome.
%
There is not necessarily an underlying ordering of these outcomes, but labels are assigned in describing the distribution (in the range $1$ to $K$).
%
\specBlock{a}{Categorical}
%
\attrIntro
\vspace{-5mm}
\begin{itemize}
  \itemsep0em
  \item \nameDescription
\end{itemize}
\vspace{-5mm}
\subnodeIntro
\begin{itemize}
  \item \xmlNode{state}, \xmlDesc{float, required parameter}, probability for outcome 1
  \begin{itemize}
          \item \xmlAttr{outcome}, \xmlDesc{float, required parameter}, outcome value.
  \end{itemize}
  \item \xmlNode{state}, \xmlDesc{float, required parameter}, probability for outcome 2
  \begin{itemize}
          \item \xmlAttr{outcome}, \xmlDesc{float, required parameter}, outcome value.
  \end{itemize}
  \item ...
  \item \xmlNode{state}, \xmlDesc{float, required parameter}, probability for outcome K
  \begin{itemize}
          \item \xmlAttr{outcome}, \xmlDesc{float, required parameter}, outcome value.
  \end{itemize}
 \end{itemize}
Example:
\begin{lstlisting}[style=XML]
<Distributions>
  ...
    <Categorical name='testCategorical'>
        <state outcome="10">0.1</state>
        <state outcome="20">0.2</state>
        <state outcome="50">0.15</state>
        <state outcome="60">0.4</state>
        <state outcome="90">0.15</state>
    </Categorical>
  ...
</Distributions>
\end{lstlisting}

\paragraph{Markov Categorical Distribution}
\label{subsec:markovCategorical}

The \textbf{MarkovCategorical} distribution is a specific discrete categorical distribution describes
a random variable that can have $K$ possible outcomes, based on the steady state probabilities provided by
Markov model.
%
\begin{itemize}
  \item \xmlNode{transition}, \xmlDesc{float, optional field}, the transition matrix of given Markov model.
  \item \xmlNode{dataFile}, \xmlDesc{string, optional xml node}. The path for the given data file, i.e. the transition matrix.
    In this node, the following attribute should be specified: 
    \begin{itemize}
      \item \xmlAttr{fileType}, \xmlDesc{string, optional field}, the type of given data file, default is `csv'.
    \end{itemize}
  \nb Either \xmlNode{transition} or \xmlNode{dataFile} is required to provide the transition matrix.
  \item \xmlNode{workingDir}, \xmlDesc{string, optional field}, the path of working directory
  \item \xmlNode{state}, \xmlDesc{required xml node}. The output from this state indicates
    the probability for outcome 1.
    In this node, the following attribute should be specified:
    \begin{itemize}
      \item \xmlAttr{outcome}, \xmlDesc{float, required field}, outcome value. 
      \item \xmlAttr{index}, \xmlDesc{integer, required field}, the index of steady state probabilities corresponding to the transition matrix. 
    \end{itemize}
  \item \xmlNode{state}, \xmlDesc{required xml node}. The output from this state indicates
    the probability for outcome 2.
    In this node, the following attribute should be specified:
    \begin{itemize}
      \item \xmlAttr{outcome}, \xmlDesc{float, required field}, outcome value. 
      \item \xmlAttr{index}, \xmlDesc{integer, required field}, the index of steady state probabilities corresponding to the transition matrix. 
    \end{itemize}
  \item ...
  \item \xmlNode{state}, \xmlDesc{required xml node}. The output from this state indicates
    the probability for outcome K.
    In this node, the following attribute should be specified:
    \begin{itemize}
      \item \xmlAttr{outcome}, \xmlDesc{float, required field}, outcome value. 
      \item \xmlAttr{index}, \xmlDesc{integer, required field}, the index of steady state probabilities corresponding to the transition matrix. 
    \end{itemize}

\end{itemize}

\textbf{Example:}

\begin{lstlisting}[style=XML]
<Simulation>
 ...
  <Distributions>
    ...
    <MarkovCategorical name="x_dist">
        <!--dataFile fileType='csv'>transitionFile</dataFile-->
        <transition>
            -1.1   0.8   0.7
            0.8    -1.4  0.2
            0.3    0.6   -0.9
        </transition>
        <state outcome='1' index='1'/>
        <state outcome='2' index='2'/>
        <state outcome='4' index='3'/>
    </MarkovCategorical>
    ...
  </Distributions>
 ...
</Simulation>
\end{lstlisting}



%%%%%% N-Dimensional Probability distributions
\subsection{N-Dimensional Probability Distributions}
\label{subsec:NdDist}
The group of $N$-Dimensional distributions allow the user to model stochastic dependences between parameters. Thus instead of using $N$ distributions for $N$ parameters, the user can define a single distribution lying in a $N$-Dimensional space.
The following $N$-Dimensional Probability Distributions are available within RAVEN:
\begin{itemize}
\item MultivariateNormal: Multivariate normal distribution (see Section~\ref{MultivariateNormal})
\item NDInverseWeight: ND Inverse Weight interpolation distribution (see Section~\ref{NDInverseWeight})
\item NDCartesianSpline: ND spline interpolation distribution (see Section~\ref{NDCartesianSpline})
\end{itemize}
For NDInverseWeight and NDCartesianSpline distributions, the user provides the sampled values of either CDF or PDF of the distribution. The sampled values can be scattered distributed (for NDInverseWeight) or over a cartesian grid (for NDCartesianSpline).

The user could specify, for each $N$-Dimensional distribution, the parameters of the random number generator function:
\begin{itemize}
\item \xmlNode{initialGridDisc}, \xmlDesc{positive integer, optional field}, user-defined initial grid discretization. This parameter specifies the number of discretizations that need to be performed, initially, for each Dimension in
order to find N-Dimensional coordinate that corresponds to the CDF represented by a random number (0-1);
\item \xmlNode{tolerance}, \xmlDesc{float, optional field}, user-defined tolerance in order to find the N-D coordinates corresponding to a random number. This tolerance is expressed in terms of CDF.
\end{itemize}
in the \xmlNode{samplerInit} block defined in sampler block \xmlNode{samplerInit} (see Section~\ref{sec:Samplers}).

\subsubsection{MultivariateNormal Distribution}
\label{MultivariateNormal}
the multivariate normal distribution or multivariate Gaussian distribution, is a generalization of the one-dimensional (univariate) normal distribution to higher dimensions.
The multivariate normal distribution is often used to describe, at least approximately, any set of (possibly) correlated real-valued random variables each of which clusters around a mean value.
The multivariate normal distribution of a $k$-dimensional random vector $\mathbf{x} = [x_1, x_2, …, x_k]$  can be written in the following notation:
$ \mathbf{x}\ \sim\ \mathcal{N}(\boldsymbol\mu,\, \boldsymbol\Sigma)$
with with $k$-dimensional mean vector

$\boldsymbol\mu= [E[x_1], E[x_2], …, E[x_k]]$

and $k \times k$ covariance matrix

$\boldsymbol\Sigma = [Cov[x_i,x_j]] , i=1,2,\ldots,k ; j=1,2,\ldots,k$

The probability distribution function for this distribution is the following:

$
f_{\mathbf x}(x_1,\ldots,x_k) =
\frac{1}{\sqrt{(2\pi)^k|\boldsymbol\Sigma|}}
\exp\left(-\frac{1}{2}({\mathbf x}-{\boldsymbol\mu})^\mathrm{T}{\boldsymbol\Sigma}^{-1}({\mathbf x}-{\boldsymbol\mu})
\right),
$

The specifications of this distribution must be defined within the xml block \xmlNode{MultivariateNormal}.
This XML node needs to contain the attributes:
\vspace{-5mm}
\begin{itemize}
\itemsep0em
\item \xmlAttr{name}, \xmlDesc{required string attribute}, user-defined identifier of this multivariate normal distribution.
%
\nb As with other objects, this is the name that can be used to refer to this specific entity from other input XML blocks.
\item \xmlAttr{method}, \xmlDesc{required string attribute}, defines which method is used to generate the multivariate normal distribution.
The only allowable methods are \xmlString{spline} and \xmlString{pca}.
%
\end{itemize}
\vspace{-5mm}

In RAVEN the MultivariateNormal distribution can be initialized through the following keywords:
\begin{itemize}
  \item \xmlNode{mu}, list of mean values of each dimension
  \item \xmlNode{covariance}, list of element values in the covariance matrix. There are two types of \xmlNode{covariance}, based on the \xmlAttr{type}:
  \begin{itemize}
    \item \xmlAttr{type}, \xmlDesc{string, optional field}, specifies the type of covariance, the default \xmlAttr{type} is \xmlString{abs}. Possible values for \xmlAttr{type} are \xmlString{abs} and \xmlString{rel}. \nb \xmlString{abs} indicates the covariance is a normal covariance matrix, while \xmlString{rel} indicates the covariance is a relative covariance matrix. In addition, method \xmlString{pca} can be combined with both types, and method \xmlString{spline} only accept the type \xmlString{abs}
  \end{itemize}
  \item \xmlNode{transformation}, \xmlDesc{XML node, optional field}, option to enable input parameter transformation using principal component analysis (PCA) approach. If this node is provided, PCA will be used to compute the principal components of input covariance matrix. The subnode \xmlNode{rank} is used to indicate the number of principal components that will be used for the input transformation. The content will specify one attribute:
  \begin{itemize}
    \item \xmlNode{rank}, \xmlDesc{positive integer, required field}, user-defined dimensionality reduction.
  \end{itemize}
\end{itemize}

Example:
\begin{lstlisting}[style=XML]
<Distributions>
  ...
    <MultivariateNormal name='MultivariateNormal_test' method='spline'>
        <mu>0.0 60.0</mu>
        <covariance>
	1.0 0.7
	0.7 1.0
        </covariance>
    </MultivariateNormal>
    <MultivariateNormal name='MultivariateNormal_abs' method='pca'>
        <mu>0.0 60.0</mu>
        <covariance type='abs'>
	1.0 0.7
	0.7 1.0
        </covariance>
    </MultivariateNormal>
    <MultivariateNormal name='MultivariateNormal_rel' method='pca'>
        <mu>0.0 60.0</mu>
        <covariance type='rel'>
	1.0 0.7
	0.7 1.0
        </covariance>
    </MultivariateNormal>
  ...
</Distributions>
\end{lstlisting}

In the following, we defined a distribution with a transformation node using PCA method. The number of principal components is
defined in \xmlNode{rank}. In this distribution, PCA is employed to restruct the multivariate normal distribution. In addition,
the size of uncorrelated variables is also determined by \xmlNode{rank}.

\begin{lstlisting}[style=XML]
<Distributions>
  ...
    <MultivariateNormal name='MultivariateNormal_test' method='pca'>
        <mu>0.0 10.0 20.0</mu>
        <covariance type="abs">
           1.0   0.7   -0.2
           0.7   1.0   0.4
           -0.2  0.4   1.0
        </covariance>
        <transformation>
          <rank>2</rank>
        </transformation>
    </MultivariateNormal>
  ...
</Distributions>
\end{lstlisting}

\subsubsection{NDInverseWeight Distribution}
\label{NDInverseWeight}
The NDInverseWeight distribution creates a $N$-Dimensional distribution given a set of points
scattered distributed. These points sample the PDF of the original distribution.
Distribution values (PDF or CDF) are calculated using the inverse weight
interpolation scheme.

\specBlock{a}{NDInverseWeight}
%
\attrsIntro
\vspace{-5mm}
\begin{itemize}
\itemsep0em
\item \nameDescription
\end{itemize}
\vspace{-5mm}


In RAVEN the NDInverseWeight distribution can be initialized through the following nodes:
\begin{itemize}
\item \xmlNode{p}, \xmlDesc{float, required parameter}, power parameter. Greater values of p assign greater influence to values closest to the interpolated point.
\item \xmlNode{data\textunderscore filename}, \xmlDesc{string, required parameter},  name of the data file containing scattered values (file type '.txt').
\begin{itemize}
\item \xmlAttr{type}, \xmlDesc{required string attribute},  indicates if the data in indicated file is PDF or CDF.
\end{itemize}
\item \xmlNode{working\textunderscore dir}, \xmlDesc{string, required parameter}, folder location of the data file
\end{itemize}

Example:
\begin{lstlisting}[style=XML]
<Distributions>
  ...
  <NDInverseWeight name='...'>
        <p>...</p>
        <dataFilename type='...'>...</dataFilename>
        <workingDir>...</workingDir>
  </NDInverseWeight>
  ...
</Distributions>
\end{lstlisting}

Each data entry contained in data\textunderscore filename is listed row by row and must be listed as follows:
\begin{itemize}
\item number of dimensions
\item number of sampled points
\item ND coordinate of each sampled point
\item value of each sampled point
\end{itemize}

As an example, the following shows the data entries contained in data\textunderscore filename for a 3-dimensional data set that contained two sampled CDF values: ([0.0,0.0,0.0], 0.1) and ([1.0, 1.0,0.0], 0.8)

Example scattered data file:
\begin{lstlisting}
3
2
0.0
0.0
0.0
1.0
1.0
0.0
0.1
0.8
\end{lstlisting}

\subsubsection{NDCartesianSpline Distribution}
\label{NDCartesianSpline}

The NDCartesianSpline distribution creates a $N$-Dimensional distribution given a set of points
regularly distributed on a cartesian grid. These points sample the PDF of the original distribution.
Distribution values (PDF or CDF) are calculated using the ND spline
interpolation scheme.


\specBlock{a}{NDCartesianSpline}
%
\attrsIntro
\vspace{-5mm}
\begin{itemize}
\itemsep0em
\item \nameDescription
\end{itemize}
\vspace{-5mm}


In RAVEN the NDCartesianSpline distribution can be initialized through the following nodes:
\begin{itemize}
\item \xmlNode{data\textunderscore filename}, \xmlDesc{string, required parameter},  name of the data file containing scattered values (file type '.txt').
\begin{itemize}
\item \xmlAttr{type}, \xmlDesc{required string attribute}, indicates if the data in indicated file is PDF or CDF.
\end{itemize}
\item \xmlNode{working\textunderscore dir}, \xmlDesc{string, required parameter}, folder location of the data file
\end{itemize}

Example:
\begin{lstlisting}[style=XML]
<Distributions>
  ...
  <NDCartesianSpline name='...'>
        <dataFilename type='...'>...</dataFilename>
        <workingDir></workingDir>
  </NDCartesianSpline>
  ...
</Distributions>
\end{lstlisting}

Each data entry contained in data \textunderscore filename is listed row by row and must be listed as follows:
\begin{itemize}
\item number of dimensions
\item number of discretization for each dimension
\item discretization values for each dimension
\item value of each sampled point
\end{itemize}

As an example, the following shows the data entries contained in data \textunderscore filename for a 2-dimensional CDF data set on the following grid $(x,y)$:
\begin{itemize}
\item first dimension (x): -0.5, 0.5
\item first dimension (y): 1.0 2.0 3.0
\end{itemize}

Example scattered data file:
\begin{lstlisting}
2
2
3
-0.5
0.5
1.0
2.0
3.0
CDF value of (-0.5,1.0)
CDF value of (+0.5,1.0)
CDF value of (-0.5,2.0)
CDF value of (+0.5,2.0)
CDF value of (-0.5,3.0)
CDF value of (+0.5,3.0)
\end{lstlisting}

\section{Samplers}
\label{sec:Samplers}

%%%%%%%%%%%%%%%%%%%%%%%%%%%%%%%%%%%%%%%%%%%%%%%%%%%%%%%%%%%%%%%%%%%%%%%%%%%%%%%%
% If you are confused by the input of this document, please make sure you see
% these defined commands first. There is no point writing the same thing over
% and over and over and over and over again, so these will help us reduce typos,
% by just editing a template sentence or paragraph.
\renewcommand{\nameDescription}
{
  \xmlAttr{name},
  \xmlDesc{required string attribute}, user-defined name of this sampler.
  \nb As with other objects, this identifier can be used to reference this
  specific entity from other input blocks in the XML.
}
\newcommand{\shapeVariableDescription}
{
  \xmlAttr{shape},
  \xmlDesc{comma-separated integers, optional field},
  determines the number of samples and shape of samples
  to be taken.  For example, \xmlAttr{shape}=``2,3'' will provide a 2 by 3
  matrix of values, while \xmlAttr{shape}=``10'' will produce a vector of 10 values.
  Omitting this optional attribute will result in a single scalar value instead.
  Each of the values in the matrix or vector will be the same as the single sampled value.
  \nb A model interface must be prepared to handle non-scalar inputs to use this option.
}
\newcommand{\shapeConstantDescription}
{
  \xmlAttr{shape},
  \xmlDesc{comma-separated integers, optional field},
  determines the shape of samples of the constant value.
  For example, \xmlAttr{shape}=``2,3'' will shape the values into a 2 by 3
  matrix, while \xmlAttr{shape}=``10'' will shape into a vector of 10 values.
  Unlike the \xmlNode{variable}, the constant requires each value be entered; the number
  of required values is equal to the product of the \xmlAttr{shape}.
  \nb A model interface must be prepared to handle non-scalar inputs to use this option.
}

\renewcommand{\specBlock}[2]
{
  The specifications of this sampler must be defined within #1 \xmlNode{#2} XML
  block.
}
\newcommand{\variableChildIntro}
{
 This \xmlNode{variable} recognizes the following child node:
}

\newcommand{\variableChildrenIntro}
{
  This \xmlNode{variable} recognizes the following child nodes:
}

\newcommand{\variableIntro}[1]
{
  In the \xmlNode{#1} input block, the user
  needs to specify the variables to sample.
  %
  As already mentioned, these variables are specified within consecutive
  \xmlNode{variable} XML blocks:
}

\newcommand{\constructionGridDescriptionOnlyCustom}
{
Based on the \xmlAttr{construction} type, the content of the \xmlNode{grid}
XML node and the requirements for other attributes change. In this case, only the following is available:
\begin{itemize}
  \item \xmlAttr{construction}\textbf{\texttt{=}}\xmlString{custom}.
    The grid will be directly specified by the user.
    This construction type requires that the \xmlNode{grid} node contains
    the actual mesh bins.
    For example, if the grid \xmlAttr{type} is \xmlString{CDF}, in the body
    of \xmlNode{grid}, the user will specify the CDF probability thresholds
    (nodalization in probability).
      All the bins are checked against the associated
      \xmlNode{distribution} bounds.
      If one or more of them falls outside the distribution's bounds, the
      code will raise an error.
      No additional attributes are needed.
\end{itemize}
}

\newcommand{\constructionGridDescription}
{
Based on the \xmlAttr{construction} type, the content of the \xmlNode{grid}
XML node and the requirements for other attributes change:
\begin{itemize}
  \item \xmlAttr{construction}\textbf{\texttt{=}}\xmlString{equal}.
    The grid is going to be constructed equally-spaced
    (\xmlAttr{type}\textbf{\texttt{=}}\xmlString{value}) or equally probable
    (\xmlAttr{type}\textbf{\texttt{=}}\xmlString{CDF}).
    This construction type requires the definition of additional attributes:
      \begin{itemize}
         \item \xmlAttr{steps}, \xmlDesc{required integer attribute}, number
           of equally spaced/probable discretization steps.
      \end{itemize}
      This construction type requires that the content of the \xmlNode{grid}
      node represents the lower and upper bounds (either
      in probability or value). Two values need to be specified; the lowest one
     will be considered as the $lowerBound$, the largest, the $upperBound$.
      The lower and upper bounds are checked against the associated
      \xmlNode{distribution} bounds.
      If one or both of them falls outside the distribution's bounds, the
      code will raise an error.
      The $stepSize$ is determined as follows:
      \\ $stepSize=(upperBound - lowerBound)/steps$
  \item \xmlAttr{construction}\textbf{\texttt{=}}\xmlString{custom}.
    The grid will be directly specified by the user.
    No additional attributes are needed.
    This construction type requires that the \xmlNode{grid} node contains
    the actual mesh bins.
    For example, if the grid \xmlAttr{type} is \xmlString{CDF}, in the body
    of \xmlNode{grid}, the user will specify the CDF probability thresholds
    (nodalization in probability).
      All the bins are checked against the associated
      \xmlNode{distribution} bounds.
      If one or more of them falls outside the distribution's bounds, the
      code will raise an error.
\end{itemize}
}
\newcommand{\variableDescription}
{
  \xmlNode{variable}, \xmlDesc{XML node,
  required parameter} can specify the following attribute:
  \begin{itemize}
    \item \xmlAttr{name}, \xmlDesc{required string attribute}, user-defined name
      of this variable.
    \item \shapeVariableDescription
  \end{itemize}
}

\newcommand{\constantVariablesDescription}
{
  \xmlNode{constant}, \xmlDesc{XML node,
  optional parameter} the user is able to input variables that need to be
  kept constant.
  For doing this, as many \xmlNode{constant} nodes as needed can be
  input, where the body of the node contains the constant value that is
  going to be injected as an additional variable.  The constant has the following attributes:
  \begin{itemize}
    \item \xmlAttr{name}, \xmlDesc{required string attribute}, user-defined name
      of this constant.
    \item \shapeConstantDescription
  \end{itemize}
}

\newcommand{\distributionDescription}
{
  \xmlNode{distribution}, \xmlDesc{string,
  required field}, name of the distribution that is associated to this variable.
  Its name needs to be contained in the \xmlNode{Distributions} block explained
  in Section \ref{sec:distributions}. In addition, if NDDistribution is used,
  the attribute \xmlAttr{dim} is required. \nb{Alternatively, this node must be omitted
  if the \xmlNode{function} node is supplied.}
}
\newcommand{\functionDescription}
{
  \xmlNode{function}, \xmlDesc{string, required field}, name of the function that
  defines the calculation of this variable from other distributed variables.  Its name
  needs to be contained in the \xmlNode{Functions} block explained in Section
  \ref{sec:functions}. This function must implement a method named ``evaluate''.
 \nb{Alternatively, this node must be ommitted
  if the \xmlNode{distribution} node is supplied.}
}




\newcommand{\gridDescriptionOnlyCustom}
{
  \xmlNode{grid}, \xmlDesc{space separated floats, required
  field}, the content of this XML node depends on the definition of the
  associated attributes:
  \begin{itemize}
  \itemsep0em
    \item \xmlAttr{type}, \xmlDesc{required string attribute}, user-defined
      discretization metric type: 1) \xmlString{CDF}, the grid will be
      specified based on cumulative distribution function probability
      thresholds, and 2) \xmlString{value}, the grid will be provided
      using variable values.
    \item \xmlAttr{construction}, \xmlDesc{required string attribute}, how
      the grid needs to be constructed, independent of its type (i.e.
      \xmlString{CDF} or \xmlString{value}).
  \end{itemize}
  \constructionGridDescriptionOnlyCustom
  \nb{The \xmlNode{grid} node is only required if a \xmlNode{distribution}
  node is supplied.  In the case of a \xmlNode{function} node, no grid
  information is requested.}
}



\newcommand{\gridDescription}
{
  \xmlNode{grid}, \xmlDesc{space separated floats, required
  field}, the content of this XML node depends on the definition of the
  associated attributes:
  \begin{itemize}
  \itemsep0em
    \item \xmlAttr{type}, \xmlDesc{required string attribute}, user-defined
      discretization metric type: 1) \xmlString{CDF}, the grid will be
      specified based on cumulative distribution function probability
      thresholds, and 2) \xmlString{value}, the grid will be provided
      using variable values.
    \item \xmlAttr{construction}, \xmlDesc{required string attribute}, how
      the grid needs to be constructed, independent of its type (i.e.
      \xmlString{CDF} or \xmlString{value}).
  \end{itemize}
  \constructionGridDescription
  \nb{The \xmlNode{grid} node is only required if a \xmlNode{distribution}
  node is supplied.  In the case of a \xmlNode{function} node, no grid
  information is requested.}
}

\newcommand{\convergenceDescription}
{
\xmlNode{Convergence}, \xmlDesc{float, required field}, Convergence
    tolerance.
    %
    The meaning of this tolerance depends on the definition of other attributes
    that might be defined in this XML node:
    \begin{itemize}
      \item \xmlAttr{limit}, \xmlDesc{optional integer attribute}, the
        maximum number of adaptive samples (iterations).
        %
        \default{infinite}.
      \item \xmlAttr{forceIteration}, \xmlDesc{optional boolean attribute},
        this attribute controls if at least a number of iterations equal to
        \textbf{limit} must be performed.
        %
        \default{False}.
      \item \xmlAttr{weight}, \xmlDesc{optional string attribute (case insensitive)}, defines on
        what the convergence check needs to be performed.
        \begin{itemize}
          \item \xmlString{CDF}, the convergence is checked in terms
            of probability (Cumulative Distribution Function). From a practical point of view,
            this means that full uncertain domain
            is discretized in a way that the probability volume of each cell is going to be equal to
           the tolerance specified in the body of the node \xmlNode{Convergence}
          \item \xmlString{value}, the convergence is checked on the
            hyper-volume in terms of variable values.From a practical point of view,
            this means that full uncertain domain
            is discretized in a way that the ``volume'' fraction of each cell is going to be equal to
           the tolerance specified in the body of the node \xmlNode{Convergence}. In other words,
           each cell volume is going to be equal to the total volume times the tolerance.
        \end{itemize}
        \default{CDF}.
      \item \xmlAttr{persistence}, \xmlDesc{optional integer attribute},
        offers an additional convergence check.
        %
        It represents the number of times the computed error needs to be
        below the inputted tolerance before convergence is reported.
        %
        \default{5}.
        \item \xmlAttr{subGridTol}, \xmlDesc{optional float attribute},
            this attribute is used to activate the multi-grid approach (adaptive meshing)
            of the constructed evaluation grid (see attribute \xmlAttr{weight}).
            In case this attribute is specified, the final grid discretization (cell's ``volume content''
             aka convergence confidence) is represented by the
            value here specified. The sampler converges on the initial coarse grid, defined by
            the tolerance specified in the body of the node \xmlNode{Convergence}.
            When the Limit Surface has been identified on the coarse grid, the sampler starts
            refining the grid until the ``volume content'' of each cell is equal to the value
            specified in this attribute (Multi-grid approach).
           \default{None}.
    \end{itemize}
    In summary, this XML node contains the information that is needed in order
    to control this sampler's convergence criterion.
}

\newcommand{\assemblerDescription}[1] %%NOTE this only applies to the adaptive sampler.  Why is it a newcommand? %%
{
  \textbf{Assembler Objects} These objects are either required or optional
    depending on the functionality of the #1 Sampler.
    %
    The objects must be listed with a rigorous syntax that, except for the XML
    node tag, is common among all the objects.
    %
    Each of these nodes must contain 2 attributes that are used to identify them
    within the simulation framework:
    \begin{itemize}
      \item \xmlAttr{class}, \xmlDesc{required string attribute}, the main
        ``class'' of the listed object.
        %
        For example, it can be \xmlString{Models}, \xmlString{Functions}, etc.
      \item \xmlAttr{type},  \xmlDesc{required string attribute}, the object
        identifier or sub-type.
        %
        For example, it can be \xmlString{ROM}, \xmlString{External}, etc.
    \end{itemize}
    The \textbf{#1} approach requires or optionally accepts the
    following object types:

}
\newcommand{\ROMDescription}[1]
{
    \begin{itemize}
      \item \xmlNode{ROM}, \xmlDesc{string, required field}, the
        body of this XML node must contain the name of an appropriate ROM defined in the
        \xmlNode{Models} block (see Section~\ref{subsec:models_ROM}).
    \end{itemize}
}

\newcommand{\restartDescription}[1]
{
    \begin{itemize}
      \item \xmlNode{Restart}, \xmlDesc{string, optional field}, the
        body of this XML node must contain the name of an appropriate \textbf{DataObject} defined in the
        \xmlNode{DataObjects} block (see Section~\ref{sec:DataObjects}).  It is used as a
        ``restart'' tool, where it accepts pre-existing solutions in the PointSet instead
        of recalculating solutions.
    \end{itemize}

    The following node is an additional option when a restart DataObject is
    provided:

    \begin{itemize}
      \item \xmlNode{restartTolerance}, \xmlDesc{float, optional field}, the
        body of this XML node must contain a valid floating point value.  If a \xmlNode{Restart} node is
        supplied for this \xmlNode{Sampler}, this node offers a way to determine how strictly matching points
        are determined.  Given a point in the input space, if that point is within a relative Euclidean
        distance (equal to the tolerance) of a restart point, the nearest restart point will be used.
        \default{1e-15}
    \end{itemize}
}

\newcommand{\variablesTransformationDescription}[1]
{
    \begin{itemize}
      \item \xmlNode{variablesTransformation}, \xmlDesc{optional field}. this XML node accepts one attribute:
      \begin{itemize}
        \item \xmlAttr{distribution}, \xmlDesc{required string attribute}, the name for the distribution defined in the XML node \xmlNode{Distributions}.
        This attribute indicates the values of \xmlNode{manifestVariables} are drawn from \xmlAttr{distribution}.
      \end{itemize}
      In addition, this XML node also accepts three childen nodes:
      \begin{itemize}
        \item \xmlNode{latentVariables}, \xmlDesc{comma separated string, required field}, user-defined latent variables that
        are used for the variables transformation. All the variables listed under this node should be also mentioned in \xmlNode{variable}.
        \item \xmlNode{manifestVariables}, \xmlDesc{comma separated string, required field}, user-defined manifest variables
        that can be used by the \xmlAttr{model}.
        \item \xmlNode{manifestVariablesIndex}, \xmlDesc{comma separated string, optional field}, user-defined manifest variables indices paired with \xmlNode{manifestVariables}.
        These indices indicate the position of manifest variables associated with multivariate normal distribution defined in the XML node \xmlNode{Distributions}.
        The indices should be postive integer. If not provided, the code will use the positions of manifest variables listed in \xmlNode{manifestVariables} as the indices.
        \item \xmlNode{method}, \xmlDesc{string, required field}, the method that is used for the variables transformation. The currently available method is '\textbf{pca}'.
      \end{itemize}
    \end{itemize}
}

\newcommand{\convergenceStudyDescription}
{
    \xmlNode{convergenceStudy}, \xmlDesc{optional node},
    if included, triggers writing state points at particular numbers of model solves for the purpose of
    a convergence study.  The study is performed by writing XML output files as described in the
    OutStreams for ROMs at the state points requested, using \xmlString{all} as the requested
    \xmlNode{what} values.
    The state points are identified when a certain
    number of model runs is passed, as specified by the \xmlNode{runStatePoints} node.
    This node has the following sub-nodes to define its parameters:
    \begin{itemize}
      \item \xmlNode{runStatePoints}, \xmlDesc{list of integers, required node},
        lists the number of model runs at which state points should be written. Note that these will be
        written when the requested number of runs is met or passed, so the actual value is often somewhat more
        than the requested value, and the exact value will be listed in the XML output.
      \item \xmlNode{baseFilename}, \xmlDesc{string, optional node},
        if specified determines the base file name for the state point outputs.  If not specified, defaults to
        \xmlString{out\_}.
      \item \xmlNode{pickle}, \xmlDesc{no text, optional node},
        if this node is included, serialized (pickled) versions of the ROM at each of the run states is also
        created in the working directory, with the format \texttt{<baseFilename><numRuns>.pk}, such as
        \texttt{out\_100.pk}.
    \end{itemize}
}
%%%%%%%%%%%%%%%%%%%%%%%%%%%%%%%%%%%%%%%%%%%%%%%%%%%%%%%%%%%%%%%%%%%%%%%%%%%%%%%%

The sampler is probably the most important entity in the RAVEN framework.
%
It performs the driving of the specific sampling strategy and, hence, determines
the effectiveness of the analysis, from both an accuracy and computational point
of view.
%
The samplers, that are available in RAVEN, can be categorized into three main
classes:
\begin{itemize}
\item \textbf{Forward} (see Section~\ref{subsec:onceThroughSamplers})
\item \textbf{Dynamic Event Tree (DET)} (see Section~\ref{subsec:DETSamplers})
\item \textbf{Adaptive} (see Section~\ref{subsec:AdaptSamplers})
\end{itemize}
Before analyzing each sampler in detail, it is important to mention that each
type has a similar syntax to input the variables to be ``sampled''.
%
In the example below, the variable \xmlString{variableName} is going to be
sampled by the Sampler \xmlString{whatever} using the distribution named\\
\xmlString{aDistribution}.
\begin{lstlisting}[style=XML]
<Simulation>
  ...
  <Samplers>
    ...
    <WhatEverSampler name='whatever'>
      ...
     <variable name='variableName'>
       ...
       <distribution>aDistribution</distribution>
       ...
     </variable>
      ...
    </WhatEverSampler>
    ...
  </Samplers>
  ...
</Simulation>
\end{lstlisting}

As reported in section \ref{sec:existingInterface}, the variable naming syntax,
for external driven codes, depends on the way the ``code interface'' has been
implemented.
%
For example, if the code has an input structure like the one reported below (YAML), the
variable name might be\xmlString{I-Level|II-Level|variable}.
%
In this way, the relative code interface (and input parser) will know which
variable needs to be perturbed and the ``recipe'' to access it.
%
As reported in \ref{sec:existingInterface}, its syntax is chosen by the
developer of the ``code interface'' and is implemented in the interface only
(no modifications are needed in the RAVEN code).

%\maljdan{Where does this type of input come from? Should the user care?}
%\alfoa{Dan, this is an example (in this case, a YAML structure)}

Example YAML based Input:
\begin{lstlisting}
[I-Level]
  [./II-Level]
    variable = xxx
  [../]
[]
\end{lstlisting}

Example XML block to define the variables and associated distributions:
\begin{lstlisting}[style=XML]
<variable name='I-Level|II-Level|variable'>
  <distribution>exampleDistribution</distribution>
</variable>
\end{lstlisting}

If the variable is associated to a multi-dimensional ND distribution, it is needed to specify which dimension of the ND distribution is associated to such variable. An example is shown below: the variable  ``variableX'' is associated to the third dimension of the ND distribution ``NDdistribution''.

\begin{lstlisting}[style=XML]
<variable name='variableX'>
     <distribution dim='3'>NDdistribution</distribution>
</variable>
\end{lstlisting}

For most codes, it is prudent that there are no redundant inputs; however there are
cases where this is not reality.  For example, if there is a variable \xmlString{inner\_radius} and
a variable \xmlString{outer\_radius}, there may be a third variable \xmlString{thickness} that
is actually derived from the previous two, as \xmlString{thickness} = \xmlString{outer\_radius} - \xmlString{inner\_radius}.
RAVEN supports this type of redundant input through a Function entity.  In this case,
instead of a \xmlNode{distribution} node in the \xmlNode{variable} block, there is a
\xmlNode{function} node, specifying the name of the function (defined in the \xmlNode{Functions} block).
In order to work properly, this function must have a method named ``evaluate''
that returns a single python float object. In this way, multiple variables can be associated with the same function.  For example,
\begin{lstlisting}[style=XML]
...
<Functions>
  <External name='torus_calcs' file='torus_calcs.py'>
    <variable>outer_radius</variable>
    <variable>inner_radius</variable>
  </External>
<Functions>
...
<Samplers>
  <WhatEverSampler name='myExampleSampler'>
    <variable name='inner_radius'>
      <distribution>inner_dist</distribution>
    </variable>
    <variable name='outer_radius'>
      <distribution>outer_dist</distribution>
    </variable>
    <variable name='thickness'>
      <function>torus_calcs</function>
    </variable>
  </WhatEverSampler>
</Samplers>
\end{lstlisting}
The corresponding function file \xmlString{torus\_calcs.py} needs the following method:
\begin{lstlisting}
def evaluate(self):
  return self.outer_radius - self.inner_radius
\end{lstlisting}
The \xmlString{thickness} parameter will still be treated as an input for the sake of csv
printing and DataObjects storage.
\\\nb It is important to notice that if the user use variables with no-Python compatible names (e.g. parenthesis, etc.),
the \xmlNode{alias} system needs to be used to alias the variables.

In the sampler class a special node exists: the \xmlNode{sampler\textunderscore init} node.
This node contains specific parameters that characterize each particular sampler.
In addition, \xmlNode{sampler\textunderscore init} might contain the information regarding the random generator function for each $N$-Dimensional distribution (specified in the \xmlNode{dist\textunderscore init} node):
\begin{itemize}
\item initial\textunderscore grid\textunderscore disc
\item tolerance
\end{itemize}

An example of \xmlNode{dist\textunderscore init} node is provided below:

\begin{lstlisting}[style=XML]
<distInit>
    <distribution name= 'ND_dist_name'>
         <initialGridDisc>5</initialGridDisc>
          <tolerance>0.2</tolerance>
     </distribution>
  </distInit>
\end{lstlisting}

In the \xmlNode{sampler\textunderscore init}  node it is possible to add also the subnode \xmlNode{globalGrid}.
The \xmlNode{globalGrid} can be used in two cases:
\begin{itemize}
\item 1D distributions: an identical grid that is associated to several distributions
\item ND distribution: a grid associated to a single ND distribution. This is the case when a stratified sampling is performed on the CDF of an ND distribution: the  \xmlNode{globalGrid} is  shared among the variables associated to the Nd distribution
\end{itemize}

%%%%%%%%%%%%%%%%%%%%%%%%%
%%%      Forward Samplers      %%%
%%%%%%%%%%%%%%%%%%%%%%%%%
\subsection{Forward Samplers}
\label{subsec:onceThroughSamplers}
The Forward sampler category collects all the strategies that perform the
sampling of the input space without exploiting, through dynamic learning
approaches, the information made available from the outcomes of calculations
previously performed (adaptive sampling) and the common system evolution
(patterns) that different sampled calculations can generate in the phase space
(dynamic event tree).
%
In the RAVEN framework, several different ``Forward'' samplers
are available:
\begin{itemize}
\item \textbf{Monte Carlo (MC)}
\item \textbf{Stratified}
\item \textbf{Grid Based}
\item \textbf{Sparse Grid Collocation}
\item \textbf{Sobol Decomposition}
\item \textbf{Response Surface Design of Experiment}
\item \textbf{Factorial Design of Experiment}
\item \textbf{Ensemble Forward Sampling strategy}
\item \textbf{Custom Sampling strategy}
\end{itemize}

From a practical point of view, these sampling strategies represent different
ways to explore the input space.
%
In the following paragraphs, the input requirements and a small explanation of
the different sampling methodologies are reported.


%%% Forward Samplers: MonteCarlo
\subsubsection{Monte Carlo}
\label{subsubsubsec:MC}
The \textbf{Monte-Carlo} sampling approach is one of the most well-known and
widely used approaches to perform exploration of the input space.
%
The main idea behind MonteCarlo sampling is to  randomly perturb the input space according
to uniform or parameter-based probability density functions.
%

\specBlock{a}{MonteCarlo}
%
\attrsIntro
\vspace{-5mm}
\begin{itemize}
\itemsep0em
\item \textbf{name}, \textit{required string attribute}, user-defined name of this Sampler. N.B. As for the other objects, this is the name that can be used to refer to this specific entity from other input blocks (xml);
%\item \textbf{limit}, \textit{required integer attribute}, number of MonteCarlo samples needs to be generated;
%\item \textbf{initial\_seed}, \textit{optional integer attribute}, initial seeding of random number generator. \textit{Default = random seed};
%\item \textbf{reseedEachIteration}, \textit{optional boolean/string attribute}, perform a re-seeding for each sample generated (True values = True, yes, y, t). \textit{Default = False};
\end{itemize}
\vspace{-5mm}

In the \textbf{MonteCarlo} input block, the user needs to specify the variables need to be sampled. As already mentioned, these variables are inputted within consecutive xml blocks called \xmlNode{variable}. In addition, the settings for this sampler need to be specified in the \xmlNode{samplerInit} XML block:
\begin{itemize}
\item \xmlNode{samplerInit},  \textit{\textbf{XML node, required parameter}}. In this xml-node,the following xml sub-nodes need to be specified:
  \begin{itemize}
    \item \xmlNode{limit}, \textit{\textbf{integer,required field}}, number of MonteCarlo samples needs to be generated;
    \item \xmlNode{initialSeed}, \textit{\textbf{integer, optional field}}, initial seeding of random number generator
    \item \xmlNode{reseedEachIteration},  \textit{\textbf{boolean/string(case insensitive), optional field}}, perform a re-seeding for each sample generated (True values = True, yes, y, t). \default{False};
    \item \xmlNode{distInit},  \textit{\textbf{integer, optional field}}, in this node the user specifies the initialization of the random number generator function for each N-Dimensional Probability Distributions (see Section~\ref{subsec:NdDist}).
    \item \xmlNode{samplingType}, \textit{\textbf{string, optional field}}, sub-type of sampling \default{None}. the user can choose to perform a Monte-Carlo sampling where the location of the samples in
the input space is uniformly distributed and not generated accordingly to the specific set of distributions. This can be specificed
in the \xmlNode{samplingType} with the kewyword ``uniform''. This option works only if all the distributions have an upper and lower
bound specified (i.e., \xmlNode{lowerBound} and \xmlNode{upperBound}). Allowed fields for this node are ``None'' and ``uniform''.
  \end{itemize}
\end{itemize}
\begin{itemize}
\item \variableDescription
 \variableChildrenIntro
 \begin{itemize}
    \item \distributionDescription
    \item \functionDescription
 \end{itemize}
 \item \constantVariablesDescription
\end{itemize}
%\end{itemize}

If the input parameters are correlated, the \textbf{MonteCarlo} sampling approach can be also used if the user specified a
multivariate distributions inside the \xmlNode{Distributions} (see Section \ref{subsec:NdDist}). Furthermore, if the
covariance matrix is provided and the input parameters is assumed to have the multivariate normal distribution, one can also use
\textbf{MonteCarlo} approach to sample the input parameters in the transformed space (aka subspace, reduced space). If this is
the case, the user needs to provide additional information, i.e. the \xmlNode{transformation} under \xmlNode{MultivariateNormal} of
\xmlNode{Distributions} (more information can be found in Section \ref{subsec:NdDist}). In addition, the node
\xmlNode{variablesTransformation} is also required for \textbf{MonteCarlo} sampling. This node is used to tranform the variables
specified by \xmlNode{latentVariables} in the transformed space of input into variables spefified by \xmlNode{manifestVariables}
in the input space. The variables listed in \xmlNode{latentVariables} should be predefined in \xmlNode{variable}, and the variables
listed in \xmlNode{manifestVariables} are used by the \xmlNode{Models}.

\variablesTransformationDescription{MonteCarlo}
\assemblerDescription{MonteCarlo}
\restartDescription{MonteCarlo}

Example:
\begin{lstlisting}[style=XML]
<Samplers>
  ...
  <MonteCarlo name='MCname'>
    <samplerInit>
      <limit>10</limit>
      <initialSeed>200286</initialSeed>
      <reseedEachIteration>false</reseedEachIteration>
      <distInit>
        <distribution name= 'ND_InverseWeight_P'>
          <initialGridDisc>10</initialGridDisc>
          <tolerance>0.2</tolerance>
        </distribution>
      </distInit>
    </samplerInit>
    <variable name='var1'>
      <distribution>aDistributionNameDefinedInDistributionBlock
      </distribution>
    </variable>
    <Restart class='DataObject' type='PointSet'>data</Restart>
  </MonteCarlo>
  ...
</Samplers>
  ...
  <PointSet name="data">
    <Input>var1</Input>
    <Output>ans</Output>
  </PointSet>
  ...
\end{lstlisting}

%%% Forward Samplers: Grid
\subsubsection{Grid}
\label{subsubsubsec:Grid}
The \textbf{Grid} sampling approach is probably the simplest exploration
approach that can be employed to explore an uncertain domain.
%
The idea is to construct an $N$-dimensional grid where each dimension is
represented by one uncertain variable.
%
This approach performs the sampling at each node of the grid.
%
The sampling of the grid consists in evaluating the answer of the system under
all possible combinations among the different variables' values with respect to
a predefined discretization metric.
%
In RAVEN two discretization metrics are available: 1) cumulative distribution
function, and 2) value.
%
Thus, the grid meshing can be input via probability or variable values.
%
Regarding the N-dimensional distributions, the user can specify for each dimension the type of grid to be used (i.e., value or CDF). Note the discretization of the CDF, only for the grid sampler, is performed on the marginal distribution for the specific variable considered.

\specBlock{a}{Grid}
%
\attrIntro
\begin{itemize}
\itemsep0em
\item \nameDescription
\end{itemize}
\variableIntro{Grid}
\begin{itemize}
\item \variableDescription
 \variableChildrenIntro
 \begin{itemize}
    \item \distributionDescription
    \item \functionDescription
    \item \gridDescription
  \end{itemize}
\item \constantVariablesDescription
\end{itemize}

If the input parameters are correlated, the \textbf{Grid} sampling approach can be also used if the user specified a
multivariate distributions inside the \xmlNode{Distributions} (see Section \ref{subsec:NdDist}). Furthermore, if the
covariance matrix is provided and the input parameters is assumed to have the multivariate normal distribution, one can also use
\textbf{Grid} approach to sample the input parameters in the transformed space (aka subspace, reduced space). This means one creates
the grids of variables listed by \xmlNode{latentVariables} in the transformed space. If this is the case, the user needs to
provide additional information, i.e. the \xmlNode{transformation} under \xmlNode{MultivariateNormal} of \xmlNode{Distributions}
(more information can be found in Section \ref{subsec:NdDist}). In addition, the node \xmlNode{variablesTransformation} is also
required for \textbf{Grid} sampling. This node is used to tranform the variables specified by \xmlNode{latentVariables} in the
transformed space of input into variables spefified by \xmlNode{manifestVariables} in the input space. The variables listed
in \xmlNode{latentVariables} should be predefined in \xmlNode{variable}, and the variables listed in \xmlNode{manifestVariables}
are used by the \xmlNode{Models}.

\variablesTransformationDescription{Grid}

\assemblerDescription{Grid}
\restartDescription{Grid}

Example:
\begin{lstlisting}[style=XML,morekeywords={construction,steps,lowerBound,upperBound}]
<Samplers>
  ...
  <Grid name='Gridname'>
    <variable name='var1'>
      <distribution>aDistributionNameDefinedInDistributionBlock1
      </distribution>
      <grid type='value' construction='equal' steps='100' >0.2 10</grid>
    </variable>
    <variable name='var2'>
      <distribution>aDistributionNameDefinedInDistributionBlock2
      </distribution>
      <grid type='CDF' construction='equal' steps='5' >0.2 0.8</grid>
    </variable>
    <variable name='var3'>
      <distribution>aDistributionNameDefinedInDistributionBlock3
      </distribution>
      <grid type='value' construction='equal' steps='100' >0.2 21.0</grid>
    </variable>
    <variable name='var4'>
      <distribution>aDistributionNameDefinedInDistributionBlock4
      </distribution>
      <grid type='CDF' construction='equal' steps='5' >0.2 1.0</grid>
    </variable>
    <variable name='var5'>
      <distribution>aDistributionNameDefinedInDistributionBlock5
      </distribution>
      <grid type='value' construction='custom'>0.2 0.5 10.0</grid>
    </variable>
    <variable name='var6'>
      <distribution>aDistributionNameDefinedInDistributionBlock6
      </distribution>
      <grid type='CDF' construction='custom'>0.2 0.5 1.0</grid>
    </variable>
    <Restart class='DataObject' type='PointSet'>data</Restart>
    <restartTolerance>1e-6</restartTolerance>
  </Grid>
  ...
</Samplers>
  ...
  <PointSet name="data">
    <Input>var1,var2,var3,var4,var5,var6</Input>
    <Output>ans</Output>
  </PointSet>
  ...
\end{lstlisting}
\nb A restart example is included here but is not necessary in general.

%%% Forward Samplers: Sparse Grid Collocation
\subsubsection{Sparse Grid Collocation}
%\talbpaul{Work in progress.}
%\senrs{Assembler section updated}
\label{subsubsubsec:SparseGridCollocation}
\textbf{Sparse Grid Collocation} builds on generic \textbf{Grid} sampling by selecting evaluation points based on characteristic quadratures as part of
stochastic collocation for generalized polynomial chaos uncertainty quantification.  In collocation you construct an N-dimensional grid, with each uncertain
variable providing an axis.  Along each axis, the points of evaluation correspond to quadrature points necessary to integrate polynomials
(see \ref{subsubsec:GaussPolynomialRom}).  In the simplest (and most  naive) case, a N-Dimensional tensor product of all possible combinations of points from
each dimension's quadrature is constructed as sampling points.  The number of necessary samples can be reduced by employing Smolyak-like sparse grid algorithms,
which use reduced combinations of polynomial orders to reduce the necessary sampling space.  \specBlock{a}{SparseGridCollocation}.

\begin{itemize}
\itemsep0em
\item \nameDescription
\item \xmlAttr{parallel}, \xmlDesc{optional string attribute}, option to disable parallel construction of the sparse grid.  Because of increasing computational expense with increasing input space dimension, RAVEN will default to parallel construction of the sparse grid. %\talbpaul{Is this what we want?}
\item \xmlAttr{outfile}, \xmlDesc{optional string attribute}, option to allow the generated sparse grid points and weights to be printed to a file with the given name.
\default{True}
\end{itemize}
\variableIntro{SparseGridCollocation}
\begin{itemize}
\item \variableDescription
 In the variable node, the following xml-node needs to be specified:
 \begin{itemize}
    \item \distributionDescription
    \item \functionDescription
 \end{itemize}
 \item \constantVariablesDescription
\end{itemize}
Because of the tight coupling between the Sampler and the ROM in stochastic collocation for generalized polynomial chaos, the Sampler needs access to the ROM via the assembler do determine the polynomials, quadratures, and importance weights to use in each dimension (see \ref{subsubsec:GaussPolynomialRom}).

  % Assembler Objects
  \assemblerDescription{SparseGridCollocation}
  \ROMDescription{SparseGridCollocation}
  \restartDescription{SparseGridCollocation}

\footnotesize
\begin{lstlisting}[style=XML]
Example:
<Samplers>
  ...
  <SparseGridCollocation name="mySG" parallel="0">
    <variable name="x1">
      <distribution>myDist1</distribution>
    </variable>
    <variable name="x2">
      <distribution>myDist2</distribution>
    </variable>
    <ROM class = 'Models' type = 'ROM' >SCROM</ROM>
    <Restart class = 'DataObjects' type = 'PointSet' >solns</Restart>
  </SparseGridCollocation>
  ...
</Samplers>
  ...
  <PointSet name="solns">
    <Input>x1,x2</Input>
    <Output>y</Output>
  </PointSet>
  ...
\end{lstlisting}
 \normalsize

In general, \textbf{SparseGridCollocation} requires uncorrelated input parameters. If the input parameters are correlated, one can transform the
correlated parameters into uncorrelated parameters; the \textbf{SparseGridCollocation} can also be used with the uncorrelated parameters
in the transformed space. Like in the \textbf{Grid} sampler, if the covariance matrix is provided
and the input parameters are assumed to have the multivariate normal distribution, the \textbf{SparseGridCollocation} can be used.
This means one creates the sparse grids of variables listed by \xmlNode{latentVariables} in the transformed space. If this is
the case, the user needs to provide additional information, i.e. the \xmlNode{transformation} under \xmlNode{MultivariateNormal}
of \xmlNode{Distributions} (more information can be found in Section \ref{subsec:NdDist}). In addition, the node
\xmlNode{variablesTransformation} is also required for \textbf{SparseGridCollocation} sampler. This node is used to tranform
the variables specified by \xmlNode{latentVariables} in the transformed space of input into variables spefified by
\xmlNode{manifestVariables} in the input space. The variables listed in \xmlNode{latentVariables} should be predefined
in \xmlNode{variable}, and the variables listed in \xmlNode{manifestVariables}
are used by the \xmlNode{Models}.

\variablesTransformationDescription{SparseGridCollocation}


\begin{lstlisting}[style=XML,morekeywords={ND,grid}]
...
<Models>
    ...
    <ExternalModel ModuleToLoad="lorentzAttractor_noK" name="PythonModule" subType="">
        <variables>sigma,rho,beta,x,y,z,time,z0,y0,z0</variables>
    </ExternalModel>
    <ROM name="SCROM" subType="GaussPolynomialRom">
        <Target>and</Target>
        <Features>x1,y1,z1</Features>
        <IndexSet>TensorProduct</IndexSet>
        <PolynomialOrder>1</PolynomialOrder>
    </ROM>
    ...
</Models>

<Distributions>
    ...
    <MultivariateNormal name='MVNDist' method='pca'>
        <transformation>
            <rank>3</rank>
        </transformation>
        <mu>0.0 1.0 2.0</mu>
        <covariance type="abs">
            1.0       0.6      -0.4
            0.6       1.0      0.2
            -0.4      0.2      0.8
        </covariance>
    </MultivariateNormal>
    ...
</Distributions>

<Samplers>
  ...
  <SparseGridCollocation name='SC'>
        <variable name='x0'>
            <distribution dim='1'>MVNDist</distribution>
        </variable>
        <variable name='y0'>
            <distribution dim='2'>MVNDist</distribution>
        </variable>
        <variable name='z0'>
            <distribution dim='3'>MVNDist</distribution>
        </variable>
        <variablesTransformation model="PythonModule">
            <latentVariables>x1,y1,z1</latentVariables>
            <manifestVariables>x0,y0,z0</manifestVariables>
            <method>pca</method>
        </variablesTransformation>
        <ROM class = 'Models' type = 'ROM' >SCROM</ROM>
        <Restart class="DataObjects" type="PointSet">solns</Restart>
  </SparseGridCollocation>
  ...
</Samplers>
  ...
  <PointSet name="solns">
    <Input>x0,y0,z0</Input>
    <Output>ans</Output>
  </PointSet>
  ...
\end{lstlisting}

 %%% Forward Samplers: Sobol
\subsubsection{Sobol}
%\senrs{Assembler section updated}
\label{subsubsubsec:SobolSampler}
The \textbf{Sobol} sampler uses high-density model reduction (HDMR) a.k.a. Sobol decomposition to approximate a function as the sum of increasing-complexity
interactions.  At its lowest level (order 1), it treats the function as a sum of the reference case plus a functional of each input dimesion separately.  At
order 2, it adds functionals to consider the pairing of each dimension with each other dimension.  The benefit to this approach is considering several functions
of small input cardinality instead of a single function with large input cardinality.  This allows reduced order models like generalized polynomial chaos
(see \ref{subsubsec:GaussPolynomialRom}) to approximate the functionals accurately with few computations runs.  This Sobol sampler uses the associated HDMRRom
(see \ref{subsubsec:HDMRRom}) to determine at what points the input space need be evaluated. Since Sobol sampler relies on SparseGridCollocation, it is also compatible with
multivariate normal distribution objects. The \xmlNode{Sobol} node supports the following attributes:

\begin{itemize}
\itemsep0em
\item \nameDescription
\item \xmlAttr{parallel}, \xmlDesc{optional string attribute}, option to disable parallel construction of the sparse grid.  Because of increasing computational expense with increasing input space dimension, RAVEN will default to parallel construction of the sparse grid.
\default{True}
\end{itemize}
\variableIntro{Sobol}
\begin{itemize}
\item \variableDescription
 In the variable node, the following xml-node needs to be specified:
 \begin{itemize}
    \item \distributionDescription
    \item \functionDescription
 \end{itemize}
\item \constantVariablesDescription
\end{itemize}

Like the \textbf{SparseGridCollocation}, if multivariate normal distribution is provided, the following node need to be specified:
\variablesTransformationDescription{Sobol}

Because of the tight coupling between the Sobol sampler and the HDMRRom, the Sampler needs access to the ROM via the assembler do determine the polynomials, quadratures, Sobol order, and importance weights to use in each dimension (see \ref{subsubsec:HDMRRom}).

  % Assembler Objects
  \assemblerDescription{Sobol}
  \ROMDescription{Sobol}
  \restartDescription{Sobol}

\footnotesize
\begin{lstlisting}[style=XML]
Example:
<Samplers>
  ...
  <Sobol name="mySobol" parallel="0">
    <variable name="x1">
      <distribution>myDist1</distribution>
    </variable>
    <variable name="x2">
      <distribution>myDist2</distribution>
    </variable>
    <ROM class = 'Models' type = 'ROM' >myHDMR</ROM>
    <Restart class="DataObjects" type="PointSet">solns</Restart>
  </Sobol>
  ...
</Samplers>
  ...
  <PointSet name="solns">
    <Input>x1,y2</Input>
    <Output>ans</Output>
  </PointSet>
  ...
\end{lstlisting}
 \normalsize

%%% Forward Samplers: Stratified
\subsubsection{Stratified}
\label{subsubsubsec:Stratified}
The \textbf{Stratified} sampling approach is a method for the exploration of the
input space that consists of dividing the uncertain domain into subgroups before
sampling.
%
In the ``stratified'' sampling, these subgroups must be:
\begin{itemize}
 \item mutually exclusive: every element in the population must be assigned to
   only one stratum (subgroup);
 \item collectively exhaustive: no population element can be excluded.
\end{itemize}

Then simple random sampling or systematic sampling is applied within each
stratum.
%
It is worthwhile to note that the well-known Latin hypercube sampling represents
a specialized version of the stratified approach, when the domain strata are
constructed in equally-probable CDF bins.

\specBlock{a}{Stratified}
%
\attrIntro
\begin{itemize}
\itemsep0em
\item \nameDescription
\end{itemize}
\variableIntro{Stratified}
\begin{itemize}
\item \variableDescription
 \variableChildrenIntro
 \begin{itemize}
    \item \distributionDescription
    \item \functionDescription
    \item \gridDescription
  \end{itemize}
\item \constantVariablesDescription
\end{itemize}
In addition, the settings for this sampler need to be specified in the \xmlNode{samplerInit} XML block:
\begin{itemize}
\item \xmlNode{samplerInit},  \textit{\textbf{XML node, required parameter}}. In this xml-node,the following xml sub-nodes need to be specified:
  \begin{itemize}
    \item \xmlNode{initialSeed}, \textit{\textbf{integer, optional field}}, initial seeding of random number generator
    \item \xmlNode{distInit},  \textit{\textbf{integer, optional field}}, in this node the user specifies the initialization of the random number generator function for each N-Dimensional Probability Distributions (see Section~\ref{subsec:NdDist}).
  \end{itemize}
\end{itemize}

As one can see, the input specifications for the \textbf{Stratified} sampler are
similar to that of the \textbf{Grid} sampler.
%
It is important to mention again that for each zone (grid mesh) only a point,
randomly selected, is picked and not all the nodal combinations (like in the
\textbf{Grid} sampling).

\assemblerDescription{Stratified}
\restartDescription{Stratified}

Example:
\begin{lstlisting}[style=XML,morekeywords={construction,steps,lowerBound,upperBound}]
<Samplers>
  ...
  <Stratified name='StratifiedName'>
    <variable name='var1'>
      <distribution>aDistributionNameDefinedInDistributionBlock1
      </distribution>
      <grid type='CDF' construction='equal' steps='5' >0.2 0.8</grid>
    </variable>
    <variable name='var2'>
      <distribution>aDistributionNameDefinedInDistributionBlock2
      </distribution>
      <grid type='value' construction='equal' steps='100' >0.2 21.0</grid>
    </variable>
    <variable name='var3'>
      <distribution>aDistributionNameDefinedInDistributionBlock3
      </distribution>
      <grid type='CDF' construction='custom'>0.2 0.5 1.0</grid>
    </variable>
  </Stratified>
  ...
</Samplers>
\end{lstlisting}

For N-dimensional  (ND) distributions, there are two different approahes to perform the stratified sampling. In the first approach,
the subgroups is determined by the joint CDF of given multivariate distributions. If this approach is used, the sampling is performed
on a grid on a CDF, while the user is required to specify the same CDF grid for all the dimensions of the ND distribution.
This is possible by defining a \xmlNode{globalGrid} node and associate such \xmlNode{globalGrid} to each variable belonging to the
ND distribution as follows.

\begin{lstlisting}[style=XML,morekeywords={ND,grid}]
<Samplers>
  ...
  <Stratified name='StratifiedName'>
        <variable name='x0'>
            <distribution dim='1'>ND_InverseWeight_P</distribution>
            <grid type='globalGrid'>name_grid1</grid>
        </variable>
        <variable name='y0,z0'>
            <distribution dim='2'>ND_InverseWeight_P</distribution>
            <grid type='globalGrid'>name_grid1</grid>
        </variable>
        <globalGrid>
            <grid name='name_grid1' type='CDF' construction='custom'>0.1 1.0 0.2</grid>
        </globalGrid>
  </Stratified>
  ...
</Samplers>
...
\end{lstlisting}

The second approach is different than the first approach. Like in the \textbf{Grid} sampling, if the covariance matrix is provided
and the input parameters is assumed to have the multivariate normal distribution, one can also use \textbf{Stratified} approach to
sample the input parameters in the transformed space (aka subspace, reduced space). This means one creates
the grids of variables listed by \xmlNode{latentVariables} in the transformed space. If this is the case, the user needs to
provide additional information, i.e. the \xmlNode{transformation} under \xmlNode{MultivariateNormal} of \xmlNode{Distributions}
(more information can be found in Section \ref{subsec:NdDist}). In addition, the node \xmlNode{variablesTransformation} is also
required for \textbf{Stratified} sampler. This node is used to tranform the variables specified by \xmlNode{latentVariables} in the
transformed space of input into variables spefified by \xmlNode{manifestVariables} in the input space. The variables listed
in \xmlNode{latentVariables} should be predefined in \xmlNode{variable}, and the variables listed in \xmlNode{manifestVariables}
are used by the \xmlNode{Models}. In addition, \xmlNode{globalGrid} will be not used for approach.

\variablesTransformationDescription{Stratified}


\begin{lstlisting}[style=XML,morekeywords={ND,grid}]
...
<Models>
    ...
    <ExternalModel ModuleToLoad="lorentzAttractor_noK" name="PythonModule" subType="">
        <variables>sigma,rho,beta,x,y,z,time,z0,y0,z0</variables>
    </ExternalModel>
    ...
</Models>

<Distributions>
    ...
    <MultivariateNormal name='MVNDist' method='pca'>
        <transformation>
            <rank>3</rank>
        </transformation>
        <mu>0.0 1.0 2.0</mu>
        <covariance type="abs">
            1.0       0.6      -0.4
            0.6       1.0      0.2
            -0.4      0.2      0.8
        </covariance>
    </MultivariateNormal>
    ...
</Distributions>

<Samplers>
  ...
  <Stratified name='StratifiedName'>
        <variable name='x0'>
            <distribution dim='1'>MVNDist</distribution>
            <grid type='CDF' construction='equal' steps='3'>0.1 0.9</grid>
        </variable>
        <variable name='y0'>
            <distribution dim='2'>MVNDist</distribution>
            <grid type='value' construction='equal' steps='3'>0.1 0.9</grid>
        </variable>
        <variable name='z0'>
            <distribution dim='3'>MVNDist</distribution>
            <grid type='CDF' construction='equal' steps='3'>0.2 0.8</grid>
        </variable>
        <variablesTransformation model="PythonModule">
            <latentVariables>x1,y1,z1</latentVariables>
            <manifestVariables>x0,y0,z0</manifestVariables>
            <method>pca</method>
        </variablesTransformation>
  </Stratified>
  ...
</Samplers>
...
\end{lstlisting}

%%% Forward Samplers: Response Surface Design
\subsubsection{Response Surface Design}
\label{subsubsubsec:RespSurfDOE}
The \textbf{Response Surface Design}, or Response Surface Modeling (RSM),
approach is one of the most common Design of Experiment (DOE) methodologies
currently in use.
%
It explores the relationships between several explanatory variables and one or
more response variables.
%
The main idea of RSM is to use a sequence of designed experiments to obtain an
optimal response.
%
RAVEN currently employs two different algorithms that can be classified within
this family of methods:
\begin{itemize}
 \item \textbf{Box-Behnken}: This methodology aims to achieve the following
  goals:
  \begin{itemize}
    \item Each factor, or independent variable, is placed at one of three
      equally spaced values, usually coded as -1, 0, +1. (At least three levels
      are needed for the following goal);
    \item The design should be sufficient to fit a quadratic model, that is, one
      squared term per factor and the products of any two factors;
    \item The ratio of the number of experimental points to the number of
      coefficients in the quadratic model should be reasonable (in fact, their
      designs keep it in the range of 1.5 to 2.6);
    \item The estimation variance should more or less depend only on the
      distance from the center (this is achieved exactly for the designs with 4
      and 7 factors), and should not vary too much inside the smallest
      (hyper)cube containing the experimental points.
  \end{itemize}
  Each design can be thought of as a combination of a two-level (full or
  fractional) factorial design with an incomplete block design.
  In each block, a certain number of factors are put through all combinations
  for the factorial design, while the other factors are kept at the central
  values.
 \item \textbf{Central Composite}: This design consists of three distinct sets
  of experimental runs:
  \begin{itemize}
    \item A factorial (perhaps fractional) design in the factors are studied,
      each having two levels;
    \item A set of center points, experimental runs whose values of each factor
      are the medians of the values used in the factorial portion.
      This point is often replicated in order to improve the precision of the
      experiment;
    \item A set of axial points, experimental runs identical to the centre
      points except for one factor, which will take on values both below and
      above the median of the two factorial levels, and typically both outside
      their range.
      All factors are varied in this way.
  \end{itemize}
  This methodology is useful for building a second order (quadratic) model for
  the response variable without needing to use a complete three-level factorial
  experiment.
\end{itemize}
All the parameters, needed for setting up the algorithms reported above, must be
defined within a \xmlNode{ResponseSurfaceDesign} block.
%
\attrIntro
\begin{itemize}
\itemsep0em
\item \nameDescription
\end{itemize}

\variableIntro{ResponseSurfaceDesign}
\begin{itemize}
\item \variableDescription
 \variableChildrenIntro
 \begin{itemize}
    \item \distributionDescription
    \item \functionDescription
     \item \gridDescriptionOnlyCustom
     \nb{Only the construction ``custom'' is available. In the \xmlNode{grid} body only the lower and upper bounds can be inputted (2 numbers only).}
 \end{itemize}
 \item \constantVariablesDescription
\item \xmlNode{ResponseSurfaceDesignSettings}, \xmlDesc{required},
In this sub-node, the user needs to specify different settings depending on the
algorithm being used:
 \begin{itemize}
  \item \xmlNode{algorithmType}, \xmlDesc{string, required field}, this XML node
    will contain the name of the algorithm to be used.
    Based on the chosen algorithm, other nodes need to be defined:
    \begin{itemize}
      \item \xmlNode{algorithmType}\texttt{BoxBehnken}\xmlNode{algorithmType/}. If Box-Behnken
        is specified, the following additional node is recognized:
     \begin{itemize}
      \item \xmlNode{ncenters}, \xmlDesc{integer, optional field}, the
        number of center points to include in the box.
        If this parameter is not specified, then a pre-determined number of
        points are automatically included.
        \default{Automatic Generation}.
     \end{itemize}
     \nb In order to employ the ``Box-Behnken'' design, at least 3 variables
     must be used.
     \item \xmlNode{algorithmType}\texttt{CentralComposite}\xmlNode{algorithmType/}. If
       Central Composite is specified, the following additional nodes will
       be recognized:
     \begin{itemize}
      \item \xmlNode{centers}, \xmlDesc{comma separated integers, optional
        field}, the number of center points to be included.
        This block needs to contain 2 integers values separated by a comma.
        The first entry represents the number of centers to be added for the
        factorial block; the second one is the one for the star block.
        \default{4,4}.
      \item \xmlNode{alpha}, \xmlDesc{string, optional field}, in this node,
        the user decides how an $\alpha$ factor needs to be determined.
        Two options are available:
        \begin{description}
          \item[\texttt{orthogonal}] for orthogonal design.
          \item[\texttt{rotatable}] for rotatable design.
        \end{description}
        \default{orthogonal}.
      \item \xmlNode{face}, \xmlDesc{string, optional field}, in this node,
        the user defines how faces should be constructed.
        Three options are available:
        \begin{description}
          \item[\texttt{circumscribed}] for circumscribed facing
          \item[\texttt{inscribed}] for inscribed facing
          \item[\texttt{faced}] for faced facing.
        \end{description}
        \default{circumscribed}.
     \end{itemize}
  \end{itemize}
  \nb In order to employ the ``Central Composite'' design, at least 2
  variables must be used.
\end{itemize}
\end{itemize}

Furthermore, if the covariance matrix is provided and the input parameters are assumed to have a multivariate normal distribution, one can use
\textbf{ResponseSurfaceDesign} approach to sample the input parameters in the transformed space (aka subspace, reduced space).
In this case, the user needs to provide additional information, i.e. the \xmlNode{transformation} under \xmlNode{MultivariateNormal} of \xmlNode{Distributions}
(more information can be found in Section \ref{subsec:NdDist}). In addition, the node \xmlNode{variablesTransformation} is also
required for \textbf{ResponseSurfaceDesign} sampling. This node is used to tranform the variables specified by \xmlNode{latentVariables} in the
transformed space of input into variables spefified by \xmlNode{manifestVariables} in the input space. The variables listed
in \xmlNode{latentVariables} should be predefined in \xmlNode{variable}, and the variables listed in \xmlNode{manifestVariables}
are used by the \xmlNode{Models}.

\variablesTransformationDescription{ResponseSurfaceDesign}
%\maljdan{Is it weird that one of these uses ncenters and the other uses centers?}
%\alfoa{The names of those parameters are different in order to avoid confusion, since the meaning (and the way ) to input them is different}

%\maljdan{This is the first example where type is an attribute and a node...This
%is confusing.}
%\alfoa{Changed.}

Example:
\begin{lstlisting}[style=XML,morekeywords={}]
<Samplers>
  ...
    <ResponseSurfaceDesign name='BoxBehnkenRespDesign'>
        <ResponseSurfaceDesignSettings>
            <algorithmType>BoxBehnken</algorithmType>
            <ncenters>1</ncenters>
        </ResponseSurfaceDesignSettings>
        <variable name='var1' >
            <distribution >Gauss1</distribution>
           <grid type='CDF' construction='custom'  >0.2 0.8</grid>
        </variable>
        <!-- N.B. at least 3 variables need to inputted
                in order to employ this algorithm
         -->
    </ResponseSurfaceDesign>
    <ResponseSurfaceDesign name='CentralCompositeRespDesign'>
        <ResponseSurfaceDesignSettings>
            <algorithmType>CentralComposite</algorithmType>
            <centers>1, 2</centers>
            <alpha>orthogonal</alpha>
            <face>circumscribed</face>
        </ResponseSurfaceDesignSettings>
        <variable name='var4' >
            <distribution >Gauss1</distribution>
            <grid type='CDF' construction='custom'  >0.2 0.8</grid>
        </variable>
        <!-- N.B. at least 2 variables need to inputted
                in order to employ this algorithm
         -->
    </ResponseSurfaceDesign>
    <ResponseSurfaceDesign name='transformedSpaceSampling'>
        <ResponseSurfaceDesignSettings>
            <algorithmType>BoxBehnken</algorithmType>
            <ncenters>1</ncenters>
        </ResponseSurfaceDesignSettings>
        <variable name='var1' >
            <distribution >Gauss1</distribution>
           <grid type='CDF' construction='custom'  >0.2 0.8</grid>
        </variable>
        ...
        <variablesTransformation model="givenModel">
          <latentVariables>var1,...</latentVariables>
          <manifestVariables>...</manifestVariables>
          <method>pca</method>
        </variablesTransformation>
    </ResponseSurfaceDesign>
  ...
</Samplers>
\end{lstlisting}

%%% Forward Samplers: Factorial Design
\subsubsection{Factorial Design}
\label{subsubsubsec:FactorialDOE}
The \textbf{Factorial Design} method is an important method to determine the
effects of multiple variables on a response.
%
A factorial design can reduce the number of samples one has to perform by
studying multiple factors simultaneously.
%
Additionally, it can be used to find both main effects (from each independent
factor) and interaction effects (when both factors must be used to explain the
outcome).
%
A factorial design tests all possible conditions.
%
Because factorial designs can lead to a large number of trials, which can
become expensive and time-consuming, they are best used for small numbers of
variables with only a few domain discretizations (1 to 3).
%
Factorial designs work well when interactions between variables are strong and
important and where every variable contributes significantly.
%
RAVEN currently employs three different algorithms that can be classified within
this family of techniques:
\begin{itemize}
  \item \textbf{General Full Factorial} explores the input space by
    investigating all possible combinations of a set of factors (variables).
  \item \textbf{2-Level Fractional-Factorial} consists of a carefully chosen
    subset (fraction) of the experimental runs of a full factorial design.
    %
    The subset is chosen so as to exploit the sparsity-of-effects principle
    exposing information about the most important features of the problem
    studied, while using a fraction of the effort of a full factorial design in
    terms of experimental runs and resources.
  \item \textbf{Plackett-Burman} identifies the most important factors early in
    the experimentation phase when complete knowledge about the system is
    usually unavailable.
    %
    It is an efficient screening method for identifying the active factors
    (variables) using as few samples as possible.
    %
    In Plackett-Burman designs, main effects have a complicated confounding
    relationship with two-factor interactions.
    %
    Therefore, these designs should be used to study main effects when it can be
    assumed that two-way interactions are negligible.
\end{itemize}
All the parameters needed for setting up the algorithms reported above must be
defined within a \xmlNode{FactorialDesign} block.
%
\attrIntro
\begin{itemize}
\itemsep0em
\item \nameDescription
\end{itemize}
\variableIntro{FactorialDesign}
\begin{itemize}
  \item \variableDescription
    \variableChildrenIntro
    \begin{itemize}
      \item \distributionDescription
      \item \functionDescription
      \item \gridDescription
    \end{itemize}
  \item \constantVariablesDescription
\end{itemize}

The main \xmlNode{FactorialDesign} block needs to contain an additional sub-node
called\\\xmlNode{FactorialSettings}.
%
In this sub-node, the user needs to specify different settings depending on the
algorithm being used:
   \begin{itemize}
    \item \xmlNode{algorithmType}, \xmlDesc{string, required field}, specifies the
      algorithm to be used.
      %
      Based on the chosen algorithm, other nodes may be defined:
      \begin{itemize}
        \item \xmlNode{algorithmType}\texttt{full}\xmlNode{algorithmType/}. Full factorial design.
          If \texttt{full} is specified, no additional nodes are necessary.
          \\
          \nb The full factorial design does not have any limitations on the
          number of discretization bins that can be used in the \xmlNode{grid}
          XML node for each \xmlNode{variable} specified.
        \item \xmlNode{algorithmType}\texttt{2levelFract}\xmlNode{algorithmType/}. Two-level
          Fractional-Factorial design.
          %
          If \\\texttt{2levelFract}  is specified, the following additional
          nodes must be specified:
          \begin{itemize}
            \item \xmlNode{gen}, \xmlDesc{space separated strings, required
              field}, specifies the confounding mapping.
              %
              For instance, in this block the user defines the decisions on a
              fraction of the full-factorial by allowing some of the factor main
              effects to be compounded with other factor interaction effects.
              %\maljdan{compounded?} \alfoa{Right Dan.}
              %
              This is done by defining an alias structure that defines,
              symbolically, these interactions.
              %
              These alias structures are written like “C = AB” or “I = ABC”, or
              “AB = CD”, etc.
              %
              These define how a column is related to the others.
            \item \xmlNode{genMap}, \xmlDesc{space separated strings, required
              field}, defines the mapping between the \xmlNode{gen} symbolic
              aliases and the variables that have been inputted in the
              \xmlNode{FactorialDesign} main block.
          \end{itemize}
          \nb The Two-levels Fractional-Factorial design is limited to 2
          discretization bins in the \xmlNode{grid} node for each
          \xmlNode{variable}.
       \item \xmlNode{algorithmType}\texttt{pb}\xmlNode{algorithmType/}. Plackett-Burman design.
         If \texttt{pb} is specified, no additional nodes are necessary.
         \\
         \nb The Plackett-Burman design does not have any limitations on the
         number of discretization bins allowed in the \xmlNode{grid} node for
         each \xmlNode{variable}.
      \end{itemize}

  \end{itemize}
Example:
\begin{lstlisting}[style=XML,morekeywords={construction,upperBound,steps}]
<Samplers>
  ...
  <FactorialDesign name='fullFactorial'>
    <FactorialSettings>
      <algorithmType>full</algorithmType>
    </FactorialSettings>
    <variable name='var1' >
      <distribution>aDistributionNameDefinedInDistributionBlock1
      </distribution>
      <grid type='value' construction='custom' >0.02 0.03 0.5</grid>
    </variable>
    <variable name='var2' >
      <distribution>aDistributionNameDefinedInDistributionBlock2
      </distribution>
      <grid type='CDF' construction='custom'>0.5 0.7 1.0</grid>
    </variable>
  </FactorialDesign>
  <FactorialDesign name='2levelFractFactorial'>
    <FactorialSettings>
      <algorithmType>2levelFract</algorithmType>
      <gen>a,b,ab</gen>
      <genMap>var1,var2,var3</genMap>
    </FactorialSettings>
    <variable name='var1' >
      <distribution>aDistributionNameDefinedInDistributionBlock3
      </distribution>
      <grid type='value' construction='custom' >0.02 0.5</grid>
    </variable>
    <variable name='var2' >
      <distribution>aDistributionNameDefinedInDistributionBlock
      </distribution>
      <grid type='CDF' construction='custom'>0.5 1.0</grid>
    </variable>
    <variable name='var3'>
      <distribution>aDistributionNameDefinedInDistributionBlock5
      </distribution>
      <grid type='value' upperBound='4' construction='equal' steps='1'>0.5</grid>
    </variable>
  </FactorialDesign>
  <FactorialDesign name='pbFactorial'>
    <FactorialSettings>
      <algorithmType>pb</algorithmType>
    </FactorialSettings>
    <variable name='var1' >
      <distribution>aDistributionNameDefinedInDistributionBlock6
      </distribution>
      <grid type='value' construction='custom' >0.02 0.5</grid>
    </variable>
    <variable name='VarGauss2' >
      <distribution>aDistributionNameDefinedInDistributionBlock7
      </distribution>
      <grid type='CDF' construction='custom'>0.5 1.0</grid>
    </variable>
  </FactorialDesign>
  ...
</Samplers>
\end{lstlisting}

%%% Forward Samplers: EnsembleForward
\subsubsection{Ensemble Forward Sampling strategy}
\label{subsubsubsec:EnsembleSampler}
The \textbf{Ensemble Forward} sampling approach allows the user to combine multiple Forward sampling strategies
into one single strategy. For example, it can happen that a variable is more suitable for a particular sampling strategy (e.g. a
stochastic event
modeled with a Monte Carlo approach) and a second variable is more suitable for another sampling method (e.g. because part of a parametric space modeled with a Grid-based approach).
\specBlock{a}{EnsembleForward}
%
\attrsIntro
\vspace{-5mm}
\begin{itemize}
\itemsep0em
\item \textbf{name}, \textit{required string attribute}, user-defined name of this Sampler. N.B. As for the other objects, this is the name that can be used to refer to this specific entity from other input blocks (xml);
\end{itemize}
\vspace{-5mm}

In the \textbf{EnsembleForward} input block, the user needs to specify the sampling strategies that he wants to combine together.
\\Currently, only the following strategies can be combined:
\begin{itemize}
  \item \xmlNode{MonteCarlo}
  \item \xmlNode{Grid}
  \item \xmlNode{Stratiefied}
  \item \xmlNode{FactorialDesign}
  \item \xmlNode{ResponseSurfaceDesign}
  \item \xmlNode{CustomSampler}
\end{itemize}
For each of the above samplers, the input specifications can be found in the relative sections.

Example:
\begin{lstlisting}[style=XML]
<Samplers>
  ...
    <EnsembleForward name="testEnsembleForward">
        <MonteCarlo name = "theMC">
            <samplerInit> <limit>4</limit> </samplerInit>
            <variable name="sigma">
                <distribution>norm</distribution>
            </variable>
        </MonteCarlo>
        <Grid name = "theGrid">
            <variable name="x0">
                <distribution>unif</distribution>
                <grid construction="custom" type="value">0.02 0.5 0.6</grid>
            </variable>
        </Grid>
        <Stratified name = "theStratified">
            <variable name="z0">
                <distribution>tri</distribution>
                <grid construction="equal" steps="2" type="CDF">0.2 0.8</grid>
            </variable>
            <variable name="y0">
                <distribution>unif</distribution>
                <grid construction="equal" steps="2" type="value">0.5 0.8</grid>
            </variable>
        </Stratified>
        <ResponseSurfaceDesign name = "theRSD">
            <ResponseSurfaceDesignSettings>
                <algorithmType>CentralComposite</algorithmType>
                <centers>1,2</centers>
                <alpha>orthogonal</alpha>
                <face>circumscribed</face>
            </ResponseSurfaceDesignSettings>
            <variable name="rho">
                <distribution>unif</distribution>
                <grid construction="custom" type="CDF">0.0 1.0</grid>
            </variable>
            <variable name="beta">
                <distribution>tri</distribution>
                <grid construction="custom" type="value">0.1 1.5</grid>
            </variable>
        </ResponseSurfaceDesign>
    </EnsembleForward>
  ...
</Samplers>
\end{lstlisting}

Care should be used when using deterministic random seeds for EnsembleForward sampling.  The EnsembleForward
sample will ignore any seeds set in any of its subset samplers; however, the global random seed can be set by
adding a \xmlNode{samplerInit} block with the \xmlNode{initialSeed} block therein, with an integer value
providing the seed.  For example,
\begin{lstlisting}[style=XML]
  <Samplers>
    ...
    <EnsembleForward name='testEnsembleForward'>
      <samplerInit>
        <initialSeed>42</initialSeed>
      </samplerInit>
    ...
     </EnsembleForward>
    ...
  </Samplers>
\end{lstlisting}
Because RAVEN has a single global random number generator, this will set the seed for the full calculation
when the Step containing a run using this ForwardSampler is begun.

Note also variables that are defined from functions, as well as constants, need to be defined outside the
samplers of the ensemble sampler. An example is shown below.

Example:
\begin{lstlisting}[style=XML]
  <Samplers>
    <EnsembleForward name='testEnsembleForward'>
      <variable name='x3'>
          <function>funct1</function>
      </variable>
      <variable name='x4,x5'>
          <function>funct2</function>
      </variable>
      <constant name='pi'>3.14159</constant>
      <MonteCarlo name='notNeeded'>
        <samplerInit>
          <limit>3</limit>
        </samplerInit>
        <variable name='x1'>
          <distribution>norm</distribution>
        </variable>
      </MonteCarlo>
      <Grid name='notNeeded'>
        <variable name='x2'>
          <distribution>unif</distribution>
          <grid construction='custom' type='value'>0.02 0.6</grid>
        </variable>
      </Grid>
     </EnsembleForward>
  </Samplers>
\end{lstlisting}

In this example note that:
\begin{itemize}
  \item variables $x1$ and $x2$ are generated by the two samplers (Monte-Carlo and Grid respectively)
  \item variable $x3$ is generated from the function $funct1$
  \item variables $x4$ and $x5$ are generated from the function $funct2$
  \item variables $x3$, $x4$ and $x5$ are defined outside the Monte-Carlo and Grid
\end{itemize}

%%% Forward Samplers: Custom Sampler
\subsubsection{Custom Sampling strategy}
\label{subsubsubsec:CustomSampler}
The \textbf{Custom} sampling approach allows the user to specify a predefined set of coordinates (in the input space) that RAVEN should use to inquire the model. For example, the user can provide a CSV file containing a list of samples that RAVEN should use.
\specBlock{a}{CustomSampler}
%
\attrsIntro
\vspace{-5mm}
\begin{itemize}
\itemsep0em
\item \textbf{name}, \textit{required string attribute}, user-defined name of this Sampler. N.B. As for the other objects, this is the name that can be used to refer to this specific entity from other input blocks (xml);
\end{itemize}
\vspace{-5mm}

In the \textbf{CustomSampler} input block, the user needs to specify the variables need to be sampled. As
already mentioned, these variables are inputted within consecutive XML blocks called \xmlNode{variable}.  Note
that if any variables are dependent on other dimensions (e.g. ``time''), the dependent dimensions need to be
listed as variables as well.

In addition, the \xmlNode{Source} from which the samples need to be retrieved needs to be specified:
\begin{itemize}
  \item \xmlNode{variable}, \xmlDesc{XML node,
    required parameter} can specify the following attribute:
    \begin{itemize}
      \item \xmlAttr{name}, \xmlDesc{required string attribute}, user-defined name of this variable.
      \item \xmlAttr{nameInSource}, \xmlDesc{optional string attribute}, name of the variable to read from in
        \xmlNode{Source}.  \default Same as \xmlAttr{name}.
      \item \shapeVariableDescription
    \end{itemize}
 \item \xmlNode{Source}, \xmlDesc{XML node,
  required parameter} will specify the following attributes:
  \begin{itemize}
    \item \xmlAttr{class}, \xmlDesc{required string attribute}, class entity of the source where the samples need to be retrieved from.
     It can be either \textbf{Files} or \textbf{DataObjects}.
     \item \xmlAttr{type}, \xmlDesc{required string attribute}, type of the source withing the previously explained ``class''.
      If \xmlAttr{class} is  \textbf{Files}, this attribute needs to be kept empty; otherwise it must be one
      of the \textbf{DataSet} objects: PointSet, HistorySet, or DataSet.
      \\ \nb If the \xmlNode{Source} \xmlAttr{class} is  \textbf{Files}, the File needs to be a standard CSV file, specified in the
      \xmlNode{Files} XML block in the RAVEN input.
      \\ In addition, it is important to notice that if in the \xmlNode{Source}  the \textbf{PointProbability} and
      \textbf{ProbabilityWeight} quantities are not found, the samples are assumed to come from a MonteCarlo (from a  statistical
      post-processing prospective).
  \end{itemize}
 \item \constantVariablesDescription
\end{itemize}

Example:
\begin{table}[h!]
\centering
\caption{samples.csv}
\begin{tabular}{ccccc}
\textbf{y}  & \textbf{x}  & \textbf{z}  & \textbf{PointProbability} & \textbf{ProbabilityWeight} \\
0.725675246 & 0.031099304 & 0.984988317 & 0.1                       & 0.2                        \\
0.565949127 & 0.028589754 & 1.13186372  & 0.1                       & 0.2                        \\
0.72567754  & 0.031099304 & 0.967209238 & 0.1                       & 0.2                        \\
0.565951633 & 0.028589754 & 1.111431662 & 0.1                       & 0.2                        \\
0.725968307 & 0.031100307 & 0.98498835  & 0.1                       & 0.2
\end{tabular}
\end{table}
\begin{lstlisting}[style=XML]
<Samplers>
  ...
  <Samplers>
    <CustomSampler name="customSamplerDataObject">
      <Source   class="DataObjects"  type="PointSet">outCustomSamplerFromFile</Source>
      <variable name="x"/>
      <variable name="y"/>
      <variable name="z"/>
    </CustomSampler>
  </Samplers>
  <Samplers>
    <CustomSampler name="customSamplerFile">
      <Source   class="Files"  type="">samples.csv</Source>
      <variable name="x"/>
      <variable name="y"/>
      <variable name="z"/>
    </CustomSampler>
  </Samplers>
  ...
</Samplers>
\end{lstlisting}

%%%%%%%%%%%%%%%%%%%%%%%%%%%%
%%% Dynamic Event Tree Samplers %%%
%%%%%%%%%%%%%%%%%%%%%%%%%%%%
\subsection{Dynamic Event Tree (DET) Samplers}
\label{subsec:DETSamplers}
The \textbf{Dynamic Event Tree} methodologies are designed to take the timing of
events explicitly into account, which can become very important especially when
uncertainties in complex phenomena are considered.
%
Hence, the main idea of this methodology is to let a system code determine the
pathway of an accident scenario within a probabilistic environment.
%
In this family of methods, a continuous monitoring of the system evolution in
the phase space is needed.
%
In order to use the DET-based methods, the generic driven code needs to have, at
least, an internal trigger system and, consequently, a ``restart'' capability.
%
In the RAVEN framework, 4 different DET samplers are available:
\begin{itemize}
\item \textbf{Dynamic Event Tree (DET)}
\item \textbf{Hybrid Dynamic Event Tree (HDET)}
\item \textbf{Adaptive Dynamic Event Tree (ADET)}
\item \textbf{Adaptive Hybrid Dynamic Event Tree (AHDET)}
\end{itemize}

The ADET and the AHDET methodologies represent a hybrid between the DET/HDET and adaptive sampling
approaches.
%
For this reason, its input requirements are reported in the Adaptive Samplers'
section (\ref{subsec:AdaptSamplers}).

%%%%%%%%% Dynamic Event Tree Samplers: Dynamic Event Tree
\subsubsection{Dynamic Event Tree}
\label{subsubsubsec:DET}
The \textbf{Dynamic Event Tree} sampling approach is a sampling strategy that is
designed to take the timing of events, in transient/accident scenarios,
explicitly into account.
%
From an application point of view, an $N$-Dimensional grid is built on the CDF
space.
%
A single simulation is spawned and a set of triggers is added to the system code
control logic.
%
Every time a trigger is activated (one of the CDF thresholds in the grid is
exceeded), a new set of simulations (branches) is spawned.
%
Each branch carries its conditional probability.
%
In the RAVEN code, the triggers are defined by specifying a grid using a
predefined discretization metric in the mode input space.
%
RAVEN provides two discretization metrics: 1) CDF, and 2) value.
%
Thus, the trigger thresholds can be entered either in probability or value
space.
%

\specBlock{a}{DynamicEventTree}
%
\attrsIntro
\begin{itemize}
  \itemsep0em
  \item \nameDescription
  \item \xmlAttr{printEndXmlSummary}, \xmlDesc{optional string/boolean attribute},
    controls the dumping of a ``summary'' of the DET performed into an external
    XML.
    %
    \default{False}.
  \item \xmlAttr{maxSimulationTime}, \xmlDesc{optional float attribute}, this
    attribute controls the maximum ``mission'' time of the simulation
    underneath.
    %
    \default{None}.
\end{itemize}
\variableIntro{DynamicEventTree}
\begin{itemize}
\item \variableDescription
  \variableChildrenIntro
  \begin{itemize}
    \item \distributionDescription
    \item \functionDescription
    \item \gridDescription
  \end{itemize}
  \item \constantVariablesDescription
\end{itemize}

Example:
\begin{lstlisting}[style=XML]
<Samplers>
  ...
  <DynamicEventTree name='DETname'>
    <variable name='var1'>
      <distribution>aDistributionNameDefinedInDistributionBlock1 </distribution>
      <grid type='value' construction='equal' steps='100' >1.0 201.0</grid>
    </variable>
    <variable name='var2'>
      <distribution>aDistributionNameDefinedInDistributionBlock2 </distribution>
      <grid type='CDF' construction='equal' steps='5'>0 1</grid>
    </variable>
    <variable name='var3'>
      <distribution>aDistributionNameDefinedInDistributionBlock3 </distribution>
      <grid type='value' construction='equal' steps='10' >11.0 21.0</grid>
    </variable>
    <variable name='var4'>
      <distribution>aDistributionNameDefinedInDistributionBlock4 </distribution>
      <grid type='CDF' construction='equal' steps='5' >0.0 1.0</grid>
    </variable>
    <variable name='var5'>
      <distribution>aDistributionNameDefinedInDistributionBlock5 </distribution>
      <grid type='value' construction='custom'>0.2 0.5 10.0</grid>
    </variable>
    <variable name='var6'>
      <distribution>aDistributionNameDefinedInDistributionBlock6 </distribution>
      <grid type='CDF' construction='custom'>0.2 0.5 1.0</grid>
    </variable>
  </DynamicEventTree>
  ...
</Samplers>
\end{lstlisting}

%%%%%%%%% Dynamic Event Tree Samplers: Hybrid Dynamic Event Tree
\subsubsection{Hybrid Dynamic Event Tree}
\label{subsubsubsec:HDET}
The \textbf{Hybrid Dynamic Event Tree} sampling approach is a sampling strategy
that represents an evolution of the Dynamic Event Tree method for the
simultaneous exploration of the epistemic and aleatory uncertain space.
%
In similar approaches, the uncertainties are generally treated by employing a
Monte-Carlo sampling approach (epistemic) and DET methodology (aleatory).
%
The HDET methodology, developed within the RAVEN code, can reproduce the
capabilities employed by this approach, but provides additional sampling
strategies to the user.
%
The epistemic or epistemic-like uncertainties can be sampled through the
following strategies:

\begin{itemize}
  \item Monte-Carlo;
  \item Grid sampling;
  \item Stratified (e.g., Latin Hyper Cube).
\end{itemize}

From a practical point of view, the user defines the parameters that need to be
sampled by one or more different approaches.
%
The HDET module samples those parameters creating an $N$-dimensional grid
characterized by all the possible combinations of the input space coordinates
coming from the different sampling strategies.
%
Each coordinate in the input space represents a separate and parallel standard
DET exploration of the uncertain domain.
%
The HDET methodology allows the user to explore the uncertain domain
employing the best approach for each variable kind.
%
The addition of a grid sampling strategy among the usable approaches allows the
user to perform a discrete parametric study under aleatory and epistemic
uncertainties.

Regarding the input requirements, the HDET sampler is a ``sub-type'' of the\\
\xmlNode{DynamicEventTree} sampler.
%
For this reason, its specifications must be defined within a
\xmlNode{DynamicEventTree} block.
%
\attrsIntro

\begin{itemize}
  \itemsep0em
  \item \nameDescription
  \item \xmlAttr{printEndXmlSummary}, \xmlDesc{optional string/boolean attribute},
    controls the dumping of a ``summary'' of the DET performed into an external
    XML.
    %
    \default{False}.
  \item \xmlAttr{maxSimulationTime}, \xmlDesc{optional float attribute}, this
    attribute controls the maximum ``mission'' time of the simulation
    underneath.
    %
    \default{None}.
\end{itemize}

\variableIntro{DynamicEventTree}

\begin{itemize}
  \item \variableDescription
  \variableChildrenIntro
  \begin{itemize}
    \item \distributionDescription
    \item \functionDescription
    \item \gridDescription
  \end{itemize}
 \item \constantVariablesDescription
\end{itemize}

In order to activate the \textbf{Hybrid Dynamic Event Tree}  sampler, the main
\xmlNode{DynamicEventTree} block needs to contain, at least, an additional
sub-node called \xmlNode{HybridSampler}.
%
As already mentioned, the user can combine the Monte-Carlo, Stratified, and Grid
approaches in order to create a ``pre-sampling'' $N$-dimensional grid, from
whose nodes a standard DET method is employed.
%
For this reason, the user can specify a maximum of three
\xmlNode{HybridSampler} sub-nodes (i.e. one for each of the available
Forward samplers).
%
This sub-node needs to contain the following attribute:
\begin{itemize}
  \item \xmlAttr{type}, \xmlDesc{required string attribute}, type of
    pre-sampling strategy to be used.
    %
    Available options are \xmlString{MonteCarlo}, \xmlString{Grid}, and
    \xmlString{Stratified}.
 \end{itemize}

Independent of the type of ``pre-sampler'' that has been specified, the
\xmlNode{HybridSampler} must contain the variables that need to be sampled.
%
As already mentioned, these variables are specified within consecutive
\xmlNode{variable} XML blocks:

\begin{itemize}
  \item \variableDescription
    \variableChildrenIntro
    \begin{itemize}
      \item \distributionDescription
      \item \functionDescription
    \end{itemize}
  \item \constantVariablesDescription
 \end{itemize}

If a pre-sampling strategy \xmlAttr{type} is either \xmlString{Grid} or
\xmlString{Stratified}, within the \xmlNode{variable} blocks, the user needs to
specify the sub-node \xmlNode{grid}.
%
As with the standard DET, the content of this XML node depends on the definition
of the associated attributes:
\begin{itemize}
\itemsep0em
\item \xmlAttr{type}, \xmlDesc{required string attribute}, user-defined
  discretization metric type:
  \begin{itemize}
    \item \xmlString{CDF}, the grid is going to be specified based on the
      cumulative distribution function probability thresholds
    \item \xmlString{value}, the grid is going to be provided using variable
      values.
  \end{itemize}
  \item \xmlAttr{construction}, \xmlDesc{required string attribute}, how the
    grid needs to be constructed, independent of its type (i.e. \xmlString{CDF}
    or \xmlString{value}).
\end{itemize}
\constructionGridDescription

Example:
\begin{lstlisting}[style=XML]
<Samplers>
  ...
  <DynamicEventTree name='HybridDETname' print_end_XML="True">
    <HybridSampler type='MonteCarlo' limit='2'>
      <variable name='var1' >
        <distribution>aDistributionNameDefinedInDistributionBlock1 </distribution>
      </variable>
      <variable name='var2' >
        <distribution>aDistributionNameDefinedInDistributionBlock2 </distribution>
        <grid type='CDF' construction='equal' steps='1' lowerBound='0.1'>0.1</grid>
      </variable>
    </HybridSampler>
    <HybridSampler type='Grid'>
      <!-- Point sampler way (directly sampling the variable) -->
      <variable name='var3' >
        <distribution>aDistributionNameDefinedInDistributionBlock3 </distribution>
        <grid type='CDF' construction='equal' steps='1' lowerBound='0.1'>0.1</grid>
      </variable>
      <variable name='var4' >
        <distribution>aDistributionNameDefinedInDistributionBlock4 </distribution>
        <grid type='CDF' construction='equal' steps='1' lowerBound='0.1'>0.1</grid>
      </variable>
    </HybridSampler>
    <HybridSampler type='Stratified'>
      <!-- Point sampler way (directly sampling the variable ) -->
      <variable name='var5' >
        <distribution>aDistributionNameDefinedInDistributionBlock5 </distribution>
        <grid type='CDF' construction='equal' steps='1' lowerBound='0.1'>0.1</grid>
      </variable>
      <variable name='var6' >
        <distribution>aDistributionNameDefinedInDistributionBlock6 </distribution>
        <grid type='CDF' construction='equal' steps='1' lowerBound='0.1'>0.1</grid>
      </variable>
    </HybridSampler>
    <!-- DYNAMIC EVENT TREE INPUT (it goes outside an inner block like HybridSamplerSettings) -->
      <Distribution name='dist7'>
        <distribution>aDistributionNameDefinedInDistributionBlock7 </distribution>
        <grid type='CDF' construction='custom'>0.1 0.8</grid>
      </Distribution>
  </DynamicEventTree>
  ...
</Samplers>
\end{lstlisting}

%%%%%%%%%%%%%%%%%%%%%%%%%
%%% Adaptive Samplers %%%
%%%%%%%%%%%%%%%%%%%%%%%%%
\subsection{Adaptive Samplers}
\label{subsec:AdaptSamplers}
The Adaptive Samplers' family provides the possibility to perform smart sampling
(also known as adaptive sampling) as an alternative to classical “Forward”
techniques.
%
The motivation is that system simulations are often computationally expensive,
time-consuming, and high dimensional with respect to the number of input
parameters.
%
Thus, exploring the space of all possible simulation outcomes is infeasible
using finite computing resources.
%
During simulation-based probabilistic risk analysis, it is important to discover
the relationship between a potentially large number of input parameters and the
output of a simulation using as few simulation trials as possible.

The description above characterizes a typical context for performing adaptive
sampling where a few observations are obtained from the simulation, a reduced
order model (ROM) is built to represent the simulation space, and new samples
are selected based on the model constructed.
%
The reduced order model (see section \ref{subsec:models_ROM}) is then updated
based on the simulation results of the sampled points.
%
In this way, an attempt is made to gain the most information possible with a
small number of carefully selected sample points, limiting the number of
expensive trials needed to understand features of the system space.
%

Currently, RAVEN provides support for the following adaptive algorithms:

\begin{itemize}
  \item Limit Surface Search
  \item Adaptive Dynamic Event Tree
  \item Adaptive Hybrid Dynamic Event Tree
  \item Adaptive Sparse Grid
  \item Adaptive Sobol Decomposition
\end{itemize}

In the following paragraphs, the input requirements and a small explanation of
the different sampling methods are reported.

%%% Adaptive Samplers: Adaptive Sampling for Limit Surface search
\subsubsection{Limit Surface Search}
\label{subsubsubsec:LimitSurfaceSearch}
The \textbf{Limit Surface Search} approach is an advanced methodology that employs
a smart sampling around transition zones that determine a change in the status
of the system (limit surface).
%
To perform such sampling, RAVEN uses ROMs for predicting, in the input space,
the location(s) of these transitions, in order to accelerate the exploration of
the input space in proximity of the limit surface.
%

\specBlock{an}{LimitSurfaceSearch}
%
\attrIntro

\begin{itemize}
  \itemsep0em
  \item \nameDescription
\end{itemize}

\variableIntro{LimitSurfaceSearch}

\begin{itemize}
    \item \variableChildrenIntro
    \begin{itemize}
      \item \distributionDescription
      \item \functionDescription
    \end{itemize}
\end{itemize}

In addition to the \xmlNode{variable} nodes, the main XML node
\xmlNode{Adaptive} needs to contain two supplementary sub-nodes:

\begin{itemize}
  \item \convergenceDescription
  \item \xmlNode{batchStrategy}, \xmlDesc{string, optional field}, defines how
    points should be selected within a batch of size $n$ where $n$ is given by
    the \xmlNode{maxBatchSize} parameter below.
    Four options are available:
    \begin{itemize}
       \item \xmlString{none} If this is specified then the
       \xmlNode{maxBatchSize} parameter below will be ignored and the
       functionality will replicate the LimitSurfaceSearch, in that the limit
       surface will be rebuilt and the points will be re-scored after each trial
       is completed.
       \item \xmlString{naive} The top $n$ candidates will be queued for
       adaptive sampling before retraining the limit surface and re-scoring the
       new candidate set.
       \item \xmlString{maxP} The topology of the limit surface given the
       scoring function values will be decomposed and the top $n$ highest
       topologically persistent features (local maxima) will be queued for
       adaptive sampling before retraining and re-scoring the new candidate set.
       \item \xmlString{maxV} The topology of the limit surface given the
       scoring function values will be decomposed and the top $n$ highest
       topological features (local maxima) will be queued for adaptive sampling
       before retraining and re-scoring the new candidate set.
    \end{itemize}
  \default{none}.
  \item \xmlNode{maxBatchSize}, \xmlDesc{integer, optional field}, specifies
  the number of points to select for adaptive sampling before retraining the
  limit surface and re-scoring the candidates. This is the equivalent of the
  $n$ parameter used in the \xmlNode{batchStrategy} description.
  \default{1}.
  \item \xmlNode{scoring}, \xmlDesc{string, optional field}, defines the scoring
    function to use on the candidate limit surface points in order to select the
    next adaptive point.
    Two options are available:
    \begin{itemize}
       \item \xmlString{distance} will scoring the candidate points by their
       distance to the closest realized point, in this way preference is given
       to unexplored regions of the limit surface.
       \item \xmlString{distancePersistence} augments the distance above by
       multiplying it with the inverse persistence of a candidate point which
       measures how many times the label of the candidate point has changed
       throughout the lifespan of the algorithm.
    \end{itemize}
  \default{distancePersistence}.
  \item \xmlNode{simplification}, \xmlDesc{float in the range [0,1], optional
  field}, specifies the percent of the scoring function range (on the candidate
  set) as the amount of topological simplification to do before extracting the
  topological features from the candidate set (local maxima). This only applies
  when the \xmlNode{batchStrategy} is set to \xmlString{maxP} or
  \xmlString{maxV}. Thus, one may end up with a batch size less than that
  specified by \xmlNode{maxBatchSize}.
  \default{0}.
  \item \xmlNode{thickness}, \xmlDesc{positive integer, optional field},
  specifies how much the limit surface should be expanded (in terms of grid
  distance) when constructing a candidate set. A value of 1 implies only the
  points bounding the limit surface.
  \default{1}.
  \item \xmlNode{threshold}, \xmlDesc{float in the range [0,1], optional field},
  once the candidates have been ranked and selected, before queueing them for
  adaptive sampling, this value is used to threshold any points whose score is
  less than this percentage of the scoring function range (on the candidate
  set). Thus, one may end up with a batch size less than that specified by
  \xmlNode{maxBatchSize}.
  \default{0}
  % Limit Surface Search Objects
  \item \assemblerDescription{LimitSurfaceSearch}
    \begin{itemize}
      \item \xmlNode{Function}, \xmlDesc{string, required field},  the
        body of this XML block needs to contain the name of an external
        function object defined within the \xmlNode{Functions} main block (see
        Section~\ref{sec:functions}).
        %
        This object represents the boolean function that defines the transition
        boundaries.
        %
        This function must implement a method called
        \texttt{\_\_residuumSign(self)}, that returns either -1 or 1, depending
        on the system conditions (see Section \ref{sec:functions}.
      \item \xmlNode{ROM}, \xmlDesc{, string, optional  field}, if used, the
        body of this XML node must contain the name of a ROM defined in the
        \xmlNode{Models} block (see Section~\ref{subsec:models_ROM}). The ROM
        here specified is going to be used as ``acceleration model'' to speed up the
        convergence of the sampling strategy. The \xmlNode{Target} XML node in the ROM
        input block (within the \xmlNode{Models} section) needs to match the name of the goal
        \xmlNode{Function} (e.g. if the goal function is named ``transitionIdentifier'', the \xmlNode{Target} of the
        ROM needs to report the same name: \xmlNode{Target}\textbf{transitionIdentifier}\xmlNode{Target}).
      \item \xmlNode{TargetEvaluation}, \xmlDesc{string, required field},
        represents the container where the system evaluations are stored.
        %
        From a practical point of view, this XML node must contain the name of
        a data object defined in the \xmlNode{DataObjects} block (see
        Section~\ref{sec:DataObjects}). The object here specified must be
        input as  \xmlNode{Output} in the Steps that employ this sampling strategy.
        %
        The Limit Surface Search sampling accepts ``DataObjects'' of type
        ``PointSet'' only.
    \end{itemize}
\end{itemize}

Example:
\begin{lstlisting}[style=XML,morekeywords={class,limit,subGridTol,weight,persistence}]
<Samplers>
  ...
  <LimitSurfaceSearch name='LSSName'>
    <ROM class='Models' type='ROM'>ROMname</ROM>
    <Function class='Functions' type='External' >FunctionName</Function>
    <TargetEvaluation class='DataObjects' type='PointSet'>DataName</TargetEvaluation>
    <Convergence limit='3000'  forceIteration='False' weight='CDF'  subGridTol='1e-4' persistence='5'>
      1e-2
    </Convergence>
    <variable name='var1'>
      <distribution>aDistributionNameDefinedInDistributionBlock1 </distribution>
    </variable>
    <variable name='var2'>
      <distribution>aDistributionNameDefinedInDistributionBlock2 </distribution>
    </variable>
    <variable name='var3'>
      <distribution>aDistributionNameDefinedInDistributionBlock3 </distribution>
    </variable>
  </LimitSurfaceSearch>
  ...
</Samplers>
\end{lstlisting}

Batch sampling Example:
\begin{lstlisting}[style=XML,morekeywords={class,limit,subGridTol,weight,persistence}]
<Samplers>
  ...
  <LimitSurfaceSearch name='LSBSName'>
    <ROM class='Models' type='ROM'>ROMname</ROM>
    <Function class='Functions' type='External' >FunctionName</Function>
    <TargetEvaluation class='DataObjects' type='PointSet'>DataName</TargetEvaluation>
    <Convergence limit='3000'  forceIteration='False' weight='CDF'  subGridTol='1e-4' persistence='5'>
      1e-2
    </Convergence>
    <scoring>distancePersistence</scoring>
    <batchStrategy>maxP</batchStrategy>
    <thickness>1</thickness>
    <maxBatchSize>4</maxBatchSize>
    <variable name='var1'>
      <distribution>aDistributionNameDefinedInDistributionBlock1 </distribution>
    </variable>
    <variable name='var2'>
      <distribution>aDistributionNameDefinedInDistributionBlock2 </distribution>
    </variable>
    <variable name='var3'>
      <distribution>aDistributionNameDefinedInDistributionBlock3 </distribution>
    </variable>
  </LimitSurfaceSearch>
  ...
</Samplers>
\end{lstlisting}

Associated External Python Module:
\begin{lstlisting}[language=python]
def __residuumSign(self):
  if self.whatEverValue < self.OtherValue :
    return  1
  else:
    return -1
\end{lstlisting}

%%% Adaptive Samplers: ADET
\subsubsection{Adaptive Dynamic Event Tree}
\label{subsubsubsec:ADET}
The \textbf{Adaptive Dynamic Event Tree} approach is an advanced methodology
employing a smart sampling around transition zones that determine a change in
the status of the system (limit surface), using the support of a Dynamic Event
Tree methodology.
%
The main idea of the application of the previously explained adaptive sampling
approach to the DET comes from the observation that the DET, when evaluated from
a limit surface perspective, is intrinsically adaptive.
%
For this reason, it appears natural to use the DET approach to perform a
goal-function oriented pre-sampling of the input space.

RAVEN uses ROMs for predicting, in the input space,
the location(s) of these transitions, in order to accelerate the exploration of
the input space in proximity of the limit surface.

\specBlock{an}{AdaptiveDynamicEventTree}
%
\attrIntro

\begin{itemize}
  \itemsep0em
  \item \nameDescription
  \item \xmlAttr{printEndXmlSummary}, \xmlDesc{optional string/boolean attribute},
    this attribute controls the dumping of a ``summary'' of the DET performed in
    to an external XML.
    %
    \default{False}.
  \item \xmlAttr{maxSimulationTime}, \xmlDesc{optional float attribute}, this
    attribute controls the maximum ``mission'' time of the simulation
    underneath.
    %
    \default{None}.
  \item \xmlAttr{mode}, \xmlDesc{optional string attribute}, controls when the
    adaptive search needs to begin.
    %
    Two options are available:
    \begin{itemize}
       \item \xmlString{post}, if this option is activated, the sampler first
         performs a standard Dynamic Event Tree analysis. At end of it, it uses
         the outcomes to start the adaptive search in conjunction with the DET
         support.
       \item \xmlString{online}, if this option is activated, the adaptive
         search starts at the beginning, during the initial standard Dynamic
         Event Tree analysis.
         %
         Whenever a transition is detected, the
         \textbf{Adaptive Dynamic Event Tree} starts its goal-oriented search
         using the DET as support;
    \end{itemize}
      \default{post}.
  \item \xmlAttr{updateGrid}, \xmlDesc{optional boolean attribute}, if true,
    each adaptive request is going to update the meshing of the initial DET
    grid.
    %
    \default{True}.
\end{itemize}
\variableIntro{AdaptiveDynamicEventTree}
\begin{itemize}
\item \variableDescription
  \variableChildrenIntro
 \begin{itemize}
    \item \distributionDescription
    \item \functionDescription
    \item \gridDescription
  \end{itemize}
  \item \constantVariablesDescription
\end{itemize}

 In addition to the \xmlNode{variable} nodes, the main
\xmlNode{AdaptiveDynamicEventTree} node needs to contain two supplementary
sub-nodes:

\begin{itemize}
  \item \convergenceDescription
  % Assembler Objects
  \item \assemblerDescription{AdaptiveDynamicEventTree}
    \begin{itemize}
      \item \xmlNode{Function}, \xmlDesc{string, required field},  the
        body of this XML block needs to contain the name of an external
        function object defined within the \xmlNode{Functions} main block (see
        Section~\ref{sec:functions}).
        %
        This object represents the boolean function that defines the transition
        boundaries.
        %
        This function must implement a method called
        \texttt{\_\_residuumSign(self)}, that returns either -1 or 1, depending
        on the system conditions (see Section \ref{sec:functions}.
      \item \xmlNode{ROM}, \xmlDesc{, string, optional  field}, if used, the
        body of this XML node must contain the name of a ROM defined in the
        \xmlNode{Models} block (see Section~\ref{subsec:models_ROM}). The ROM
        here specified is going to be used as ``acceleration model'' to speed up the
        convergence of the sampling strategy. The \xmlNode{Target} XML node in the ROM
        input block (within the \xmlNode{Models} section) needs to match the name of the goal
        \xmlNode{Function} (e.g. if the goal function is named ``transitionIdentifier'', the \xmlNode{Target} of the
        ROM needs to report the same name: \xmlNode{Target}\textbf{transitionIdentifier}\xmlNode{Target}).
      \item \xmlNode{TargetEvaluation}, \xmlDesc{string, required field},
        represents the container where the system evaluations are stored.
        %
        From a practical point of view, this XML node must contain the name of
        a data object defined in the \xmlNode{DataObjects} block (see
        Section~\ref{sec:DataObjects}).
        %
        The adaptive sampling accepts ``DataObjects'' of type
        ``PointSet'' only.
    \end{itemize}
\end{itemize}


Example:
\begin{lstlisting}[style=XML]
<Samplers>
  ...
  <AdaptiveDynamicEventTree name = 'AdaptiveName'>
    <ROM class = 'Models' type = 'ROM'ROMname</ROM>
    <Function class = 'Functions' type = 'External'>FunctionName</Function>
    <TargetEvaluation class = 'DataObjects' type = 'PointSet'>DataName</TargetEvaluation>
    <Convergence limit = '3000' subGridTol= '0.001' forceIteration = 'False' weight = 'CDF' subGriTol='''1e-5' persistence = '5'>
      1e-2
    </Convergence>
    <variable name = 'var1'>
        <distribution>
         aDistributionNameDefinedInDistributionBlock1
        </distribution>
        <grid type='CDF' construction='custom'>0.1 0.8</grid>
    </variable>
    <variable name = 'var2'>
        <distribution>
          aDistributionNameDefinedInDistributionBlock2
        </distribution>
        <grid type='CDF' construction='custom'>0.1 0.8</grid>
    </variable>
    <variable name = 'var3'>
        <distribution>
          aDistributionNameDefinedInDistributionBlock3
        </distribution>
        <grid type='CDF' construction='custom'>0.1 0.8</grid>
    </variable>
  </AdaptiveDynamicEventTree>
  ...
</Samplers>
\end{lstlisting}

Associated External Python Module:
\begin{lstlisting}[language=python]
def __residuumSign(self):
  if self.whatEverValue < self.OtherValue:
    return  1
  else:
    return -1
\end{lstlisting}


%%% Adaptive Samplers: AHDET
\subsubsection{Adaptive Hybrid Dynamic Event Tree}
\label{subsubsubsec:AHDET}
The \textbf{Adaptive Hybrid Dynamic Event Tree} approach is an advanced methodology
employing a smart sampling around transition zones that determine a change in
the status of the system (limit surface), using the support of the Hybrid Dynamic Event
Tree methodology. Practically, this methodology represents a conjunction between the previously
described Adaptive DET and the Hybrid DET method for the treatment of the epistemic variables.

Regarding the input requirements, the AHDET sampler is a ``sub-type'' of the\\
\xmlNode{AdaptiveDynamicEventTree} sampler.
%
For this reason, its specifications must be defined within a
\xmlNode{AdaptiveDynamicEventTree} block.

\specBlock{an}{AdaptiveDynamicEventTree}
%
\attrIntro

\begin{itemize}
  \itemsep0em
  \item \nameDescription
  \item \xmlAttr{printEndXmlSummary}, \xmlDesc{optional string/boolean attribute},
    this attribute controls the dumping of a ``summary'' of the DET performed in
    to an external XML.
    %
    \default{False}.
  \item \xmlAttr{maxSimulationTime}, \xmlDesc{optional float attribute}, this
    attribute controls the maximum ``mission'' time of the simulation
    underneath.
    %
    \default{None}.
  \item \xmlAttr{mode}, \xmlDesc{optional string attribute}, controls when the
    adaptive search needs to begin.
    %
    Two options are available:
    \begin{itemize}
       \item \xmlString{post}, if this option is activated, the sampler first
         performs a standard Dynamic Event Tree analysis. At end of it, it uses
         the outcomes to start the adaptive search in conjunction with the DET
         support.
       \item \xmlString{online}, if this option is activated, the adaptive
         search starts at the beginning, during the initial standard Dynamic
         Event Tree analysis.
         %
         Whenever a transition is detected, the
         \textbf{Adaptive Dynamic Event Tree} starts its goal-oriented search
         using the DET as support;
    \end{itemize}
      \default{post}.
  \item \xmlAttr{updateGrid}, \xmlDesc{optional boolean attribute}, if true,
    each adaptive request is going to update the meshing of the initial DET
    grid.
    %
    \default{True}.
\end{itemize}

\variableIntro{AdaptiveDynamicEventTree}
\begin{itemize}
\item \variableDescription
  \variableChildrenIntro
 \begin{itemize}
    \item \distributionDescription
    \item \functionDescription
    \item \gridDescription
  \end{itemize}
  \item \constantVariablesDescription
\end{itemize}

In addition to the \xmlNode{variable} nodes, the main
\xmlNode{AdaptiveDynamicEventTree} node needs to contain two supplementary
sub-nodes:

\begin{itemize}
  \item \convergenceDescription
  % Assembler Objects
  \item \assemblerDescription{AdaptiveDynamicEventTree}
    \begin{itemize}
      \item \xmlNode{Function}, \xmlDesc{string, required field},  the
        body of this XML block needs to contain the name of an external
        function object defined within the \xmlNode{Functions} main block (see
        Section~\ref{sec:functions}).
        %
        This object represents the boolean function that defines the transition
        boundaries.
        %
        This function must implement a method called
        \texttt{\_\_residuumSign(self)}, that returns either -1 or 1, depending
        on the system conditions (see Section \ref{sec:functions}.
      \item \xmlNode{ROM}, \xmlDesc{, string, optional  field}, if used, the
        body of this XML node must contain the name of a ROM defined in the
        \xmlNode{Models} block (see Section~\ref{subsec:models_ROM}). The ROM
        here specified is going to be used as ``acceleration model'' to speed up the
        convergence of the sampling strategy. The \xmlNode{Target} XML node in the ROM
        input block (within the \xmlNode{Models} section) needs to match the name of the goal
        \xmlNode{Function} (e.g. if the goal function is named ``transitionIdentifier'', the \xmlNode{Target} of the
        ROM needs to report the same name: \xmlNode{Target}\textbf{transitionIdentifier}\xmlNode{Target}).
      \item \xmlNode{TargetEvaluation}, \xmlDesc{string, required field},
        represents the container where the system evaluations are stored.
        %
        From a practical point of view, this XML node must contain the name of
        a data object defined in the \xmlNode{DataObjects} block (see
        Section~\ref{sec:DataObjects}).
        %
        The adaptive sampling accepts ``DataObjects'' of type
        ``PointSet'' only.
    \end{itemize}
\end{itemize}

As it can be noticed, the basic specifications of the Adaptive Hybrid Dynamic Event Tree
method are consistent with the ones for the ADET methodology.
In order to activate the \textbf{Adaptive Hybrid Dynamic Event Tree}  sampler, the main
\xmlNode{AdaptiveDynamicEventTree} block needs to contain an additional
sub-node called \xmlNode{HybridSampler}.
This sub-node needs to contain the following attribute:
\begin{itemize}
  \item \xmlAttr{type}, \xmlDesc{required string attribute}, type of
    pre-sampling strategy to be used.
    %
    Up to now only one option is available:
    \begin{itemize}
      \item \xmlString{LimitSurface}. With this option, the epistemic variables here listed are going to be part of the LS search.
                                                        This means that the discretization of the domain of these variables is determined by the
                                                        \xmlNode{Convergece} node.
    \end{itemize}
 \end{itemize}
Independent of the type of HybridSampler that has been specified, the
\xmlNode{HybridSampler} must contain the variables that need to be sampled.
%
As already mentioned, these variables are specified within consecutive
\xmlNode{variable} XML blocks:
\begin{itemize}
  \item \variableDescription
    \variableChildrenIntro
    \begin{itemize}
      \item \distributionDescription
      \item \functionDescription
    \end{itemize}
  \item \constantVariablesDescription
 \end{itemize}


Example:
\begin{lstlisting}[style=XML]
<Samplers>
  ...
  <AdaptiveDynamicEventTree name = 'AdaptiveName'>
    <ROM class = 'Models' type = 'ROM'ROMname</ROM>
    <Function class = 'Functions' type = 'External'>FunctionName</Function>
    <TargetEvaluation class = 'DataObjects' type = 'PointSet'>DataName</TargetEvaluation>
    <Convergence limit = '3000' subGridTol= '0.001' forceIteration = 'False' weight = 'CDF' subGriTol='''1e-5' persistence = '5'>
      1e-2
    </Convergence>
    <HybridSampler type='LimitSurface'>
       <variable name = 'epistemicVar1'>
          <distribution>
            aDistributionNameDefinedInDistributionBlock1
          </distribution>
      </variable>
       <variable name = 'epistemicVar2'>
          <distribution>
            aDistributionNameDefinedInDistributionBlock2
          </distribution>
      </variable>
    </HybridSampler>
    <variable name = 'var1'>
        <distribution>
         aDistributionNameDefinedInDistributionBlock3
        </distribution>
        <grid type='CDF' construction='custom'>0.1 0.8</grid>
    </variable>
    <variable name = 'var2'>
        <distribution>
          aDistributionNameDefinedInDistributionBlock4
        </distribution>
        <grid type='CDF' construction='custom'>0.1 0.8</grid>
    </variable>
    <variable name = 'var3'>
        <distribution>
          aDistributionNameDefinedInDistributionBlock5
        </distribution>
        <grid type='CDF' construction='custom'>0.1 0.8</grid>
    </variable>

  </AdaptiveDynamicEventTree>
  ...
</Samplers>
\end{lstlisting}

Associated External Python Module:
\begin{lstlisting}[language=python]
def __residuumSign(self):
  if self.whatEverValue < self.OtherValue:
    return  1
  else:
    return -1
\end{lstlisting}


%%% Adaptive Samplers: Adaptive Sparse Grid Collocation
\subsubsection{Adaptive Sparse Grid}
\label{subsubsubsec:AdaptiveSparseGrid}
The \textbf{Adaptive Sparse Grid} approach is an advanced methodology that employs
an intelligent search for the most suitable sparse grid quadrature to characterize a model.
%
To perform such sampling, RAVEN adaptively builds an index set and generates sparse grids
in a similar manner to Sparse Grid Collocation samplers.  In each iterative step, the adaptive
index set determines the next possible quadrature orders to add in each dimension, and
determines the index set point that would offer the largest impact to one of the convergence
metrics.  This process continues until the total impact of all the potential index set points is
less than tolerance.  For many models, this function converges after fewer runs than a traditional
Sparse Grid Collocation sampling.  However, it should be noted that this algorithm fails
in the event that the partial derivative of the response surface with respect to any single
input dimension is zero at the origin of the input domain.  For example, the adaptive
algorithm fails for the model $f(x)=x\cdot y$.
%

\specBlock{an}{Adaptive Sparse Grid}
%
\attrIntro

\begin{itemize}
  \itemsep0em
  \item \nameDescription
\end{itemize}

\variableIntro{Adaptive Sparse Grid}

\begin{itemize}
  \item \variableDescription
    \variableChildrenIntro
    \begin{itemize}
      \item \distributionDescription
    \item \functionDescription
    \end{itemize}
    \item \constantVariablesDescription
\end{itemize}

In addition to the \xmlNode{variable} nodes, the main XML node
\xmlNode{AdaptiveSparseGrid} needs to contain the following supplementary sub-nodes:

\begin{itemize}
  \item \xmlNode{Convergence}, \xmlDesc{float, required field}, Convergence
    tolerance.
    %
    The meaning of this tolerance depends on the \xmlAttr{target} attribute of this node.
    \begin{itemize}
      \item \xmlAttr{target}, \xmlDesc{required string attribute}, the metric for convergence.
        The following metrics are available: \xmlString{variance}, which
        converges the sparse quadrature integration of the second moment of the model.%; and
        %\xmlString{coeffs}, which integrates the L2 norm of the coefficients of the polynomial
        %moments from a GaussPolynomialRom construction using the sparse grid.
        %
      \item \xmlAttr{maxPolyOrder}, \xmlDesc{optional integer attribute},
        limits the maximum size equivalent polynomial for any one dimension.
        %
        \default{10}.
      \item \xmlAttr{persistence}, \xmlDesc{optional integer attribute}, defines the number of
        index set points that are required to be found before calculation can exit.  Setting this to a higher
        value can help if the adaptive process is not finding significant indices on its own.
        %
        \default{2}.
    \end{itemize}
    In summary, this XML node contains the information that is needed in order
    to control this sampler's convergence criterion.
  \item \convergenceStudyDescription
  \item \xmlNode{logFile}, \xmlDesc{optional node},
    if included, the log file onto which the adaptive step progress can be printed.  The log includes the
    values of included polynomial coefficients as well as the expected impacts of polynomial coefficients not
    yet included.  This is different from
    the convergenceStudy print, which will give statistical moments at certain steps.
  \item \xmlNode{maxRuns}, \xmlDesc{optional node},
    if included, the adaptive sampler will track the number of computational solves necessary to construct the
    associated GaussPolynomialROM.  If at any point the number of solves exceeds the value given, it will not
    initiate any additional solves, and will exit when existing solves finish.
\end{itemize}
  % Adaptive Sparse Grid Objects
  %\assemblerDescription{Adaptive Sparse Grid}
  %\ROMDescription{Adaptive Sparse Grid}
  \assemblerDescription{Adaptive Sparse Grid}
       \ROMDescription{Adaptive Sparse Grid}
        %
        \begin{itemize}
      \item \xmlNode{TargetEvaluation}, \xmlDesc{string, required field},
        represents the container where the system evaluations are stored.
        %
        From a practical point of view, this XML node must contain the name of
        a data object defined in the \xmlNode{DataObjects} block (see
        Section~\ref{sec:DataObjects}).
        %
        The Adaptive Sparse Grid sampling accepts ``DataObjects'' of type
        ``PointSet'' only.
   % \end{itemize}
\end{itemize}

Example:
\begin{lstlisting}[style=XML,morekeywords={class,limit,subGridTol,weight,persistence}]
<Samplers>
  ...
  <AdaptiveSparseGrid name="ASG" verbosity='debug'>
    <Convergence target='coeffs'>1e-2</Convergence>
    <variable name="x1">
      <distribution>UniDist</distribution>
    </variable>
    <variable name="x2">
      <distribution>UniDist</distribution>
    </variable>
    <ROM class = 'Models' type = 'ROM'>gausspolyrom</ROM>
    <TargetEvaluation class = 'DataObjects' type = 'PointSet'>solns</TargetEvaluation>
  </AdaptiveSparseGrid>
  ...
</Samplers>
\end{lstlisting}

Like in the \textbf{SparseGridCollocation} sampler, if the covariance matrix is provided
and the input parameters are assumed to have the multivariate normal distribution, the \textbf{AdaptiveSparseGrid} can be also used.
This means one creates the sparse grids of variables listed by \xmlNode{latentVariables} in the transformed space. If this is
the case, the user needs to provide additional information, i.e. the \xmlNode{transformation} under \xmlNode{MultivariateNormal}
of \xmlNode{Distributions} (more information can be found in Section \ref{subsec:NdDist}). In addition, the node
\xmlNode{variablesTransformation} is also required for \textbf{AdaptiveSparseGrid} sampler. This node is used to tranform
the variables specified by \xmlNode{latentVariables} in the transformed space of input into variables spefified by
\xmlNode{manifestVariables} in the input space. The variables listed in \xmlNode{latentVariables} should be predefined
in \xmlNode{variable}, and the variables listed in \xmlNode{manifestVariables}
are used by the \xmlNode{Models}.

\variablesTransformationDescription{AdaptiveSparseGrid}


\begin{lstlisting}[style=XML,morekeywords={ND,grid}]
...
<Models>
    ...
    <ExternalModel ModuleToLoad="lorentzAttractor_noK" name="PythonModule" subType="">
        <variables>sigma,rho,beta,x,y,z,time,x0,y0,z0</variables>
    </ExternalModel>
    <ROM name="gausspolyrom" subType="GaussPolynomialRom">
        <Target>ans</Target>
        <Features>x1,y1,z1</Features>
        <IndexSet>TensorProduct</IndexSet>
        <PolynomialOrder>1</PolynomialOrder>
    </ROM>
    ...
</Models>

<Distributions>
    ...
    <MultivariateNormal name='MVNDist' method='pca'>
        <transformation>
            <rank>3</rank>
        </transformation>
        <mu>0.0 1.0 2.0</mu>
        <covariance type="abs">
            1.0       0.6      -0.4
            0.6       1.0      0.2
            -0.4      0.2      0.8
        </covariance>
    </MultivariateNormal>
    ...
</Distributions>

<Samplers>
  ...
  <AdaptiveSparseGrid name='ASC'>
        <variable name='x0'>
            <distribution dim='1'>MVNDist</distribution>
        </variable>
        <variable name='y0'>
            <distribution dim='2'>MVNDist</distribution>
        </variable>
        <variable name='z0'>
            <distribution dim='3'>MVNDist</distribution>
        </variable>
        <variablesTransformation model="PythonModule">
            <latentVariables>x1,y1,z1</latentVariables>
            <manifestVariables>x0,y0,z0</manifestVariables>
            <method>pca</method>
        </variablesTransformation>
        <ROM class = 'Models' type = 'ROM'>gausspolyrom</ROM>
        <TargetEvaluation class = 'DataObjects' type = 'PointSet'>solns</TargetEvaluation>
  </AdaptiveSparseGrid>
  ...
</Samplers>
...
\end{lstlisting}

\subsubsection{Adaptive Sobol Decomposition}
\label{subsubsubsec:AdaptiveSobol}
The \textbf{Adaptive Sobol Decomposition} approach is an advanced methodology that decomposes an uncertainty
space into subsets and adaptively includes the most influential ones.  For example, for a response function
$f(a,b,c)$, the full list of subsets include $(a), (b), (c), (a,b), (a,c), (b,c), (a,b,c)$.  A Gauss Polynomial ROM is
constructed for each included subset using the Adaptive Sparse Grid sampler.  The importance of each subset is
estimated based on the importance of preceding subsets; that is, the impact of $(a,b)$ on the representation
of $f$ is estimated using the impact of $(a)$ and $(b)$.  Because of the excellent performance of Gauss
Polynomial ROMs for small-dimension spaces, this sampler used to construct an HDMR ROM can be very efficient.
Note that the ROM specified for this sampler \emph{must} be an HDMRRom specified in the Models block.
%

\specBlock{an}{Adaptive Sobol}
%
\attrIntro

\begin{itemize}
  \itemsep0em
  \item \nameDescription
\end{itemize}

\variableIntro{Adaptive Sobol}

\begin{itemize}
  \item \variableDescription
    \variableChildrenIntro
    \begin{itemize}
      \item \distributionDescription
    \item \functionDescription
    \end{itemize}
    \item \constantVariablesDescription
\end{itemize}

In addition to the \xmlNode{variable} nodes, the main XML node
\xmlNode{AdaptiveSobol} needs to contain the following supplementary sub-nodes:

\begin{itemize}
  \item \xmlNode{Convergence}, \xmlDesc{required node}, Convergence
    properties.
    This node contains the following properties that can be set by sub-nodes:
    %
    \begin{itemize}
      \item \xmlNode{relTolerance}, \xmlDesc{required float}, the relative tolerance to converge.
        This will compare to the estimate of subset polynomial errors and additional subset polynomials over
        the variance of the expansion so far to determine convergence.
      \item \xmlNode{maxRuns}, \xmlDesc{optional integer field},
        a limit for the number of model calls.  Once this limit is reached, no additional subsets
        will be generated or considered; however, existing subsets will continue to be trained.  If not
        specified, no limit on solves is imposed.
      \item \xmlNode{maxSobolOrder}, \xmlDesc{optional integer field},
        the largest polynomials orders to use in subset GaussPolynomialRom objects.  If specified, polynomial
        indices with a value larger than the value given will be rejected during adaptive construction.
      \item \xmlNode{progressParam}, \xmlDesc{optional float field}, a favoritism parameter ranging between
        0 and 2.  At 0, the algorithm will always prefer adding polynomials to adding new subsets in the HDMR
        expansion.  At 2, the opposite is true.  Default is 1.
      \item \xmlNode{logFile}, \xmlDesc{optional string field},
        a file to which adaptive progress is recorded.  If specified, each adaptive step will trigger printing
        progress to the file given, including the estimated error at the step, the next adaptive step to take,
        the coefficient of each polynomial within each gPC expansion, and the actual and expected Sobol
        sensitivities of each HDMR subset. Default is no printing.
      \item \xmlNode{subsetVerbosity}, \xmlDesc{optional string field}, the verbosity for components
        constructed during the adaptive HDMR process.  Options are \emph{silent}, \emph{quiet}, \emph{all}, or
        \emph{debug}, in order of
        verbosity.  If an invalid entry is provided, will resort to default.  Default is \emph{quiet}.
        %
    \end{itemize}
    In summary, this XML node contains the information that is needed in order
    to control this sampler's convergence criterion.
  \item \convergenceStudyDescription
  Like the \textbf{Sobol}, if multivariate normal distribution is provided, the following node need to be specified:
  \item \variablesTransformationDescription{AdaptiveSobol}

\end{itemize}
  % Adaptive Sobol
  \assemblerDescription{AdaptiveSobol}
       \ROMDescription{AdaptiveSobol}
        %
        \begin{itemize}
      \item \xmlNode{TargetEvaluation}, \xmlDesc{string, required field},
        represents the container where the system evaluations are stored.
        %
        From a practical point of view, this XML node must contain the name of
        a data object defined in the \xmlNode{DataObjects} block (see
        Section~\ref{sec:DataObjects}).
        %
        The Adaptive Sobol sampling accepts ``DataObjects'' of type
        ``PointSet'' only.
   % \end{itemize}
\end{itemize}

Example:
\begin{lstlisting}[style=XML,morekeywords={class,limit,subGridTol,weight,persistence}]
<Samplers>
  ...
  <AdaptiveSobol name="AS" verbosity='debug'>
    <Convergence>
      <relTolerance>1e-5</relTolerance>
      <maxRuns>150</maxRuns>
      <maxSobolOrder>3</maxSobolOrder>
      <progressParam>1</progressParam>
      <logFile>progress.txt</logFile>
      <subsetVerbosity>silent</subsetVerbosity>
    </Convergence>
    <variable name="x1">
      <distribution>UniDist</distribution>
    </variable>
    <variable name="x2">
      <distribution>UniDist</distribution>
    </variable>
    <ROM class = 'Models' type = 'ROM'>hdmrrom</ROM>
    <TargetEvaluation class = 'DataObjects' type = 'PointSet'>solns</TargetEvaluation>
  </AdaptiveSobol>
  ...
</Samplers>
\end{lstlisting}

\section{Optimizers}
\label{sec:Optimizers}

%%%%%%%%%%%%%%%%%%%%%%%%%%%%%%%%%%%%%%%%%%%%%%%%%%%%%%%%%%%%%%%%%%%%%%%%%%%%%%%%
% If you are confused by the input of this document, please make sure you see
% these defined commands first. There is no point writing the same thing over
% and over and over and over and over again, so these will help us reduce typos,
% by just editing a template sentence or paragraph.
\renewcommand{\nameDescription}
{
  \xmlAttr{name},
  \xmlDesc{required string attribute}, user-defined name of this optimizer.
  \nb As with other objects, this identifier can be used to reference this
  specific entity from other input blocks in the XML.
}
\renewcommand{\specBlock}[2]
{
  The specifications of this optimizer must be defined within #1 \xmlNode{#2} XML
  block.
}


%%%%%%%%%%%%%%%%%%%%%%%%%%%%%%%%%%%%%%%%%%%%%%%%%%%%%%%%%%%%%%%%%%%%%%%%%%%%%%%%

The optimizer is another important entity in the RAVEN framework. It performs the driving of a specific goal function
over the model for value optimization. The difference between an optimizer and a sampler is that the former does not require
 sampling over a distribution, although certain specific optimizers may utilize stochastic approach to locate the optimality.
The optimizers currently available in RAVEN can be categorized into the following class(es):
\begin{itemize}
\item \textbf{Gradient Based Optimizer} (see Section~\ref{subsec:gradientBasedOptimizers})
\end{itemize}

Before analyzing each optimizer in detail, it is important to mention that each type needs to be contained in the main XML
node \xmlNode{Optimizers}, as reported below:

\textbf{Example:}

\begin{lstlisting}[style=XML]
<Simulation>
  ...
  <Optimizers>
    ...
    <WhatEverOptimizer name='whatever'>
      ...
    </WhatEverOptimizer>
    ...
  </Optimizers>
  ...
</Simulation>
\end{lstlisting}

It should be noted that gradient-based optimizers will not function without including a
\xmlNode{SolutionExport} HistorySet in the \xmlNode{MultiRun} step using the optimizer.

%%%%%%%%%%%%%%%%%%%%%%%%%
%%%      Gradient Based Optimizers      %%%
%%%%%%%%%%%%%%%%%%%%%%%%%
\subsection{Gradient Based Optimizers}
\label{subsec:gradientBasedOptimizers}
The Gradient Based Optimizer category collects all the strategies that perform the optimization based on gradient information,
 either directly provided or estimated by optimization strategy. In the RAVEN framework, currently implemented optimizer in this
 category are:
\begin{itemize}
\item \textbf{Simultaneous Perturbation Stochastic Approximation (SPSA)}
\item \textbf{Finite Difference Gradient Optimizer Forward (FiniteDifferenceGradientOptimizer)}
\end{itemize}

From a practical point of view, these optimization strategies represent different ways to estimate the gradient based on information
from previously performed model evaluation. In the following paragraphs, the input requirements and a small explanation of the
different sampling methodologies are reported.

Note that in addition to the input variables and response variable as well as other model outputs, several
other parameters are available to
request for the output of a Gradient-Based Optimizer run.  They include the following:
\begin{itemize}
  \item \xmlString{varsUpdate}, is the iteration number for each new optimal point;
  \item \xmlString{stepSize}, is the step size used to go from the previous optimal point to the current step;
  \item \xmlString{gradient\_var}, where \emph{var} is replaced by an input variable name, provides the
    gradient in the \emph{var} direction followed to arrive at the current optimal point (evaluated at the
    previous optimal point);
  \item \xmlString{convergenceRel}, the last-calculated relative convergence of the loss function value
    (see the description of the \xmlNode{convergence} node for more details);
  \item \xmlString{convergenceAbs}, the last-calculated absolute convergence of the loss function value
    (see the description of the \xmlNode{convergence} node for more details);
  \item \xmlString{convergenceGrad}, the last-calculated norm of the gradient used to arrive at the
    current point (see the description of the \xmlNode{convergence} node for more details).
\end{itemize}
Note that none of these additional parameters will be provided to the output DataObject by default; they must
be specifically requested by listing them in the output space when defining the optimizing step's output data
object.  Also note that if any of these parameters are not available (for instance, on the first iteration),
their output value will be set to -1, as this value is nonsensical for the step size and convergence values.

Example:
\begin{lstlisting}[style=XML]
<Optimizers>
  ...
  <AnyGradientBasedOptimizer name="anyname">
    <initialization>
      <limit>300</limit>
    </initialization>
    <TargetEvaluation class="DataObjects" type="PointSet">TEdataObjectName</TargetEvaluation>
    <convergence>
      <iterationLimit>50</iterationLimit>
      <relativeThreshold>1e-3</relativeThreshold>
      <absoluteThreshold>1e-1</absoluteThreshold>
      <gradientThreshold>1e-5</gradientThreshold>
      <persistence>1</persistence>
    </convergence>
    <parameter>
      <numGradAvgIterations>3</numGradAvgIterations>
      <normalize>False</normalize>
    </parameter>
    <variable name="var1">
      <upperBound>100</upperBound>
      <lowerBound>-100</lowerBound>
      <initial>0</initial>
    </variable>
    <objectVar>c</objectVar>
  </SPSA>
  ...
</Optimizers>
..
<DataObjects>
  ...
  <PointSet name='optOut'>
    <Input>x,y</Input>
    <Output>z</Output>
  </PointSet>
  <HistorySet name='opt_export'>
    <Input>trajID</Input>
    <Output>
      x,y,z,varsUpdate,stepSize,
      gradient_x,gradient_y,
      convergenceAbs,convergenceRel,convergenceGrad
    </Output>
  </HistorySet>
  ...
</DataObjects>
\end{lstlisting}

%%% Gradient Based Optimizers: SPSA
\subsubsection{Simultaneous Perturbation Stochastic Approximation (SPSA)}
\label{subsubsubsec:SPSA}
The \textbf{SPSA} optimization approach is one of the optimization strategies that are based on gradient estimation. The main
idea is to simultaneously perturb all decision variables in order to estimate the gradient. Consequently a minimal number of two
model evaluations are required in order to approximate the gradient. The theory behind SPSA can be found in
\cite{spall1998implementation}.

In addition to the algorithm in \cite{spall1998implementation}, current implementation of \textbf{SPSA} can also handles
constrained optimization problem. This paragraph briefly describes how current implementation ensures the input satisfies the
constraints. When when updating the variables (not perturbing), if constraint is violated, \textbf{SPSA} does the following in
sequence:
\begin{itemize}
\item Try to find, through bisection method, the longest fraction of gradient vector so that the variable update satisfies the
constraints;
\item When such fraction cannot be found, then find a random vector orthogonal to gradient vector so that, by using this
orthogonal vector as gradient, the variable update satisfies the constraints. Rotate the orthogonal vector towards the gradient,
through bisection methods, until constraints can no longer be satisfied;
\item If all above cannot return a constraint satisfying variable update, then do not update the variables and the \textbf{SPSA} will
terminate.
\end{itemize}

It is important to notice that the gradient and the feature space is always normalized. This means that the gradient is going to be
normalized with respect to its norm (versor of the gradient); hence, the optimization advancement is not going to be influenced by the
magnitude of the gradient, but just on its ``direction'' information content. All the following parameters that can be optionally be inputted
should be calibrated with this information in mind.
%

\specBlock{a}{SPSA}
%
\attrsIntro
\vspace{-5mm}
\begin{itemize}
\itemsep0em
\item \xmlAttr{name}, \xmlDesc{required string attribute}, user-defined name of this optimizer. \nb As for the other objects, this is
the name that can be used to refer to this specific entity from other input blocks (xml);
\end{itemize}
\vspace{-5mm}

In the \xmlNode{SPSA} input block, the user needs to specify the objective variable to be optimized, the decision variables, the
DataObject storing previously performed model evaluation, as well as convergence criteria. In addition, the settings for this
optimization can be specified in the \xmlNode{initialization} and \xmlNode{parameter} XML blocks:
\begin{itemize}
\item \xmlNode{initialization},  \xmlDesc{XML node, optional parameter}. In this xml-node,the following xml sub-nodes can be
specified:
  \begin{itemize}
    \item \xmlNode{limit}, \xmlDesc{integer,optional field}, number of samples to be generated, which is same as the number of
    model evaluation. \default{2000}.
    \item \xmlNode{initialSeed}, \xmlDesc{integer, optional field}, initial seeding of random number generator for stochastic
    perturbations;
    \item \xmlNode{type},  \xmlDesc{string (case insensitive), optional field}, specifies whether this optimizer performs maximization
    or minimization. Available options are \xmlString{max} and \xmlString{min}.
    \default{Min};
    \item \xmlNode{thresholdTrajRemoval}, \xmlDesc{float, optional field}, this will be used to determine the convergence of different
    optimization trajectories on each other when multiple trajectories is handled by \xmlNode{SPSA}.  When one
    trajectory comes within tolerance of a point on another trajectory, the first will be removed in interest
    of the second.  Note that this value is
    calculated as Euclidean distance in a normalized 0 to 1 cubic domain, not the original input space domain.
    \default{0.05}
    \item \xmlNode{writeSteps},  \xmlDesc{string, optional field}, specifies how often the current optimal
      point should be stored to the solution export.  Options are \xmlString{every}, in which case each new
      optimal point will be stored in the solution export; or \xmlString{final}, in which case only the most
      optimal point found during the simulation will be stored in the solution export.
    \default{every};

  \end{itemize}
\end{itemize}
\begin{itemize}
\item \xmlNode{TargetEvaluation}, \xmlDesc{XML node, required parameter},
        represents the container where the model evaluations are stored.
        %
        From a practical point of view, this XML node must contain the name of
        a data object defined in the \xmlNode{DataObjects} block (see
        Section~\ref{sec:DataObjects}). The object here specified must be
        input as  \xmlNode{Output} in the Steps that employ this optimization strategy.
        %
        The \xmlNode{SPSA} optimizer accepts ``DataObjects'' of type ``PointSet'' only;
        \item \xmlNode{objectVar}, \xmlDesc{XML node, required parameter}. The objective variable to be optimized. This variable must be
          output of the DataObject specified in \xmlNode{TargetEvaluation}.
\item \xmlNode{Sampler}, \xmlDesc{XML node, optional parameter},
        represents a Sampler (Forward) that can be used to initialize the starting points for the trajectories of some of the variables.
        %
        From a practical point of view, this XML node must contain the name of
        a Sampler (Forward) defined in the \xmlNode{Samplers} block (see
        Section~\ref{subsec:onceThroughSamplers}). The Sampler will be used to initialize the trajectories' initial points for some
        of the variables. For example, if the Sampler here specified ``samples'' only 2 variables over 5, the  \xmlNode{initial} XML node (see below) is required
        only for the remaining 3 variables.
\item \xmlNode{Function}, \xmlDesc{XML node, optional parameter},
        indicates the external function where the constraints are stored. From a practical point of view, this XML node must contain the
        name of a function defined in the \xmlNode{Functions} block (see Section~\ref{sec:functions}). This external function must
        contain a method called ``constrain'', which returns 1 for inputs satisfying the constraints and 0 otherwise.
\item \xmlNode{Preconditioner}, \xmlDesc{XML node, optional parameter},
        provides a model that can be used as a preconditioner in Multilevel optimization
        calculations.  Only
        affects optimizers with a \xmlNode{multilevel} node.  As many preconditioners as
        desired can be added
        to the optimizer, each defined with a \xmlNode{Preconditioner} node.
        %
        From a practical point of view, this XML node must contain the name of
        an \xmlNode{ExternalModel} defined in the \xmlNode{Models} block (see
        Section~\ref{subsec:models_externalModel}).
        %
        In multilevel optimization, the preconditioner is attached to a particular subspace.  Whenever
        subspaces that are ``higher'' (early in \xmlNode{sequence}) are perturbed, before moving to a lower
        subspace, the preconditioner will be called to provide a new value for each variable in the lower
        subspace.
        %
        For example, if an input space is divided into one subspace \xmlString{subx} with the input variable $x$ and
        another subspace \xmlString{suby} with input variable $y$, and if the sequence is specified as
        \xmlString{subx,suby}, and a preconditioner is attached to subspace \xmlString{suby}, then when a step
        is taken for subspace \xmlString{subx}, the preconditioner will provide a new value for $y$ before
        starting a convergence search for $y$.
\item \xmlNode{multilevel}, \xmlDesc{XML node, optional node}, engages the optimizer in \emph{multilevel}
        mode.  When in multilevel mode, the input space is divided into multiple subspaces.  The subspaces are
        then aligned in a sequence, and optimizing follows the following procedure:
        \begin{enumerate}
          \item Hold all variable values in all subspaces constant EXCEPT the last subspace in the sequence.
          \item Converge the optimizer considering only input variables in the last listed subspace.
          \item Hold all variables in the last subspace constant, and take a single optimizing step in the
            second-to-last subspace.
          \item If the second to last subspace is converged, go up one more subspace and take a step, then
            repeat the process thus far.
          \item If the second to last subspace is not converged, go back to the last subspace and converge it
            again.
          \item Et cetera.
        \end{enumerate}
        Once the outermost subspace is converged, the entire space is considered converged.
        %
        Note that multilevel optimization is not in general better than not using it.  Multilevel works
        especially well when some variables in the input space are connected and have relatively
        difficult-to-converge optimization, while other variables in the input space are easily converged.  In
        this case, the difficult-to-converge variables should make up the last subspace in the sequence, while
        the easily-converging variables should make up the outer subspace.
        %
        RAVEN places no limit on the
        number of subspaces that are defined, but each variable should only exist in a single subspace.
        %
        The \xmlNode{multilevel} node requires the definition of subspaces and the sequence as follows:
        \begin{itemize}
          \item \xmlNode{sequence}, \xmlDesc{comma-separated string, required parameter}, lists the order in
            which subspaces should be converged.  Each subspace is listed as identified by its \xmlAttr{name}
            parameter in the \xmlNode{subspace} definition.  Note that the first subspace listed will be the
            slowest to converge and converge only once, and the last subspace listed will be converged
            frequently and quickly.
          \item \xmlNode{subspace}, \xmlDesc{comma-separated string, required parameter}, lists the variables
            included in this subspace.  This node additionally has the following attributes:
            \begin{itemize}
              \item \xmlAttr{name}, \xmlDesc{string, require parameter}, provides the identifier that RAVEN
                will use for this subspace group, both in the \xmlNode{sequence} node as well as in log
                prints.
              \item \xmlAttr{precond}, \xmlDesc{model name, optional parameter}, provides the option to attach
                a preconditioner to this subset, chosen from the \xmlNode{Preconditioner} nodes defined within
                the optimzer, and identified by the text of those nodes.  See the documentation for the
                preconditioner node above for details on how they affect the calculation flow.
            \end{itemize}
        \end{itemize}

\end{itemize}
\begin{itemize}
\item \variableDescription
 The variable specified here must be input of the DataObject specified in \xmlNode{TargetEvaluation}.
 \variableChildrenIntro
 \begin{itemize}
    \item \xmlNode{upperBound}, \xmlDesc{float, required field}, the upper bound of this variable;
    \item \xmlNode{lowerBound}, \xmlDesc{float, required field}, the lower bound of this variable;
    \item \xmlNode{initial}, \xmlDesc{comma separated strings, optional field}, the initial value(s) for this variable. If there are more
    than one initial values specified for a variable, then all the variables need to have the same number of initial values. In this case,
    \xmlNode{SPSA} optimizer will maintain multiple trajectories to fully utilize potential parallel computing capability.
    Every input variable must have an initial value specified either through this node, or through a
    preconditioner in multilevel optimization or through a linked Sampler (see above).
  \end{itemize}
\item \constantVariablesDescription
\item \xmlNode{convergence}, \xmlDesc{XML node, optional parameter} will specify parameters associated with optimization
convergence. This node accepts the following sub-nodes:
  \begin{itemize}
  \item \xmlNode{iterationLimit}, \xmlDesc{integer, optional field}, user-defined maximum number of optimization iterations. \default{650}.
  \item \xmlNode{persistence}, \xmlDesc{integer, optional field}, number of consecutive successful
    convergences required before completing calculation (per trajectory). Any value less than 1 will be
    treated as 1.  Float values are rounded down to the nearest integer. \default{1}.
  \item \xmlNode{relativeThreshold}, \xmlDesc{float, optional field}, specifies the convergence criteria to determine the optimality
  in a ``relative'' sense: when the relative change of the objective variable in two successive model evaluations is smaller than
  this specified threshold, the \xmlNode{SPSA} optimizer is in convergence and terminates the simulation.
      \default{1e-3}
  \item \xmlNode{absoluteThreshold}, \xmlDesc{float, optional field}, specifies the convergence criteria to determine the optimality,
  in an ``absolute'' sense: when the absolute change of objective variable in two successive model evaluations is smaller
  than this specified threshold, the \xmlNode{SPSA} optimizer is in convergence and terminates the simulation.
      \default{0.0}
  \item \xmlNode{gradientThreshold}, \xmlDesc{float, optional field}, specifies the convergence criteria to determine the optimality,
   as function of the L2 norm of the gradient (useful for unconstrained problems): when the L2 norm od the gradient falls below this threshold, the \xmlNode{SPSA} optimizer is in convergence and terminates the simulation.
      \default{1e-3}
  \item \xmlNode{minStepSize}, \xmlDesc{float, optional field}, specifies the minimum allowable step size in
    the normalized input space, ranging from 0 (no movement) to 1 (spans any dimension).
      \default{1e-9}
  \item \xmlNode{gainGrowthFactor}, \xmlDesc{float, optional field}, specifies the rate at which the step size
    should grow when it does grow, for instance when multiple steps are in the same direction.  Increasing
    this will increase the likelihood that an optimization path travels quickly across the domain along a
    consistent gradient.
      \default{2}
  \item \xmlNode{gainShrinkFactor}, \xmlDesc{float, optional field}, specifies the rate at which the step size
    should shrink when it does shrink, for instance when switching directions on successive steps.  Increasing
    this will slow convergence, but decrease the likelihood of achieving false convergence due to small step
    sizes.
      \default{same value as gainGrowthFactor}
  \end{itemize}
\item \xmlNode{parameter}, \xmlDesc{XML node, optional parameter} will accepts the following sub-nodes:
  \begin{itemize}
  \item \xmlNode{numGradAvgIterations}, \xmlDesc{integer, optional field} is the number of iterations for gradient estimation. When this
        parameter is $>1$, multiple gradient evaluations are going to be performed. Since the main goal for this parameter is to
        get a better gradient estimation (performing a denoising), the current point $x_k$ is evaluated multiple times in order to be able to
        converge in average.
        \default{1}

  \item \xmlNode{stochasticDistribution}, \xmlDesc{string, optional field} determines the process used to find
        gradient evaluation perturbation points as part of SPSA. Choice include the following:
        \begin{itemize}
          \item \xmlString{Hypersphere}, which chooses from all possible directions with equal probability,
          \item \xmlString{Bernoulli}, which limits directions closely to the diagonal directions (corners of
                                       a hypercube).
        \end{itemize}
        \default{Hypersphere}

  \item Optimizer Gradient Evaluation parameters:
    \begin{itemize}
      \item \xmlNode{gamma}, \xmlDesc{float, optional field} Inverse exponent for gradient evaluation distance. Increasing this
        parameter will greatly decrease the distance between points sampled in evaluating the gradient
        \cite{spall1998implementation}. A practical suggestion for $\gamma$ is 0.101 (paired with an
        $\alpha$ value of 0.602); however, the asymptotic limit is $\gamma=1/6$ ($\alpha=1$). \default{0.101}
      \item \xmlNode{c}, \xmlDesc{float, optional field} Step size coefficient.  This term determines the
        nominal step size, and increasing it will directly increase the distance between points sampled in evaluating the gradient
        \cite{spall1998implementation}. It is suggested this parameter be approximately equal to the standard
        deviation of the measurement noise in the response for stochastic responses.  For regular responses,
        it can be a small arbitrary value. \default{0.005}
    \end{itemize}

  \item Optimizer Step Size parameters:
    \begin{itemize}
      \item \xmlNode{a}, \xmlDesc{float, optional field} Nominal optimizer step size parameter.  Increasing
        this parameter will directly increase the distance traversed in each optimizer step
        \cite{spall1998implementation}. In contrast to $A$, this parameter will be unchanged by increasing
        iterations, and so will be more impacting as the optimization algorithm iterates. \default{0.16}
      \item \xmlNode{alpha}, \xmlDesc{float, optional field} Inverse exponent for optimizer step size.
        Increasing this parameter will greatly decrease the distance traversed in each optimizer step
        \cite{spall1998implementation}. Values less than 1 for $\alpha$ usually yield better performance by
        keeping a large step size. See the description of $\gamma$ above for some suggested values. \default{0.602}
      \item \xmlNode{A}, \xmlDesc{float, optional field} Nominal step damping stability parameter.  Increasing this
        parameter will directly decrease the distance traversed in each optimizer step
        \cite{spall1998implementation}. This parameter will have greater affect in reducing step size early in
        the calculation, and reduced affect as iterations increase. \default{\xmlNode{limit} divided by 10}
    \end{itemize}
  \item \xmlNode{innerBisectionThreshold}, \xmlDesc{float, optional field} a parameter specifying the convergence threshold of the
  bisection method used in constraint handling (See above). This parameter shall be in the open inverval $(0,1)$.
        \default{0.01}
  \item \xmlNode{innerLoopLimit}, \xmlDesc{integer, optional field} a parameter specifying the number of orthogonal vectors to try
  when handling the constraints (See above).
        \default{1000}
  \end{itemize}
\end{itemize}


Example:
\begin{lstlisting}[style=XML]
<Optimizers>
  ...
  <SPSA name="SPSAname">
    <initialization>
      <limit>300</limit>
      <type>min</type>
      <initialSeed>30</initialSeed>
    </initialization>
    <TargetEvaluation class="DataObjects" type="PointSet">dataObjectName</TargetEvaluation>
    <convergence>
      <iterationLimit>50</iterationLimit>
      <relativeThreshold>1e-3</relativeThreshold>
      <absoluteThreshold>1e-1</absoluteThreshold>
      <gradientThreshold>1e-5</gradientThreshold>
      <persistence>1</persistence>
    </convergence>
    <parameter>
      <numGradAvgIterations>3</numGradAvgIterations>
    </parameter>
    <variable name="var1">
      <upperBound>100</upperBound>
      <lowerBound>-100</lowerBound>
      <initial>0</initial>
    </variable>
    <objectVar>c</objectVar>
  </SPSA>
  ...
</Optimizers>
\end{lstlisting}

%%% Gradient Based Optimizers: FiniteDifferenceGradientOptimizer
\subsubsection{Finite Difference Gradient Optimizer (FiniteDifferenceGradientOptimizer)}
\label{subsubsubsec:FiniteDifferenceGradientOptimizer}
The \textbf{FiniteDifferenceGradientOptimizer} optimization approach is  the simplest Gradient based approach since it is based on the
first order evaluation of the Gradient.  A minimal number of $n variable$
model evaluations are required in order to get a first order approximation of the gradient.

Current implementation of \textbf{FiniteDifferenceGradientOptimizer} can also handles
constrained optimization problem. This paragraph briefly describes how current implementation ensures the input satisfies the
constraints. When when updating the variables (not perturbing), if constraint is violated, \textbf{FiniteDifferenceGradientOptimizer} does the following in
sequence:
\begin{itemize}
\item Try to find, through bisection method, the longest fraction of gradient vector so that the variable update satisfies the
constraints;
\item When such fraction cannot be found, then find a random vector orthogonal to gradient vector so that, by using this
orthogonal vector as gradient, the variable update satisfies the constraints. Rotate the orthogonal vector towards the gradient,
through bisection methods, until constraints can no longer be satisfied;
\item If all above cannot return a constraint satisfying variable update, then do not update the variables and the \textbf{FiniteDifferenceGradientOptimizer} will
terminate.
\end{itemize}

It is important to notice that the gradient and the feature space is always normalized. This means that the gradient is going to be
normalized with respect to its norm (versor of the gradient); hence, the optimization advancement is not going to be influenced by the
magnitude of the gradient, but just on its ``direction'' information content. All the following parameters that can be optionally be inputted
should be calibrated with this information in mind.
%

\specBlock{a}{FiniteDifferenceGradientOptimizer}
%
\attrsIntro
\vspace{-5mm}
\begin{itemize}
\itemsep0em
\item \xmlAttr{name}, \xmlDesc{required string attribute}, user-defined name of this optimizer. \nb As for the other objects, this is
the name that can be used to refer to this specific entity from other input blocks (xml);
\end{itemize}
\vspace{-5mm}

In the \xmlNode{FiniteDifferenceGradientOptimizer} input block, the user needs to specify the objective variable to be optimized, the decision variables, the
DataObject storing previously performed model evaluation, as well as convergence criteria. In addition, the settings for this
optimization can be specified in the \xmlNode{initialization} and \xmlNode{parameter} XML blocks:
\begin{itemize}
\item \xmlNode{initialization},  \xmlDesc{XML node, optional parameter}. In this xml-node,the following xml sub-nodes can be
specified:
  \begin{itemize}
    \item \xmlNode{limit}, \xmlDesc{integer,optional field}, number of samples to be generated, which is same as the number of
    model evaluation. \default{2000}.
    \item \xmlNode{initialSeed}, \xmlDesc{integer, optional field}, initial seeding of random number generator for stochastic
    perturbations;
    \item \xmlNode{type},  \xmlDesc{string (case insensitive), optional field}, specifies whether this optimizer performs maximization
    or minimization. Available options are \xmlString{max} and \xmlString{min}.
    \default{Min};
    \item \xmlNode{thresholdTrajRemoval}, \xmlDesc{float, optional field}, this will be used to determine the convergence of different
    optimization trajectories on each other when multiple trajectories is handled by \xmlNode{FiniteDifferenceGradientOptimizer}.  When one
    trajectory comes within tolerance of a point on another trajectory, the first will be removed in interest
    of the second.  Note that this value is
    calculated as Euclidean distance in a normalized 0 to 1 cubic domain, not the original input space domain.
    \default{0.05}
  \end{itemize}
\end{itemize}
\begin{itemize}
\item \xmlNode{TargetEvaluation}, \xmlDesc{XML node, required parameter},
        represents the container where the model evaluations are stored.
        %
        From a practical point of view, this XML node must contain the name of
        a data object defined in the \xmlNode{DataObjects} block (see
        Section~\ref{sec:DataObjects}). The object here specified must be
        input as  \xmlNode{Output} in the Steps that employ this optimization strategy.
        %
        The \xmlNode{FiniteDifferenceGradientOptimizer} optimizer accepts ``DataObjects'' of type ``PointSet'' only;
        \item \xmlNode{objectVar}, \xmlDesc{XML node, required parameter}. The objective variable to be optimized. This variable must be
          output of the DataObject specified in \xmlNode{TargetEvaluation}.
\item \xmlNode{Sampler}, \xmlDesc{XML node, optional parameter},
        represents a Sampler (Forward) that can be used to initialize the starting points for the trajectories of some of the variables.
        %
        From a practical point of view, this XML node must contain the name of
        a Sampler (Forward) defined in the \xmlNode{Samplers} block (see
        Section~\ref{subsec:onceThroughSamplers}). The Sampler will be used to initialize the trajectories' initial points for some
        of the variables. For example, if the Sampler here specified ``samples'' only 2 variables over 5, the  \xmlNode{initial} XML node (see below) is required
        only for the remaining 3 variables.
\item \xmlNode{Function}, \xmlDesc{XML node, optional parameter},
        indicates the external function where the constraints are stored. From a practical point of view, this XML node must contain the
        name of a function defined in the \xmlNode{Functions} block (see Section~\ref{sec:functions}). This external function must
        contain a method called ``constrain'', which returns 1 for inputs satisfying the constraints and 0 otherwise.
\item \xmlNode{Preconditioner}, \xmlDesc{XML node, optional parameter},
        provides a model that can be used as a preconditioner in Multilevel optimization
        calculations.  Only
        affects optimizers with a \xmlNode{multilevel} node.  As many preconditioners as
        desired can be added
        to the optimizer, each defined with a \xmlNode{Preconditioner} node.
        %
        From a practical point of view, this XML node must contain the name of
        an \xmlNode{ExternalModel} defined in the \xmlNode{Models} block (see
        Section~\ref{subsec:models_externalModel}).
        %
        In multilevel optimization, the preconditioner is attached to a particular subspace.  Whenever
        subspaces that are ``higher'' (early in \xmlNode{sequence}) are perturbed, before moving to a lower
        subspace, the preconditioner will be called to provide a new value for each variable in the lower
        subspace.
        %
        For example, if an input space is divided into one subspace \xmlString{subx} with the input variable $x$ and
        another subspace \xmlString{suby} with input variable $y$, and if the sequence is specified as
        \xmlString{subx,suby}, and a preconditioner is attached to subspace \xmlString{suby}, then when a step
        is taken for subspace \xmlString{subx}, the preconditioner will provide a new value for $y$ before
        starting a convergence search for $y$.
\item \xmlNode{multilevel}, \xmlDesc{XML node, optional node}, engages the optimizer in \emph{multilevel}
        mode.  When in multilevel mode, the input space is divided into multiple subspaces.  The subspaces are
        then aligned in a sequence, and optimizing follows the following procedure:
        \begin{enumerate}
          \item Hold all variable values in all subspaces constant EXCEPT the last subspace in the sequence.
          \item Converge the optimizer considering only input variables in the last listed subspace.
          \item Hold all variables in the last subspace constant, and take a single optimizing step in the
            second-to-last subspace.
          \item If the second to last subspace is converged, go up one more subspace and take a step, then
            repeat the process thus far.
          \item If the second to last subspace is not converged, go back to the last subspace and converge it
            again.
          \item Et cetera.
        \end{enumerate}
        Once the outermost subspace is converged, the entire space is considered converged.
        %
        Note that multilevel optimization is not in general better than not using it.  Multilevel works
        especially well when some variables in the input space are connected and have relatively
        difficult-to-converge optimization, while other variables in the input space are easily converged.  In
        this case, the difficult-to-converge variables should make up the last subspace in the sequence, while
        the easily-converging variables should make up the outer subspace.
        %
        RAVEN places no limit on the
        number of subspaces that are defined, but each variable should only exist in a single subspace.
        %
        The \xmlNode{multilevel} node requires the definition of subspaces and the sequence as follows:
        \begin{itemize}
          \item \xmlNode{sequence}, \xmlDesc{comma-separated string, required parameter}, lists the order in
            which subspaces should be converged.  Each subspace is listed as identified by its \xmlAttr{name}
            parameter in the \xmlNode{subspace} definition.  Note that the first subspace listed will be the
            slowest to converge and converge only once, and the last subspace listed will be converged
            frequently and quickly.
          \item \xmlNode{subspace}, \xmlDesc{comma-separated string, required parameter}, lists the variables
            included in this subspace.  This node additionally has the following attributes:
            \begin{itemize}
              \item \xmlAttr{name}, \xmlDesc{string, require parameter}, provides the identifier that RAVEN
                will use for this subspace group, both in the \xmlNode{sequence} node as well as in log
                prints.
              \item \xmlAttr{precond}, \xmlDesc{model name, optional parameter}, provides the option to attach
                a preconditioner to this subset, chosen from the \xmlNode{Preconditioner} nodes defined within
                the optimzer, and identified by the text of those nodes.  See the documentation for the
                preconditioner node above for details on how they affect the calculation flow.
              \item \xmlAttr{holdOutputSpace}, \xmlDesc{parameter names, optional comma separated parameter}, provides the option to identify some output parameters that
              need to be kept on hold at this subspace optimization level. This capability is
              currently implemented for the \textit{EnsembleModel} only.  In other words,
              all the models that have, in their output spaces, the parameter here specified
              will not be re-run for the iteration $i$ but the solution at iteration $i-1$ will be
              used.
            \end{itemize}
        \end{itemize}

\end{itemize}
\begin{itemize}
\item \variableDescription
 The variable specified here must be input of the DataObject specified in \xmlNode{TargetEvaluation}.
 \variableChildrenIntro
 \begin{itemize}
    \item \xmlNode{upperBound}, \xmlDesc{float, required field}, the upper bound of this variable;
    \item \xmlNode{lowerBound}, \xmlDesc{float, required field}, the lower bound of this variable;
    \item \xmlNode{initial}, \xmlDesc{comma separated strings, optional field}, the initial value(s) for this variable. If there are more
    than one initial values specified for a variable, then all the variables need to have the same number of initial values. In this case,
    \xmlNode{FiniteDifferenceGradientOptimizer} optimizer will maintain multiple trajectories to fully utilize potential parallel computing capability.
    Every input variable must have an initial value specified either through this node, or through a
    preconditioner in multilevel optimization or through a linked Sampler (see above).
  \end{itemize}
\end{itemize}
\constantVariablesDescription
\begin{itemize}
\item \xmlNode{convergence}, \xmlDesc{XML node, optional parameter} will specify parameters associated with optimization
convergence. This node accepts the following sub-nodes:
  \begin{itemize}
  \item \xmlNode{iterationLimit}, \xmlDesc{integer, optional field}, user-defined maximum number of optimization iterations. \default{650}.
  \item \xmlNode{persistence}, \xmlDesc{integer, optional field}, number of consecutive successful
    convergences required before completing calculation (per trajectory). Any value less than 1 will be
    treated as 1.  Float values are rounded down to the nearest integer. \default{1}.
  \item \xmlNode{relativeThreshold}, \xmlDesc{float, optional field}, specifies the convergence criteria to determine the optimality
  in a ``relative'' sense: when the relative change of the objective variable in two successive model evaluations is smaller than
  this specified threshold, the \xmlNode{FiniteDifferenceGradientOptimizer} optimizer is in convergence and terminates the simulation.
      \default{1e-3}
  \item \xmlNode{absoluteThreshold}, \xmlDesc{float, optional field}, specifies the convergence criteria to determine the optimality,
  in an ``absolute'' sense: when the absolute change of objective variable in two successive model evaluations is smaller
  than this specified threshold, the \xmlNode{FiniteDifferenceGradientOptimizer} optimizer is in convergence and terminates the simulation.
      \default{0.0}
  \item \xmlNode{gradientThreshold}, \xmlDesc{float, optional field}, specifies the convergence criteria to determine the optimality,
   as function of the L2 norm of the gradient (useful for unconstrained problems): when the L2 norm od the gradient falls below this threshold, the \xmlNode{FiniteDifferenceGradientOptimizer} optimizer is in convergence and terminates the simulation.
      \default{1e-3}
  \item \xmlNode{minStepSize}, \xmlDesc{float, optional field}, specifies the minimum allowable step size in
    the normalized input space, ranging from 0 (no movement) to 1 (spans any dimension).
      \default{1e-9}
  \item \xmlNode{gainGrowthSize}, \xmlDesc{float, optional field}, specifies the rate at which the step size
    should grow when it does grow, for instance when multiple steps are in the same direction.  Increasing
    this will increase the likelihood that an optimization path travels quickly across the domain along a
    consistent gradient.
      \default{2}
  \item \xmlNode{gainShrinkSize}, \xmlDesc{float, optional field}, specifies the rate at which the step size
    should shrink when it does shrink, for instance when switching directions on successive steps.  Increasing
    this will slow convergence, but decrease the likelihood of achieving false convergence due to small step
    sizes.
      \default{same value as grainGrowthSize}
  \end{itemize}
\item \xmlNode{parameter}, \xmlDesc{XML node, optional parameter} will accepts the following sub-nodes:
  \begin{itemize}
  \item \xmlNode{numGradAvgIterations}, \xmlDesc{integer, optional field} is the number of iterations for gradient estimation. When this
        parameter is $>1$, multiple gradient evaluations are going to be performed. Since the main goal for this parameter is to
        get a better gradient estimation (performing a denoising), the current point $x_k$ is evaluated multiple times in order to be able to
        converge in average.
        \default{1}

  \item Optimizer Gradient Evaluation parameters:
    \begin{itemize}
      \item \xmlNode{gamma}, \xmlDesc{float, optional field} Inverse exponent for gradient evaluation distance. Increasing this
        parameter will greatly decrease the distance between points sampled in evaluating the gradient
        \cite{spall1998implementation}. A practical suggestion for $\gamma$ is 0.101 (paired with an
        $\alpha$ value of 0.602); however, the asymptotic limit is $\gamma=1/6$ ($\alpha=1$). \default{0.101}
      \item \xmlNode{c}, \xmlDesc{float, optional field} Step size coefficient.  This term determines the
        nominal step size, and increasing it will directly increase the distance between points sampled in evaluating the gradient.
        It is suggested this parameter be approximately equal to the standard
        deviation of the measurement noise in the response for stochastic responses.  For regular responses,
        it can be a small arbitrary value. \default{0.005}
    \end{itemize}

  \item Optimizer Step Size parameters:
    \begin{itemize}
      \item \xmlNode{a}, \xmlDesc{float, optional field} Nominal optimizer step size parameter.  Increasing
        this parameter will directly increase the distance traversed in each optimizer step
        \cite{spall1998implementation}. In contrast to $A$, this parameter will be unchanged by increasing
        iterations, and so will be more impacting as the optimization algorithm iterates. \default{0.16}
      \item \xmlNode{alpha}, \xmlDesc{float, optional field} Inverse exponent for optimizer step size.
        Increasing this parameter will greatly decrease the distance traversed in each optimizer step
        \cite{spall1998implementation}. Values less than 1 for $\alpha$ usually yield better performance by
        keeping a large step size. See the description of $\gamma$ above for some suggested values. \default{0.602}
      \item \xmlNode{A}, \xmlDesc{float, optional field} Nominal step damping stability parameter.  Increasing this
        parameter will directly decrease the distance traversed in each optimizer step
        \cite{spall1998implementation}. This parameter will have greater affect in reducing step size early in
        the calculation, and reduced affect as iterations increase. \default{\xmlNode{limit} divided by 10}
    \end{itemize}
  \item \xmlNode{innerBisectionThreshold}, \xmlDesc{float, optional field} a parameter specifying the convergence threshold of the
  bisection method used in constraint handling (See above). This parameter shall be in the open inverval $(0,1)$.
        \default{0.01}
  \item \xmlNode{innerLoopLimit}, \xmlDesc{integer, optional field} a parameter specifying the number of orthogonal vectors to try
  when handling the constraints (See above).
        \default{1000}
  \end{itemize}
\end{itemize}


Example:
\begin{lstlisting}[style=XML]
<Optimizers>
  ...
  <FiniteDifferenceGradientOptimizer name="SPSAname">
    <initialization>
      <limit>300</limit>
      <type>min</type>
      <initialSeed>30</initialSeed>
    </initialization>
    <TargetEvaluation class="DataObjects" type="PointSet">dataObjectName</TargetEvaluation>
    <convergence>
      <iterationLimit>50</iterationLimit>
      <relativeThreshold>1e-3</relativeThreshold>
      <absoluteThreshold>1e-1</absoluteThreshold>
      <gradientThreshold>1e-5</gradientThreshold>
      <persistence>1</persistence>
    </convergence>
    <parameter>
      <numGradAvgIterations>3</numGradAvgIterations>
    </parameter>
    <variable name="var1">
      <upperBound>100</upperBound>
      <lowerBound>-100</lowerBound>
      <initial>0</initial>
    </variable>
    <objectVar>c</objectVar>
  </FiniteDifferenceGradientOptimizer>
  ...
</Optimizers>
\end{lstlisting}

\section{DataObjects}
\label{sec:DataObjects}

As seen in the previous chapters, different entities in the RAVEN
code interact with each other in order to create, ideally, an infinite number of
different calculation flows.
%
These interactions are made possible through a data handling system that each
entity understands.
%
This system is called the ``DataObjects'' framework.

The \xmlNode{DataObjects} tag is a container of data objects of various types that can
be constructed during the execution of a particular calculation flow.
%
These data objects can be used as input or output for a particular
\textbf{Model} (see Roles' meaning in section \ref{sec:models}), etc.
%
Currently, RAVEN supports 3 different data types, each with a particular
conceptual meaning.
%
These data types are instantiated as sub-nodes in the \xmlNode{DataObjects} block of
an input file:
\begin{itemize}
  \item \xmlNode{PointSet} is a collection of individual objects, each
  describing the state of the system at a certain point (e.g. in time).
  %
  It can be considered a mapping between multiple sets of parameters in the
  input space and the resulting sets of outcomes in the output space at a
  particular point (e.g. in time).
  %
  \item \xmlNode{HistorySet} is a collection of individual objects each
  describing the temporal evolution of the state of the system within a certain
  input domain.
  %
  It can be considered a mapping between multiple sets of parameters in the
  input space and the resulting sets of temporal evolution in the output
  space.
  %
   \item \xmlNode{DataSet} is a generalization of the previously described DataObject,
   aimed to contain a mixture of data (scalars, arrays, etc.). The variables here stored
   can be independent (i.e. scalars) or dependent (arrays) on certain dimensions (e.g. time, coordinates, etc.).
  %
  It can be considered a mapping between multiple sets of parameters in the
  input space (both dependent and/or independent) and the resulting sets of evolution in the output
  space (both dependent and/or independent).
  %
  \nb \textcolor{red} {\textbf{The  \xmlNode{DataSet} is currently usable in the  \xmlNode{EnsembleModel} only (see \ref{subsec:models_EnsembleModel} )}}
\end{itemize}

In summary, the DataObjects accept the following data in their input/output spaces:
\begin{table}[h]
\centering
\caption{DataObjects' accepted data formats.}
\label{DataObjectDataFormatTable}
\begin{tabular}{|c|c|c|}
\hline
\textbf{DataObject}                        & \textbf{Input Space} & \textbf{Output Space} \\ \hline
{\color[HTML]{FE0000} \textit{PointSet}}   & scalars              & scalars               \\ \hline
{\color[HTML]{FE0000} \textit{HistorySet}} & scalars              & vectors               \\ \hline
{\color[HTML]{FE0000} \textit{DataSet}}    & any                  & any                   \\ \hline
\end{tabular}
\end{table}


As noted above, each data object represents a mapping between a set of
parameters and the resulting outcomes.
%
The data objects are defined within the main XML block called \xmlNode{DataObjects}:
\begin{lstlisting}[style=XML]
<Simulation>
   ...
  <DataObjects>
    <PointSet name='***'>...</PointSet>
    <HistorySet name='***'>...</HistorySet>
    <DataSet name='***'>...</DataSet>
  </DataObjects>
   ...
</Simulation>
\end{lstlisting}

Independently on the type of data, the respective XML node has the following
available attributes:
\vspace{-5mm}
\begin{itemize}
  \itemsep0em
  \item \xmlAttr{name}, \xmlDesc{required string attribute}, is a user-defined
  identifier for this data object.
    %
  \nb As with other objects, this name can be used to refer to this specific
  entity from other input blocks in the XML.
  %
%  % Regarding the time attribute, we need to take a better decision... Now it is very confusing.
%  \item \xmlAttr{time}, \xmlDesc{optional float or string attribute}, time
%    attribute.
%    %
%    Here, the user can specify either the time (value) at which the outcomes
%    need to be taken (History-like object, it represents the time from which the
%    outcomes' evolution need to be tracked) or a string  that can be either
%    ``end'', at the end of the history, or ``all'', consider.
%    %
%    \default{random seed};
%  \item \xmlAttr{inputTs}, \xmlDesc{optional integer attribute}, used to
%  specify at which ``time step'' the input space needs to be retrieved.
%  %
%  \nb If the user wants to take conditions from the end of the simulation, s/he
%  can directly input ``-1.''
%  %
%  \default{0}
%  \item \xmlAttr{operator}, \xmlDesc{optional string attribute}, is aimed at
%  performing simple operations on the data to be stored.
%  %
%  %
%  The 3 options currently available are:
%  \begin{itemize}
%    \item \xmlString{max}
%    \item \xmlString{min}
%    \item \xmlString{average}
%  \end{itemize}
%  %
%  \default{None}

  \item \xmlAttr{hierarchical}, \xmlDesc{optional boolean attribute},
  This flag is going to ``control'' the printing/plotting of the DataObject in
  case a hierarchical structure is determined (e.g.
  data coming from Dynamic Event Tree-like approaches):
  \begin{itemize}
    \item if \textbf{True} all the branches of the tree are going to be printed/plotted independently
               (i.e. all the branches are going to be exposed independently)
    \item if \textbf{False} all the branches are going to be walked back and reconstructed in order to create independent histories
  \end{itemize}
  %
  \default{False}
\end{itemize}
\vspace{-5mm}
In each XML node (e.g. \xmlNode{PointSet}, \xmlNode{HistorySet} or  \xmlNode{DataSet}), the user
specifies the following sub-nodes:
\begin{itemize}
  \item \xmlNode{Input}, \xmlDesc{comma separated string, required field}, lists
  the input parameters to which this data is connected.
  %
  \item \xmlNode{Output}, \xmlDesc{comma separated string, required field}, lists
  the output parameters to which this data is connected.
  %
  \item \xmlNode{Index}, \xmlDesc{comma separated string, required for \xmlNode{DataSet}}, lists
  the dependent variables that depend on this index (specified through the attribute  \xmlAttr{var}).
  This XML node requires the following attribute:
   \begin{itemize}
     \item \xmlAttr{var}, \xmlDesc{required string attribute}, the dimension name of this index (e.g. time)
   \end{itemize}
 \item \xmlNode{options}, \xmlDesc{optional node}, contains additional option nodes
   for data objects.  This node contains the following subnodes:
   \begin{itemize}
     \item \xmlNode{pivotParameter}, \xmlDesc{optional, string}, specifies the \textit{pivotParameter} for a
       \xmlNode{HistorySet}. The pivot parameter is the shared index of the output variables in the data
       object.  \default time
    \item \xmlNode{inputRow}, \xmlDesc{integer, optional field}, used to
         specify  the row (in a CSV file or HDF5 table) from which the input space
        needs to be retrieved (e.g. the time-step);
    %
    \item \xmlNode{outputRow}, \xmlDesc{integer, optional field}, used to
         specify  the row (in the CSV file or HDF5 table) from which the output space
        needs to be retrieved (e.g. the time-step). If this node is inputted, the nodes
         \xmlNode{operator} and  \xmlNode{outputPivotValue} can not be inputted (mutually exclusive).
       \\\nb This XML node is available for DataObjects of type \xmlNode{PointSet} only;
    %
    \item \xmlNode{operator}, \xmlDesc{string, optional field}, is aimed to perform
         simple operations on the data to be stored.
         The 3 options currently available are:
         \begin{itemize}
            \item \xmlString{max}
            \item \xmlString{min}
            \item \xmlString{average}
         \end{itemize}
         If this node is inputted, the nodes
         \xmlNode{outputRow} and  \xmlNode{outputPivotValue} can not be inputted (mutually exclusive).
         \\\nb This XML node is available for DataObjects of type \xmlNode{PointSet} only;
   \end{itemize}

  %
\end{itemize}

The  \xmlNode{PointSet} and  \xmlNode{HistorySet} objects are a specialization of the  \xmlNode{DataSet}. In
the \xmlNode{PointSet}, the input and output space are all exclusively scalar values.  These values might be
extracted from a vector of values for each entry using the \xmlNode{options} node, but the end result is a
single scalar per input or output variable.

For the \xmlNode{HistorySet}, all inputs must be scalar, and all outputs must share an index (the
\textit{pivotParameter}.  There cannot be scalars in any of the outputs. The pivotParameter can be changed
through the corresponding node in the \xmlNode{options} node.


  %
  %%%%%%%%%%%%%%%%%%%%%%%%%%%%%%%%%%%%%%%%%%%%%%%%%%%%%%%%%%%%%%%%%%%%%%%%%%%%%%
  %%%% This feature is being disabled until the DataObjects handle data in a
  %%%% more encapsulated fashion. When the data can handle this all internally
  %%%% then we can re-add this feature. As of now, determining the rows
  %%%% associated to the outputPivotValue or inputPivotValue requires knowing
  %%%% information outside of the "value" passed into
  %%%% DataObject.updateOutputValue or DataObject.updateInputValue, thus the
  %%%% caller has to do this computation, but currently the caller occurs in ~50
  %%%% different places according to my grep of "updateOutputValue"
  %%%% -- DPM 8/29/2017
  % \item \xmlNode{pivotParameter}, \xmlDesc{string, optional field} the name of
  %   the parameter whose values need to be used as reference for the values
  %   specified in the XML nodes \xmlNode{inputPivotValue},
  %   \xmlNode{outputPivotValue}, or \xmlNode{inputPivotValue} (if inputted).
  %   This field can be used, for example, if the driven code output file uses  a
  %   different name for the variable ``time'' or to specify a different reference
  %   parameter (e.g. PRESSURE). Default value is \xmlString{time}.
  %   \\\nb The variable specified here should be monotonic; the code does not
  %   check for eventual oscillation and is going to take the first occurance for
  %   the values specified in the XML nodes \xmlNode{inputPivotValue},
  %   \xmlNode{outputPivotValue}, and  \xmlNode{inputPivotValue};
  % %
  % \item \xmlNode{inputPivotValue}, \xmlDesc{float, optional field}, the value of the \xmlNode{pivotParameter} at which the input space needs to be retrieved
  %   If this node is inputted, the node  \xmlNode{inputRow} can not be inputted (mutually exclusive).
  %   %
  % \item \xmlNode{outputPivotValue}. This node can be either a float or a list of floats, depending on the type of DataObjects:
  %  \begin{itemize}
  %     \item if \xmlNode{HistorySet},\xmlNode{outputPivotValue}, \xmlDesc{list of floats, optional field},  list of values of the
  %                         \xmlNode{pivotParameter} at which the output space needs to be retrieved;
  %     \item if \xmlNode{PointSet},\xmlNode{outputPivotValue}, \xmlDesc{float, optional field},  the value of the \xmlNode{pivotParameter}
  %        at which the output space needs to be retrieved. If this node is inputted, the node  \xmlNode{outputRow} can not be inputted (mutually exclusive);
  %  \end{itemize}
  %%%%%%%%%%%%%%%%%%%%%%%%%%%%%%%%%%%%%%%%%%%%%%%%%%%%%%%%%%%%%%%%%%%%%%%%%%%%%%
  %
Note that if the optional nodes in the block \xmlNode{options} are not inputted, the following default are applied:
\begin{itemize}
   \item the Input space (scalars) is retrieved from the first row in the CSVs files or HDF5 tables (if the parameters specified are not
      among the variables sampled by RAVEN); In case of the  \xmlNode{DataSet}, if any of the input space variables depend on an \xmlNode{Index}, they
      are going to be linked to the \xmlNode{Index} variable
   \item  the output space defaults are as follows:
   \begin{itemize}
       \item if \xmlNode{PointSet}, the output space is retrieved from the last row in the CSVs files or HDF5 tables;
       \item if \xmlNode{HistorySet}, the output space is represented by all the rows found in  the CSVs or HDF5 tables.
       \item if \xmlNode{DataSet}, the output space of the variables that do not depends on any index is retrieved from the last row in the CSVs files or HDF5 tables;
       on the contrary, the output space of the variables that depends on indexes is represented by all the rows found in  the CSVs or HDF5 tables (if they match
       with the indexes' dimension)
    \end{itemize}
\end{itemize}


\begin{lstlisting}[style=XML,morekeywords={operator,hierarchical,name,var}]
  <DataObjects>
    <PointSet name='outTPS1'>
      <options>
       <inputRow>1</inputRow>
       <outputRow>-1</outputRow>
      </options>
      <Input>pipe_Area,pipe_Dh,Dummy1</Input>
      <Output>pipe_Hw,pipe_Tw,time</Output>
    </PointSet>
    <HistorySet name='stories1'>
        <options>
            <pivotParameter>TIME</pivotParameter>
            <inputRow>1</inputRow>
            <outputRow>-1</outputRow>
        </options>
      <Input>pipe_Area,pipe_Dh</Input>
      <Output>pipe_Hw,pipe_Tw,time</Output>
    </HistorySet>
    <DataSet name='aDataSet'>
      <Input>pipe_Area,pipe_Dh</Input>
      <Output>pipe_Hw,pipe_Tw</Output>
      <Index var="time">pipe_Hw,pipe_Tw</Index>
    </DataSet>
  </DataObjects>
\end{lstlisting}

\section{Databases}
\label{sec:Databases}
The RAVEN framework provides the capability to store and retrieve data to/from
an external database.
%
Currently RAVEN has support for only a database type called \textbf{HDF5}.
%
This database, depending on the data format it is receiving, will organize
itself in a ``parallel'' or ``hierarchical'' fashion.
%
The user can create as many database objects as needed.
%
The Database objects are defined within the main XML block called
\xmlNode{Databases}:
\begin{lstlisting}[style=XML]
<Simulation>
  ...
  <Databases>
    ...
    <HDF5 name="aDatabaseName1" readMode="overwrite"/>
    <HDF5 name="aDatabaseName2" readMode="overwrite"/>
    ...
  </Databases>
  ...
</Simulation>
\end{lstlisting}
The specifications of each Database of type HDF5 needs to be defined within the
XML block \xmlNode{HDF5}, that recognizes the following attributes:
\vspace{-5mm}
\begin{itemize}
  \itemsep0em
  \item \xmlAttr{name}, \xmlDesc{required string attribute}, a user-defined
  identifier of this object.
  %
  \nb As with other objects, this is name can be used to reference this specific
  entity from other input blocks in the XML.
  \item \xmlAttr{readMode}, \xmlDesc{required string attribute}, defines whether an existing database should
    be read when loaded (\xmlString{read}) or overwritten (\xmlString{overwrite}).
    \nb if in \xmlString{read} mode and the database is not found, RAVEN will read in
    the data as empty and raise a warning, NOT an error.
  %
  \item \xmlAttr{directory}, \xmlDesc{optional string attribute}, this attribute
  can be used to specify a particular directory path where the database will be
  created or read from.  If an absolute path is given, RAVEN will respect it; otherwise,
  the path will be assumed to be relative to the \xmlNode{WorkingDir} from the \xmlNode{RunInfo} block.
  RAVEN recognizes path expansion tools such as tildes (\emph{user dir}), single dots (\emph{current dir}),
  and double dots (\emph{parent dir}).
  %
  \default{workingDir/DatabaseStorage}.  The \xmlNode{workingDir} is
   the one defined within the \xmlNode{RunInfo} XML block (see Section~\ref{sec:RunInfo}).
  \item \xmlAttr{filename}, \xmlDesc{optional string attribute}, specifies the
  filename of the HDF5 that will be created in the \xmlAttr{directory}.
  %
  \nb When this attribute is not specified, the newer database filename will be
  named \texttt{name}.h5, where \textit{name} corresponds to the \xmlAttr{name}
  attribute of this object.
  %
  \default{None}
  \item \xmlAttr{compression}, \xmlDesc{optional string attribute}, compression
  algorithm to be used.
  %
  Available are:
  \begin{itemize}
    \item \xmlString{gzip}, best where portability is required.
    %
    Good compression, moderate speed.
    %
    \item \xmlString{lzf}, Low to moderate compression, very fast.
    %
  \end{itemize}
  \default{None}
\end{itemize}

In addition, the \xmlNode{HDF5} recognizes the following subnodes:
\begin{itemize}
  \itemsep0em
  \item \xmlNode{variables}, \xmlDesc{optional, comma-separated string}, allows only a pre-specified set of variables to be
    included in the HDF5 when it is written to.  If this node is not included, by default the HDF5 will
    include ALL of the input/output variables as a result of the step it is part of.  If included, only the
    comma-separated variable names will be included if found.

    \nb RAVEN will not error if one of the requested variables is not found; instead, it will silently pass.
    It is recommended that a small trial run is performed, loading the HDF5 back into a data object, to check
    that the correct variables are saved to the HDF5 before performing large-scale calculations.
\end{itemize}


Example:
\begin{lstlisting}[style=XML,morekeywords={directory,filename}]
<Databases>
  <HDF5 name="aDatabaseName1" directory=''path_to_a_dir'' compression=''lzf'' readMode='overwrite'/>
  <HDF5 name="aDatabaseName2" filename=''aDatabaseName2.h5'' readMode='read'/>
</Databases>
\end{lstlisting}

\input{OutStreamSystem.tex}
\section{Models}
\label{sec:models}

%%%%%%%%%%%%%%%%%%%%%%%%%%%%%%%%%%%%%%%%%%%%%%%%%%%%%%%%%%%%%%%%%%%%%%%%%%%%%%%%
% If you are confused by the input of this document, please make sure you see
% these defined commands first. There is no point writing the same thing over
% and over and over and over and over again, so these will help us reduce typos,
% by just editing a template sentence or paragraph.

% These should be organized according to whic section they are most often used
% e.g. kernelDescription should go under a heading for SVMs.
% This will take a bit of work, but if later things are added it will make
% finding the appropriate parameters easier.

\renewcommand{\nameDescription}
{
  \xmlAttr{name},
  \xmlDesc{required string attribute}, user-defined identifier of this model.
  \nb As with other objects, this identifier can be used to reference this
  specific entity from other input blocks in the XML.
}


\newcommand{\assemblerAttrDescription}[1]
{
    This XML node needs to contain the attributes:
    \begin{itemize}
      \item \xmlAttr{class}, \xmlDesc{required string attribute}, the main
        ``class'' of the #1.
        %
      \item \xmlAttr{type},  \xmlDesc{required string attribute}, the sub-type of the #1.
        %
    \end{itemize}
}


\newcommand{\aliasSystemDescription}[1]
{
  \xmlNode{alias} \xmlDesc{string, optional field} specifies alias for
  any variable of interest in the input or output space for the #1.
  %
  These aliases can be used anywhere in the RAVEN input to refer to the #1
  variables.
  %
  In the body of this node the user specifies the name of the variable that the model is going to use
  (during its execution).
  %
  The actual alias, usable throughout the RAVEN input, is instead defined in the
  \xmlAttr{variable} attribute of this tag.
  \\The user can specify aliases for both the input and the output space. As sanity check, RAVEN
  requires an additional required attribute \xmlAttr{type}. This attribute can be either ``input'' or ``output''.
  %
  \nb The user can specify as many aliases as needed.
  %
  \default{None}
}

\renewcommand{\specBlock}[2]{
  The specifications of this model must be defined within #1 \xmlNode{#2} XML
  block.
}
%
\renewcommand{\subnodeIntro}
{
  This model can be initialized with the following child:
}
\renewcommand{\subnodesIntro}
{
  This model can be initialized with the following children:
}

\newcommand{\ppType}[2]
{
  In order to use the \textit{#1} PP, the user needs to set the
  \xmlAttr{subType} of a \xmlNode{PostProcessor} node:

  \xmlNode{PostProcessor \xmlAttr{subType}=\xmlString{#2}/}.

   Several sub-nodes are available:
}

\newcommand{\skltype}[2]
{
  In order to use the \textit{#1}, the user needs to set the
  sub-node:

  \xmlNode{SKLtype}\texttt{#2}\xmlNode{/SKLtype}.

  In addition to this XML node, several others are available:
}

\newcommand{\nIterDescriptionA}[1]
{
  \xmlNode{n\_iter}, \xmlDesc{integer, optional field}, is the maximum number of
  iterations.
  %
  \ifthenelse{\equal{#1}{}}{}{\default{#1}}
}

\newcommand{\nIterDescriptionB}[1]
{
  \xmlNode{n\_iter}, \xmlDesc{int, optional field}, specifies the number of
  passes over the training data (aka epochs).
  %
  \ifthenelse{\equal{#1}{}}{}{\default{#1}}
}

\newcommand{\tolDescriptionA}[1]
{
  \xmlNode{tol}, \xmlDesc{float, optional field}, stop the algorithm if the convergence error felt below the tolerance specified here.
  %
  \ifthenelse{\equal{#1}{}}{}{\default{#1}}
}

\newcommand{\tolDescriptionB}[1]
{
  \xmlNode{tol}, \xmlDesc{float, optional field}, specifies the tolerance for
  the optimization: if the updates are smaller than tol, the optimization code
  checks the dual gap for optimality and continues until it is smaller than tol.
  %
  \ifthenelse{\equal{#1}{}}{}{\default{#1}}
}

\newcommand{\tolDescriptionC}[1]
{
  \xmlNode{tol}, \xmlDesc{float, optional field}, specifies the tolerance for
  stopping criteria.
  %
  \ifthenelse{\equal{#1}{}}{}{\default{#1}}
}

\newcommand{\fitInterceptDescription}[1]
{
  \xmlNode{fit\_intercept}, \xmlDesc{boolean, optional field}, determines
  whether to calculate the intercept for this model.
  %
  If set to False, no intercept will be used in the calculations (e.g. data is
  expected to be already centered).
  %
  \ifthenelse{\equal{#1}{}}{}{\default{#1}}
}

\newcommand{\normalizeDescription}[1]
{
  \xmlNode{normalize}, \xmlDesc{boolean, optional field}, if True, the
  regressors X will be normalized before regression.
  %
  \ifthenelse{\equal{#1}{}}{}{\default{#1}}
}

\newcommand{\verDescriptionA}[1]
{
  \xmlNode{verbose}, \xmlDesc{boolean, optional field}, use verbose mode
  when fitting the model.
  %
  \ifthenelse{\equal{#1}{}}{}{\default{#1}}
}

\newcommand{\verDescriptionB}[1]
{
  \xmlNode{verbose}, \xmlDesc{boolean or integer, optional field}, use verbose
  mode when fitting the model.
  %
  \ifthenelse{\equal{#1}{}}{}{\default{#1}}
}

\newcommand{\maxIterDescription}[1]
{
\xmlNode{max\_iter}, \xmlDesc{integer, optional field}, specifies the
  maximum number of iterations.
  %
  \ifthenelse{\equal{#1}{}}{}{\default{#1}}
}

\newcommand{\warmStartDescription}[1]
{
  \xmlNode{warm\_start}, \xmlDesc{boolean, optional field}, when set to
  True, the model will reuse the solution of the previous call to fit as
  initialization, otherwise, it will just erase the previous solution.
  %
  \ifthenelse{\equal{#1}{}}{}{\default{#1}}
}

\newcommand{\positiveDescription}[1]
{
  \xmlNode{positive}, \xmlDesc{boolean, optional field}, when set to True, this
  forces the coefficients to be positive.
  %
  \ifthenelse{\equal{#1}{}}{}{\default{#1}}
}

\newcommand{\precomputeDescription}[1]
{
  \xmlNode{precompute}, \xmlDesc{boolean or string, optional field}, determines
  whether to use a precomputed Gram matrix to speed up calculations.
  %
  If set to `auto,' RAVEN will decide.
  %
  The Gram matrix can also be passed as an argument.
  %
  Available options are [True | False | `auto' | array-like].
  %
  \ifthenelse{\equal{#1}{}}{}{\default{#1}}
}

\newcommand{\nAlphasDescription}[1]
{
  \xmlNode{max\_n\_alphas}, \xmlDesc{integer, optional field}, specifies the
  maximum number of points on the path used to compute the residuals in the
  cross-validation.
  %
  \ifthenelse{\equal{#1}{}}{}{\default{#1}}
}

\newcommand{\shuffleDescription}[1]
{
  \xmlNode{shuffle}, \xmlDesc{boolean, optional field}, specifies whether
  or not the training data should be shuffled after each epoch.
  %
  \ifthenelse{\equal{#1}{}}{}{\default{#1}}
}

\newcommand{\randomStateDescription}[1]
{
  \xmlNode{random\_state}, \xmlDesc{int seed, RandomState instance, or None},
  sets the seed of the pseudo random number generator to use when shuffling the
  data.
  %
  \ifthenelse{\equal{#1}{}}{}{\default{#1}}
}

\newcommand{\solverDescription}%[1]
{
  \xmlNode{solver}, \xmlDesc{\{`auto', `svd', `cholesky', `lsqr',
  `sparse\_cg'\}}, specifies the solver to use in the computational routines:
  \begin{itemize}
    \item `auto' chooses the solver automatically based on the type of data.
    %
    \item `svd' uses a singular value decomposition of X to compute the ridge
    coefficients.
    %
    More stable for singular matrices than `cholesky.'
    %
    \item `cholesky' uses the standard scipy.linalg.solve function to obtain a
    closed-form solution.
    %
    \item `sparse\_cg' uses the conjugate gradient solver as found in
    scipy.sparse.linalg.cg.
    %
    As an iterative algorithm, this solver is more appropriate than
    `cholesky' for large-scale data (possibility to set tol and max\_iter).
    %
    \item `lsqr' uses the dedicated regularized least-squares routine
    scipy.sparse.linalg.lsqr.
    %
    It is the fatest but may not be available in old scipy versions.
    %
    It also uses an iterative procedure.
    %
  \end{itemize}
  All three solvers support both dense and sparse data.
  %
  %\ifthenelse{\equal{#1}{}}{}{\default{#1}}
}

%%%%%%%%%%%%%%%%%%%%%%%%% Common Regression Parameters %%%%%%%%%%%%%%%%%%%%%%%%%
%%%%%%%%%%%%%%%%%%%%%%% End Common Regression Parameters %%%%%%%%%%%%%%%%%%%%%%%

%%%%%%%%%%%%%%%%%%%%%%%%%%%% Common SVM Parameters %%%%%%%%%%%%%%%%%%%%%%%%%%%%%
\newcommand{\CSVMDescription}[1]
{
  \xmlNode{C}, \xmlDesc{float, optional field}, sets the penalty parameter C
  of the error term.
  %
  \ifthenelse{\equal{#1}{}}{}{\default{#1}}
  %
}

\newcommand{\kernelDescription}[1]
{
  \xmlNode{kernel}, \xmlDesc{string, optional}, specifies the kernel type
  to be used in the algorithm.
  %
  It must be one of:
  \begin{itemize}
    \item `linear'
    \item `poly'
    \item `rbf'
    \item `sigmoid'
    \item `precomputed'
    \item a callable object
  \end{itemize}
  %
  If a callable is given it is used to pre-compute the kernel matrix.
  %
  \ifthenelse{\equal{#1}{}}{}{\default{#1}}
  %
}

\newcommand{\degreeDescription}[1]
{
  \xmlNode{degree}, \xmlDesc{int, optional field}, determines the degree
  of the polynomial kernel function (`poly').
  %
  Ignored by all other kernels.
  %
  \ifthenelse{\equal{#1}{}}{}{\default{#1}}
}

\newcommand{\gammaDescription}[1]
{
  \xmlNode{gamma}, \xmlDesc{float, optional field}, sets the
  kernel coefficient for the kernels `rbf,' `poly,' and `sigmoid.'
  %
  If gamma is `auto' then 1/n\_features will be used instead.
  %
  \ifthenelse{\equal{#1}{}}{}{\default{#1}}
}

\newcommand{\coefZeroDescription}[1]
{
  \xmlNode{coef0}, \xmlDesc{float, optional field}, is an independent term in
  kernel function.
  %
  It is only significant in `poly' and `sigmoid.'
  %
  \ifthenelse{\equal{#1}{}}{}{\default{#1}}
}

\newcommand{\probabilityDescription}[1]
{
  \xmlNode{probability}, \xmlDesc{boolean, optional field}, determines whether
  or not to enable probability estimates.
  %
  This must be enabled prior to calling fit, and will slow down that method.
  %
  \ifthenelse{\equal{#1}{}}{}{\default{#1}}
}

\newcommand{\shrinkingDescription}[1]
{
  \xmlNode{shrinking}, \xmlDesc{boolean, optional field}, determines whether or
  not to use the shrinking heuristic.
  %
  \ifthenelse{\equal{#1}{}}{}{\default{#1}}
}

\newcommand{\cacheSizeDescription}[1]
{
  \xmlNode{cache\_size}, \xmlDesc{float, optional field}, specifies the
  size of the kernel cache (in MB).
  %
  \ifthenelse{\equal{#1}{}}{}{\default{#1}}
}

\newcommand{\classWeightDescription}[1]
{
  \xmlNode{class\_weight}, \xmlDesc{{dict, `auto'}, optional}, sets the
  parameter C of class i to class\_weight[i]*C for SVC.
  %
  If not given, all classes are assumed to have weight one.
  %
  The `auto' mode uses the values of y to automatically adjust weights inversely
  proportional to class frequencies.
  %
  \ifthenelse{\equal{#1}{}}{}{\default{#1}}
}

\newcommand{\tolSVMDescription}[1]
{
  \tolDescriptionC{#1}
}

\newcommand{\verSVMDescription}[1]
{
  \verDescriptionA{False}
  %
  \nb This setting takes advantage of a per-process runtime setting in libsvm
  that, if enabled, may not work properly in a multithreaded context.
}

\newcommand{\randomStateSVMDescription}[1]
{
  \xmlNode{random\_state}, \xmlDesc{int seed, RandomState instance, or
  None}, represents the seed of the pseudo random number generator to use when
  shuffling the data for probability estimation.
  %
  \ifthenelse{\equal{#1}{}}{}{\default{#1}}
}

%%%%%%%%%%%%%%%%%%%%%%%%%% End Common SVM Parameters %%%%%%%%%%%%%%%%%%%%%%%%%%%

%%%%%%%%%%%%%%%%%%%%%%%%% Common Multi-Class Parameters %%%%%%%%%%%%%%%%%%%%%%%%
\newcommand{\estimatorDescription}[1]
{
  \xmlNode{estimator}, \xmlDesc{boolean, required field},
  %
  An estimator object implementing fit and one of decision\_function or
  predict\_proba.
  %
  This XML node needs to contain the following attribute:
  \vspace{-5mm}
  \begin{itemize}
    \itemsep0em
    \item \xmlAttr{estimatorType}, \xmlDesc{required string attribute}, this
    attribute is another reduced order mode type that needs to be used for the
    construction of the multi-class algorithms.
    %
    Each sub-sequential node depends on the chosen ROM.
  \end{itemize}
  %
  \ifthenelse{\equal{#1}{}}{}{\default{#1}}
}
%%%%%%%%%%%%%%%%%%%%%%% End Common Multi-Class Parameters %%%%%%%%%%%%%%%%%%%%%%

%%%%%%%%%%%%%%%%%%%%%%%%%% Common Bayesian Parameters %%%%%%%%%%%%%%%%%%%%%%%%%%
\newcommand{\alphaBayesDescription}[1]
{
  \xmlNode{alpha}, \xmlDesc{float, optional field}, specifies an additive
  (Laplace/Lidstone) smoothing parameter (0 for no smoothing).
  %
  \ifthenelse{\equal{#1}{}}{}{\default{#1}}
}

\newcommand{\fitPriorDescription}[1]
{
  \xmlNode{fit\_prior}, \xmlDesc{boolean, required field}, determines whether to
  learn class prior probabilities or not.
  %
  If false, a uniform prior will be used.
  %
  \ifthenelse{\equal{#1}{}}{}{\default{#1}}
}

\newcommand{\classPriorDescription}[1]
{
  \xmlNode{class\_prior}, \xmlDesc{array-like float (n\_classes), optional
  field}, specifies prior probabilities of the classes.
  %
  If specified, the priors are not adjusted according to the data.
  %
  \ifthenelse{\equal{#1}{}}{}{\default{#1}}
}

%%%%%%%%%%%%%%%%%%%%%%%% End Common Bayesian Parameters %%%%%%%%%%%%%%%%%%%%%%%%

%%%%%%%%%%%%%%%%%%%%%%%%%% Common Neighbor Parameters %%%%%%%%%%%%%%%%%%%%%%%%%%
\newcommand{\nNeighborsDescription}[1]
{
  \xmlNode{n\_neighbors}, \xmlDesc{integer, optional field}, specifies the
  number of neighbors to use by default for `k\_neighbors' queries.
  %
  \ifthenelse{\equal{#1}{}}{}{\default{#1}}
}

\newcommand{\radiusDescription}[1]
{
  \xmlNode{radius}, \xmlDesc{float, optional field}, specifies the range of
  parameter space to use by default for `radius\_neighbors' queries.
  %
  \ifthenelse{\equal{#1}{}}{}{\default{#1}}
}

\newcommand{\weightsDescription}[1]
{
  \xmlNode{weights}, \xmlDesc{string, optional field}, specifies the weight
  function used in prediction.
  %
  Possible values:
  \begin{itemize}
    \item \textit{uniform} : uniform weights.
    %
    All points in each neighborhood are weighted equally;
    \item \textit{distance} : weight points by the inverse of their distance.
    %
    In this case, closer neighbors of a query point will have a greater
    influence than neighbors which are further away.
    %
  \end{itemize}
  %
  \ifthenelse{\equal{#1}{}}{}{\default{#1}}
}

\newcommand{\metricDescription}[1]
{
  \xmlNode{metric}, \xmlDesc{string, optional field}, sets the distance metric
  to use for the tree.
  %
  The Minkowski metric with p=2 is equivalent to the standard Euclidean metric.
  %
  \ifthenelse{\equal{#1}{}}{}{\default{#1}}
}

\newcommand{\algorithmDescription}[1]
{
  \xmlNode{algorithm}, \xmlDesc{string, optional field}, specifies the algorithm
  used to compute the nearest neighbors:
  \begin{itemize}
    \item \textit{ball\_tree} will use BallTree.
    \item \textit{kd\_tree} will use KDtree.
    \item \textit{brute} will use a brute-force search.
    \item \textit{auto} will attempt to decide the most appropriate algorithm
    based on the values passed to fit method.
    %
  \end{itemize}
  \nb Fitting on sparse input will override the setting of this parameter, using
  brute force.
  %
  \ifthenelse{\equal{#1}{}}{}{\default{#1}}
}

\newcommand{\leafSizeDescription}[1]
{
  \xmlNode{leaf\_size}, \xmlDesc{integer, optional field}, sets the leaf size
  passed to the BallTree or KDTree.
  %
  This can affect the speed of the construction and query, as well as the memory
  required to store the tree.
  %
  The optimal value depends on the nature of the problem.
  %
  \ifthenelse{\equal{#1}{}}{}{\default{#1}}
}

\newcommand{\pDescription}[1]
{
  \xmlNode{p}, \xmlDesc{integer, optional field}, is a parameter for the
  Minkowski metric.
  %
  When $p = 1$, this is equivalent to using manhattan distance (L1), and
  euclidean distance (L2) for $p = 2$.
  %
  For arbitrary p, minkowski distance (L\_p) is used.
  %
  \ifthenelse{\equal{#1}{}}{}{\default{#1}}
}

\newcommand{\outlierLabelDescription}[1]
{
  \xmlNode{outlier\_label}, \xmlDesc{integer, optional field}, is a label, which
  is given for outlier samples (samples with no neighbors on a given radius).
  %
  If set to None, ValueError is raised, when an outlier is detected.
  %
  \ifthenelse{\equal{#1}{}}{}{\default{#1}}
}

%%%%%%%%%%%%%%%%%%%%%%%% End Common Neighbor Parameters %%%%%%%%%%%%%%%%%%%%%%%%

%%%%%%%%%%%%%%%%%%%%%%%%%%%% Common Tree Parameters %%%%%%%%%%%%%%%%%%%%%%%%%%%%
\newcommand{\criterionDescription}[1]
{
  \xmlNode{criterion}, \xmlDesc{string, optional field}, specifies the function
  used to measure the quality of a split.
  %
  Supported criteria are ``gini'' for the Gini impurity and ``entropy'' for the
  information gain.
  %
  \ifthenelse{\equal{#1}{}}{}{\default{#1}}
}

\newcommand{\criterionDescriptionDT}[1]
{
  \xmlNode{criterion}, \xmlDesc{string, optional field}, specifies the function
  used to measure the quality of a split.
  %
  The only supported criterion is ``mse'' for mean squared error.
  %
  \ifthenelse{\equal{#1}{}}{}{\default{#1}}
}

\newcommand{\splitterDescription}[1]
{
  \xmlNode{splitter}, \xmlDesc{string, optional field}, specifies the strategy
  used to choose the split at each node.
  %
  Supported strategies are ``best'' to choose the best split and ``random'' to
  choose the best random split.
  %
  \ifthenelse{\equal{#1}{}}{}{\default{#1}}
}

\newcommand{\maxFeaturesDescription}[1]
{
  \xmlNode{max\_features}, \xmlDesc{int, float or string, optional field}, sets
  the number of features to consider when looking for the best split:
  \begin{itemize}
    \item If int, then consider max\_features features at each split.
    %
    \item If float, then max\_features is a percentage and int(max\_features *
    n\_features) features are considered at each split.
    %
    \item If ``auto,'' then max\_features=sqrt(n\_features).
    \item If ``sqrt,'' then max\_features=sqrt(n\_features).
    \item If ``log2,'' then max\_features=log2(n\_features).
    \item If None, then max\_features=n\_features.
    %
  \end{itemize}
  \nb The search for a split does not stop until at least one valid partition of
  the node samples is found, even if it requires to effectively inspect more
  than max\_features features.
  %
  \ifthenelse{\equal{#1}{}}{}{\default{#1}}
}

\newcommand{\maxDepthDescription}[1]
{
  \xmlNode{max\_depth}, \xmlDesc{integer, optional field}, determines the
  maximum depth of the tree.
  %
  If None, then nodes are expanded until all leaves are pure or until all leaves
  contain less than min\_samples\_split samples.
  %
  Ignored if max\_samples\_leaf is not None.
  %
  \ifthenelse{\equal{#1}{}}{}{\default{#1}}
}

\newcommand{\minSamplesSplitDescription}[1]
{
  \xmlNode{min\_samples\_split}, \xmlDesc{integer, optional field}, sets the
  minimum number of samples required to split an internal node.
  %
  \ifthenelse{\equal{#1}{}}{}{\default{#1}}
}

\newcommand{\minSamplesLeafDescription}[1]
{
  \xmlNode{min\_samples\_leaf}, \xmlDesc{integer, optional field}, sets the
  minimum number of samples required to be at a leaf node.
  %
  \ifthenelse{\equal{#1}{}}{}{\default{#1}}
}

\newcommand{\maxLeafNodesDescription}[1]
{
  \xmlNode{max\_leaf\_nodes}, \xmlDesc{integer, optional field}, grow a tree
  with max\_leaf\_nodes in best-first fashion.
  %
  Best nodes are defined by relative reduction in impurity.
  %
  If None then unlimited number of leaf nodes.
  %
  If not None then max\_depth will be ignored.
  %
  \ifthenelse{\equal{#1}{}}{}{\default{#1}}
}
%%%%%%%%%%%%%%%%%%%%%%%%%% End Common Tree Parameters %%%%%%%%%%%%%%%%%%%%%%%%%%

%%%%%%%%%%%%%%%%%%%%%% Common Gaussian Process Parameters %%%%%%%%%%%%%%%%%%%%%%
\newcommand{\blankbDescription}[1]
{
  %
  \ifthenelse{\equal{#1}{}}{}{\default{#1}}
}
%%%%%%%%%%%%%%%%%%%% End Common Gaussian Process Parameters %%%%%%%%%%%%%%%%%%%%

%%%%%%%%%%%%%%%%%%%%%% Common Multi-layer Perceptron Parameters %%%%%%%%%%%%%%%%%%%%%%
\newcommand{\hiddenLayerSizesMLPDescription}[1]
{
  \xmlNode{hidden\_layer\_sizes}, \xmlDesc{tuple, length = n\_layers - 2, optional field}, the ith
  element represents the number of neurons in the ith hidden layer.
  %
  \ifthenelse{\equal{#1}{}}{}{\default{#1}}
}

\newcommand{\activationMLPDescription}[1]
{
  \xmlNode{activation}, \xmlDesc{string, optional field}, activation function for the hidden layer.
  \begin{itemize}
    \item \textit{identity}, no-op activation, useful to implement linear bottleneck, returns
      f(x) = x
    \item \textit{logistic}, the logistic sigmoid function, returns f(x) = 1/(1+exp(-x))
    \item \textit{tanh}, the hyperbolic tan function, returns f(x) = tanh(x)
    \item \textit{relu}, the rectified linear function, returns f(x) = max(0, x)
  \end{itemize}

  %
  \ifthenelse{\equal{#1}{}}{}{\default{#1}}
}

\newcommand{\solverMLPDescription}[1]
{
  \xmlNode{solver}, \xmlDesc{string, optional field}, The solver for weight optimization.
  \begin{itemize}
    \item \textit{lbfgs}, is an optimizer in the family of quasi-Newton methods
    \item \textit{sgd}, refers to stochastic gradient descent
    \item \textit{adam}, refers to a stochastic gradient-based optimizer proposed by
      Kingma, Diederik, and Jimmy Ba.
  \end{itemize}
  \nb The default solver \textit{adam} works pretty well on relatively large datasets
  (with thousands of training samples or more) in terms of both training time and validation
  score. For small datasets, however, \textit{lbfgs} can converge faster and perform better.
  %
  \ifthenelse{\equal{#1}{}}{}{\default{#1}}
}


\newcommand{\alphaMLPDescription}[1]
{
  \xmlNode{alpha}, \xmlDesc{float, optional field}, L2 penalty parameter
  %
  \ifthenelse{\equal{#1}{}}{}{\default{#1}}
}

\newcommand{\batchSizeMLPDescription}[1]
{
  \xmlNode{batch\_size}, \xmlDesc{int or `auto', optional field}, Size of minibatches for stochastic
  optimizers. If the solver is \textit{lbfgs}, the classifier will not use minibatch. When
  set to \textit{auto}, batch\_size = min(200, n\_samples)
  %
  \ifthenelse{\equal{#1}{}}{}{\default{#1}}
}


\newcommand{\learningRateMLPDescription}[1]
{
  \xmlNode{learning\_rate}, \xmlDesc{string, optional field}, Learning rate schedule for weight updates
  \begin{itemize}
    \item \textit{constant}, is a constant learning rate given by \textit{learning\_rate\_init}
    \item \textit{invscaling}, gradually decreases the learning rate at each time step `t' using an
      inverse scaling exponent of `power\_t'. effective\_learning\_rate = learning\_rate\_init / pow(t, power\_t)
    \item \textit{adaptive}, keeps the learning rate constant to `learning\_rate\_init' as long as
      training loss keeps decreasing. Each time two consecutive epochs fail to decrease training loss by at least
       tol, or fail to increase validation score by at least tol if `early\_stopping' is on, the current
       learning rate is divided by 5.
       \nb Only used when \textit{solver = `sgd'}
  \end{itemize}
  %
  \ifthenelse{\equal{#1}{}}{}{\default{#1}}
}

\newcommand{\learningRateInitMLPDescription}[1]
{
  \xmlNode{learning\_rate\_init}, \xmlDesc{float, optional field}, The initial learning rate used. It
  controls the step-size in updating the weights. Only used when solver = `sgd' or `adam'
  %
  \ifthenelse{\equal{#1}{}}{}{\default{#1}}
}


\newcommand{\powerTMLPDescription}[1]
{
  \xmlNode{power\_t}, \xmlDesc{float, optional field}, the exponent for inverse scaling learning rate.
  It is used in updating effective learning rate when the learning\_rate is set to `invscaling'. only
  used when solver = `sgd'
  %
  \ifthenelse{\equal{#1}{}}{}{\default{#1}}
}

\newcommand{\maxIterMLPDescription}[1]
{
  \xmlNode{max\_iter}, \xmlDesc{int, optional field}, maximum number of iterations. The solver iterates until
  convergence (determined by `tol') or this number of iterations. For stochastic solvers (`sgd', `adam'), note
  that this determines the number of epochs (how many times each data point will be used), not the number
  of gradient steps
  %
  \ifthenelse{\equal{#1}{}}{}{\default{#1}}
}



\newcommand{\shuffleMLPDescription}[1]
{
  \xmlNode{shuffle}, \xmlDesc{bool, optional field}, whether to shuffle samples in each iteration. Only used
  when solver = `sgd' or `adam'
  %
  \ifthenelse{\equal{#1}{}}{}{\default{#1}}
}

\newcommand{\randomStateMLPDescription}[1]
{
  \xmlNode{random\_state}, \xmlDesc{int, RandomState instance or None, optional field}
  if int, random\_state is the seed used by the random number generator; If RandomState
  instance, random\_state is the random number generator; If None, the random number
  generator is the RandomState instance used by np.random.
  %
  \ifthenelse{\equal{#1}{}}{}{\default{#1}}
}
\newcommand{\tolMLPDescription}[1]
{
  \xmlNode{tol}, \xmlDesc{float, optional field}, tolerance for optimization. When the loss or
  score is not improving by at least tol for two consecutive iterations, unless learning\_rate
  is set to `adaptive', convergence is considered to be reached and training stops.
  %
  \ifthenelse{\equal{#1}{}}{}{\default{#1}}
}
\newcommand{\verboseMLPDescription}[1]
{
  \xmlNode{verbose}, \xmlDesc{bool, optional field}, whether to print progress messages or stdout
  %
  \ifthenelse{\equal{#1}{}}{}{\default{#1}}
}
\newcommand{\warmStartMLPDescription}[1]
{
  \xmlNode{warm\_start}, \xmlDesc{bool, optional field}, when set to True, reuse the solution of 
  previous call to fit as initialization, otherise, just erase the previous solution.
  %
  \ifthenelse{\equal{#1}{}}{}{\default{#1}}
}
\newcommand{\momentumMLPDescription}[1]
{
  \xmlNode{momentum}, \xmlDesc{float, optional field}, momentum for gradient descent update.
  Should be between 0 and 1. Only used when solver = `sgd'.
  %
  \ifthenelse{\equal{#1}{}}{}{\default{#1}}
}
\newcommand{\nesterovsMomentumMLPDescription}[1]
{
  \xmlNode{nesterovs\_momentum}, \xmlDesc{bool, optional field}, whether to use Nesterov's momentum.
  Only used when solver='sgd' and momentum > 0.
  %
  \ifthenelse{\equal{#1}{}}{}{\default{#1}}
}
\newcommand{\earlyStoppingMLPDescription}[1]
{
  \xmlNode{early\_stopping}, \xmlDesc{bool, optional field}, whether to use early stopping to terminate
  training when validation score is not improving. If set to true, it will automatically set aside 10\%
  of training data as validation and terminate training when validation score is not improving by at
  least tol for two consecutive epochs. Only effective when solver=`sgd' or `adam'.
  %
  \ifthenelse{\equal{#1}{}}{}{\default{#1}}
}
\newcommand{\validationFractionMLPDescription}[1]
{
  \xmlNode{validation\_fraction}, \xmlDesc{float, optional field}, the proportion of training data to set
  aside as validation set for early stopping. Must be between 0 and 1. Only used if early\_stopping is True.
  %
  \ifthenelse{\equal{#1}{}}{}{\default{#1}}
}
\newcommand{\epsilonMLPDescription}[1]
{
  \xmlNode{epsilon}, \xmlDesc{float, optional field}, value for numerical stability in adam. Only used
  when solver = `adam'
  %
  \ifthenelse{\equal{#1}{}}{}{\default{#1}}
}

\newcommand{\betaAMLPDescription}[1]
{
  \xmlNode{beta\_1}, \xmlDesc{float, optional field}, exponential decay rate for estimates of first moment
  vector in adam. should be in [0, 1). only used when solver = `adam'
  %
  \ifthenelse{\equal{#1}{}}{}{\default{#1}}
}
\newcommand{\betaBMLPDescription}[1]
{
  \xmlNode{beta\_2}, \xmlDesc{float, optional field}, exponentail decay rate for estimates of second moment
  vector in adam. should be in [0, 1). only used when solver = `adam'
  %
  \ifthenelse{\equal{#1}{}}{}{\default{#1}}
}

%%%%%%%%%%%%%%%%%%%% End Common Multi-layer Perceptron Parameters %%%%%%%%%%%%%%%%%%%%

%%%%%%%%%%%%%%%%%%%%%%%%%%%%%%%%%%%%%%%%%%%%%%%%%%%%%%%%%%%%%%%%%%%%%%%%%%%%%%%%

In RAVEN, \textbf{Models} are important entities.
%
A model is an object that employs a mathematical representation of a
phenomenon, either of a physical or other nature (e.g. statistical operators,
etc.).
%
From a practical point of view, it can be seen, as a ``black box'' that, given
an input, returns an output.
%

RAVEN has a strict classification of the different types of models.
%
Each ``class'' of models is represented by the definition reported above, but it
can be further classified based on its particular functionalities:
\begin{itemize}
  \item \xmlNode{Code} represents an external system code that employs a high
  fidelity physical model.
  \item \xmlNode{Dummy} acts as ``transfer'' tool.
  %
  The only action it performs is transferring the the information in the input
  space (inputs) into the output space (outputs).
  %
  For example, it can be used to check the effect of a sampling strategy, since
  its outputs are the sampled parameters' values (input space) and a counter
  that keeps track of the number of times an evaluation has been requested.
  %
  \item \xmlNode{ROM}, or reduced order model, is a mathematical model trained
  to predict a response of interest of a physical system.
  %
  Typically, ROMs trade speed for accuracy representing a faster, rough estimate
  of the underlying phenomenon.
  %
  The ``training'' process is performed by sampling the response of a physical
  model with respect to variation of its parameters subject to probabilistic
  behavior.
  %
  The results (outcomes of the physical model) of the sampling are fed into
  the algorithm representing the ROM that tunes itself to replicate those
  results.
  \item \xmlNode{ExternalModel}, as its name suggests, is an entity existing
  outside the RAVEN framework that is embedded in the RAVEN code at run time.
  %
  This object allows the user to create a Python module that will be treated as
  a predefined internal model object.
  \item \xmlNode{EnsembleModel} is model that is able to combine \textbf{Code},
  \textbf{ExternalModel} and \textbf{ROM} models. It is aimed to create a chain
  of Models (whose execution order is determined by the Input/Output relationships among them).
  If the relationships among the models evolve in a non-linear system, a Picard's Iteration scheme
  is employed.
  \item \xmlNode{PostProcessor} is a container of all the actions that can
  manipulate and process a data object in order to extract key information,
  such as statistical quantities, clustering, etc.
  %
\end{itemize}
Before analyzing each model in detail, it is important to mention that each
type needs to be contained in the main XML node \xmlNode{Models}, as reported
below:

\textbf{Example:}
\begin{lstlisting}[style=XML]
<Simulation>
  ...
  <Models>
    ...
    <WhatEverModel name='whatever'>
      ...
    </WhatEverModel>
    ...
  </Models>
  ...
</Simulation>
\end{lstlisting}
In the following sub-sections each \textbf{Model} type is fully analyzed and
described.
%
%%%%%%%%%%%%%%%%%%%%%%%%
%%%%%%  Code  Model   %%%%%%
%%%%%%%%%%%%%%%%%%%%%%%%
%<Code name='MyRAVEN' subType='RAVEN'><executable>%FRAMEWORK_DIR%/../RAVEN-%METHOD%</executable></Code>
%<alias variable='internal_variable_name'>Material|Fuel|thermal_conductivity</alias>
\subsection{Code}
\label{subsec:models_code}
The \textbf{Code} model represents an external system
software employing a high fidelity physical model.
%
The link between RAVEN and the driven code is performed at run-time, through
coded interfaces that are the responsible for transferring information from the
code to RAVEN and vice versa.
%
In Section~\ref{sec:existingInterface}, all of the available interfaces are
reported and, for advanced users, Section~\ref{sec:newCodeCoupling} explains how
to couple a new code.


\specBlock{a}{Code}
%
\attrsIntro
%
\vspace{-5mm}
\begin{itemize}
  \itemsep0em
  \item \nameDescription
  \item \xmlAttr{subType}, \xmlDesc{required string attribute}, specifies the
  code that needs to be associated to this Model.
  %
  \nb See Section~\ref{sec:existingInterface} for a list of currently supported
  codes.
  %
\end{itemize}
\vspace{-5mm}

\subnodesIntro
%
\begin{itemize}
  \item \xmlNode{executable} \xmlDesc{string, required field} specifies the path
  of the executable to be used.

  \item \xmlNode{walltime}  \xmlDesc{string, required field} specifies the maximum 
  allowed run time of the code; if the code running time is greater than the specified 
  walltime then the code run is stopped. The stopped run is then considered as if it chrashed.
  %
  \nb Either an absolute or relative path can be used.
  \item \aliasSystemDescription{Code}
  %
  \item \xmlNode{clargs} \xmlDesc{string, optional field} allows addition of
  command-line arguments to the execution command.  If the code interface
  specified in \xmlNode{Code} \xmlAttr{subType} does not specify how to
  determine the input file(s), this node must be used to specify them.
  There are several types of \xmlNode{clargs}, based on the \xmlAttr{type}:
  \begin{itemize}
    \item \xmlAttr{type} \xmlDesc{string, required field} specifies the type of
    command-line argument to add.  Options include \xmlString{input},
    \xmlString{output}, \xmlString{prepend}, \xmlString{postpend}, and
    \xmlString{text}.
    %
    \item \xmlAttr{arg} \xmlDesc{string, optional field} specifies the flag to
    be used before the entry.  For example, \xmlAttr{arg=}\xmlString{-i} would
    place a \texttt{-i} before the entry in the execution command.  Required for
    the \xmlString{output} \xmlAttr{type}.
    %
    \item \xmlAttr{extension} \xmlDesc{string, optional field} specifies the type
    of file extension to use (for example, \texttt{-i} or \texttt{-o}).  This links the
    \xmlNode{Input} file in the \xmlNode{Step} to this location in the execution
    command.  Required for \xmlString{input} \xmlAttr{type}.
  \end{itemize}
  The execution command is combined in the order \xmlString{prepend},
  \xmlNode{executable}, \xmlString{input}, \xmlString{output}, \xmlString{text},
  \xmlString{postpend}.
  %
  \item \xmlNode{fileargs} \xmlDesc{string, optional field} like \xmlNode{clargs},
  but allows editing of input files to specify the output filename and/or auxiliary
  file names.
  %
  The location in the input files to edit using these arguments are identified in
  the input file using the prefix-postfix notation, which defaults to
  \texttt{\$RAVEN-var\$} for variable keyword \emph{var}.  The variable keyword
  is then listed in the \xmlNode{fileargs} node in the attribute \xmlAttr{arg} to
  couple it in Raven.
  %
  If the code interface specified in \xmlNode{Code} \xmlAttr{subType}
  does not specify how to name the output file, that must be specified either through
  \xmlNode{clargs} or \xmlNode{filargs}, with \xmlAttr{type} \xmlString{output}.
  The attributes required for \xmlNode{fileargs} are as follows:
  \begin{itemize}
    \item \xmlAttr{type} \xmlDesc{string, required field} specifies the type
    of entry to replace in the file.  Possible values for \xmlNode{fileargs}
    \xmlAttr{type} are \xmlString{input} and \xmlString{output}.
    %
    \item \xmlAttr{arg} \xmlDesc{string, required field} specifies the Raven
    variable with which to replace the file of interest.  This should match
    the entry in the template input file; that is, if \texttt{\$RAVEN-auxinp\$}
    is in the input file, the arg for the corresponding input file should be
    \xmlString{auxinp}.
    %
    \item \xmlAttr{extension} \xmlDesc{string, optional field} specifies the
    extension of the input file that should replace the Raven variable in
    the input file.  This attribute is required for the \xmlString{input} \xmlAttr{type}
    and ignored for the \xmlString{output} \xmlAttr{type}.
    \nb{Currently, there can only be a one-to-one pairing between input files
    and extensions; that is, multiple Raven-editable input files cannot have the
    same extension.}
  \end{itemize}
\end{itemize}
\textbf{Example:}
\begin{lstlisting}[style=XML,morekeywords={subType,name,variable}]
<Simulation>
  ...
  <Models>
    ...
    <Code name='aUserDefinedName' subType='RAVEN_Driven_code'>
      <executable>path_to_executable</executable>
      <alias variable='internal_RAVEN_input_variable_name1' type="input">
         External_Code_input_Variable_Name_1
      </alias>
      <alias variable='internal_RAVEN_input_variable_name2' type='input'>
         External_Code_input_Variable_Name_2
      </alias>
      <alias variable='internal_RAVEN__output_variable_name' type='output'>
         External_Code_output_Variable_Name_2
      </alias>
      <clargs type='prepend' arg='python'/>
      <clargs type='input' arg='-i' extension='.i'/>
      <fileargs type='input' arg='aux' extension='.two'
      <fileargs type='output' arg='out' />
    </Code>
    ...
  </Models>
  ...
</Simulation>
\end{lstlisting}

%%%%%%%%%%%%%%%%%%%%%%%%
%%%%%% Dummy Model  %%%%%%
%%%%%%%%%%%%%%%%%%%%%%%%
\subsection{Dummy}
\label{subsec:models_dummy}
The \textbf{Dummy} model is an object that acts as a pass-through tool.
%
The only action it performs is transferring the information in the input
space (inputs) to the output space (outputs).
%
For example, it can be used to check the effect of a particular sampling
strategy, since its outputs are the sampled parameters' values (input space) and
a counter that keeps track of the number of times an evaluation has been
requested.
%

\specBlock{a}{Dummy}.
%
\attrsIntro
%
\vspace{-5mm}
\begin{itemize}
  \itemsep0em
  \item \nameDescription
  %
  \item \xmlAttr{subType}, \xmlDesc{required string attribute}, this attribute
  must be kept empty.
\end{itemize}
\vspace{-5mm}

\subnodesIntro
%
\begin{itemize}
  \item \aliasSystemDescription{Dummy}
  \\Since the \textbf{Dummy} model represents a transfer function only, the usage of the alias is relatively meaningless.
\end{itemize}

Given a particular \textit{Step} using this model, if this model is linked to
a \textit{Data} with the role of \textbf{Output}, it expects one of the output
parameters will be identified by the keyword ``OutputPlaceHolder'' (see
Section~\ref{sec:steps}).

\textbf{Example:}
\begin{lstlisting}[style=XML,morekeywords={subType}]
<Simulation>
  ...
  <Models>
    ...
    <Dummy name='aUserDefinedName1' subType=''/>

    <Dummy name='aUserDefinedName2' subType=''>
      <alias variable="a_RAVEN_input_variable" type="input">
      another_name_for_this_variable_in_the_model
      </alias>
    </Dummy>
    ...
  </Models>
  ...
</Simulation>
\end{lstlisting}



%%%%%%%%%%%%%%%%%%%%%%
%%%%% ROM Model  %%%%%%%
%%%%%%%%%%%%%%%%%%%%%%
\newcommand{\zNormalizationPerformed}[1]
{
  \textcolor{red}{\\It is important to NOTE that RAVEN uses a Z-score normalization of the training data before
  constructing the \textit{#1} ROM:
\begin{equation}
  \mathit{\mathbf{X'}} = \frac{(\mathit{\mathbf{X}}-\mu )}{\sigma }
\end{equation}
 }
}

\newcommand{\zNormalizationNotPerformed}[1]
{
  \textcolor{red}{
  \\It is important to NOTE that RAVEN does not pre-normalize the training data before
  constructing the \textit{#1} ROM.}
}

\subsection{ROM}
\label{subsec:models_ROM}
A Reduced Order Model (ROM) is a mathematical model consisting of a fast
solution trained to predict a response of interest of a physical system.
%
The ``training'' process is performed by sampling the response of a physical
model with respect to variations of its parameters subject, for example, to
probabilistic behavior.
%
The results (outcomes of the physical model) of the sampling are fed into the
algorithm representing the ROM that tunes itself to replicate those results.
%
RAVEN supports several different types of ROMs, both internally developed and
imported through an external library called ``scikit-learn''~\cite{SciKitLearn}.

Currently in RAVEN, the ROMs are classified into several sub-types that, once chosen,
provide access to several different algorithms.
%
These sub-types are specified in the \xmlAttr{subType} attribute and should be
one of the following:
\begin{itemize}
  \item \xmlString{GaussPolynomialRom}, for both static and time-dependent regression
  \item \xmlString{HDMRRom}, for both static and time-dependent regression
  \item \xmlString{NDinvDistWeight}, for both static and time-dependent regression
  \item \xmlString{NDSpline}, for both static and time-dependent regression
  \item \xmlString{SciKitLearn}, for both static and time-dependent regression and classification
  \item \xmlString{MSR}, for both static and time-dependent regression
  \item \xmlString{ARMA}, for time-dependent stochastic regression (time series generator)
  \item \xmlString{PolyExponential}, for time-dependent regression
  \item \xmlString{DMD}, for time-dependent regression
\end{itemize}

\specBlock{a}{ROM}
%
\attrsIntro
%
\vspace{-5mm}
\begin{itemize}
  \itemsep0em
  \item \nameDescription
  \item \xmlAttr{subType}, \xmlDesc{required string attribute}, defines which of
  the sub-types should be used, choosing among the previously reported
  types.
  %
  This choice conditions the subsequent the required and/or optional
  \xmlNode{ROM} sub-nodes.
  %
\end{itemize}
\vspace{-5mm}

In the \xmlNode{ROM} input block, the following XML sub-nodes are required,
independent of the \xmlAttr{subType} specified:
%
\begin{itemize}
  %
   \item \xmlNode{Features}, \xmlDesc{comma separated string, required field}, 
     specifies the names of the features of this ROM.
   \nb These parameters are going to be requested for the training of this object
    (see Section~\ref{subsec:stepRomTrainer});
    \item \xmlNode{Target}, \xmlDesc{comma separated string, required field},
      contains a comma separated list of the targets of this ROM. These parameters 
      are the Figures of Merit (FOMs) this ROM is supposed to predict.
    \nb These parameters are going to be requested for the training of this
    object (see Section \ref{subsec:stepRomTrainer}).
\end{itemize}

If a time-dependent ROM is requested, a \textbf{HistorySet} should be provided.
The temporal vairable specified in the \textbf{HistorySet} should be also listed
as sub-nodes inside \xmlNode{ROM}
%
\begin{itemize}
  \item \xmlNode{pivotParameter}, \xmlDesc{string, optional parameter}, specifies the pivot
    variable (e.g. time, etc) used in the input HistorySet.
    \default{time}
\end{itemize}
%
In addition, if the user wants to use the alias system, the following XML block can be inputted:
\begin{itemize}
  \item \aliasSystemDescription{ROM}
\end{itemize}


The types and meaning of the remaining sub-nodes depend on the sub-type
specified in the attribute \xmlAttr{subType}.

%
Note that if an HistorySet is provided in the training step then a temporal ROM is created, i.e. a ROM that generates not a single value prediction of each element indicated in the  \xmlNode{Target} block but its full temporal profile.
\\
\textcolor{red}{\\\textbf{It is important to NOTE that RAVEN uses a Z-score normalization of the training data before constructing most of the
Reduced Order Models (e.g. most of the SciKitLearn-based ROMs):}}
\begin{equation}
  \mathit{\mathbf{X'}} = \frac{(\mathit{\mathbf{X}}-\mu )}{\sigma }
\end{equation}
\\In the following sections the specifications of each ROM type are reported, highlighting when a \textbf{Z-score normalization} is performed by RAVEN before constructing the ROM or when it is not performed.

%
%%%%% ROM Model - NDspline  %%%%%%%
\subsubsection{NDspline}
\label{subsubsec:NDspline}
The NDspline sub-type contains a single ROM type, based on an $N$-dimensional
spline interpolation/extrapolation scheme.
%
In spline interpolation, the regressor is a special type of piece-wise
polynomial called tensor spline.
%
The interpolation error can be made small even when using low degree polynomials
for the spline.
%
Spline interpolation avoids the problem of Runge's phenomenon, in which
oscillation can occur between points when interpolating using higher degree
polynomials.
%

In order to use this ROM, the \xmlNode{ROM} attribute \xmlAttr{subType} needs to
be \xmlString{NDspline} (see the example below).
%
No further XML sub-nodes are required.
%
\nb This ROM type must be trained from a regular Cartesian grid.
%
Thus, it can only be trained from the outcomes of a grid sampling strategy.

\zNormalizationPerformed{NDspline}

\textbf{Example:}
\begin{lstlisting}[style=XML,morekeywords={name,subType}]
<Simulation>
  ...
  <Models>
    ...
    <ROM name='aUserDefinedName' subType='NDspline'>
       <Features>var1,var2,var3</Features>
       <Target>result1,result2</Target>
     </ROM>
    ...
  </Models>
  ...
</Simulation>
\end{lstlisting}

%%%%% ROM Model - GaussPolynomialRom  %%%%%%%
\subsubsection{pickledROM}
\label{subsubsec:pickledROM}
It is not uncommon for a reduced-order model (ROM) to be created and trained in one RAVEN run, then
serialized to file (\emph{pickled}), then loaded into another RAVEN run to be used as a model.  When this is
the case, a \xmlNode{ROM} with subtype \xmlString{pickledROM} is used to hold the place of the ROM that will
be loaded from file.  The notation for this ROM is much less than a typical ROM; it only requires a name and
its subtype.

Note that when loading ROMs from file, RAVEN will not perform any checks on the expected inputs or outputs of
a ROM; it is expected that a user know at least the I/O of a ROM before trying to use it as a model.
However, RAVEN does require that pickled ROMs be trained before pickling in the first place.

Initially, a pickledROM is not usable.  It cannot be trained or sampled; attempting to do so will raise an
error.  An \xmlNode{IOStep} is used to load the ROM from file, at which point the ROM will have all the same
characteristics as when it was pickled in a previous RAVEN run.

\textbf{Example:}
For this example the ROM has already been created and trained in another RAVEN run, then pickled to a file
called \texttt{rom\_pickle.pk}.  In the example, the file is identified in \xmlNode{Files}, the model is
defined in \xmlNode{Models}, and the model loaded in \xmlNode{Steps}.
{\footnotesize
\begin{lstlisting}[style=XML,morekeywords={name,subType}]
<Simulation>
  ...
  <Files>
    <Input name="rompk" type="">rom_pickle.pk</Input>
  </Files>
  ...
  <Models>
    ...
    <ROM name="myRom" subType="pickledROM"/>
    ...
  </Models>
  ...
  <Steps>
    ...
    <IOStep name="loadROM">
      <Input class="Files" type="">rompk</Input>
      <Output class="Models" type="ROM">myRom</Output>
    </IOStep>
    ...
  </Steps>
  ...
</Simulation>
\end{lstlisting}
}


\subsubsection{GaussPolynomialRom}
\label{subsubsec:GaussPolynomialRom}
The GaussPolynomialRom sub-type contains a single ROM type, based on a
characteristic Gaussian polynomial fitting scheme: generalized polynomial chaos
expansion (gPC).
%
In gPC, sets of polynomials orthogonal with respect to the distribution of uncertainty
are used to represent the original model.  The method converges moments of the original
model faster than Monte Carlo for small-dimension uncertainty spaces ($N<15$).
%
In order to use this ROM, the \xmlNode{ROM} attribute \xmlAttr{subType} needs to
be \xmlString{GaussPolynomialRom} (see the example below).
%
The GaussPolynomialRom is dependent on specific sampling; thus, this ROM cannot be trained unless a
SparseGridCollocation or similar Sampler specifies this ROM in its input and is sampled in a MultiRun step.
%
In addition to the common \xmlNode{Target} and \xmlNode{Features}, this ROM requires
two more nodes and can accept multiple entries of a third optional node.
\begin{itemize}
  \item \xmlNode{IndexSet}, \xmlDesc{string, required field},
  specifies the rules by which to construct multidimensional polynomials.  The options are
  \xmlString{TensorProduct}, \xmlString{TotalDegree},\\
  \xmlString{HyperbolicCross}, and \xmlString{Custom}.
  %
  Total degree is efficient for
  uncertain inputs with a large degree of regularity, while hyperbolic cross is more efficient
  for low-regularity input spaces.
  %
  If \xmlString{Custom} is chosen, the \xmlNode{IndexPoints} is required.
  %
  \item \xmlNode{PolynomialOrder}, \xmlDesc{integer, required field},
  indicates the maximum polynomial order in any one dimension to use in the
  polynomial chaos expansion. \nb If non-equal importance weights are supplied in the optional
  \xmlNode{Interpolation} node, the actual polynomial order in dimensions with high
  importance might exceed this value; however, this value is still used to limit the
  relative overall order.
  %
  \item \xmlNode{SparseGrid},\xmlDesc{string, optional field}, allows specification of the multidimensional
    quadrature construction strategy.  Options are \xmlString{smolyak} and \xmlString{tensor}.  Default is
    \xmlString{smolyak}.
  \item \xmlNode{IndexPoints}, \xmlDesc{list of tuples, required field},
  used to specify the index set points in a \xmlString{Custom} index set.  The tuples are
  entered as comma-seprated values between parenthesis, with each tuple separated by a comma.
  Any amount of whitespace is acceptable.  For example, \xmlNode{IndexPoints}\verb'(0,1),(0,2),(1,1),(4,0)'\xmlNode{/IndexPoints}
  \nb{Using custom index sets
  does not guarantee accurate convergence.}
  %
  \item \xmlNode{Interpolation}, \xmlDesc{string, optional field},
  offers the option to specify quadrature, polynomials, and importance weights for the given
  variable name.  The ROM accepts any number of \xmlNode{Interpolation} nodes up to the
  dimensionality of the input space.  This node accepts several attributes, all of which are
  optional and default to
  the code-defined optimal choices based on the input dimension uncertainty distribution:
  \begin{itemize}
    \item \xmlAttr{quad}, \xmlDesc{string, optional field},
      specifies the quadrature type to use for collocation in this dimension.  The default options
      depend on the uncertainty distribution of the input dimension, as shown in Table
      \ref{tab:gpcCompatible}. Additionally, Clenshaw Curtis quadrature can be used for any
      distribution that doesn't include an infinite bound.
      \default{see Table \ref{tab:gpcCompatible}.}
      \nb For an uncertain distribution aside from the four listed on Table
      \ref{tab:gpcCompatible}, this ROM
      makes use of the uniform-like range of the distribution's CDF to apply quadrature that is
      suited uniform uncertainty (Legendre).  It converges more slowly than the four listed, but are
      viable choices.  Choosing polynomial type Legendre for any non-uniform distribution will
      enable this formulation automatically.
    \item \xmlAttr{poly}, \xmlDesc{string,optional field},
      specifies the interpolating polynomial family to use for the polynomial expansion in this
      dimension.  The default options depend on the quadrature type chosen, as shown in Table
      \ref{tab:gpcCompatible}.  Currently, no polynomials are available outside the
      default. \default{see Table \ref{tab:gpcCompatible}.}
    \item  \xmlAttr{weight}, \xmlDesc{float, optional field},
      delineates the importance weighting of this dimension.  A larger importance weight will
      result in increased resolution for this dimension at the cost of resolution in lower-weighted
      dimensions.  The algorithm normalizes weights at run-time.\default{1}.
  \end{itemize}
  %
\end{itemize}
\begin{table}[htb]
  \centering
  \begin{tabular}{c | c c}
    Unc. Distribution & Default Quadrature & Default Polynomials \\ \hline
    Uniform & Legendre & Legendre \\
    Normal & Hermite & Hermite \\ \hline
    Gamma & Laguerre & Laguerre \\
    Beta & Jacobi & Jacobi \\ \hline
    Other & Legendre* & Legendre*
  \end{tabular}
  \caption{GaussPolynomialRom defaults}
  \label{tab:gpcCompatible}
\end{table}
%
\nb This ROM type must be trained from a collocation quadrature set.
%
Thus, it can only be trained from the outcomes of a SparseGridCollocation sampler.
Also, this ROM must be referenced in the SparseGridCollocation sampler in order to
accurately produce the necessary sparse grid points to train this ROM.

\zNormalizationNotPerformed{GaussPolynomialRom}

\textbf{Example:}
{\footnotesize
\begin{lstlisting}[style=XML,morekeywords={name,subType}]
<Simulation>
  ...
  <Samplers>
    ...
    <SparseGridCollocation name="mySG" parallel="0">
      <variable name="x1">
        <distribution>myDist1</distribution>
      </variable>
      <variable name="x2">
        <distribution>myDist2</distribution>
      </variable>
      <ROM class = 'Models' type = 'ROM' >myROM</ROM>
    </SparseGridCollocation>
    ...
  </Samplers>
  ...
  <Models>
    ...
    <ROM name='myRom' subType='GaussPolynomialRom'>
      <Target>ans</Target>
      <Features>x1,x2</Features>
      <IndexSet>TotalDegree</IndexSet>
      <PolynomialOrder>4</PolynomialOrder>
      <Interpolation quad='Legendre' poly='Legendre' weight='1'>x1</Interpolation>
      <Interpolation quad='ClenshawCurtis' poly='Jacobi' weight='2'>x2</Interpolation>
    </ROM>
    ...
  </Models>
  ...
</Simulation>
\end{lstlisting}
}

When Printing this ROM via a Print OutStream (see \ref{sec:printing}), the available metrics are:
\begin{itemize}
  \item \xmlString{mean}, the mean value of the ROM output within the input space it was trained,
  \item \xmlString{variance}, the variance of the ROM output within the input space it was trained,
  \item \xmlString{samples}, the number of distinct model runs required to construct the ROM,
  \item \xmlString{indices}, the Sobol sensitivity indices (in percent), Sobol total indices, and partial variances,
  \item \xmlString{polyCoeffs}, the polynomial expansion coefficients (PCE moments) of the ROM.  These are
    listed by each polynomial combination, with the polynomial order tags listed in the order of the variables
    shown in the XML print.
\end{itemize}

%%%%% ROM Model - HDMRRom  %%%%%%%
\subsubsection{HDMRRom}
\label{subsubsec:HDMRRom}
The HDMRRom sub-type contains a single ROM type, based on a Sobol decomposition scheme.
%
In Sobol decomposition, also known as high-density model reduction (HDMR, specifically Cut-HDMR),
a model is approximated as as the sum of increasing-complexity interactions.  At its lowest level (order 1), it treats the function as a sum of the reference case plus a functional of each input dimesion separately.  At order 2, it adds functionals to consider the pairing of each dimension with each other dimension.  The benefit to this approach is considering several functions of small input cardinality instead of a single function with large input cardinality.  This allows reduced order models like generalized polynomial chaos (see \ref{subsubsec:GaussPolynomialRom}) to approximate the functionals accurately with few computations runs.
%
In order to use this ROM, the \xmlNode{ROM} attribute \xmlAttr{subType} needs to
be \xmlString{HDMRRom} (see the example below).
%
The HDMRRom is dependent on specific sampling; thus, this ROM cannot be trained unless a
Sobol or similar Sampler specifies this ROM in its input and is sampled in a MultiRun step.
%
In addition to the common \xmlNode{Target} and \xmlNode{Features}, this ROM requires
the same nodes as the GaussPolynomialRom (see \ref{subsubsec:GaussPolynomialRom}.
Additionally, this ROM requires the \xmlNode{SobolOrder} node.
\begin{itemize}
  \item \xmlNode{SobolOrder}, \xmlDesc{integer, required field},
  indicates the maximum cardinality of the input space used in the subset functionals.  For example, order 1
  includes only functionals of each independent dimension separately, while order 2 considers pair-wise
  interactions.
  %
\end{itemize}
\nb This ROM type must be trained from a Sobol decomposition training set.
%
Thus, it can only be trained from the outcomes of a Sobol sampler.
Also, this ROM must be referenced in the Sobol sampler in order to
accurately produce the necessary sparse grid points to train this ROM.
Experience has shown order 2 Sobol decompositions to include the great majority of
  uncertainty in most models.

\zNormalizationNotPerformed{HDMRRom}

\textbf{Example:}
{\footnotesize
\begin{lstlisting}[style=XML,morekeywords={name,subType}]
  <Samplers>
    ...
    <Sobol name="mySobol" parallel="0">
      <variable name="x1">
        <distribution>myDist1</distribution>
      </variable>
      <variable name="x2">
        <distribution>myDist2</distribution>
      </variable>
      <ROM class = 'Models' type = 'ROM' >myHDMR</ROM>
    </Sobol>
    ...
  </Samplers>
  ...
  <Models>
    ...
    <ROM name='myHDMR' subType='HDMRRom'>
      <Target>ans</Target>
      <Features>x1,x2</Features>
      <SobolOrder>2</SobolOrder>
      <IndexSet>TotalDegree</IndexSet>
      <PolynomialOrder>4</PolynomialOrder>
      <Interpolation quad='Legendre' poly='Legendre' weight='1'>x1</Interpolation>
      <Interpolation quad='ClenshawCurtis' poly='Jacobi' weight='2'>x2</Interpolation>
    </ROM>
    ...
  </Models>
\end{lstlisting}
}

When Printing this ROM via an OutStream (see \ref{sec:printing}), the available metrics are:
\begin{itemize}
  \item \xmlString{mean}, the mean value of the ROM output within the input space it was trained,
  \item \xmlString{variance}, the ANOVA-calculated variance of the ROM output within the input space it
    was trained.
  \item \xmlString{samples}, the number of distinct model runs required to construct the ROM,
  \item \xmlString{indices}, the Sobol sensitivity indices (in percent), Sobol total indices, and partial variances.
\end{itemize}

%%%%% ROM Model - MSR  %%%%%%%
\subsubsection{MSR}
\label{subsubsec:MSR}
The MSR sub-type contains a class of ROMs that perform a topological
decomposition of the data into approximately monotonic regions and fits weighted
linear patches to the identified monotonic regions of the input space. Query
points have estimated probabilities that they belong to each cluster. These
probabilities can eitehr be used to give a smooth, weighted prediction based on
the associated linear models, or a hard classification to a particular local
linear model which is then used for prediction. Currently, the probability
prediction can be done using kernel density estimation (KDE) or through a
one-versus-one support vector machine (SVM).
%

In order to use this ROM, the \xmlNode{ROM} attribute \xmlAttr{subType} needs to
be \xmlString{MSR} (see the associated example).
%
\subnodesIntro
%
\begin{itemize}
  \item \xmlNode{persistence}, \xmlDesc{string, optional field}, specifies how
  to define the hierarchical simplification by assigning a value to each local
  minimum and maximum according to the one of the strategy options below:
  \begin{itemize}
    \item \texttt{difference} - The function value difference between the
    extremum and its closest-valued neighboring saddle.
    \item \texttt{probability} - The probability integral computed as the
    sum of the probability of each point in a cluster divided by the count of
    the cluster.
    \item \texttt{count} - The count of points that flow to or from the
    extremum.
    % \item \xmlString{area} - The area enclosed by the manifold that flows to
    % or from the extremum.
  \end{itemize}
  \default{\texttt{difference}}
  \item \xmlNode{gradient}, \xmlDesc{string, optional field}, specifies the
  method used for estimating the gradient, available options are:
  \begin{itemize}
    \item \texttt{steepest}
    %\item \texttt{maxflow} \textit{(disabled)}
  \end{itemize}
  \default{\texttt{steepest}}
  \item \xmlNode{simplification}, \xmlDesc{float, optional field}, specifies the
  amount of noise reduction to apply before returning labels.
  \default{0}
  \item \xmlNode{graph} \xmlDesc{, string, optional field}, specifies the type
  of neighborhood graph used in the algorithm, available options are:
  \begin{itemize}
    \item \texttt{beta skeleton}
    \item \texttt{relaxed beta skeleton}
    \item \texttt{approximate knn}
    %\item \texttt{delaunay} \textit{(disabled)}
  \end{itemize}
  \default{\texttt{beta skeleton}}
  \item \xmlNode{beta}, \xmlDesc{float in the range: (0,2], optional field}, is
  only used when the \xmlNode{graph} is set to \texttt{beta skeleton} or
  \texttt{relaxed beta skeleton}.
  \default{1.0}
  \item \xmlNode{knn}, \xmlDesc{integer, optional field}, is the number of
  neighbors when using the \texttt{approximate knn} for the \xmlNode{graph}
  sub-node and used to speed up the computation of other graphs by using the
  approximate knn graph as a starting point for pruning. -1 means use a fully
  connected graph.
  \default{-1}
  % \item \xmlNode{weighted}, \xmlDesc{boolean, optional}, a flag that specifies
  % whether the regression models should be probability weighted.
  % \default{False}
  \item \xmlNode{partitionPredictor}, \xmlDesc{string, optional}, a flag that
  specifies how the predictions for query point classification should be
  performed. Available options are:
  \begin{itemize}
    \item \texttt{kde}
    \item \texttt{svm}
  \end{itemize}
  \default{kde}
  \item \xmlNode{smooth}, if this node is present, the ROM will blend the
  estimates of all of the local linear models weighted by the probability the
  query point is classified as belonging to that partition of the input space.
  \item \xmlNode{kernel}, \xmlDesc{string, optional field}, this option is only
  used when the \xmlNode{partitionPredictor} is set to \texttt{kde} and
  specifies the type of kernel to use in the kernel density estimation.
  Available options are:
  \begin{itemize}
    \item \texttt{uniform}
    \item \texttt{triangular}
    \item \texttt{gaussian}
    \item \texttt{epanechnikov}
    \item \texttt{biweight} or \texttt{quartic}
    \item \texttt{triweight}
    \item \texttt{tricube}
    \item \texttt{cosine}
    \item \texttt{logistic}
    \item \texttt{silverman}
    \item \texttt{exponential}
  \end{itemize}
  \default{gaussian}
  \item \xmlNode{bandwidth}, \xmlDesc{float or string, optional field}, this
  option is only used when the \xmlNode{partitionPredictor} is set to
  \texttt{kde} and specifies the scale of the fall-off. A higher bandwidth
  implies a smooother blending. If set to \texttt{variable}, then the bandwidth
  will be set to the distance of the $k$-nearest neighbor of the query point
  where $k$ is set by the \xmlNode{knn} parameter.
  \default{1.}
\end{itemize}

\zNormalizationNotPerformed{MSR}

\textbf{Example:}
\begin{lstlisting}[style=XML,morekeywords={name,subType}]
<Simulation>
  ...
  <Models>
    ...
    </ROM>
    <ROM name='aUserDefinedName' subType='MSR'>
       <Features>var1,var2,var3</Features>
       <Target>result1,result2</Target>
       <!-- <weighted>true</weighted> -->
       <simplification>0.0</simplification>
       <persistence>difference</persistence>
       <gradient>steepest</gradient>
       <graph>beta skeleton</graph>
       <beta>1</beta>
       <knn>8</knn>
       <partitionPredictor>kde</partitionPredictor>
       <kernel>gaussian</kernel>
       <smooth/>
       <bandwidth>0.2</bandwidth>
     </ROM>
    ...
  </Models>
  ...
</Simulation>
\end{lstlisting}

%%%%% ROM Model - NDinvDistWeight  %%%%%%%
\subsubsection{NDinvDistWeight}
\label{subsubsec:NDinvDistWeight}
The NDinvDistWeight sub-type contains a single ROM type, based on an
$N$-dimensional inverse distance weighting formulation.
%
Inverse distance weighting (IDW) is a type of deterministic method for
multivariate interpolation with a known scattered set of points.
%
The assigned values to unknown points are calculated via a weighted average of
the values available at the known points.
%

In order to use this Reduced Order Model, the \xmlNode{ROM} attribute
\xmlAttr{subType} needs to be xmlString{NDinvDistWeight} (see the example
below).
%
\subnodeIntro

\begin{itemize}
  \item \xmlNode{p}, \xmlDesc{integer, required field}, must be greater than
  zero and represents the ``power parameter''.
  %
  For the choice of value for \xmlNode{p},it is necessary to consider the degree
  of smoothing desired in the interpolation/extrapolation, the density and
  distribution of samples being interpolated, and the maximum distance over
  which an individual sample is allowed to influence the surrounding ones (lower
  $p$ means greater importance for points far away).
  %
\end{itemize}

\zNormalizationPerformed{NDinvDistWeight}

\textbf{Example:}
\begin{lstlisting}[style=XML,morekeywords={name,subType}]
<Simulation>
  ...
  <Models>
    ...
    <ROM name='aUserDefinedName' subType='NDinvDistWeight'>
      <Features>var1,var2,var3</Features>
      <Target>result1,result2</Target>
      <p>3</p>
     </ROM>
    ...
  </Models>
  ...
</Simulation>
\end{lstlisting}


%%%%% ROM Model - SciKitLearn  %%%%%%%
\subsubsection{SciKitLearn}
\label{subsubsec:SciKitLearn}
The SciKitLearn sub-type represents the container of several ROMs available in
RAVEN through the external library scikit-learn~\cite{SciKitLearn}.
%

In order to use this Reduced Order Model, the \xmlNode{ROM} attribute
\xmlAttr{subType} needs to be \\ \xmlString{SciKitLearn} (i.e.
\xmlAttr{subType}\textbf{\texttt{=}}\xmlString{SciKitLearn}).
%
The specifications of a \xmlString{SciKitLearn} ROM depend on the value assumed
by the following sub-node within the main \xmlNode{ROM} XML node:
\begin{itemize}
  \item \xmlNode{SKLtype}, \xmlDesc{vertical bar (\texttt{|}) separated string,
  required field}, contains a string that represents the ROM type to be used.
  %
  As mentioned, its format is:\\
  \xmlNode{SKLtype}\texttt{mainSKLclass|algorithm}\xmlNode{/SKLtype} where the
  first word (before the ``\texttt{|}'' symbol) represents the main class of
  algorithms, and the second word (after the ``\texttt{|}'' symbol) represents
  the specific algorithm.
  %
\end{itemize}
Based on the \xmlNode{SKLtype} several different algorithms are available.
%
\nb for HistorySet's \xmlString{SciKitLearn} performs the task given in \xmlNode{SKLtype} for
each time step, and only synchronized HistorySet can be used as input to this ROM. For
unsynchronized HistorySet, use \xmlString{HistorySetSync} method in \xmlString{Interfaced}
post-processor to synchronize the input data before using \xmlString{SciKitLearn}.

In the following paragraphs a brief explanation and the input requirements are
reported for each of them.
%
%%%%% ROM Model - SciKitLearn: Linear Models %%%%%%%
\paragraph{Linear Models}
\label{LinearModels}
The LinearModels' algorithms implement generalized linear models.
%
They include Ridge regression, Bayesian regression, lasso, and elastic net
estimators computed with least angle regression and coordinate descent.
%
This class also implements stochastic gradient descent related algorithms.
%
In the following, all of the linear models available in RAVEN are reported.
%
\subparagraph{Linear Model: Automatic Relevance Determination Regression}
\mbox{}
\\The \textit{Automatic Relevance Determination} (ARD) regressor is a
hierarchical Bayesian approach where hyperparameters explicitly represent the
relevance of different input features.
%
These relevance hyperparameters determine the range of variation for the
parameters relating to a particular input, usually by modelling the width of a
zero-mean Gaussian prior on those parameters.
%
If the width of the Gaussian is zero, then those parameters are constrained to
be zero, and the corresponding input cannot have any effect on the predictions,
therefore making it irrelevant.
%
ARD optimizes these hyperparameters to discover which inputs are relevant.
%
\skltype{Automatic Relevance Determination regressor}{linear\_model|ARDRegression}.
\begin{itemize}
  \item \nIterDescriptionA{300}
  \item \tolDescriptionA{1.e-3}
  \item \xmlNode{alpha\_1}, \xmlDesc{float, optional field}, is a shape
  hyperparameter for the Gamma distribution prior over the $\alpha$ parameter.
  \default{ 1.e-6}
  %
  \item \xmlNode{alpha\_2}, \xmlDesc{float, optional field}, inverse scale
  hyperparameter (rate parameter) for the Gamma distribution prior over the
  $\alpha$ parameter.
  \default{ 1.e-6}
  %
  \item \xmlNode{lambda\_1}, \xmlDesc{float, optional field}, shape
  hyperparameter for the Gamma distribution prior over the $\lambda$ parameter.
  \default{ 1.e-6}
  %
  \item \xmlNode{lambda\_2}, \xmlDesc{float, optional field}, inverse scale
  hyperparameter (rate parameter) for the Gamma distribution prior over the
  $\lambda$ parameter.
  \default{ 1.e-6}
  %
  \item \xmlNode{compute\_score}, \xmlDesc{boolean, optional field}, if True,
  compute the objective function at each step of the model.
  \default{False}
  %
  \item \xmlNode{threshold\_lambda}, \xmlDesc{float, optional field}, specifies
  the threshold for removing (pruning) weights with high precision from the
  computation.
  \default{ 1.e+4}
  %
  \item \fitInterceptDescription{True}
  %
  \item \normalizeDescription{False}
  %
  \item \verDescriptionA{False}
\end{itemize}

\zNormalizationNotPerformed{ARDRegression}
%%%%%%%%
\subparagraph{Linear Model: Bayesian ridge regression}
\mbox{}

The \textit{Bayesian ridge regression} estimates a probabilistic model of the
regression problem as described above.
%
The prior for the parameter $w$ is given by a spherical Gaussian:
\begin{equation}
p(w|\lambda) =\mathcal{N}(w|0,\lambda^{-1}\bold{I_{p}})
\end{equation}
The priors over $\alpha$ and $\lambda$ are chosen to be gamma distributions, the
conjugate prior for the precision of the Gaussian.
%
The resulting model is called Bayesian ridge regression, and is similar to the
classical ridge regression.
%
The parameters $w$, $\alpha$, and $\lambda$ are estimated jointly during the fit
of the model.
%
The remaining hyperparameters are the parameters of the gamma priors over
$\alpha$ and $\lambda$.
%
These are usually chosen to be non-informative.
%
The parameters are estimated by maximizing the marginal log likelihood.
%
\skltype{Bayesian ridge regressor}{linear\_model|BayesianRidge}
\begin{itemize}
  \item \nIterDescriptionA{300}
  \item \tolDescriptionA{1.e-3}
  \item \xmlNode{alpha\_1}, \xmlDesc{float, optional field}, is a shape
  hyperparameter for the Gamma distribution prior over the $\alpha$ parameter.
  \default{ 1.e-6}
  %
  \item \xmlNode{alpha\_2}, \xmlDesc{float, optional field}, inverse scale
  hyperparameter (rate parameter) for the Gamma distribution prior over the
  $\alpha$ parameter.
  \default{ 1.e-6}
  %
  \item \xmlNode{lambda\_1}, \xmlDesc{float, optional field}, shape
  hyperparameter for the Gamma distribution prior over the $\lambda$ parameter.
  \default{ 1.e-6}
  %
  \item \xmlNode{lambda\_2}, \xmlDesc{float, optional field}, inverse scale
  hyperparameter (rate parameter) for the Gamma distribution prior over the
  $\lambda$ parameter.
  \default{ 1.e-6}
  %
  \item \xmlNode{compute\_score}, \xmlDesc{boolean, optional field}, if True,
  compute the objective function at each step of the model.
  \default{False}
  %
  \item \fitInterceptDescription{True}
  \item \normalizeDescription{False}
  \item \verDescriptionA{False}
\end{itemize}

\zNormalizationNotPerformed{BayesianRidge}
%%%%%%%
\subparagraph{Linear Model: Elastic Net}
\mbox{}
\\The \textit{Elastic Net} is a linear regression technique with combined L1 and
L2 priors as regularizers.
%
It minimizes the objective function:
\begin{equation}
1/(2*n_{samples}) *||y - Xw||^2_2+alpha*l1\_ratio*||w||_1 + 0.5 *alpha*(1 - l1\_ratio)*||w||^2_2
\end{equation}

\skltype{Elastic Net regressor}{linear\_model|ElasticNet}
\begin{itemize}
  \item \xmlNode{alpha}, \xmlDesc{float, optional field}, specifies a constant
  that multiplies the penalty terms.
  %
  $alpha = 0$ is equivalent to an ordinary least square, solved by the
  \textbf{LinearRegression} object.
  \default{1.0}
  %
  \item \xmlNode{l1\_ratio}, \xmlDesc{float, optional field}, specifies the
  ElasticNet mixing parameter, with $0 <= l1\_ratio <= 1$.
  %
  For $l1\_ratio = 0$ the penalty is an L2 penalty.
  %
  For $l1\_ratio = 1$ it is an L1 penalty.
  %
  For $0 < l1\_ratio < 1$, the penalty is a combination of L1 and L2.
  %
  \default{0.5}
  \item \fitInterceptDescription{True}
  \item \normalizeDescription{False}
  \item \maxIterDescription{1000}
  \item \tolDescriptionB{1.e-4}
  \item \warmStartDescription{False}
  \item \positiveDescription{False}
  %
\end{itemize}
\zNormalizationNotPerformed{ElasticNet}
%%%%%%%%
\subparagraph{Linear Model: Elastic Net CV}
\mbox{}
\\The \textit{Elastic Net CV} is a linear regression similar to the Elastic Net
model but with an iterative fitting along a regularization path.
%
The best model is selected by cross-validation.
%

\skltype{Elastic Net CV regressor}{linear\_model|ElasticNetCV}
\begin{itemize}
  \item \xmlNode{l1\_ratio}, \xmlDesc{float, optional field},
  %
  Float flag between 0 and 1 passed to ElasticNet (scaling between l1 and l2
  penalties).
  %
  For $l1\_ratio = 0$ the penalty is an L2 penalty.
  %
  For $l1\_ratio = 1$ it is an L1 penalty.
  %
  For $0 < l1\_ratio < 1$, the penalty is a combination of L1 and L2 This
  parameter can be a list, in which case the different values are tested by
  cross-validation and the one giving the best prediction score is used.
  %
  Note that a good choice of list of values for $l1\_ratio$ is often to put more
  values close to 1 (i.e. Lasso) and less close to 0 (i.e. Ridge), as in [.1,
  .5, .7, .9, .95, .99, 1].
  %
  \default{0.5}
  \item \xmlNode{eps}, \xmlDesc{float, optional field}, specifies the length of
  the path.
  %
  eps=1e-3 means that $alpha\_min / alpha\_max = 1e-3$.
  %
  \default{0.001}
  \item \xmlNode{n\_alphas}, \xmlDesc{integer, optional field}, is the number of
  alphas along the regularization path used for each $l1\_ratio$.
  %
  \default{100}
  \item \precomputeDescription{'auto'}
  \item \maxIterDescription{1000}
  \item \tolDescriptionB{1.e-4}
  %
  \item \positiveDescription{False}
  %
\end{itemize}
\zNormalizationNotPerformed{ElasticNetCV}
%%%%%%
\subparagraph{Linear Model: Least Angle Regression model}
\mbox{}
\\The \textit{Least Angle Regression model} (LARS) is a regression algorithm for
high-dimensional data.
%
The LARS algorithm provides a means of producing an estimate of which variables
to include, as well as their coefficients, when a response variable is
determined by a linear combination of a subset of potential covariates.
%

\skltype{Least Angle Regression model}{linear\_model|Lars}
\begin{itemize}
  \item \xmlNode{n\_nonzero\_coefs}, \xmlDesc{integer, optional field},
  represents the target number of non-zero coefficients.
  %
  \default{500}
  \item \fitInterceptDescription{True}
  \item \verDescriptionA{False}
  \item \precomputeDescription{'auto'}
  \item \normalizeDescription{True}
  \item \xmlNode{eps}, \xmlDesc{float, optional field}, represents the machine
  precision regularization in the computation of the Cholesky diagonal factors.
  %
  Increase this for very ill-conditioned systems.
  %
  Unlike the \xmlNode{tol} parameter in some iterative optimization-based
  algorithms, this parameter does not control the tolerance of the optimization.
  %
  \default{2.2204460492503131e-16}
  \item \xmlNode{fit\_path}, \xmlDesc{boolean, optional field}, if True the
  full path is stored in the coef\_path\_attribute.
  %
  If you compute the solution for a large problem or many targets, setting
  fit\_path to False will lead to a speedup, especially with a small alpha.
  %
  \default{True}
  %
\end{itemize}
\zNormalizationNotPerformed{Lars}
%%%%%%
\subparagraph{Linear Model: Cross-validated Least Angle Regression model}
\mbox{}
\\The \textit{Cross-validated Least Angle Regression model} is a regression
algorithm for high-dimensional data.
%
It is similar to the LARS method, but the best model is selected by
cross-validation.
%
\skltype{Cross-validated Least Angle Regression model}{linear\_model|LarsCV}
\begin{itemize}
  \item \fitInterceptDescription{True}
  \item \verDescriptionA{False}
  \item \normalizeDescription{True}
  \item \precomputeDescription{'auto'}
  \item \maxIterDescription{500}
  \item \nAlphasDescription{1000}
  \item \xmlNode{eps}, \xmlDesc{float, optional field}, represents the
  machine-precision regularization in the computation of the Cholesky diagonal
  factors.
  %
  Increase this for very ill-conditioned systems.
  %
  Unlike the \textit{tol} parameter in some iterative optimization-based
  algorithms, this parameter does not control the tolerance of the optimization.
  %
  \default{2.2204460492503131e-16}
\end{itemize}
\subparagraph{Linear Model trained with L1 prior as regularizer (aka the Lasso)}
\mbox{}
\\The \textit{Linear Model trained with L1 prior as regularizer (Lasso)} is a
shrinkage and selection method for linear regression.
%
It minimizes the usual sum of squared errors, with a bound on the sum of the
absolute values of the coefficients.
%
\skltype{Linear Model trained with L1 prior as regularizer
  (Lasso)}{linear\_model|Lasso}
\begin{itemize}
  \item \xmlNode{alpha}, \xmlDesc{float, optional field}, sets a constant
  multiplier for the L1 term.
  %
  alpha = 0 is equivalent to an ordinary least square, solved by the
  LinearRegression object.
  %
  For numerical reasons, using alpha = 0 with the Lasso object is not advised
  and you should instead use the LinearRegression object.
  %
  \default{1.0}
  %
  \item \fitInterceptDescription{True}
  \item \normalizeDescription{False}
  \item \precomputeDescription{False}
  \nb For sparse input this option is always True to preserve sparsity.
  \item \maxIterDescription{1000}
  \item \tolDescriptionB{1.e-4}
  \item \warmStartDescription{False}
  \item \positiveDescription{False}
\end{itemize}
\zNormalizationNotPerformed{LarsCV}
\subparagraph{Lasso linear model with iterative fitting along a regularization
  path}
\mbox{}

The \textit{Lasso linear model with iterative fitting along a regularization
path} is an algorithm of the Lasso family, that computes the linear regressor weights,
identifying the regularization path in an iterative fitting (see http://www.jstatsoft.org/v33/i01/paper)

\skltype{Lasso linear model with iterative fitting along a regularization path
regressor}{linear\_model|LassoCV}
\begin{itemize}
  \item \xmlNode{eps}, \xmlDesc{float, optional field}, represents the length of
  the path.
  %
  eps=1e-3 means that alpha\_min / alpha\_max = 1e-3.
  %
  \default{1.0e-3}
  %
  \item \xmlNode{n\_alphas}, \xmlDesc{int, optional field}, sets the number of
  alphas along the regularization path.
  %
  \default{100}
  %
  \item \xmlNode{alphas}, \xmlDesc{numpy array, optional field}, lists the
  locations of the alphas used to compute the models.
  %
  \default{None}
  %
  If None, alphas are set automatically.
  \item \precomputeDescription{'auto'}
  \item \maxIterDescription{1000}
  \item \tolDescriptionB{1.e-4}
  \item \verDescriptionB{False}
  \item \positiveDescription{False}
\end{itemize}
\zNormalizationNotPerformed{LassoCV}
\subparagraph{Lasso model fit with Least Angle Regression}
\mbox{}

\textit{Lasso model fit with Least Angle Regression} (aka Lars)
It is a Linear Model trained with an L1 prior as regularizer.
In order to use the \textit{Least Angle Regression model regressor}, the user needs to set the sub-node
%
\skltype{Least Angle Regression model
regressor}{linear\_model|LassoLars}

\begin{itemize}
  \item \xmlNode{alpha}, \xmlDesc{float, optional field}, specifies a constant
  that multiplies the penalty terms.
  %
  $alpha = 0$ is equivalent to an ordinary least square, solved by the
  \textbf{LinearRegression} object.
  \default{1.0}
  %
  \item \fitInterceptDescription{True}
  \item \verDescriptionB{False}
  \item \normalizeDescription{True}
  \item \precomputeDescription{'auto'}
  \item \maxIterDescription{500}
  \item \xmlNode{eps}, \xmlDesc{float, optional field}, sets the machine
  precision regularization in the computation of the Cholesky diagonal factors.
  %
  Increase this for very ill-conditioned systems.
  %
  \default{2.2204460492503131e-16}
\end{itemize}
\zNormalizationNotPerformed{LassoLars}
\subparagraph{Cross-validated Lasso, using the LARS algorithm}
\mbox{}

The \textit{Cross-validated Lasso, using the LARS algorithm} is a
cross-validated Lasso, using the LARS algorithm.

\skltype{Cross-validated Lasso, using the LARS algorithm
regressor}{linear\_model|LassoLarsCV}

\begin{itemize}
  \item \fitInterceptDescription{True}
  \item \verDescriptionB{False}
  \item \normalizeDescription{True}
  \item \precomputeDescription{'auto'}
  \item \maxIterDescription{500}
  \item \nAlphasDescription{1000}
  \item \xmlNode{eps}, \xmlDesc{float, optional field}, specifies the machine
  precision regularization in the computation of the Cholesky diagonal factors.
  %
  Increase this for very ill-conditioned systems.
  %
  \default{2.2204460492503131e-16}
\end{itemize}
\zNormalizationNotPerformed{LassoLarsCV}
\subparagraph{Lasso model fit with Lars using BIC or AIC for model selection}
\mbox{}

The \textit{Lasso model fit with Lars using BIC or AIC for model selection} is
a Lasso model fit with Lars using BIC or AIC for model selection.
%\maljdan{redundant}
The optimization objective for Lasso is:
$(1 / (2 * n\_samples)) * ||y - Xw||^2_2 + alpha * ||w||_1$
AIC is the Akaike information criterion and BIC is the Bayes information
criterion.
%
Such criteria are useful in selecting the value of the regularization parameter
by making a trade-off between the goodness of fit and the complexity of the
model.
%
A good model explains the data well while maintaining simplicity.
%
\skltype{Lasso model fit with Lars using BIC or AIC for
  model selection regressor}{linear\_model|LassoLarsIC}
\begin{itemize}
  \item \xmlNode{criterion}, \xmlDesc{`bic' | `aic' }, specifies the type of
  criterion to use.
  %
  \default{'aic'}
  %
  \item \fitInterceptDescription{True}
  \item \verDescriptionB{False}
  \item \normalizeDescription{True}
  \item \precomputeDescription{'auto'}
  \item \maxIterDescription{500}
  \item \xmlNode{eps}, \xmlDesc{float, optional field}, represents the machine
  precision regularization in the computation of the Cholesky diagonal factors.
  %
  Increase this for very ill-conditioned systems.
  %
  Unlike the tol parameter in some iterative optimization-based algorithms, this
  parameter does not control the tolerance of the optimization.
  %
  %
  \default{2.2204460492503131e-16}
\end{itemize}
\zNormalizationNotPerformed{LassoLarsIC}
\subparagraph{Ordinary least squares Linear Regression}
\mbox{}

The \textit{Ordinary least squares Linear Regression} is a method for
estimating the unknown parameters in a linear regression model, with the goal of
minimizing the differences between the observed responses in some arbitrary
dataset and the responses predicted by the linear approximation of the data.
%
\skltype{Ordinary least squares Linear
Regressor}{linear\_model|LinearRegression}

\begin{itemize}
  \item \fitInterceptDescription{True}
  \item \normalizeDescription{False}
\end{itemize}
\zNormalizationNotPerformed{LinearRegression}
\subparagraph{Logistic Regression}
\mbox{}
\\The \textit{Logistic Regression} implements L1 and L2 regularized logistic
regression using the liblinear library.
%
It can handle both dense and sparse input.
%
This regressor uses C-ordered arrays or CSR matrices containing 64-bit floats
for optimal performance; any other input format will be converted (and copied).
%
\skltype{Logistic Regressor}{linear\_model|LogisticRegression}
\begin{itemize}
  \item \xmlNode{penalty}, \xmlDesc{string, `l1' or `l2'}, specifies the norm
  used in the penalization.
  %
  \default{'l2'}
  %}
  \item \xmlNode{dual}, \xmlDesc{boolean}, specifies the dual or primal
  formulation.
  %
  Dual formulation is only implemented for the l2 penalty.
  %
  Prefer dual=False when n\_samples $>$ n\_features.
  %
  \default{False}
  %
  \item \xmlNode{C}, \xmlDesc{float, optional field}, is the inverse of the
  regularization strength; must be a positive float.
  %
  Like in support vector machines, smaller values specify stronger
  regularization.
  %
  \default{1.0}
  \item \xmlNode{fit\_intercept}, \xmlDesc{boolean}, specifies if a constant
  (a.k.a. bias or intercept) should be added to the decision function.
  %
  \default{True}
  \item \xmlNode{intercept\_scaling}, \xmlDesc{float, optional field}, when
  self.fit\_intercept is True, instance vector x becomes [x,
  self.intercept\_scaling], i.e. a ``synthetic'' feature with constant value
  equal to intercept\_scaling is appended to the instance vector.
  %
  The intercept becomes intercept\_scaling * synthetic feature
  weight.
  \nb The synthetic feature weight is subject to l1/l2 regularization as are all
  other features.
  %
  To lessen the effect of regularization on synthetic feature weight (and
  therefore on the intercept) intercept\_scaling has to be increased.
  \default{1.0}
  \item \xmlNode{class\_weight}, \xmlDesc{dict, or 'balanced', optional}
  Weights associated with classes in the form \{class\_label: weight\}. If not given, all classes are supposed to have weight one.
  %
  The ``balanced'' mode uses the values of y to automatically adjust weights inversely proportional to class frequencies in the
  input data as n\_samples / (n\_classes * np.bincount(y))
  %
  Note that these weights will be multiplied with sample\_weight (passed through the fit method) if sample\_weight is specified.
  %
  New in version 0.17: class\_weight=’balanced’ instead of deprecated class\_weight=’auto’.
  %
  \default{None}
  %
  \item \randomStateDescription{None}
  \item \tolDescriptionC{0.0001}
\end{itemize}
\zNormalizationPerformed{LogisticRegression}
\subparagraph{Multi-task Lasso model trained with L1/L2 mixed-norm as
  regularizer}
\mbox{}
\\The \textit{Multi-task Lasso model trained with L1/L2 mixed-norm as
  regularizer} is a regressor where the optimization objective for Lasso is:
$(1 / (2 * n\_samples)) * ||Y - XW||^2_{Fro} + alpha * ||W||_{21}$
Where:
$||W||_{21} = \sum_i \sqrt{\sum_j w_{ij}^2}$
i.e. the sum of norm of each row.
%
\skltype{Multi-task Lasso model trained with L1/L2
  mixed-norm as regularizer regressor}{linear\_model|MultiTaskLasso}
\begin{itemize}
  \item \xmlNode{alpha}, \xmlDesc{float, optional field}, sets the constant
  multiplier for the L1/L2 term.
  %
  \default{1.0}
  \item \fitInterceptDescription{True}
  \item \normalizeDescription{False}
  \item \maxIterDescription{1000}
  \item \tolDescriptionB{1.e-4}
  \item \warmStartDescription{False}
\end{itemize}
\zNormalizationNotPerformed{MultiTaskLasso}
\subparagraph{Multi-task Elastic Net model trained with L1/L2 mixed-norm as
  regularizer}
\mbox{}

The \textit{Multi-task Elastic Net model trained with L1/L2 mixed-norm as
  regularizer} is a regressor where the optimization objective for
MultiTaskElasticNet is:
$(1 / (2 * n\_samples)) * ||Y - XW||^{Fro}_2
+ alpha * l1\_ratio * ||W||_{21}
+ 0.5 * alpha * (1 - l1\_ratio) * ||W||_{Fro}^2$
Where:
$||W||_{21} = \sum_i \sqrt{\sum_j w_{ij}^2}$
i.e. the sum of norm of each row.
%
\skltype{Multi-task ElasticNet model trained with L1/L2
  mixed-norm as regularizer regressor}{linear\_model|MultiTaskElasticNet}
\begin{itemize}
  \item \xmlNode{alpha}, \xmlDesc{float, optional field}, represents a constant
  multiplier for the L1/L2 term.
  %
  \default{1.0}
  \item \xmlNode{l1\_ratio}, \xmlDesc{float}, represents the Elastic Net mixing
  parameter, with $0 < l1\_ratio \leq 1$.
  %
  For $l1\_ratio = 0$ the penalty is an L1/L2 penalty.
  %
  For $l1\_ratio = 1$ it is an L1 penalty.
  %
  For $0 < l1\_ratio < 1$, the penalty is a combination of L1/L2
  and L2.
  %
  \default{0.5}
  %
  \item \fitInterceptDescription{True}
  \item \normalizeDescription{False}
  \item \maxIterDescription{}
  \item \tolDescriptionB{1.e-4}
  \item \warmStartDescription{False}
\end{itemize}
\zNormalizationNotPerformed{MultiTaskElasticNet}
\subparagraph{Orthogonal Mathching Pursuit model (OMP)}
\mbox{}

The \textit{Orthogonal Mathching Pursuit model (OMP)} is a type of sparse
approximation which involves finding the ``best matching'' projections of
multidimensional data onto an over-complete dictionary, $D$.
%
\skltype{Orthogonal Mathching Pursuit model (OMP)
regressor}{linear\_model|OrthogonalMatchingPursuit}
\begin{itemize}
  \item \xmlNode{n\_nonzero\_coefs}, \xmlDesc{int, optional field}, represents
  the desired number of non-zero entries in the solution.
  %
  If None, this value is set to 10\% of n\_features.
  %
  \default{None}
  \item \xmlNode{tol}, \xmlDesc{float, optional field}, specifies the maximum
  norm of the residual.
  %
  If not None, overrides n\_nonzero\_coefs.
  %
  \default{None}
  %
  \item \fitInterceptDescription{True}
  \item \normalizeDescription{True}
  \item \xmlNode{precompute}, \xmlDesc{\{True, False, `auto'\}}, specifies
  whether to use a precomputed Gram and Xy matrix to speed up calculations.
  %
  Improves performance when n\_targets or n\_samples is very large.
  %
  \nb If you already have such matrices, you can pass them directly to the
  fit method.
  %
  \default{`auto'}
\end{itemize}
\zNormalizationNotPerformed{OrthogonalMatchingPursuit}
\subparagraph{Cross-validated Orthogonal Mathching Pursuit model (OMP)}
\mbox{}

The \textit{Cross-validated Orthogonal Mathching Pursuit model (OMP)} is a
regressor similar to OMP which has good performance in sparse recovery.
%
\skltype{Cross-validated Orthogonal Mathching Pursuit model (OMP)
regressor}{linear\_model|OrthogonalMatchingPursuitCV}
\begin{itemize}
  \item \fitInterceptDescription{True}
  \item \normalizeDescription{True}
  \item \maxIterDescription{None}
  %
  Maximum number of iterations to perform, therefore maximum features to
  include 10\% of n\_features but at least 5 if available.
  %
  \item \xmlNode{cv}, \xmlDesc{cross-validation generator, optional},
  %
  see sklearn.cross\_validation.
  %
  \default{None}
  \item \verDescriptionB{False}
\end{itemize}
\zNormalizationNotPerformed{OrthogonalMatchingPursuitCV}
\subparagraph{Passive Aggressive Classifier}
\mbox{}
\\The \textit{Passive Aggressive Classifier} is a principled approach to linear
classification that advocates minimal weight updates i.e., the least required
to correctly classify the current training instance.
%
\skltype{Passive Aggressive
Classifier}{linear\_model|PassiveAggressiveClassifier}
\begin{itemize}
  \item \xmlNode{C}, \xmlDesc{float}, specifies the maximum step size
  (regularization).
  %
  \default{1.0}
  %
  \item \fitInterceptDescription{True}
  \item \nIterDescriptionB{5}
  \item \shuffleDescription{True}
  \item \randomStateDescription{None}
  \item \verDescriptionB{0}
  \item \xmlNode{loss}, \xmlDesc{string, optional field}, the loss function to
  be used:
  \begin{itemize}
    \item hinge: equivalent to PA-I (http://jmlr.csail.mit.edu/papers/volume7/crammer06a/crammer06a.pdf)
    \item squared\_hinge: equivalent to PA-II (http://jmlr.csail.mit.edu/papers/volume7/crammer06a/crammer06a.pdf)
  \end{itemize}
  %
  \default{'hinge'}
  %
  \item \warmStartDescription{False}
\end{itemize}
\zNormalizationPerformed{PassiveAggressiveClassifier}
\subparagraph{Passive Aggressive Regressor}
\mbox{}
\\The \textit{Passive Aggressive Regressor} is similar to the Perceptron in that
it does not require a learning rate.
%
However, contrary to the Perceptron, this regressor includes a regularization
parameter, $C$.

\skltype{Passive Aggressive Regressor}{linear\_model|PassiveAggressiveRegressor}
\begin{itemize}
  \item \xmlNode{C}, \xmlDesc{float}, sets the maximum step size
  (regularization).
  %
  \default{1.0}
  %
  \item \xmlNode{epsilon}, \xmlDesc{float}, if the difference between the
  current prediction and the correct label is below this threshold, the model is
  not updated.
  %
  \default{0.1}
  %
  \item \fitInterceptDescription{True}
  \item \nIterDescriptionB{5}
  \item \shuffleDescription{True}
  \item \randomStateDescription{None}
  \item \verDescriptionB{0}
  \item \xmlNode{loss}, \xmlDesc{string, optional field}, specifies the loss
  function to be used:
  \begin{itemize}
    \item epsilon\_insensitive: equivalent to PA-I in the reference paper (http://jmlr.csail.mit.edu/papers/volume7
    /crammer06a/crammer06a.pdf).
    \item squared\_epsilon\_insensitive: equivalent to PA-II in the reference paper (http://jmlr.csail.mit.edu/papers
    /volume7/crammer06a/crammer06a.pdf).
  \end{itemize}
  %
  \default{'epsilon\_insensitive'}
  %
  \item \warmStartDescription{False}
\end{itemize}
\zNormalizationPerformed{PassiveAggressiveRegressor}
\subparagraph{Perceptron}
\mbox{}

The \textit{Perceptron} method is an algorithm for supervised classification of
an input into one of several possible non-binary outputs.
%
It is a type of linear classifier, i.e. a classification algorithm that makes
its predictions based on a linear predictor function combining a set of weights
with the feature vector.
%
The algorithm allows for online learning, in that it processes elements in the
training set one at a time.
%
\skltype{Perceptron classifier}{linear\_model|Perceptron}
\begin{itemize}
  \item \xmlNode{penalty}, \xmlDesc{None, `l2' or `l1' or `elasticnet'}, defines
  the penalty (aka regularization term) to be used.
  %
  \default{None}
  %
  \item \xmlNode{alpha}, \xmlDesc{float}, sets the constant multiplier for the
  regularization term if regularization is used.
  %
  \default{0.0001}
  \item \fitInterceptDescription{True}
  \item \nIterDescriptionB{5}
  \item \shuffleDescription{True}
  \item \randomStateDescription{0}
  \item \verDescriptionB{0}
  \item \xmlNode{eta0}, \xmlDesc{double, optional field}, defines the constant
  multiplier for the updates.
  %
  \default{1.0}
  %
  \item \xmlNode{class\_weight}, \xmlDesc{dict, \{class\_label: weight\} or “balanced” or None, optional}
  Preset for the class\_weight fit parameter.
  %
  Weights associated with classes. If not given, all classes are supposed to have weight one.
  %
  The “balanced” mode uses the values of y to automatically adjust weights inversely proportional to class
  frequencies in the input data as n\_samples / (n\_classes * np.bincount(y))
  %
  \item \warmStartDescription{False}
\end{itemize}
\zNormalizationPerformed{PassiveAggressiveRegressor}
\subparagraph{Linear least squares with l2 regularization}
\mbox{}
\\The \textit{Linear least squares with l2 regularization} solves a regression
model where the loss function is the linear least squares function and the
regularization is given by the l2-norm.
%
Also known as Ridge Regression or Tikhonov regularization.
%
This estimator has built-in support for multivariate regression (i.e., when y
is a 2d-array of shape [n\_samples, n\_targets]).
%
\skltype{Linear least squares with l2 regularization}{linear\_model|Ridge}
\begin{itemize}
  \item \xmlNode{alpha}, \xmlDesc{float, array-like},
  %
  shape = [n\_targets] Small positive values of alpha improve the
  conditioning of the problem and reduce the variance of the estimates.
  %
  Alpha corresponds to $(2*C)^-1$ in other linear models such as
  LogisticRegression or LinearSVC.
  %
  If an array is passed, penalties are assumed to be specific to the targets.
  %
  Hence they must correspond in number.
  %
  \default{1.0}
  %
  \item \fitInterceptDescription{True}
  \item \maxIterDescription{determined by scipy.sparse.linalg.}
  \item \normalizeDescription{False}
  \item \solverDescription
  \default{`auto'}
\end{itemize}
\zNormalizationNotPerformed{Ridge}
%TODO document copy_X
%TODO document tol
%TODO document random_state

\subparagraph{Classifier using Ridge regression}
\mbox{}

The \textit{Classifier using Ridge regression} is a classifier based on linear
least squares with l2 regularization.
\skltype{Classifier using Ridge regression}{linear\_model|RidgeClassifier}

\begin{itemize}
  \item \xmlNode{alpha}, \xmlDesc{float}, small positive values of alpha improve
  the conditioning of the problem and reduce the variance of the estimates.
  %
  Alpha corresponds to $(2*C)^-1$ in other linear models such as
  LogisticRegression or LinearSVC.
  %
  \default{1.0}
  %
  \item \xmlNode{class\_weight}, \xmlDesc{dict, optional field}, specifies
  weights associated with classes in the form {class\_label: weight}.
  %
  If not given, all classes are assumed to have weight one.
  %
  \default{None}
  %
  \item \fitInterceptDescription{True}
  \item \maxIterDescription{determined by scipy.sparse.linalg.}
  \item \normalizeDescription{False}
  \item \solverDescription
  \default{`auto'}
  \item \xmlNode{tol}, \xmlDesc{float}, defines the required precision of the
  solution.
  \default{0.001}
\end{itemize}
\zNormalizationNotPerformed{RidgeClassifier}
%TODO document random_state
%TODO document copy_X

\subparagraph{Ridge classifier with built-in cross-validation}
\mbox{}
\\The \textit{Ridge classifier with built-in cross-validation} performs
Generalized Cross-Validation, which is a form of efficient leave-one-out
cross-validation.
%
Currently, only the n\_features $>$ n\_samples case is handled efficiently.
%
\skltype{Ridge classifier with built-in cross-validation
classifier}{linear\_model|RidgeClassifierCV}
\begin{itemize}
  \item \xmlNode{alphas}, \xmlDesc{numpy array of shape [n\_alphas]}, is an
  array of alpha values to try.
  %
  Small positive values of alpha improve the conditioning of the problem and
  reduce the variance of the estimates.
  %
  Alpha corresponds to $(2*C)^{-1}$ in other linear models such as
  LogisticRegression or LinearSVC.
  %
  \default{(0.1, 1.0, 10.0)}
  %
  \item \fitInterceptDescription{True}
  \item \normalizeDescription{False}
  \item \xmlNode{scoring}, \xmlDesc{string, callable or None, optional}, is a
  string (see model evaluation documentation) or a scorer callable object /
  function with signature scorer(estimator, X, y).
  %
  \default{None}
  \item \xmlNode{cv}, \xmlDesc{cross-validation generator, optional},
  %
  If None, Generalized Cross-Validation (efficient leave-one-out) will be used.
  %
  \default{None}
  %
  \item \xmlNode{class\_weight}, \xmlDesc{dic, optional field}, specifies
  weights associated with classes in the form {class\_label:weight}.
  %
  If not given, all classes are supposed to have weight one.
  %
  \default{None}
  %
\end{itemize}
\zNormalizationNotPerformed{RidgeClassifierCV}
\subparagraph{Ridge regression with built-in cross-validation}
\mbox{}

The \textit{Ridge regression with built-in cross-validation} performs
Generalized Cross-Validation, which is a form of efficient leave-one-out
cross-validation.
%
\skltype{Ridge regression with built-in cross-validation regressor}{linear\_model|RidgeCV}
\begin{itemize}
  \item \xmlNode{alphas}, \xmlDesc{numpy array of shape [n\_alphas]}, specifies
  an array of alpha values to try.
  %
  Small positive values of alpha improve the conditioning of the problem and
  reduce the variance of the estimates.
  %
  Alpha corresponds to $(2*C)^{-1}$ in other linear models such as
  LogisticRegression or LinearSVC.
  %
  \default{(0.1, 1.0, 10.0)}
  \item \fitInterceptDescription{True}
  \item \normalizeDescription{False}
  \item \xmlNode{scoring}, \xmlDesc{string, callable or None, optional}, is a
  string (see model evaluation documentation) or a scorer callable object /
  function with signature scorer(estimator, X, y).
  %
  \default{None}
  %
  \item \xmlNode{cv}, \xmlDesc{cross-validation generator, optional field}, if
  None, Generalized Cross-Validation (efficient leave-one-out) will be used.
  %
  \default{None}
  %
  \item \xmlNode{gcv\_mode}, \xmlDesc{\{None, `auto,' `svd,' `eigen'\}, optional
  field}, is a flag indicating which strategy to use when performing Generalized
  Cross-Validation.
  %
  Options are:
	\begin{itemize}
    \item `auto:' use svd if n\_samples > n\_features or when X is a
    sparse matrix, otherwise use eigen
  	\item `svd:' force computation via singular value decomposition of $X$
    (does not work for sparse matrices)
	  \item `eigen:' force computation via eigendecomposition of $X^T X$
	\end{itemize}
	The `auto' mode is the default and is intended to pick the cheaper
  option of the two depending upon the shape and format of the training data.
  %
  \default{None}
  \item \xmlNode{store\_cv\_values}, \xmlDesc{boolean}, is a flag indicating if
  the cross-validation values corresponding to each alpha should be stored in
  the cv\_values\_attribute (see below).
  %
  This flag is only compatible with cv=None (i.e. using Generalized
  Cross-Validation).
  %
  \default{False}
\end{itemize}
\zNormalizationNotPerformed{RidgeCV}
\subparagraph{Linear classifiers (SVM, logistic regression, a.o.) with SGD
training}
\mbox{}

The \textit{Linear classifiers (SVM, logistic regression, a.o.) with SGD
training} implements regularized linear models with stochastic gradient
descent (SGD) learning: the gradient of the loss is estimated for each sample at
a time and the model is updated along the way with a decreasing strength
schedule (aka learning rate).
%
SGD allows minibatch (online/out-of-core) learning, see the partial\_fit method.
%
This implementation works with data represented as dense or sparse arrays of
floating point values for the features.
%
The model it fits can be controlled with the loss parameter; by default, it fits
a linear support vector machine (SVM).
%
The regularizer is a penalty added to the loss function that shrinks model
parameters towards the zero vector using either the squared Euclidean norm L2 or
the absolute norm L1 or a combination of both (Elastic Net).
%
If the parameter update crosses the 0.0 value because of the regularizer, the
update is truncated to 0.0 to allow for learning sparse models and achieves
online feature selection.
%
\skltype{Linear classifiers (SVM, logistic regression, a.o.) with SGD
training}{linear\_model|SGDClassifier}
\begin{itemize}
  \item \xmlNode{loss}, \xmlDesc{str, `hinge,' `log,' `modified\_huber,'
  `squared\_hinge,' `perceptron,' or a regression loss: `squared\_loss,'
  `huber,' `epsilon\_insensitive,' or `squared\_epsilon\_insensitive'},
  %
  dictates the loss function to be used.
  %
  The available options are:
  \begin{itemize}
    \item `hinge' gives a linear SVM.
    \item `log' loss gives logistic regression, a probabilistic classifier.
    \item `modified\_huber' is another smooth loss that brings tolerance to
    outliers as well as probability estimates.
    \item `squared\_hinge' is like hinge but is quadratically penalized.
    \item `perceptron' is the linear loss used by the perceptron algorithm.
  \end{itemize}
  The other losses are designed for regression but can be useful in
  classification as well; see SGDRegressor for a description.
  %
  \default{`hinge'}
  %
  \item \xmlNode{penalty}, \xmlDesc{str, `l2' or `l1' or `elasticnet'}, defines
  the penalty (aka regularization term) to be used.
  %
  `l2' is the standard regularizer for linear SVM models.
  %
  `l1' and `elasticnet' might bring sparsity to the model (feature
  selection) not achievable with `l2.'
  %
  \default{`l2'}
  \item \xmlNode{alpha}, \xmlDesc{float}, is the constant multiplier for the
  regularization term.
  %
  \default{0.0001}
  \item \xmlNode{l1\_ratio}, \xmlDesc{float}, is the Elastic Net mixing
  parameter, with 0 <= l1\_ratio <= 1.
  %
  l1\_ratio=0 corresponds to L2 penalty, l1\_ratio=1 to L1.
  %
  \default{0.15}
  %
  \item \fitInterceptDescription{True}
  \item \nIterDescriptionB{5}
  \item \shuffleDescription{True}
  \item \randomStateSVMDescription{None}
  \item \verDescriptionB{0}
  \item \xmlNode{epsilon}, \xmlDesc{float, optional field}, varies meaning
  depending on the value of \xmlNode{loss}. If loss is `huber',
  `epsilon\_insensitive' or `squared\_epsilon\_insensitive' then this is the
  epsilon in the epsilon-insensitive loss functions. For ‘huber’,
  determines the threshold at which it becomes less important to get the
  prediction exactly right. For `epsilon\_insensitive, any differences between
  the current prediction and the correct label are ignored if they are less than
  this threshold.
  %
  \default{0.1}
  %
  \item \xmlNode{learning\_rate}, \xmlDesc{string, optional field}, specifies
  the learning rate:
  \begin{itemize}
    \item `constant:' eta = eta0
    \item `optimal:' eta = 1.0 / (t + t0)
    \item `invscaling:' eta = eta0 / pow(t, power\_t)
  \end{itemize}
  \default{`optimal'}
  %
  \item \xmlNode{eta0}, \xmlDesc{double}, specifies the initial learning rate
  for the `constant' or `invscaling' schedules.
  %
  The default value is 0.0 as eta0 is not used by the default schedule
  `optimal.'
  %
  \default{0.0}
  %
  \item \xmlNode{power\_t}, \xmlDesc{double}, represents the exponent for
  the inverse scaling learning rate.
  %
  \default{0.5}
  %
  \item \xmlNode{class\_weight}, \xmlDesc{dict, {class\_label}}, is the preset
  for the class\_weight fit parameter.
  %
  Weights associated with classes.
  %
  If not given, all classes are assumed to have weight one.
  %
  The ``auto'' mode uses the values of y to automatically adjust weights
  inversely proportional to class frequencies.
  %
  \default{None}
  %
  \item \warmStartDescription{False}
  %
\end{itemize}
\zNormalizationPerformed{SGDClassifier}
%TODO document average

\subparagraph{Linear model fitted by minimizing a regularized empirical loss
with SGD}
\mbox{}
\\The \textit{Linear model fitted by minimizing a regularized empirical loss
with SGD} is a model where SGD stands for Stochastic Gradient Descent: the
gradient of the loss is estimated each sample at a time and the model is updated
along the way with a decreasing strength schedule (aka learning rate).
%
The regularizer is a penalty added to the loss function that shrinks model
parameters towards the zero vector using either the squared euclidean norm L2 or
the absolute norm L1 or a combination of both (Elastic Net).
%
If the parameter update crosses the 0.0 value because of the regularizer, the
update is truncated to 0.0 to allow for learning sparse models and achieving
online feature selection.
%
This implementation works with data represented as dense numpy arrays of
floating point values for the features.
%
\skltype{Linear model fitted by minimizing a regularized empirical loss with SGD}{linear\_model|SGDRegressor}
\begin{itemize}
  \item \xmlNode{loss}, \xmlDesc{str, `squared\_loss,' `huber,'
  `epsilon\_insensitive,' or `squared\_epsilon\_insensitive'}, specifies the
  loss function to be used.
  %
  Defaults to `squared\_loss' which refers to the ordinary least squares fit.
  %
  `huber' modifies `squared\_loss' to focus less on getting outliers correct by
  switching from squared to linear loss past a distance of epsilon.
  %
  `epsilon\_insensitive' ignores errors less than epsilon and is linear past
  that; this is the loss function used in SVR.
  %
  `squared\_epsilon\_insensitive' is the same but becomes squared loss past a
  tolerance of epsilon.
  %
  \default{`squared\_loss'}
  \item \xmlNode{penalty}, \xmlDesc{str, `l2' or `l1' or `elasticnet'}, sets
  the penalty (aka regularization term) to be used.
  %
  Defaults to `l2' which is the standard regularizer for linear SVM models.
  %
  `l1' and `elasticnet' might bring sparsity to the model (feature
  selection) not achievable with `l2'.
  %
  \default{`l2'}
  %
  \item \xmlNode{alpha}, \xmlDesc{float},
  %
  Constant that multiplies the regularization term.
  %
  Defaults to 0.0001
  \item \xmlNode{l1\_ratio}, \xmlDesc{float}, is the Elastic Net mixing
  parameter, with $0 \leq l1\_ratio \leq 1$.
  %
  l1\_ratio=0 corresponds to L2 penalty, l1\_ratio=1 to L1.
  %
  \default{0.15}
  %
  \item \fitInterceptDescription{True}
  \item \nIterDescriptionB{5}
  \item \shuffleDescription{True}
  \item \randomStateDescription{None}
  \item \verDescriptionB{0}
  %
  \item \xmlNode{epsilon}, \xmlDesc{float}, sets the epsilon in the
  epsilon-insensitive loss functions; only if loss is `huber,'
  `epsilon\_insensitive,' or `squared\_epsilon\_insensitive.'
  %
  For `huber', determines the threshold at which it becomes less important
  to get the prediction exactly right.
  %
  For epsilon-insensitive, any differences between the current prediction and
  the correct label are ignored if they are less than this threshold.
  %
  \default{0.1}
  %
  \item \xmlNode{learning\_rate}, \xmlDesc{string, optional field},
  Learning rate:
  \begin{itemize}
    \item constant: eta = eta0
    \item optimal: eta = 1.0/(t+t0)
    \item invscaling: eta= eta0 / pow(t, power\_t)
  \end{itemize}
  \default{invscaling}
  \item \xmlNode{eta0}, \xmlDesc{double}, specifies the initial learning rate.
  %
  \default{0.01}
  %
  \item \xmlNode{power\_t}, \xmlDesc{double, optional field}, specifies the
  exponent for inverse scaling learning rate.
  %
  \default{0.25}
  %
  \item \warmStartDescription{False}
  %
\end{itemize}
\zNormalizationPerformed{SGDRegressor}
%TODO document average

%%%%% ROM Model - SciKitLearn: Support Vector Machines %%%%%%%
\paragraph{Support Vector Machines}
\label{SVM}
In machine learning, \textbf{Support Vector Machines} (SVMs, also support vector
networks) are supervised learning models with associated learning algorithms
that analyze data and recognize patterns, used for classification and regression
analysis.
%
Given a set of training examples, each marked as belonging to one of two
categories, an SVM training algorithm builds a model that assigns new examples
into one category or the other, making it a non-probabilistic binary linear
classifier.
%
An SVM model is a representation of the examples as points in space, mapped so
that the examples of the separate categories are divided by a clear gap that is
as wide as possible.
%
New examples are then mapped into that same space and predicted to belong to a
category based on which side of the gap they fall on.
%
In addition to performing linear classification, SVMs can efficiently perform a
non-linear classification using what is called the kernel trick, implicitly
mapping their inputs into high-dimensional feature spaces.
%
\zNormalizationPerformed{SVM-based}

In the following, all the SVM models available in RAVEN are reported.

\subparagraph{Linear Support Vector Classifier}
\mbox{}
\\The \textit{Linear Support Vector Classifier} is similar to SVC with parameter
kernel=`linear', but implemented in terms of liblinear rather than libsvm,
so it has more flexibility in the choice of penalties and loss functions and
should scale better (to large numbers of samples).
%
This class supports both dense and sparse input and the multiclass support is
handled according to a one-vs-the-rest scheme.
%
\skltype{Linear Support Vector Classifier}{svm|LinearSVC}
\begin{itemize}
  \item \CSVMDescription{1.0}
  \item \xmlNode{loss}, \xmlDesc{string, `hinge' or `squared\_hinge'}, specifies the loss
  function.
  %
  `hinge' is the hinge loss (standard SVM) while `squared\_hinge' is the squared hinge
  loss.
  %
  \default{`squared\_hinge'}
  %
  \item \xmlNode{penalty}, \xmlDesc{string, `l1' or `l2'}, specifies the norm
  used in the penalization.
  %
  The `l2' penalty is the standard used in SVC.
  %
  The `l1' leads to coef\_vectors that are sparse.
  %
  \default{`l2'}
  %
  \item \xmlNode{dual}, \xmlDesc{boolean}, selects the algorithm to either solve
  the dual or primal optimization problem.
  %
  Prefer dual=False when n\_samples $>$ n\_features.
  %
  \default{True}
  %
  \item \tolSVMDescription{1e-4}
  %
  \item \xmlNode{multi\_class}, \xmlDesc{string, `ovr' or `crammer\_singer'},
  %
  Determines the multi-class strategy if y contains more than two classes.
  %
  ovr trains n\_classes one-vs-rest classifiers, while
  crammer\_singer optimizes a joint objective over all classes.
  %
  While crammer\_singer is interesting from a theoretical perspective as it is
  consistent, it is seldom used in practice and rarely leads to better accuracy
  and is more expensive to compute.
  %
  If crammer\_singer is chosen, the options loss, penalty and dual
  will be ignored.
  %
  \default{`ovr'}
  %
  \item \fitInterceptDescription{True}
  %
  \item \xmlNode{intercept\_scaling}, \xmlDesc{float, optional field}, when
  True, the instance vector x becomes [x,self.intercept\_scaling], i.e. a
  ``synthetic'' feature with constant value equals to intercept\_scaling is
  appended to the instance vector.
  %
  The intercept becomes intercept\_scaling * synthetic feature
  weight.
  \nb The synthetic feature weight is subject to l1/l2 regularization as are all
  other features.
  %
  To lessen the effect of regularization on the synthetic feature weight (and
  therefore on the intercept) intercept\_scaling has to be increased.
  %
  \default{1}
  %
  \item \classWeightDescription{None}
  \item \verDescriptionB{0}
  %
  \nb This setting takes advantage of a per-process runtime setting in liblinear
  that, if enabled, may not work properly in a multithreaded context.
  %
  \item \randomStateSVMDescription{None}
\end{itemize}

\subparagraph{C-Support Vector Classification}
\mbox{}
\\The \textit{C-Support Vector Classification} is a based on libsvm.
%
The fit time complexity is more than quadratic with the number of samples which
makes it hard to scale to datasets with more than a couple of 10000 samples.
%
The multiclass support is handled according to a one-vs-one scheme.
%
\skltype{C-Support Vector Classifier}{svm|SVC}
\begin{itemize}
  \item \CSVMDescription{1.0}
  \item \kernelDescription{`rbf'}
  \item \degreeDescription{3.0}
  \item \gammaDescription{`auto'}
  \item \coefZeroDescription{0.0}
  \item \probabilityDescription{False}
  \item \shrinkingDescription{True}
  \item \tolSVMDescription{1e-3}
  \item \cacheSizeDescription{}
  \item \classWeightDescription{None}
  \item \verSVMDescription{False}
  \item \maxIterDescription{-1}
    %TODO: Should decision_function_shape be documented?
  \item \randomStateSVMDescription{None}
  %
\end{itemize}

\subparagraph{Nu-Support Vector Classification}
\mbox{}

The \textit{Nu-Support Vector Classification} is similar to SVC but uses a
parameter to control the number of support vectors.
%
The implementation is based on libsvm.
%
\skltype{Nu-Support Vector Classifier}{svm|NuSVC}
\begin{itemize}
  \item \xmlNode{nu}, \xmlDesc{float, optional field}, is an upper bound on the
  fraction of training errors and a lower bound of the fraction of support
  vectors.
  %
  Should be in the interval (0, 1].
  %
  \default{0.5}
  %
  \item \kernelDescription{`rbf'}
  \item \degreeDescription{3}
  \item \gammaDescription{`auto'}
  \item \coefZeroDescription{0.0}
  \item \probabilityDescription{False}
  \item \shrinkingDescription{True}
  \item \tolSVMDescription{1e-3}
  \item \cacheSizeDescription{}
  \item \verSVMDescription{False}
  \item \maxIterDescription{-1}
    %TODO document decision_function_shape
  \item \randomStateSVMDescription{None}
  %
\end{itemize}

\subparagraph{Support Vector Regression}
\mbox{}
\\The \textit{Support Vector Regression} is an epsilon-Support Vector
Regression.
%
The free parameters in this model are C and epsilon.
%
The implementations is a based on libsvm.
%
\skltype{Support Vector Regressor}{svm|SVR}
\begin{itemize}
  \item \CSVMDescription{1.0}
  \item \xmlNode{epsilon}, \xmlDesc{float, optional field}, specifies the
  epsilon-tube within which no penalty is associated in the training loss
  function with points predicted within a distance epsilon from the actual
  value.
  %
  \default{0.1}
  %
  \item \kernelDescription{`rbf'}
  \item \degreeDescription{3.0}
  \item \gammaDescription{`auto'}
  \item \coefZeroDescription{0.0}
  \item \shrinkingDescription{True}
  \item \tolSVMDescription{1e-3}
  \item \cacheSizeDescription{}
  \item \verSVMDescription{False}
  \item \maxIterDescription{-1}
  %
\end{itemize}
 %%%%% ROM Model - SciKitLearn: MultiClass %%%%%%%
\paragraph{Multi Class}
\label{Multiclass}
Multiclass classification means a classification task with more than two
classes; e.g., classify a set of images of fruits which may be oranges, apples,
or pears.
%
Multiclass classification makes the assumption that each sample is assigned to
one and only one label: a fruit can be either an apple or a pear but not both at
the same time.

\zNormalizationNotPerformed{multi-class-based}

%
In the following, all the multi-class models available in RAVEN are reported.
%
%%%%%%%%%
\subparagraph{One-vs-the-rest (OvR) multiclass/multilabel strategy}
\mbox{}

The \textit{One-vs-the-rest (OvR) multiclass/multilabel strategy}, also known
as one-vs-all, consists in fitting one classifier per class.
%
For each classifier, the class is fitted against all the other classes.
%
In addition to its computational efficiency (only n\_classes classifiers are
needed), one advantage of this approach is its interpretability.
%
Since each class is represented by one and one classifier only, it is possible
to gain knowledge about the class by inspecting its corresponding classifier.
%
This is the most commonly used strategy and is a fair default choice.

\skltype{One-vs-the-rest (OvR) multiclass/multilabel classifier}{multiClass|OneVsRestClassifier}
\begin{itemize}
  \item \estimatorDescription{}
\end{itemize}
%Should n_jobs be documented?

%%%%%%%%%%%%
\subparagraph{One-vs-one multiclass strategy}
\mbox{}

The \textit{One-vs-one multiclass strategy} consists in fitting one classifier
per class pair.
%
At prediction time, the class which received the most votes is selected.
%
Since it requires to fit n\_classes * (n\_classes - 1) / 2 classifiers, this
method is usually slower than one-vs-the-rest, due to its O(n\_classes$^2$)
complexity.
%
However, this method may be advantageous for algorithms such as kernel
algorithms which do not scale well with n\_samples.
%
This is because each individual learning problem only involves a small subset of
the data whereas, with one-vs-the-rest, the complete dataset is used n\_classes
times.

\skltype{One-vs-one multiclass classifier}{multiClass|OneVsOneClassifier}
\begin{itemize}
  \item \estimatorDescription{}
\end{itemize}
%Should n_jobs be documented?

%%%%%%%%%%%%%
\subparagraph{Error-Correcting Output-Code multiclass strategy}
\mbox{}
\\The \textit{Error-Correcting Output-Code multiclass strategy} consists in
representing each class with a binary code (an array of 0s and 1s).
%
At fitting time, one binary classifier per bit in the code book is fitted.
%
At prediction time, the classifiers are used to project new points in the class
space and the class closest to the points is chosen.
%
The main advantage of these strategies is that the number of classifiers used
can be controlled by the user, either for compressing the model ($0 < code\_
size < 1$) or for making the model more robust to errors ($code\_ size > 1$).

\skltype{Error-Correcting Output-Code multiclass classifier}{multiClass|OutputCodeClassifier}
\begin{itemize}
  \item \estimatorDescription{}
  \item \xmlNode{code\_size}, \xmlDesc{float, required field}, represents the
  percentage of the number of classes to be used to create the code book.
  %
  A number between 0 and 1 will require fewer classifiers than one-vs-the-rest.
  %
  A number greater than 1 will require more classifiers than one-vs-the-rest.
  %
\end{itemize}
%Should random_state and n_jobs be documented?

%%%%%%%%%%%%%
%\subparagraph{fit a one-vs-the-rest strategy}
%pass
%\subparagraph{Make predictions using the one-vs-the-rest strategy}
%pass
%\subparagraph{ Fit a one-vs-one strategy}
%pass
%\subparagraph{Make predictions using the one-vs-one strategy}
%pass
%\subparagraph{Fit an error-correcting output-code strategy}
%pass
%\subparagraph{Make predictions using the error-correcting output-code strategy}
%pass

 %%%%% ROM Model - SciKitLearn: naiveBayes %%%%%%%
\paragraph{Naive Bayes}
\label{naiveBayes}
Naive Bayes methods are a set of supervised learning algorithms based on
applying Bayes' theorem with the ``naive'' assumption of independence between
every pair of features.
%
Given a class variable y and a dependent feature vector x\_1 through x\_n,
Bayes' theorem states the following relationship:
\begin{equation}
P(y \mid x_1, \dots, x_n) = \frac{P(y) P(x_1, \dots x_n \mid y)}
{P(x_1, \dots, x_n)}
\end{equation}
Using the naive independence assumption that
\begin{equation}
P(x_i | y, x_1, \dots, x_{i-1}, x_{i+1}, \dots, x_n) = P(x_i | y),
\end{equation}
for all i, this relationship is simplified to
\begin{equation}
P(y \mid x_1, \dots, x_n) = \frac{P(y) \prod_{i=1}^{n} P(x_i \mid y)}
{P(x_1, \dots, x_n)}
\end{equation}
Since $P(x_1, \dots, x_n)$ is constant given the input, we can use the following
classification rule:
\begin{equation}
P(y \mid x_1, \dots, x_n) \propto P(y) \prod_{i=1}^{n} P(x_i \mid y)
\Downarrow
\end{equation}
\begin{equation}
\hat{y} = \arg\max_y P(y) \prod_{i=1}^{n} P(x_i \mid y),
\end{equation}
and we can use Maximum A Posteriori (MAP) estimation to estimate $P(y)$ and
$P(x_i \mid y)$; the former is then the relative frequency of class $y$ in the
training set.
%
The different naive Bayes classifiers differ mainly by the assumptions they make
regarding the distribution of $P(x_i \mid y)$.

In spite of their apparently over-simplified assumptions, naive Bayes
classifiers have worked quite well in many real-world situations, famously
document classification and spam filtering.
%
They require a small amount of training data to estimate the necessary
parameters.
%
(For theoretical reasons why naive Bayes works well, and on which types of data
it does, see the references below.)
Naive Bayes learners and classifiers can be extremely fast compared to more
sophisticated methods.
%
The decoupling of the class conditional feature distributions means that each
distribution can be independently estimated as a one dimensional distribution.
%
This in turn helps to alleviate problems stemming from the curse of
dimensionality.

On the flip side, although naive Bayes is known as a decent classifier, it is
known to be a bad estimator, so the probability outputs from predict\_proba are
not to be taken too seriously.
%
In the following, all the Naive Bayes available in RAVEN are reported.
%
%%%%%%%
\subparagraph{Gaussian Naive Bayes}
\mbox{}
\\The \textit{Gaussian Naive Bayes strategy} implements the Gaussian Naive Bayes
algorithm for classification.
%
The likelihood of the features is assumed to be Gaussian:
\begin{equation}
P(x_i \mid y) = \frac{1}{\sqrt{2\pi\sigma^2_y}} \exp\left(-\frac{(x_i -
  \mu_y)^2}{2\sigma^2_y}\right)
\end{equation}
The parameters $\sigma_y$ and $\mu_y$ are estimated using maximum likelihood.

In order to use the \textit{Gaussian Naive Bayes strategy}, the user needs to
set the sub-node:

\xmlNode{SKLtype}\texttt{naiveBayes|GaussianNB}\xmlNode{/SKLtype}.

There are no additional sub-nodes available for this method.
%

\zNormalizationPerformed{GaussianNB}
%%%%%%%%%%%%
\subparagraph{Multinomial Naive Bayes}
\mbox{}
\\The \textit{Multinomial Naive Bayes} implements the naive Bayes algorithm for
multinomially distributed data, and is one of the two classic naive Bayes
variants used in text classification (where the data is typically represented
as word vector counts, although tf-idf vectors are also known to work well in
practice).
%
The distribution is parametrized by vectors $\theta_y =
(\theta_{y1},\ldots,\theta_{yn})$ for each class $y$, where $n$ is the number of
features (in text classification, the size of the vocabulary) and $\theta_{yi}$
is the probability $P(x_i \mid y)$ of feature $i$ appearing in a sample
belonging to class $y$.
%
The parameters $\theta_y$ are estimated by a smoothed version of maximum
likelihood, i.e. relative frequency counting:
\begin{equation}
\hat{\theta}_{yi} = \frac{ N_{yi} + \alpha}{N_y + \alpha n}
\end{equation}
where $N_{yi} = \sum_{x \in T} x_i$ is the number of times feature $i$ appears
in a sample of class y in the training set T, and
$N_{y} = \sum_{i=1}^{|T|} N_{yi}$ is the total count of all features for class
$y$.
%
The smoothing priors $\alpha \ge 0$ account for features not present in the
learning samples and prevents zero probabilities in further computations.
%
Setting $\alpha = 1$ is called Laplace smoothing, while $\alpha < 1$ is called
Lidstone smoothing.
%
\skltype{Multinomial Naive Bayes strategy}{naiveBayes|MultinomialNB}
\begin{itemize}
  \item \alphaBayesDescription{1.0}
  \item \fitPriorDescription{True}
  \item \classPriorDescription{None}
\end{itemize}
\zNormalizationNotPerformed{MultinomialNB}
%%%%%%%%%%%%
\subparagraph{Bernoulli Naive Bayes}
\mbox{}
\\The \textit{Bernoulli Naive Bayes} implements the naive Bayes training and
classification algorithms for data that is distributed according to multivariate
Bernoulli distributions; i.e., there may be multiple features but each one is
assumed to be a binary-valued (Bernoulli, boolean) variable.
%
Therefore, this class requires samples to be represented as binary-valued
feature vectors; if handed any other kind of data, a \textit{Bernoulli Naive
Bayes} instance may binarize its input (depending on the binarize parameter).
%
The decision rule for Bernoulli naive Bayes is based on
\begin{equation}
P(x_i \mid y) = P(i \mid y) x_i + (1 - P(i \mid y)) (1 - x_i)
\end{equation}
which differs from multinomial NB's rule in that it explicitly penalizes the
non-occurrence of a feature $i$ that is an indicator for class $y$, where the
multinomial variant would simply ignore a non-occurring feature.
%
In the case of text classification, word occurrence vectors (rather than word
count vectors) may be used to train and use this classifier.
%
\textit{Bernoulli Naive Bayes} might perform better on some datasets, especially
those with shorter documents.
%
It is advisable to evaluate both models, if time permits.
%
\skltype{Bernoulli Naive Bayes strategy}{naiveBayes|BernoulliNB}
\begin{itemize}
  \item \alphaBayesDescription{1.0}
  \item \xmlNode{binarize}, \xmlDesc{float, optional field},
  %
  Threshold for binarizing (mapping to booleans) of sample features.
  %
  If None, input is presumed to already consist of binary vectors.
  %
  \default{0.0}
  \item \fitPriorDescription{True}
  \item \classPriorDescription{None}
  %
\end{itemize}
\zNormalizationPerformed{BernoulliNB}
%%%%%%%%%%%%%%%%%%%%%%%%%%%%%%%%%%%%%%%%
 %%%%% ROM Model - SciKitLearn: Neighbors %%%%%%%
\paragraph{Neighbors}
\label{Neighbors}

The \textit{Neighbors} class provides functionality for unsupervised and
supervised neighbor-based learning methods.
%
The unsupervised nearest neighbors method is the foundation of many other
learning methods, notably manifold learning and spectral clustering.
%
Supervised neighbors-based learning comes in two flavors: classification for
data with discrete labels, and regression for data with continuous labels.

The principle behind nearest neighbor methods is to find a predefined number of
training samples closest in distance to the new point, and predict the label
from these.
%
The number of samples can be a user-defined constant (k-nearest neighbor
learning), or vary based on the local density of points (radius-based neighbor
learning).
%
The distance can, in general, be any metric measure: standard Euclidean distance
is the most common choice.
%
Neighbor-based methods are known as non-generalizing machine learning methods,
since they simply ``remember'' all of its training data (possibly transformed
into a fast indexing structure such as a Ball Tree or KD Tree.).

\zNormalizationPerformed{Neighbors-based}

In the following, all the Neighbors' models available in RAVEN are reported.
%
%%%%%%%%%%%%%%%
\subparagraph{K Neighbors Classifier}
\mbox{}
\\The \textit{K Neighbors Classifier} is a type of instance-based learning or
non-generalizing learning: it does not attempt to construct a general internal
model, but simply stores instances of the training data.
%
Classification is computed from a simple majority vote of the nearest neighbors
of each point: a query point is assigned the data class which has the most
representatives within the nearest neighbors of the point.
%
It implements learning based on the $k$ nearest neighbors of each query point,
where $k$ is an integer value specified by the user.

\skltype{K Neighbors Classifier}{neighbors|KNeighborsClassifier}
\begin{itemize}
  \item \nNeighborsDescription{5}
  \item \weightsDescription{uniform}
  \item \algorithmDescription{auto}
  \item \leafSizeDescription{30}
  \item \metricDescription{minkowski}
  \item \pDescription{2}
    %TODO document metric_params
    %TODO document n_jobs?
\end{itemize}
%%%%%%%%%%%%%%%
\subparagraph{Radius Neighbors Classifier}
\mbox{}
\\The \textit{Radius Neighbors Classifier} is a type of instance-based learning
or non-generalizing learning: it does not attempt to construct a general
internal model, but simply stores instances of the training data.
%
Classification is computed from a simple majority vote of the nearest neighbors
of each point: a query point is assigned the data class which has the most
representatives within the nearest neighbors of the point.
%
It implements learning based on the number of neighbors within a fixed radius
$r$ of each training point, where $r$ is a floating-point value specified by the
user.

\skltype{Radius Neighbors Classifier}{neighbors|RadiusNeighbors}
\begin{itemize}
  \item \radiusDescription{1.0}
  \item \weightsDescription{uniform}
  \item \algorithmDescription{auto}
  \item \leafSizeDescription{30}
  \item \metricDescription{minkowski}
  \item \pDescription{2}
  \item \outlierLabelDescription{None}
    %TODO document metric_params
\end{itemize}

%%%%%%%%%%%%%%%
\subparagraph{K Neighbors Regressor}
\mbox{}

The \textit{K Neighbors Regressor} can be used in cases where the data labels
are continuous rather than discrete variables.
%
The label assigned to a query point is computed based on the mean of the labels
of its nearest neighbors.
%
It implements learning based on the $k$ nearest neighbors of each query point,
where $k$ is an integer value specified by the user.

\skltype{K Neighbors Regressor}{neighbors|KNeighborsRegressor}
\begin{itemize}
  \item \nNeighborsDescription{5}
  \item \weightsDescription{uniform}
  \item \algorithmDescription{auto}
  \item \leafSizeDescription{30}
  \item \metricDescription{minkowski}
  \item \pDescription{2}
    %TODO document metric_params
    %TODO document n_jobs?
\end{itemize}

%%%%%%%%%%%%%%%
\subparagraph{Radius Neighbors Regressor}
\mbox{}

The \textit{Radius Neighbors Regressor} can be used in cases where the data
labels are continuous rather than discrete variables.
%
The label assigned to a query point is computed based on the mean of the labels
of its nearest neighbors.
%
It implements learning based on the neighbors within a fixed radius $r$ of the
query point, where $r$ is a floating-point value specified by the user.

\skltype{Radius Neighbors Regressor}{neighbors|RadiusNeighborsRegressor}
\begin{itemize}
  \item \radiusDescription{1.0}
  \item \weightsDescription{uniform}
  \item \algorithmDescription{auto}
  \item \leafSizeDescription{30}
  \item \metricDescription{minkowski}
  \item \pDescription{2}
    %TODO document metric_params
\end{itemize}
%%%%%%%%%%%%%%%
\subparagraph{Nearest Centroid Classifier}
\mbox{}

The \textit{Nearest Centroid classifier} is a simple algorithm that represents
each class by the centroid of its members.
%
It also has no parameters to choose, making it a good baseline classifier.
%
It does, however, suffer on non-convex classes, as well as when classes have
drastically different variances, as equal variance in all dimensions is assumed.

\skltype{Nearest Centroid Classifier}{neighbors|NearestCentroid}
\begin{itemize}
  \item \xmlNode{shrink\_threshold}, \xmlDesc{float, optional field}, defines
  the threshold for shrinking centroids to remove features.
  %
  \default{None}
  %
  %TODO document metric
\end{itemize}
%\subparagraph{Ball Tree}
%pass
%\subparagraph{K-D Tree}
%pass


The \textit{Quadratic Discriminant Analysis} is a classifier with a quadratic
decision boundary, generated by fitting class conditional densities to the data
and using Bayes' rule.
%
The model fits a Gaussian density to each class.

\skltype{Quadratic Discriminant Analysis Classifier}{qda|QDA}
\begin{itemize}
  \item \xmlNode{priors}, \xmlDesc{array-like (n\_classes), optional field},
  specifies the priors on the classes.
  %
  \default{None}
  \item \xmlNode{reg\_param}, \xmlDesc{float, optional field}, regularizes the
  covariance estimate as (1-reg\_param)*Sigma +
  reg\_param*Identity(n\_features).
  %
  \default{0.0}
  %
\end{itemize}
\zNormalizationNotPerformed{QDA}
 %%%%% ROM Model - SciKitLearn: Tree %%%%%%%
\paragraph{Tree}
\label{tree}

Decision Trees (DTs) are a non-parametric supervised learning method used for
classification and regression.
%
The goal is to create a model that predicts the value of a target variable by
learning simple decision rules inferred from the data features.
%
\begin{itemize}
  \item Some advantages of decision trees are:
  \item Simple to understand and to interpret.
  %
  Trees can be visualized.
  %
  \item Requires little data preparation.
  %
  Other techniques often require data normalization, dummy variables need to be
  created and blank values to be removed.
  %
  Note however, that this module does not support missing values.
  %
  \item The cost of using the tree (i.e., predicting data) is logarithmic in the
  number of data points used to train the tree.
  %
  \item Able to handle both numerical and categorical data.
  %
  Other techniques are usually specialized in analyzing datasets that have only
  one type of variable.
  %
  \item Able to handle multi-output problems.
  %
  \item Uses a white box model.
  %
  If a given situation is observable in a model, the explanation for the
  condition is easily explained by boolean logic.
  %
  By contrast, in a black box model (e.g., in an artificial neural network),
  results may be more difficult to interpret.
  %
  \item Possible to validate a model using statistical tests.
  %
  That makes it possible to account for the reliability of the model.
  %
  \item Performs well even if its assumptions are somewhat violated by the true
  model from which the data were generated.
  %
\end{itemize}
The disadvantages of decision trees include:
\begin{itemize}
  \item Decision-tree learners can create over-complex trees that do not
  generalise the data well.
  %
  This is called overfitting.
  %
  Mechanisms such as pruning (not currently supported), setting the minimum
  number of samples required at a leaf node or setting the maximum depth of the
  tree are necessary to avoid this problem.
  %
  \item Decision trees can be unstable because small variations in the data
  might result in a completely different tree being generated.
  %
  This problem is mitigated by using decision trees within an ensemble.
  %
  \item The problem of learning an optimal decision tree is known to be
  NP-complete under several aspects of optimality and even for simple concepts.
  %
  Consequently, practical decision-tree learning algorithms are based on
  heuristic algorithms such as the greedy algorithm where locally optimal
  decisions are made at each node.
  %
  Such algorithms cannot guarantee to return the globally optimal decision tree.
  %
  This can be mitigated by training multiple trees in an ensemble learner, where
  the features and samples are randomly sampled with replacement.
  %
  \item There are concepts that are hard to learn because decision trees do not
  express them easily, such as XOR, parity or multiplexer problems.
  %
  \item Decision tree learners create biased trees if some classes dominate.
  %
  It is therefore recommended to balance the dataset prior to fitting with the
  decision tree.
  %
\end{itemize}

\zNormalizationPerformed{tree-based}

In the following, all the tree-based algorithms available in RAVEN are reported.

%%%%%%%%%%%%%%%
\subparagraph{Decision Tree Classifier}
\mbox{}
\\The \textit{Decision Tree Classifier} is a classifier that is based on the
decision tree logic.

\skltype{Decision Tree Classifier}{tree|DecisionTreeClassifier}
\begin{itemize}
  \item \criterionDescription{gini}
  \item \splitterDescription{best}
  \item \maxFeaturesDescription{None}
  \item \maxDepthDescription{None}
  \item \minSamplesSplitDescription{2}
  \item \minSamplesLeafDescription{1}
    %TODO document min_weight_fraction_leaf
  \item \maxLeafNodesDescription{None}
    %TODO document class_weight
    %TODO document random_state
    %TODO document presort
\end{itemize}

%%%%%%%%%%%%%%%%
\subparagraph{Decision Tree Regressor}
\mbox{}
\\The \textit{Decision Tree Regressor} is a Regressor that is based on the
decision tree logic.
%
\skltype{Decision Tree Regressor}{tree|DecisionTreeRegressor}
\begin{itemize}
  \item \criterionDescriptionDT{mse}
  \item \splitterDescription{best}
  \item \maxFeaturesDescription{None}
  \item \maxDepthDescription{None}
  \item \minSamplesSplitDescription{2}
  \item \minSamplesLeafDescription{1}
    %TODO document min_weight_fraction_leaf
  \item \maxLeafNodesDescription{None}
    %TODO document random_state
    %TODO document presort
\end{itemize}

%%%%%%%%%%%%%%%%
\subparagraph{Extra Tree Classifier}
\mbox{}
\\The \textit{Extra Tree Classifier} is an extremely randomized tree classifier.
%
Extra-trees differ from classic decision trees in the way they are built.
%
When looking for the best split to separate the samples of a node into two
groups, random splits are drawn for each of the max\_features randomly selected
features and the best split among those is chosen.
%
When max\_features is set 1, this amounts to building a totally random decision
tree.

\skltype{Extra Tree Classifier}{tree|ExtraTreeClassifier}

\begin{itemize}
  \item \criterionDescription{gini}
  \item \splitterDescription{random}
  \item \maxFeaturesDescription{auto}
  \item \maxDepthDescription{None}
  \item \minSamplesSplitDescription{2}
  \item \minSamplesLeafDescription{1}
    %TODO document min_weight_fraction_leaf
  \item \maxLeafNodesDescription{None}
    %TODO document random_state
    %TODO document class_weight
  %
\end{itemize}

%%%%%%%%%%%%
\subparagraph{Extra Tree Regressor}
\mbox{}

The \textit{Extra Tree Regressor} is an extremely randomized tree regressor.
%
Extra-trees differ from classic decision trees in the way they are built.
%
When looking for the best split to separate the samples of a node into two
groups, random splits are drawn for each of the max\_features randomly selected
features and the best split among those is chosen.
%
When max\_features is set 1, this amounts to building a totally random decision
tree.

\skltype{Extra Tree Regressor}{tree|ExtraTreeRegressor}

\begin{itemize}
  \item \criterionDescriptionDT{mse}
  \item \splitterDescription{random}
  \item \maxFeaturesDescription{auto}
  \item \maxDepthDescription{None}
  \item \minSamplesSplitDescription{2}
  \item \minSamplesLeafDescription{1}
    %TODO document min_weight_fraction_leaf
  \item \maxLeafNodesDescription{None}
    %TODO document random_state
\end{itemize}

%%%%%%%%%%%%%%%%%%%%%%%%%%%%%%%%%%%%%%%%%%%
 %%%%% ROM Model - SciKitLearn: Gaussian Process %%%%%%%
\paragraph{Gaussian Process}
\label{GP}
Gaussian Processes for Machine Learning (GPML) is a generic supervised learning
method primarily designed to solve regression problems.
%
The advantages of Gaussian Processes for Machine Learning are:
\begin{itemize}
  \item The prediction interpolates the observations (at least for regular
  correlation models).
  \item The prediction is probabilistic (Gaussian) so that one can compute
  empirical confidence intervals and exceedance probabilities that might be used
  to refit (online fitting, adaptive fitting) the prediction in some region of
  interest.
  \item Versatile: different linear regression models and correlation models can
  be specified.
  %
  Common models are provided, but it is also possible to specify custom models
  provided they are stationary.
  %
\end{itemize}
The disadvantages of Gaussian Processes for Machine Learning include:
\begin{itemize}
  \item It is not sparse.
  %
  It uses the whole samples/features information to perform the prediction.
  \item It loses efficiency in high dimensional spaces – namely when the
  number of features exceeds a few dozens.
  %
  It might indeed give poor performance and it loses computational efficiency.
  \item Classification is only a post-processing, meaning that one first needs
  to solve a regression problem by providing the complete scalar float precision
  output $y$ of the experiment one is attempting to model.
  %
\end{itemize}

\skltype{Gaussian Process Regressor}{GaussianProcess|GaussianProcess}

\begin{itemize}
  \item \xmlNode{regr}, \xmlDesc{string, optional field}, is a regression
  function returning an array of outputs of the linear regression functional
  basis.
  %
  The number of observations n\_samples should be greater than the size p of
  this basis.
  %
  Available built-in regression models are `constant,' `linear,' and
  `quadratic.'
  %
  \default{constant}
  \item \xmlNode{corr}, \xmlDesc{string, optional field}, is a stationary
  autocorrelation function returning the autocorrelation between two points $x$
  and $x'$.
  %
  Default assumes a squared-exponential autocorrelation model.
  %
  Built-in correlation models are `absolute\_exponential,'
  `squared\_exponential,' `generalized\_exponential,' `cubic,' and `linear.'
  %
  \default{squared\_exponential}
  \item \xmlNode{beta0}, \xmlDesc{float, array-like, optional field}, specifies
  the regression weight vector to perform Ordinary Kriging (OK).
  %
  \default{None}
  \item \xmlNode{storage\_mode}, \xmlDesc{string, optional field}, specifies
  whether the Cholesky decomposition of the correlation matrix should be stored
  in the class (storage\_mode = `full') or not (storage\_mode = `light').
  %
  \default{full}
  \item \verDescriptionA{False}
  \item \xmlNode{theta0}, \xmlDesc{float, array-like, optional field}, is an
  array with shape (n\_features, ) or (1, ).
  %
  This represents the parameters in the autocorrelation model.
  %
  If thetaL and thetaU are also specified, theta0 is considered as the starting
  point for the maximum likelihood estimation of the best set of parameters.
  %
  \default{[1e-1]}
  \item \xmlNode{thetaL}, \xmlDesc{float, array-like, optional field}, is an
  array with shape matching that defined by \xmlNode{theta0}.
  %
  Lower bound on the autocorrelation parameters for maximum likelihood
  estimation.
  %
  \default{None}
  \item \xmlNode{thetaU}, \xmlDesc{float, array-like, optional field}, is an
  array with shape matching that defined by \xmlNode{theta0}.
  %
  Upper bound on the autocorrelation parameters for maximum likelihood
  estimation.
  %
  \default{None}
  \item \xmlNode{normalize}, \xmlDesc{boolean, optional field}, if True, the
  input $X$ and observations $y$ are centered and reduced w.r.t. means and
  standard deviations estimated from the n\_samples observations provided.
  %
  \default{True}
  \item \xmlNode{nugget}, \xmlDesc{float, optional field},specifies a nugget
  effect to allow smooth predictions from noisy data.
  %
  The nugget is added to the diagonal of the assumed training covariance.
  %
  In this way it acts as a Tikhonov regularization in the problem.
  %
  In the special case of the squared exponential correlation function, the
  nugget mathematically represents the variance of the input values.
  %
  \default{10 * MACHINE\_EPSILON}
  \item \xmlNode{optimizer}, \xmlDesc{string, optional field}, specifies the
  optimization algorithm to be used.
  %
  Available optimizers are: 'fmin\_cobyla', 'Welch'.
  %
  \default{fmin\_cobyla}
  \item \xmlNode{random\_start}, \xmlDesc{integer, optional field}, sets the
  number of times the Maximum Likelihood Estimation should be performed from
  a random starting point.
  %
  The first MLE always uses the specified starting point (theta0), the next
  starting points are picked at random according to an exponential distribution
  (log-uniform on [thetaL, thetaU]).
  %
  \default{1}
  \item \xmlNode{random\_state}, \xmlDesc{integer, optional field}, is the seed
  of the internal random number generator.
  %
  \default{None}
  %
\end{itemize}

\zNormalizationNotPerformed{GaussianProcess}

\textbf{Example:}
\begin{lstlisting}[style=XML,morekeywords={name,subType}]
<Simulation>
  ...
  <Models>
    ...
   <ROM name='aUserDefinedName' subType='SciKitLearn'>
     <Features>var1,var2,var3</Features>
     <Target>result1</Target>
     <SKLtype>linear_model|LinearRegression</SKLtype>
     <fit_intercept>True</fit_intercept>
     <normalize>False</normalize>
   </ROM>
    ...
  </Models>
  ...
</Simulation>
\end{lstlisting}


%%%%%%%%%%%%%%%%%%%%%%%%%%%%%%%%%%%%%%%%%%%
 %%%%% ROM Model - SciKitLearn: Neural Network Models %%%%%%%
\paragraph{Neural Network Models}
\label{DNN}
It has been more than 70 years since Warren McCulloch and Water Pitts modeled the first
artificial neural network (ANN) that mimicked the way brains work. These days, deep learning
based on ANN is showing outstanding results for solving a wide variety of robotic tasks in
the areas of perception, planning, localization, and control.
%
\textbf{Multi-layer Perceptron (MLP)} is a supervised learning algorithm that can learn
a non-linear function approximator for either classifcation or regression. It is different
from logistic regression, in that between the input and output layer, there can be one
or more non-linear layers, called hidden layers.
%
The advantages of Multi-layer Perceptron are:
\begin{itemize}
  \item Capability to learn non-linear models
  \item Capability to learn models in real-time (online learning)
\end{itemize}
The disadvantages of Multi-layer Perceptron include:
\begin{itemize}
  \item MLP with hidden layers have a non-convex loss function where there exists more than
    one local minimum. Therefore different random weight initializations can lead to different
    validation accuracy.
  \item MLP requires tuning a number of hyperparameters such as the number of hidden neurons, layers
    and iterations.
  \item MLP is sensitive to feature scaling
\end{itemize}

\zNormalizationPerformed{Multi-layer Perceptron}

In the following, Multi-layer perceptron classification and regression algorithms available in RAVEN are reported.

%%%%%%%%%%%%%%%
\subparagraph{MLPClassifier}
\mbox{}
\\The \textit{MLPClassifier} implements a multi-layer perceptron algorithm that trains using \textbf{Backpropagation}
More precisely, it trains using some form of gradient descent and the gradients are calculated using Backpropagation.
For classification, it minimizes the Cross-Entropy loss function, and it supports multi-class classification.

\skltype{MLPClassifier}{neural\_network|MLPClassifier}
\begin{itemize}
  \item \hiddenLayerSizesMLPDescription{(100,)}
  \item \activationMLPDescription{`relu'}
  \item \solverMLPDescription{`adam'}
  \item \alphaMLPDescription{0.0001}
  \item \batchSizeMLPDescription{`auto'}
  \item \learningRateMLPDescription{`constant'}
  \item \learningRateInitMLPDescription{0.001}
  \item \powerTMLPDescription{0.5}
  \item \maxIterMLPDescription{200}
  \item \shuffleMLPDescription{True}
  \item \randomStateMLPDescription{None}
  \item \tolMLPDescription{1e-4}
  \item \verboseMLPDescription{False}
  \item \warmStartMLPDescription{False}
  \item \momentumMLPDescription{0.9}
  \item \nesterovsMomentumMLPDescription{True}
  \item \earlyStoppingMLPDescription{False}
  \item \validationFractionMLPDescription{0.1}
  \item \betaAMLPDescription{0.9}
  \item \betaBMLPDescription{0.999}
  \item \epsilonMLPDescription{1e-8}
\end{itemize}

%%%%%%%%%%%%%%%
\subparagraph{MLPRegressor}
\mbox{}
\\The \textit{MLPRegressor} implements a multi-layer perceptron algorithm that trains using \textbf{Backpropagation} with
no activation function in the output layer, which can also be seen as using the identity function as activation function.
Therefore, it uses the square error as the loss function, and the output is a set of continuous values.
\textit{MLPRegressor} also supports multi-output regression, in which a sample can have more than one target.

\skltype{MLPRegressor}{neural\_network|MLPRegressor}
\begin{itemize}
  \item \hiddenLayerSizesMLPDescription{(100,)}
  \item \activationMLPDescription{`relu'}
  \item \solverMLPDescription{`adam'}
  \item \alphaMLPDescription{0.0001}
  \item \batchSizeMLPDescription{`auto'}
  \item \learningRateMLPDescription{`constant'}
  \item \learningRateInitMLPDescription{0.001}
  \item \powerTMLPDescription{0.5}
  \item \maxIterMLPDescription{200}
  \item \shuffleMLPDescription{True}
  \item \randomStateMLPDescription{None}
  \item \tolMLPDescription{1e-4}
  \item \verboseMLPDescription{False}
  \item \warmStartMLPDescription{False}
  \item \momentumMLPDescription{0.9}
  \item \nesterovsMomentumMLPDescription{True}
  \item \earlyStoppingMLPDescription{False}
  \item \validationFractionMLPDescription{0.1}
  \item \betaAMLPDescription{0.9}
  \item \betaBMLPDescription{0.999}
  \item \epsilonMLPDescription{1e-8}
\end{itemize}

%%%% ROM Model - ARMA  %%%%%%%
\subsubsection{ARMA}
\label{subsubsec:arma}
The ARMA sub-type contains a single ROM type, based on an autoregressive moving average time series model with
Fourier signal processing, sometimes referred to as a FARMA.
%
ARMA is a type of time dependent model that characterizes the autocorrelation between time series data. The mathematic description of ARMA is given as
\begin{equation*}
x_t = \sum_{i=1}^p\phi_ix_{t-i}+\alpha_t+\sum_{j=1}^q\theta_j\alpha_{t-j},
\end{equation*}
where $x$ is a vector of dimension $n$, and $\phi_i$ and $\theta_j$ are both $n$ by $n$ matrices. When $q=0$, the above is
autoregressive (AR); when $p=0$, the above is moving average (MA).
While plans for optimizing $p$ and $q$ are under consideration, currently setting $Pmin$ and $Pmax$ to the
same value is required, and similarly for $Qmin$ and $Qmax$.
%The user is allowed to provide upper and lower limits for $p$
%and $q$ (see below), and the training process will choose the optimal $p$ and $q$ that fall into the user-specified range.
When
training an ARMA, the input needs to be a synchronized HistorySet. For unsynchronized data, use PostProcessor methods to
synchronize the data before training an ARMA.

The ARMA model implemented allows an option to use Fourier series to detrend the time series before fitting to ARMA model to
train. The Fourier trend will be stored in the trained ARMA model for data generation. The following equation
describes the detrending
process.
\begin{equation*}
x_t = y_t - \sum_m\left\{\sum_{k=1}^{K_m}a_k\sin(2\pi kf_mt)+\sum_{k=1}^{K_m}b_k\cos(2\pi kf_mt)\right\},
\end{equation*}
where $K_m$ and $f_m$ are user-defined parameters.

By default, each target in the training will be considered independent and have an unique ARMA for each
target.  Correlated targets can be specified through the \xmlNode{correlate} node, at which point
the correlated targets will be trained together using a vector ARMA (or VARMA). Due to limitations in
the VARMA, in order to seed samples the VARMA must be trained with the node \xmlNode{seed}, which acts
independently from the global random seed used by other RAVEN entities.

Both the ARMA and VARMA make use of the \texttt{statsmodels} python package.

%
In order to use this Reduced Order Model, the \xmlNode{ROM} attribute
\xmlAttr{subType} needs to be \xmlString{ARMA} (see the example
below).
%
\subnodeIntro

\begin{itemize}
  \item \xmlNode{pivotParameter}, \xmlDesc{string, required field}, defines the pivot variable (e.g., time) that is non-decreasing in
  the input HistorySet.
  \item \xmlNode{Features}, \xmlDesc{comma separated string, required field}, defines the features (e.g., scaling). Note that only
  one feature is allowed for \xmlString{ARMA} and in current implementation this is used for evaluation only.
  \item \xmlNode{Target}, \xmlDesc{comma separated string, required field}, defines the variable(s) of the
    time series.  Should include the pivot parameter (or Index).
  \item \xmlNode{correlate}, \xmlDesc{comma separated string, optional field}, indicates the listed variables
    should be considered as influencing each other, and trained together instead of independently.  This node
    can only be listed once, so all variables that are desired for correlation should be included.  \nb The
    correlated VARMA takes notably longer to train than the independent ARMAs for the same number of targets.
  \item \xmlNode{seed}, \xmlDesc{integer, optional field}, provides seed for ONLY the VARMA sampling.  Has no
    effect on ARMA single-target sampling or other RAVEN entities.  Must be provided before training; it cannot
    be changed once trained.
  \default{True}
  \item \xmlNode{reseedCopies}, \xmlDesc{boolean, optional field}, if True then whenever the ARMA is copied, a
    random reseeding will be performed to ensure different histories.
  \default{True}
  \item \xmlNode{Pmax}, \xmlDesc{integer, optional field}, defines the maximum value of $p$.
  \default{3}
  \item \xmlNode{Pmin}, \xmlDesc{integer, optional field}, defines the minimum value of $p$.
  \default{0}
  \item \xmlNode{Qmax}, \xmlDesc{integer, optional field}, defines the maximum value of $q$.
  \default{3}
  \item \xmlNode{Qmin}, \xmlDesc{integer, optional field}, defines the minimum value of $q$.
  \default{0}
  \item \xmlNode{Fourier}, \xmlDesc{integers, optional field}, must be positive integers. This defines the based period (with unit of second) that would be used for Fourier detrending, i.e., this field defines $1/f_m$ in the above equation. When this filed is not specified, the ARMA considers no Fourier detrend.
  \item \xmlNode{FourierOrder}, \xmlDesc{integers, optional field}, must be positive integers. The number of integers specified in this field should be exactly same as the number of base periods specified in the node \xmlNode{Fourier}. This field defines $K_m$ in the above equation.
  \item \xmlNode{outTruncation}, \xmlDesc{string, optional field}, defines whether and how the output time series is truncated. Currently available options are: positive, negative.
  \default{None}
\end{itemize}

\textbf{Example:}
\begin{lstlisting}[style=XML,morekeywords={name,subType}]
<Simulation>
  ...
  <Models>
    ...
    <ROM name='aUserDefinedName' subType='ARMA'>
      <pivotParameter>Time</pivotParameter>
      <Features>scaling</Features>
      <Target>Speed1,Speed2</Target>
      <Pmax>5</Pmax>
      <Pmin>1</Pmin>
      <Qmax>4</Qmax>
      <Qmin>1</Qmin>
      <Fourier>604800,86400</Fourier>
      <FourierOrder>2, 4</FourierOrder>
     </ROM>
    ...
  </Models>
  ...
</Simulation>
\end{lstlisting}

%%%% ROM Model - PolyExponential  %%%%%%%
\subsubsection{PolyExponential}
\label{subsubsec:polyexponential}
The PolyExponential sub-type contains a single ROM type, aimed to construct a time-dependent (or any other monotonic
variable) surrogate model based on polynomial sum of exponential term. This surrogate have the form:
%
\begin{equation}
  SM(X,z) = \sum_{i=1}^{N} P_{i}(X) \times \exp ( - Q_{i}(X) \times z )
\end{equation}
where:
\begin{itemize}
  \item $\mathbf{z}$ is the independent  monotonic variable (e.g. time)
  \item $\mathbf{X}$  is the vector of the other independent (parametric) variables  (Features)
  \item $\mathbf{P_{i}}(X)$ is a polynomial of rank M function of the parametric space X
  \item  $\mathbf{Q_{i}}(X)$ is a polynomial of rank M function of the parametric space X
  \item  $\mathbf{N}$ is the number of requested exponential terms.
\end{itemize}
It is crucial to notice that this model is quite suitable for FOMs whose drivers are characterized by an exponential-like behavior.
In addition, it is important to notice that the exponential terms' coefficients are computed running a genetic-algorithm optimization
problem, which is quite slow in case of increasing number of ``numberExpTerms''.
%
In order to use this Reduced Order Model, the \xmlNode{ROM} attribute
\xmlAttr{subType} needs to be set equal to \xmlString{PolyExponential} (see the example
below).
%
\subnodeIntro

\begin{itemize}
  \item \xmlNode{pivotParameter}, \xmlDesc{string, optional field}, defines the pivot variable (e.g., time) that represents the
  independent monotonic variable
  \default{time}
  \item \xmlNode{Features}, \xmlDesc{comma separated string, required field}, defines the features (i.e. input parameters) of this
  model
  \item \xmlNode{Target}, \xmlDesc{comma separated string, required field}, defines output FOMs that are going to be predicted
  \item \xmlNode{numberExpTerms}, \xmlDesc{integer, optional field}, the number of exponential terms to be used ($N$ above)
   \default{3}
  \item \xmlNode{coeffRegressor}, \xmlDesc{string, optional field}, defines which regressor to use for interpolating the
   exponential coefficient. Available are ``spline'',``poly'' and ``nearest''.
    \default{spline}
  \item \xmlNode{polyOrder}, \xmlDesc{integer, optional field}, the polynomial order to be used for interpolating the exponential
  coefficients. Only valid in case of  \xmlNode{coeffRegressor} set to ``poly''.
   \default{2}
  \item \xmlNode{tol}, \xmlDesc{float, optional field}, relative tolerance of the optimization problem (differential evolution optimizer)
   \default{1e-3}
  \item \xmlNode{maxNumberIter}, \xmlDesc{integer, optional field}, maximum number of iterations (generations) for the
  optimization problem  (differential evolution optimizer)
   \default{5000}
\end{itemize}

\textbf{Example:}
\begin{lstlisting}[style=XML,morekeywords={name,subType}]
<Simulation>
  ...
  <Models>
    ...
   <ROM name='PolyExp' subType='PolyExponential'>
     <Target>time,decay_heat, xe135_dens</Target>
     <Features>enrichment,bu</Features>
     <pivotParameter>time</pivotParameter>
     <numberExpTerms>5</numberExpTerms>
     <max_iter>1000000</max_iter>
     <tol>0.000001</tol>
  </ROM>
    ...
  </Models>
  ...
</Simulation>
\end{lstlisting}
Once the ROM is trained (\textbf{Step} \xmlNode{RomTrainer}), its coefficients can be exported into an XML file
via an \xmlNode{OutStream} of type \xmlAttr{Print}. The following variable/parameters can be exported (i.e. \xmlNode{what} node
in \xmlNode{OutStream} of type \xmlAttr{Print}):
\begin{itemize}
  \item \xmlNode{expTerms}, see XML input specifications above, inquired pre-pending the keyword ``output|'' (e.g. output| expTerms)
  \item \xmlNode{coeffRegressor}, see XML input specifications above
  \item \xmlNode{polyOrder}, see XML input specifications above
  \item \xmlNode{features}, see XML input specifications above
  \item \xmlNode{timeScale}, XML node containing the array of the training time steps values
  \item \xmlNode{coefficients}, XML node containing the exponential terms' coefficients for each realization
\end{itemize}


 See the following example:
\begin{lstlisting}[style=XML,morekeywords={name,subType}]
<Simulation>
  ...
  <OutStreams>
    ...
    <Print name = 'dumpAllCoefficients'>
      <type>xml</type>
      <source>PolyExp</source>
      <!--
        here the <what> node is omitted. All the available params/coefficients
        are going to be printed out
      -->
    </Print>
    <Print name = 'dumpSomeCoefficients'>
      <type>xml</type>
      <source>PolyExp</source>
      <what>coefficients,timeScale</what>
    </Print>
    ...
  </OutStreams>
  ...
</Simulation>
\end{lstlisting}

%%%% ROM Model - DMD  %%%%%%%
\subsubsection{DMD}
\label{subsubsec:polyexponential}
The DMD sub-type contains a single ROM type, aimed to construct a time-dependent (or any other monotonic
variable) surrogate model based on Dynamic Mode Decomposition (ref. \cite{Schmid2010DMD} and \cite{Vega2017HODMD}).
This surrogate is aimed to perform a ``dimensionality reduction regression'', where, given time series (or any monotonic-dependent
variable) of data, a set of modes each of which is associated with a fixed oscillation frequency and decay/growth rate is computed
in order to represent the data-set.
%
In order to use this Reduced Order Model, the \xmlNode{ROM} attribute
\xmlAttr{subType} needs to be set equal to \xmlString{DMD} (see the example
below).
%
\subnodeIntro

\begin{itemize}
  \item \xmlNode{dmdType}, \xmlDesc{string, optional field}, the type of Dynamic Mode Decomposition to apply. Available are:
   \begin{itemize}
     \item \textit{dmd}, for classical DMD
     \item \textit{hodmd}, for high order DMD.
   \end{itemize}
   \default{dmd}
  \item \xmlNode{pivotParameter}, \xmlDesc{string, optional field}, defines the pivot variable (e.g., time) that represents the
  independent monotonic variable
  \default{time}
  \item \xmlNode{Features}, \xmlDesc{comma separated string, required field}, defines the features (i.e. input parameters) of this
  model
  \item \xmlNode{Target}, \xmlDesc{comma separated string, required field}, defines output FOMs that are going to be predicted
  \item \xmlNode{rankSVD}, \xmlDesc{integer, optional field}, defines the truncation rank to be used for the SVD.
     Available options are:
     \begin{itemize}
     \item \textit{-1}, no truncation is performed
     \item \textit{0}, optimal rank is internally computed
     \item \textit{>1}, this rank is going to be used for the truncation
   \end{itemize}
   \default{-1}
  \item \xmlNode{energyRankSVD}, \xmlDesc{float, optional field},  energy level ($0.0 < float < 1.0$) used to compute the rank such
    as computed rank is the number of the biggest singular values needed to reach the energy identified by
    \xmlNode{energyRankSVD}. This node has always priority over  \xmlNode{rankSVD}
    \default{None}
  \item \xmlNode{rankTLSQ}, \xmlDesc{integer, optional field}, $int > 0$ that defines the truncation rank to be used for the total
   least square problem. If not inputted, no truncation is applied
   \default{None}
   \item \xmlNode{exactModes}, \xmlDesc{bool, optional field}, True if the exact modes need to be computed (eigenvalues and
   eigenvectors),   otherwise the projected ones (using the left-singular matrix after SVD).
  \default{True}
  \item \xmlNode{optimized}, \xmlDesc{float, optional field}, True if the amplitudes need to be computed minimizing the error
   between the modes and all the time-steps or False, if only the 1st timestep only needs to be considered
   \default{True}

\end{itemize}

\textbf{Example:}
\begin{lstlisting}[style=XML,morekeywords={name,subType}]
<Simulation>
  ...
  <Models>
    ...
   <ROM name='DMD' subType='DMD'>
      <Target>time,totals_watts, xe135_dens</Target>
      <Features>enrichment,bu</Features>
      <dmdType>dmd</dmdType>
      <pivotParameter>time</pivotParameter>
      <rankSVD>0</rankSVD>
      <rankTLSQ>5</rankTLSQ>
      <exactModes>False</exactModes>
      <optimized>True</optimized>
    </ROM
    ...
  </Models>
  ...
</Simulation>
\end{lstlisting}
Once the ROM  is trained (\textbf{Step} \xmlNode{RomTrainer}), its parameters/coefficients can be exported into an XML file
via an \xmlNode{OutStream} of type \xmlAttr{Print}. The following variable/parameters can be exported (i.e. \xmlNode{what} node
in \xmlNode{OutStream} of type \xmlAttr{Print}):
\begin{itemize}
  \item \xmlNode{rankSVD}, see XML input specifications above
  \item \xmlNode{energyRankSVD}, see XML input specifications above
  \item \xmlNode{rankTLSQ}, see XML input specifications above
  \item \xmlNode{exactModes}, see XML input specifications above
  \item \xmlNode{optimized}, see XML input specifications above
  \item \xmlNode{features}, see XML input specifications above
  \item \xmlNode{timeScale}, XML node containing the array of the training time steps values
  \item \xmlNode{dmdTimeScale}, XML node containing the array of time scale in the DMD space (can be used as mapping
  between the  \xmlNode{timeScale} and \xmlNode{dmdTimeScale})
  \item \xmlNode{eigs}, XML node containing the eigenvalues (imaginary and real part)
  \item \xmlNode{amplitudes}, XML node containing the amplitudes (imaginary and real part)
  \item \xmlNode{modes}, XML node containing the dynamic modes (imaginary and real part)
\end{itemize}


 See the following example:
\begin{lstlisting}[style=XML,morekeywords={name,subType}]
<Simulation>
  ...
  <OutStreams>
    ...
    <Print name = 'dumpAllCoefficients'>
      <type>xml</type>
      <source>DMD</source>
      <!--
        here the <what> node is omitted. All the available params/coefficients
        are going to be printed out
      -->
    </Print>
    <Print name = 'dumpSomeCoefficients'>
      <type>xml</type>
      <source>PolyExp</source>
      <what>eigs,amplitudes,modes</what>
    </Print>
    ...
  </OutStreams>
  ...
</Simulation>
\end{lstlisting}



%%%%%%%%%%%%%%%%%%%%%%%%
%%%%%%  External  Model   %%%%%%
%%%%%%%%%%%%%%%%%%%%%%%%
\subsection{External Model}
\label{subsec:models_externalModel}
As the name suggests, an external model is an entity that is embedded in the
RAVEN code at run time.
%
This object allows the user to create a python module that is going to be
treated as a predefined internal model object.
%
In other words, the \textbf{External Model} is going to be treated by RAVEN as a
normal external Code (e.g. it is going to be called in order to compute an
arbitrary quantity based on arbitrary input).
%

The specifications of an External Model must be defined within the XML block
\xmlNode{ExternalModel}.
%
This XML node needs to contain the attributes:

\vspace{-5mm}
\begin{itemize}
  \itemsep0em
  \item \xmlAttr{name}, \xmlDesc{required string attribute}, user-defined name
  of this External Model.
  %
  \nb As with the other objects, this is the name that can be used to refer to
  this specific entity from other input blocks in the XML.
  \item \xmlAttr{subType}, \xmlDesc{required string attribute}, must be kept
  empty.
  \item \xmlAttr{ModuleToLoad}, \xmlDesc{required string attribute}, file name
  with its absolute or relative path.
  %
  \nb If a relative path is specified, the code first checks relative
  to the working directory, then it checks with respect to where the
  user runs the code.  Using the relative path with respect to where the
  code is run is not recommended.
  %
\end{itemize}
\vspace{-5mm}

In order to make the RAVEN code aware of the variables the user is going to
manipulate/use in her/his own python Module, the variables need to be specified
in the \xmlNode{ExternalModel} input block.
%
The user needs to input, within this block, only the variables that RAVEN needs
to be aware of (i.e. the variables are going to directly be used by the code)
and not the local variables that the user does not want to, for example, store
in a RAVEN internal object.
%
These variables are specified within a \xmlNode{variables} block:
\begin{itemize}
  \item \xmlNode{variables}, \xmlDesc{string, required parameter}.
  %
  Comma-separated list of variable names.
  %
  Each variable name needs to match a variable used/defined in the external python
  model.
  %
\end{itemize}

In addition, if the user wants to use the alias system, the following XML block can be inputted:
\begin{itemize}
  \item \aliasSystemDescription{ExternalModel}
\end{itemize}


When the external function variables are defined, at run time, RAVEN initializes
them and tracks their values during the simulation.
%
Each variable defined in the \xmlNode{ExternalModel} block is available in the
module (each method implemented) as a python ``self.''
%

In the External Python module, the user can implement all the methods that are
needed for the functionality of the model, but only the following methods, if
present, are called by the framework:
\begin{itemize}
  \item \texttt{\textbf{def \_readMoreXML}}, \xmlDesc{OPTIONAL METHOD}, can be
  implemented by the user if the XML input that belongs to this External Model
  needs to be extended to contain other information.
  %
  The information read needs to be stored in ``self'' in order to be available
  to all the other methods (e.g. if the user needs to add a couple of newer XML
  nodes with information needed by the algorithm implemented in the ``run''
  method).
  \item \texttt{\textbf{def initialize}}, \xmlDesc{OPTIONAL METHOD}, can
  implement all the actions need to be performed at the initialization stage.
  \item \texttt{\textbf{def createNewInput}}, \xmlDesc{OPTIONAL METHOD}, creates
  a new input with the information coming from the RAVEN framework.
  %
  In this function the user can retrieve the information coming from the RAVEN
  framework, during the employment of a calculation flow, and use them to
  construct a new input that is going to be transferred to the ``run'' method.
  \item \texttt{\textbf{def run}}, \xmlDesc{REQUIRED METHOD}, is the actual
  location where the user needs to implement the model action (e.g. resolution
  of a set of equations, etc.).
  %
  This function is going to receive the Input (or Inputs) generated either by
  the External Model ``createNewInput'' method or the internal RAVEN one.
\end{itemize}

In the following sub-sections, all the methods are going to be analyzed in
detail.

\subsubsection{Method: \texttt{def \_readMoreXML}}
\label{subsubsec:externalReadMoreXML}
As already mentioned, the \textbf{readMoreXML} method can be implemented by the
user if the XML input that belongs to this External Model needs to be extended
to contain other information.
%
The information read needs to be stored in ``self'' in order to be available to
all the other methods (e.g. if the user needs to add a couple of newer XML nodes
with information needed by the algorithm implemented in the ``run'' method).
%
If this method is implemented in the \textbf{External Model}, RAVEN is going to
call it when the node \xmlNode{ExternalModel} is found parsing the XML input
file.
%
The method receives from RAVEN an attribute of type ``xml.etree.ElementTree'',
containing all the sub-nodes and attribute of the XML block \xmlNode{ExternalModel}.
%

Example XML:
\begin{lstlisting}[style=XML,morekeywords={subType,ModuleToLoad}]
<Simulation>
  ...
  <Models>
     ...
    <ExternalModel name='AnExtModule' subType='' ModuleToLoad='path_to_external_module'>
       <variables>sigma,rho,outcome</variables>
       <!--
          here we define other XML nodes RAVEN does not read automatically.
          We need to implement, in the external module 'AnExtModule' the readMoreXML method
        -->
        <newNodeWeNeedToRead>
            whatNeedsToBeRead
        </newNodeWeNeedToRead>
    </ExternalModel>
     ...
  </Models>
  ...
</Simulation>
\end{lstlisting}

Corresponding Python function:
\begin{lstlisting}[language=python]
def _readMoreXML(self,xmlNode):
  # the xmlNode is passed in by RAVEN framework
  # <newNodeWeNeedToRead> is unknown (in the RAVEN framework)
  # we have to read it on our own
  # get the node
  ourNode = xmlNode.find('newNodeWeNeedToRead')
  # get the information in the node
  self.ourNewVariable = ourNode.text
  # end function
\end{lstlisting}


\subsubsection{Method: \texttt{def initialize}}
\label{subsubsec:externalInitialize}
The \textbf{initialize} method can be implemented in the \textbf{External Model}
in order to initialize some variables needed by it.
%
For example, it can be used to compute a quantity needed by the ``run'' method
before performing the actual calculation).
%
If this method is implemented in the \textbf{External Model}, RAVEN is going to
call it at the initialization stage of each ``Step'' (see section
\ref{sec:steps}.
%
RAVEN will communicate, thorough a set of method attributes, all the information
that are generally needed to perform a initialization:
\begin{itemize}
  \item runInfo, a dictionary containing information regarding how the
  calculation is set up (e.g. number of processors, etc.).
  %
  It contains the following attributes:
  \begin{itemize}
    \item \texttt{DefaultInputFile} -- default input file to use
    \item \texttt{SimulationFiles} -- the xml input file
    \item \texttt{ScriptDir} -- the location of the pbs script interfaces
    \item \texttt{FrameworkDir} -- the directory where the framework is located
    \item \texttt{WorkingDir} -- the directory where the framework should be
    running
    \item \texttt{TempWorkingDir} -- the temporary directory where a simulation
    step is run
    \item \texttt{NumMPI} -- the number of mpi process by run
    \item \texttt{NumThreads} -- number of threads by run
    \item \texttt{numProcByRun} -- total number of core used by one run (number
    of threads by number of mpi)
    \item \texttt{batchSize} -- number of contemporaneous runs
    \item \texttt{ParallelCommand} -- the command that should be used to submit
    jobs in parallel (mpi)
    \item \texttt{numNode} -- number of nodes
    \item \texttt{procByNode} -- number of processors by node
    \item \texttt{totalNumCoresUsed} -- total number of cores used by driver
    \item \texttt{queueingSoftware} -- queueing software name
    \item \texttt{stepName} -- the name of the step currently running
    \item \texttt{precommand} -- added to the front of the command that is run
    \item \texttt{postcommand} -- added after the command that is run
    \item \texttt{delSucLogFiles} -- if a simulation (code run) has not failed,
    delete the relative log file (if True)
    \item \texttt{deleteOutExtension} -- if a simulation (code run) has not
    failed, delete the relative output files with the listed extension (comma
    separated list, for example: `e,r,txt')
    \item \texttt{mode} -- running mode, curently the only mode supported is
      mpi (but custom modes can be created)
    \item \textit{expectedTime} -- how long the complete input is expected to
    run
    \item \textit{logfileBuffer} -- logfile buffer size in bytes
  \end{itemize}
  \item inputs, a list of all the inputs that have been specified in the
  ``Step'' using this model.
  %
\end{itemize}
In the following an example is reported:
\begin{lstlisting}[language=python]
def initialize(self,runInfo,inputs):
 # Let's suppose we just need to initialize some variables
  self.sigma = 10.0
  self.rho   = 28.0
  # end function
\end{lstlisting}


\subsubsection{Method: \texttt{def createNewInput}}
\label{subsubsec:externalcreateNewInput}

The \textbf{createNewInput} method can be implemented by the user to create a
new input with the information coming from the RAVEN framework.
%
In this function, the user can retrieve the information coming from the RAVEN
framework, during the employment of a calculation flow, and use them to
construct a new input that is going to be transferred to the ``run'' method.
%
The new input created needs to be returned to RAVEN (i.e. ``return NewInput'').
\\This method expects that the new input is returned in a Python ``dictionary''.
%
RAVEN communicates, thorough a set of method attributes, all the information
that are generally needed to create a new input:
%myInput,samplerType,**Kwargs
\begin{itemize}
  \item \texttt{inputs}, \xmlDesc{python list}, a list of all the inputs that
  have been defined in the ``Step'' using this model.
  \item \texttt{samplerType}, \xmlDesc{string}, the type of Sampler, if a
  sampling strategy is employed; will be None otherwise.
  \item \texttt{Kwargs}, \xmlDesc{dictionary}, a dictionary containing several
  pieces of information (that can change based on the ``Step'' type).
  %
  If a sampling strategy is employed, this dictionary contains another
  dictionary identified by the keyword ``SampledVars'', in which the variables
  perturbed by the sampler are reported.
\end{itemize}
\nb If the ``Step'' that is using this Model has as input(s) an object of main
class type ``DataObjects'' (see Section~\ref{sec:DataObjects}), the internal ``createNewInput''
method is going to convert it in a dictionary of values.
%

Here we present an example:
\begin{lstlisting}[language=python]
def createNewInput(self,inputs,samplerType,**Kwargs):
  # in here the actual createNewInput of the
  # model is implemented
  if samplerType == 'MonteCarlo':
    avariable = inputs['something']*inputs['something2']
  else:
    avariable = inputs['something']/inputs['something2']
  return avariable*Kwargs['SampledVars']['aSampledVar']
\end{lstlisting}

\subsubsection{Method: \texttt{def run}}
\label{subsubsec:externalRun}
As stated previously, the only method that \emph{must} be present in an
External Module is the \textbf{run} function.
%
In this function, the user needs to implement the algorithm that RAVEN will
execute.
%
The \texttt{\textbf{run}} method is generally called after having inquired the
``createNewInput'' method (either the internal or the user-implemented one).
%
The only attribute this method is going to receive is a Python list of inputs
(the inputs coming from the \texttt{createNewInput} method).
%
If the user wants RAVEN to collect the results of this method, the outcomes of
interest need to be stored in ``self.''
%
\nb RAVEN is trying to collect the values of the variables listed only in the
\xmlNode{ExternalModel} XML block.
%

In the following an example is reported:
\begin{lstlisting}[language=python]
def run(self,Input):
  # in here the actual run of the
  # model is implemented
  input = Input[0]
  self.outcome = self.sigma*self.rho*input[``whatEver'']
\end{lstlisting}

%\subsection{Projector}
%\label{sec:models_projector}
%
%Description

%Summary

%Example

%%%%%%%%%%%%%%%%%%%%%%%%%%%%%%%%%%
%%%%%%         PostProcessor         %%%%%%%%%
%%%%%%%%%%%%%%%%%%%%%%%%%%%%%%%%%%
\subsection{PostProcessor}
\label{sec:models_postProcessor}
A Post-Processor (PP) can be considered as an action performed on a set of data
or other type of objects.
%
Most of the post-processors contained in RAVEN, employ a mathematical operation
on the data given as ``input''.
%
RAVEN supports several different types of PPs.

Currently, the following types are available in RAVEN:
\begin{itemize}
  \itemsep0em
  \item \textbf{BasicStatistics}
  \item \textbf{ComparisonStatistics}
  \item \textbf{ImportanceRank}
  \item \textbf{SafestPoint}
  \item \textbf{LimitSurface}
  \item \textbf{LimitSurfaceIntegral}
  \item \textbf{External}
  \item \textbf{TopologicalDecomposition}
  \item \textbf{RavenOutput}
  \item \textbf{DataMining}
  \item \textbf{Metric}
  \item \textbf{CrossValidation}
  \item \textbf{DataClassifier}
  %\item \textbf{PrintCSV}
  %\item \textbf{LoadCsvIntoInternalObject}
\end{itemize}

The specifications of these types must be defined within the XML block
\xmlNode{PostProcessor}.
%
This XML node needs to contain the attributes:
\vspace{-5mm}
\begin{itemize}
  \itemsep0em
  \item \xmlAttr{name}, \xmlDesc{required string attribute}, user-defined
  identifier of this post-processor.
  %
  \nb As with other objects, this is the name that can be used to refer to this
  specific entity from other input XML blocks.
  \item \xmlAttr{subType}, \xmlDesc{required string attribute}, defines which of
  the post-processors needs to be used, choosing among the previously reported
  types.
  %
  This choice conditions the subsequent required and/or optional
  \xmlNode{PostProcessor} sub nodes.
  %
\end{itemize}
\vspace{-5mm}

As already mentioned, all the types and meaning of the remaining sub-nodes
depend on the post-processor type specified in the attribute \xmlAttr{subType}.
%
In the following sections the specifications of each type are reported.

%%%%% PP BasicStatistics %%%%%%%
\subsubsection{BasicStatistics}
\label{BasicStatistics}
The \textbf{BasicStatistics} post-processor is the container of the algorithms
to compute many of the most important statistical quantities. It is important to notice that this
post-processor can accept as input both \textit{\textbf{PointSet}} and \textit{\textbf{HistorySet}}
data objects, depending on the type of statistics the user wants to compute:
\begin{itemize}
  \item \textit{\textbf{PointSet}}: Static Statistics;
  \item \textit{\textbf{HistorySet}}: Dynamic Statistics. Depending on a ``pivot parameter'' (e.g. time)
  the post-processor is going to compute the statistics for each value of it (e.g. for each time step).
  In case an \textbf{HistorySet} is provided as Input, the Histories needs to be synchronized (use
    \textit{\textbf{Interfaced}} post-processor of type  \textbf{HistorySetSync}).
\end{itemize}
%
\ppType{BasicStatistics post-processor}{BasicStatistics}
\begin{itemize}
  \item \xmlNode{"metric"}, \xmlDesc{comma separated string or node list, required field},
    specifications for the metric to be calculated.  The name of each node is the requested metric.  There are
    two forms for specifying the requested parameters of the metric.  For scalar values such as
    \xmlNode{expectedValue} and \xmlNode{variance}, the text of the node is a comma-separated list of the
    parameters for which the metric should be calculated.  For matrix values such as \xmlNode{sensitivty} and
    \xmlNode{covariance}, the matrix node requires two sub-nodes, \xmlNode{targets} and \xmlNode{features},
    each of which is a comma-separated list of the targets for which the metric should be calculated, and the
    features for which the metric should be calculated for that target.  See the example below.

    \nb When defining the metrics to use, it is possible to have multiple nodes with the same name.  For
    example, if a problem has inputs $W$, $X$, $Y$, and $Z$, and the responses are $A$, $B$, and $C$, it is possible that
    the desired metrics are the \xmlNode{sensitivity} of $A$ and $B$ to $X$ and $Y$, as well as the
    \xmlNode{sensitivity} of $C$ to $W$ and $Z$, but not the sensitivity of $A$ to $W$.   In this event, two
    copies of the \xmlNode{sensitivity} node are added to the input.  The first has targets $A,B$ and features
    $X,Y$, while the second node has target $C$ and features $W,Z$.  This could reduce some computation effort
    in problems with many responses or inputs.  An example of this is shown below.
  %
  \\ Currently the scalar quantities available for request are:
  \begin{itemize}
    \item \textbf{expectedValue}: expected value or mean
    \item \textbf{minimum}: The minimum value of the samples.
    \item \textbf{maximum}: The maximum value of the samples.
    \item \textbf{median}: median
    \item \textbf{variance}: variance
    \item \textbf{sigma}: standard deviation
    \item \textbf{percentile}: the percentile. If this quantity is inputted as \textit{percentile} the $5\%$ and $95\%$ percentile(s) are going to be computed.
                               Otherwise the user can specify this quantity with a parameter \textit{percent='X'}, where the \textit{X} represents the requested
                               percentile (a floating point value between 0.0 and 100.0)
    \item \textbf{variationCoefficient}: coefficient of variation, i.e. \textbf{sigma}/\textbf{expectedValue}. \nb If the \textbf{expectedValue} is zero,
    the \textbf{variationCoefficient} will be \textbf{INF}.
    \item \textbf{skewness}: skewness
    \item \textbf{kurtosis}: excess kurtosis (also known as Fisher's kurtosis)
    \item \textbf{samples}: the number of samples in the data set used to determine the statistics.
  \end{itemize}
  The matrix quantities available for request are:
  \begin{itemize}
    \item \textbf{sensitivity}: matrix of sensitivity coefficients, computed via linear regression method.
    \item \textbf{covariance}: covariance matrix
    \item \textbf{pearson}: matrix of correlation coefficients
    \item \textbf{NormalizedSensitivity}: matrix of normalized sensitivity
    coefficients. \nb{It is the matrix of normalized VarianceDependentSensitivity}
    \item \textbf{VarianceDependentSensitivity}: matrix of sensitivity coefficients dependent on the variance of the variables
  \end{itemize}
  This XML node needs to contain the attribute:
  \begin{itemize}
    \itemsep0em
    \item \xmlAttr{prefix}, \xmlDesc{required string attribute}, user-defined prefix for the given \textbf{metric}.
      For scalar quantifies, RAVEN will define a variable with name defined as:  ``prefix'' + ``\_'' + ``parameter name''.
      For example, if we define ``mean'' as the prefix for \textbf{expectedValue}, and parameter ``x'', then variable
      ``mean\_x'' will be defined by RAVEN.
      For matrix quantities, RAVEN will define a variable with name defined as: ``prefix'' + ``\_'' + ``target parameter name'' + ``\_'' + ``feature parameter name''.
      For example, if we define ``sen'' as the prefix for \textbf{sensitivity}, target ``y'' and feature ``x'', then
      variable ``sen\_y\_x'' will be defined by RAVEN.
      \nb These variable will be used by RAVEN for the internal calculations. It is also accessible by the user through
      \textbf{DataObjects} and \textbf{OutStreams}. 
  \end{itemize}
   %
  \nb If the weights are present in the system then weighted quantities are calculated automatically. In addition, if a matrix quantity is requested (e.g. Covariance matrix, etc.), only the weights in the output space are going to be used for both input and output space (the computation of the joint probability between input and output spaces is not implemented yet).
  \\
  \nb Certain ROMs provide their own statistical information (e.g., those using
  the sparse grid collocation sampler such as: \xmlString{GaussPolynomialRom}
  and \xmlString{HDMRRom}) which can be obtained by printing the ROM to file
  (xml). For these ROMs, computing the basic statistics on data generated from
  one of these sampler/ROM combinations may not provide the information that the
  user expects.
  \\
  %
   \item \xmlNode{pivotParameter}, \xmlDesc{string, optional field}, name of the parameter that needs
   to be used for the computation of the Dynamic BasicStatistics (e.g. time). This node needs to
   be inputted just in case an \textbf{HistorySet} is used as Input. It represents the reference
   monotonic variable based on which the statistics is going to be computed (e.g. time-dependent
   statistical moments).
    \default{None}
  %
  \item \xmlNode{biased}, \xmlDesc{string (boolean), optional field}, if \textit{True} biased
  quantities are going to be calculated, if \textit{False} unbiased.
  \default{False}
  %
\end{itemize}
\textbf{Example (Static Statistics):}  This example demonstrates how to request the expected value of
\xmlString{x01} and \xmlString{x02}, along with the sensitivity of both \xmlString{x01} and \xmlString{x02} to
\xmlString{a} and \xmlString{b}.
\begin{lstlisting}[style=XML,morekeywords={name,subType,debug}]
<Simulation>
  ...
  <Models>
    ...
    <PostProcessor name='aUserDefinedName' subType='BasicStatistics' verbosity='debug'>
      <expectedValue prefix='mean'>x01,x02</expectedValue>
      <sensitivity prefix='sen'>
        <targets>x01,x02</targets>
        <features>a,b</features>
      </sensitivity>
    </PostProcessor>
    ...
  </Models>
  ...
</Simulation>
\end{lstlisting}

In this case, the RAVEN variables ``mean\_x01, mean\_x02, sen\_x01\_a, sen\_x02\_a, sen\_x01\_b, sen\_x02\_b''
will be created and accessible for the RAVEN entities \textbf{DataObjects} and \textbf{OutStreams}. 

\textbf{Example (Static, multiple matrix nodes):} This example shows how multiple nodes can specify
particular metrics multiple times to include different target/feature combinations.  This postprocessor
calculates the expected value of $A$, $B$, and $C$, as well as the sensitivity of both $A$ and $B$ to $X$ and
$Y$ as well as the sensitivity of $C$ to $W$ and $Z$.
\begin{lstlisting}[style=XML,morekeywords={name,subType,debug}]
<Simulation>
  ...
  <Models>
    ...
    <PostProcessor name='aUserDefinedName' subType='BasicStatistics' verbosity='debug'>
      <expectedValue prefix='mean'>A,B,C</expectedValue>
      <sensitivity prefix='sen1'>
        <targets>A,B</targets>
        <features>x,y</features>
      </sensitivity>
      <sensitivity prefix='sen2'>
        <targets>C</targets>
        <features>w,z</features>
      </sensitivity>
    </PostProcessor>
    ...
  </Models>
  ...
</Simulation>
\end{lstlisting}
\textbf{Example (Dynamic Statistics):}
\begin{lstlisting}[style=XML,morekeywords={name,subType,debug}]
<Simulation>
  ...
  <Models>
    ...
    <PostProcessor name='aUserDefinedNameForDynamicPP' subType='BasicStatistics' verbosity='debug'>
      <expectedValue prefix='mean'>x01,x02</expectedValue>
      <sensitivity prefix='sen'>
        <targets>x01,x02</targets>
        <features>a,b</features>
      </sensitivity>
      <pivotParameter>time</pivotParameter>
    </PostProcessor>
    ...
  </Models>
  ...
</Simulation>
\end{lstlisting}

%%%%% PP ComparisonStatistics %%%%%%%
\subsubsection{ComparisonStatistics}
\label{ComparisonStatistics}
The \textbf{ComparisonStatistics} post-processor computes statistics
for comparing two different dataObjects.  This is an experimental
post-processor, and it will definitely change as it is further
developed.

There are four nodes that are used in the post-processor.

\begin{itemize}
\item \xmlNode{kind}: specifies information to use for comparing the
  data that is provided.  This takes either uniformBins which makes
  the bin width uniform or equalProbability which makes the number
  of counts in each bin equal.  It can take the following attributes:
  \begin{itemize}
  \item \xmlAttr{numBins} which takes a number that directly
    specifies the number of bins
  \item \xmlAttr{binMethod} which takes a string that specifies the
    method used to calculate the number of bins.  This can be either
    square-root or sturges.
  \end{itemize}
\item \xmlNode{compare}: specifies the data to use for comparison.
  This can either be a normal distribution or a dataObjects:
  \begin{itemize}
  \item \xmlNode{data}: This will specify the data that is used.  The
    different parts are separated by $|$'s.
  \item \xmlNode{reference}: This specifies a reference distribution
    to be used.  It takes distribution to use that is defined in the
    distributions block.  A name parameter is used to tell which
    distribution is used.
  \end{itemize}
\item \xmlNode{fz}: If the text is true, then extra comparison
  statistics for using the $f_z$ function are generated.  These take
  extra time, so are not on by default.
\item \xmlNode{interpolation}: This switches the interpolation used
  for the cdf and the pdf functions between the default of quadratic
  or linear.
\end{itemize}

The \textbf{ComparisonStatistics} post-processor generates a variety
of data.  First for each data provided, it calculates bin boundaries,
and counts the numbers of data points in each bin.  From the numbers
in each bin, it creates a cdf function numerically, and from the cdf
takes the derivative to generate a pdf.  It also calculates statistics
of the data such as mean and standard deviation. The post-processor
can generate a CSV file only.

The post-processor uses the generated pdf and cdf function to
calculate various statistics.  The first is the cdf area difference which is:
\begin{equation}
  cdf\_area\_difference = \int_{-\infty}^{\infty}{\|CDF_a(x)-CDF_b(x)\|dx}
\end{equation}
This given an idea about how far apart the two pieces of data are, and
it will have units of $x$.

The common area between the two pdfs is calculated.  If there is
perfect overlap, this will be 1.0, if there is no overlap, this will
be 0.0.  The formula used is:
\begin{equation}
  pdf\_common\_area = \int_{-\infty}^{\infty}{\min(PDF_a(x),PDF_b(x))}dx
\end{equation}

The difference pdf between the two pdfs is calculated.  This is calculated as:
\begin{equation}
  f_Z(z) = \int_{-\infty}^{\infty}f_X(x)f_Y(x-z)dx
\end{equation}
This produces a pdf that contains information about the difference
between the two pdfs.  The mean can be calculated as (and will be
calculated only if fz is true):
\begin{equation}
  \bar{z} = \int_{-\infty}^{\infty}{z f_Z(z)dz}
\end{equation}
The mean can be used to get an signed difference between the pdfs,
which shows how their means compare.

The variance of the difference pdf can be calculated as (and will be
calculated only if fz is true):
\begin{equation}
  var = \int_{-\infty}^{\infty}{(z-\bar{z})^2 f_Z(z)dz}
\end{equation}

The sum of the difference function is calculated if fz is true, and is:
\begin{equation}
  sum = \int_{-\infty}^{\infty}{f_z(z)dz}
\end{equation}
This should be 1.0, and if it is different that
points to approximations in the calculation.


\textbf{Example:}
\begin{lstlisting}[style=XML]
<Simulation>
   ...
   <Models>
      ...
      <PostProcessor name="stat_stuff" subType="ComparisonStatistics">
      <kind binMethod='sturges'>uniformBins</kind>
      <compare>
        <data>OriData|Output|tsin_TEMPERATURE</data>
        <reference name='normal_410_2' />
      </compare>
      <compare>
        <data>OriData|Output|tsin_TEMPERATURE</data>
        <data>OriData|Output|tsout_TEMPERATURE</data>
      </compare>
      </PostProcessor>
      <PostProcessor name="stat_stuff2" subType="ComparisonStatistics">
        <kind numBins="6">equalProbability</kind>
        <compare>
          <data>OriData|Output|tsin_TEMPERATURE</data>
        </compare>
        <Distribution class='Distributions' type='Normal'>normal_410_2</Distribution>
      </PostProcessor>
      ...
   </Models>
   ...
   <Distributions>
      <Normal name='normal_410_2'>
         <mean>410.0</mean>
         <sigma>2.0</sigma>
      </Normal>
   </Distributions>
</Simulation>
\end{lstlisting}

%%%%% PP ImportanceRank %%%%%%%
\subsubsection{ImportanceRank}
\label{ImportanceRank}
The \textbf{ImportanceRank} post-processor is specifically used
to compute sensitivity indices and importance indices with respect to input parameters
associated with multivariate normal distributions. In addition, the user can also request the transformation
matrix and the inverse transformation matrix when the PCA reduction is used.
%
\ppType{ImportanceRank}{ImportanceRank}
%
\begin{itemize}
  \item \xmlNode{what}, \xmlDesc{comma separated string, required field},
  %
  List of quantities to be computed.
  %
  Currently the quantities available are:
  \begin{itemize}
    \item \xmlString{SensitivityIndex}: used to measure the impact of sensitivities on the model.
    \item \xmlString{ImportanceIndex}: used to measure the impact of sensitivities and input uncertainties on the model.
    \item \xmlString{PCAIndex}: the indices of principal component directions, used to measure the impact
    of principal component directions on input covariance matrix.
    \nb \xmlString{PCAIndex} can be only requested when subnode \xmlNode{latent} is defined in \xmlNode{features}.
    \item \xmlString{transformation}: the transformation matrix used to map the latent variables to the manifest variables in the original input space.
    \item \xmlString{InverseTransformation}: the inverse transformation matrix used to map the manifest variables to the latent variables in the transformed space.
    \item \xmlString{ManifestSensitivity}: the sensitivity coefficients of \xmlNode{target} with respect to \xmlNode{manifest} variables defined in \xmlNode{features}.

    \nb In order to request \xmlString{transformation} matrix or \xmlString{InverseTransformation} matrix or \xmlString{ManifestSensitivity},
    the subnodes \xmlNode{latent} and \xmlNode{manifest} under \xmlNode{features} are required (more details can be found in the following).
    %
  \end{itemize}
  %
  \nb For each computed quantity, RAVEN will define a unique variable name so that the data can be accessible by the users
  through RAVEN entities \textbf{DataObjects} and \textbf{OutStreams}. These variable names are defined as follows:
  \begin{itemize}
    \item \xmlString{SensitivityIndex}: `sensitivityIndex' + `\_' + `targetVariableName' + `\_' + `latentFeatureVariableName'
    \item \xmlString{ImportanceIndex}: `importanceIndex' + `\_' + `targetVariableName' + `\_' + `latentFeatureVariableName'
    \item \xmlString{PCAIndex}: `pcaIndex' + `\_' + `latentFeatureVariableName'
    \item \xmlString{transformation}: `transformation' + `\_' + `manifestFeatureVariableName' + `\_' + `latentFeatureVariableName'
    \item \xmlString{InverseTransformation}: `inverseTransformation' + `\_' + `latentFeatureVariableName' + `\_' + `manifestFeatureVariableName'
    \item \xmlString{ManifestSensitivity}: `manifestSensitivity' + `\_' + `targetVariableName' + `\_' + `manifestFeatureVariableName'
  \end{itemize}
  %
  If all the quantities need to be computed, the user can input in the body of \xmlNode{what} the string \xmlString{all}.
  \nb \xmlString{all} equivalent to \xmlString {SensitivityIndex, ImportanceIndex, PCAIndex}.

  Since the transformation and InverseTransformation matrix can be very large, they are not printed with option \xmlString{all}.
  In order to request the transformation matrix (or inverse transformation matrix) from this post processor,
  the user need to specify \xmlString{transformation} or \xmlString{InverseTransformation} in \xmlNode{what}. In addition,
  both  \xmlNode{manifest} and \xmlNode{latent} subnodes are required and should be defined in node \xmlNode{features}. For example, let $\mathbf{L, P}$ represent
  the transformation and inverse transformation matrices, respectively. We will define vectors $\mathbf x$ as manifest variables and vectors $\mathbf y$
  as latent variables. If a absolute covariance matrix is used in given distribution, the following equation will be used:

  $
  \mathbf{\delta x} = \mathbf L * \mathbf y
  $

  $
  \mathbf y = \mathbf P * \mathbf \delta \mathbf x
  $

  If a relative covariance matrix is used in given distribution, the following equation will be used:

  $
  \frac{\mathbf \delta \mathbf x}{\mathbf \mu} = \mathbf L * \mathbf y
  $

  $
  \mathbf y = \mathbf P * {\frac{\mathbf \delta \mathbf x}{\mathbf \mu}}
  $

  where $\mathbf{\delta x}$ denotes the changes in the input vector $\mathbf x$, and $\mathbf \mu$ denotes the mean values of the input vector $\mathbf x$.

  %
  %
  \item \xmlNode{features}, \xmlDesc{XML node, required parameter}, used to specify the information for the input variables.
  In this xml-node, the following xml sub-nodes need to be specified:
    \begin{itemize}
      \item \xmlNode{manifest},\xmlDesc{XML node, optional parameter}, used to indicate the input variables belongs to the original input space.
      It can accept the following child node:
        \begin{itemize}
          \item \xmlNode{variables},\xmlDesc{comma separated string, required field}, lists manifest variables.
          \item \xmlNode{dimensions}, \xmlDesc{comma separated integer, optional field}, lists the dimensions corresponding to the manifest variables.
          If not provided, the dimensions are determined by the order indices of given manifest variables.
        \end{itemize}
      \item \xmlNode{latent},\xmlDesc{XML node, optional parameter}, used to indicate the input variables belongs to the transformed space.
      It can accept the following child node:
        \begin{itemize}
          \item \xmlNode{variables},\xmlDesc{comma separated string, required field}, lists latent variables.
          \item \xmlNode{dimensions}, \xmlDesc{comma separated integer, optional field}, lists the dimensions corresponding to the latent variables.
          If not provided, the dimensions are determined by the order indices of given latent variables.
        \end{itemize}
      \nb At least one of the subnodes, i.e. \xmlNode{manifest} and \xmlNode{latent} needs to be specified.
    \end{itemize}
  %
  \item \xmlNode{targets}, \xmlDesc{comma separated string, required field}, lists output responses.
  %
  \item \xmlNode{mvnDistribution}, \xmlDesc{string, required field}, specifies the
  multivariate normal distribution name. The \xmlNode{MultivariateNormal} node must be present.
\end{itemize}
  %
  %
  Here is an example to show the user how to request the transformation matrix, the inverse transformation matrix, the
  manifest sensitivities and other quantities.
  %

\textbf{Example:}
\begin{lstlisting}[style=XML,morekeywords={name,subType,debug}]
<Simulation>
  ...
  <Models>
    ...
    <PostProcessor name='aUserDefinedName' subType='ImportanceRank'>
      <what>SensitivityIndex,ImportanceIndex,Transformation, InverseTransformation,ManifestSensitivity</what>
      <features>
        <manifest>
          <variables>x1,x2</variables>
          <dimensions>1,2</dimensions>
        </manifest>
        <latent>
          <variables>latent1</variables>
          <dimensions>1</dimensions>
        </latent>
      </features>
      <targets>y</targets>
      <mvnDistribution>MVN</mvnDistribution>
    </PostProcessor>
    ...
  </Models>
  ...
</Simulation>
\end{lstlisting}

The calculation results can be accessible via variables ``sensitivityIndex\_y\_latent1, importanceIndex\_y\_latent1,
manifestSensitivity\_y\_x1, manifestSensitivity\_y\_x2, transformation\_x1\_latent1, transformation\_x2\_latent1,
inverseTransformation\_latnet1\_x1, inverseTransformation\_laent1\_x2'' through RAVEN entities \textbf{DataObjects}
and \textbf{OutStreams}.

%%%%% PP SafestPoint %%%%%%%
\subsubsection{SafestPoint}
\label{SafestPoint}
The \textbf{SafestPoint} post-processor provides the coordinates of the farthest
point from the limit surface that is given as an input.
%
The safest point coordinates are expected values of the coordinates of the
farthest points from the limit surface in the space of the ``controllable''
variables based on the probability distributions of the ``non-controllable''
variables.

The term ``controllable'' identifies those variables that are under control
during the system operation, while the ``non-controllable'' variables are
stochastic parameters affecting the system behaviour randomly.

The ``SafestPoint'' post-processor requires the set of points belonging to the
limit surface, which must be given as an input.
%
The probability distributions as ``Assembler Objects'' are required in the
``Distribution'' section for both ``controllable'' and ``non-controllable''
variables.

The sampling method used by the ``SafestPoint'' is a ``value'' or ``CDF'' grid.
%
At present only the ``equal'' grid type is available.

\ppType{Safest Point}{SafestPoint}

\begin{itemize}
  \item \xmlNode{Distribution}, \xmlDesc{Required}, represents the probability
  distributions of the ``controllable'' and ``non-controllable'' variables.
  %
  These are \textbf{Assembler Objects}, each of these nodes must contain 2
  attributes that are used to identify those within the simulation framework:
        \begin{itemize}
    \item \xmlAttr{class}, \xmlDesc{required string attribute}, is the main
    ``class'' the listed object is from.
                \item \xmlAttr{type}, \xmlDesc{required string attribute}, is the object
    identifier or sub-type.
        \end{itemize}
             \item  \xmlNode{outputName}, \xmlDesc{string, required field}, specifies the name of the output variable where the probability is going to be stored.
               \nb This variable name must be listed in the \xmlNode{Output} field of the Output DataObject
        \item \xmlNode{controllable}, \xmlDesc{XML node, required field},  lists the controllable variables.
  %
  Each variable is associated with its name and the two items below:
        \begin{itemize}
                \item \xmlNode{distribution} names the probability distribution associated
    with the controllable variable.
    %
                \item \xmlNode{grid} specifies the \xmlAttr{type}, \xmlAttr{steps}, and
    tolerance of the sampling grid.
    %
        \end{itemize}
        \item \xmlNode{non-controllable}, \xmlDesc{XML node, required field}, lists the non-controllable variables.
  %
  Each variable is associated with its name and the two items below:
        \begin{itemize}
                \item \xmlNode{distribution} names the probability distribution associated
    with the non-controllable variable.
    %
                \item \xmlNode{grid} specifies the \xmlAttr{type}, \xmlAttr{steps}, and
    tolerance of the sampling grid.
    %
                \end{itemize}
\end{itemize}

\textbf{Example:}
\begin{lstlisting}[style=XML,morekeywords={name,subType,class,type,steps}]
<Simulation>
  ...
    <Models>
    ...
    <PostProcessor name='SP' subType='SafestPoint'>
      <Distribution  class='Distributions'  type='Normal'>x1_dst</Distribution>
      <Distribution  class='Distributions'  type='Normal'>x2_dst</Distribution>
      <Distribution  class='Distributions'  type='Normal'>gammay_dst</Distribution>
      <controllable>
        <variable name='x1'>
          <distribution>x1_dst</distribution>
          <grid type='value' steps='20'>1</grid>
        </variable>
        <variable name='x2'>
          <distribution>x2_dst</distribution>
          <grid type='value' steps='20'>1</grid>
        </variable>
      </controllable>
      <non-controllable>
        <variable name='gammay'>
          <distribution>gammay_dst</distribution>
          <grid type='value' steps='20'>2</grid>
        </variable>
      </non-controllable>
    </PostProcessor>
    ...
  </Models>
  ...
</Simulation>
\end{lstlisting}
%%%%% PP LimitSurface %%%%%%%
\subsubsection{LimitSurface}
\label{LimitSurface}
The \textbf{LimitSurface} post-processor is aimed to identify the transition
zones that determine a change in the status of the system (Limit Surface).

\ppType{LimitSurface}{LimitSurface}

\begin{itemize}
  \item \xmlNode{parameters}, \xmlDesc{comma separated string, required field},
  lists the parameters that define the uncertain domain and from which the LS
  needs to be computed.
  \item \xmlNode{tolerance}, \xmlDesc{float, optional field}, sets the absolute
  value (in CDF) of the convergence tolerance.
 %
  This value defines the coarseness of the evaluation grid.
 %
 \default{1.0e-4}
  \item \xmlNode{side}, \xmlDesc{string, optional field}, in this node the user can specify
  which side of the limit surface needs to be computed. Three options are available:
  \\ \textit{negative},  Limit Surface corresponding to the goal function value of ``-1'';
  \\ \textit{positive}, Limit Surface corresponding to the goal function value of ``1'';
  \\ \textit{both}, either positive and negative Limit Surface is going to be computed.
  %
  %
\default{negative}
  % Assembler Objects
  \item \textbf{Assembler Objects} These objects are either required or optional
  depending on the functionality of the Adaptive Sampler.
  %
  The objects must be listed with a rigorous syntax that, except for the xml
  node tag, is common among all the objects.
  %
  Each of these nodes must contain 2 attributes that are used to map those
  within the simulation framework:
   \begin{itemize}
    \item \xmlAttr{class}, \xmlDesc{required string attribute}, is the main
    ``class'' of the listed object.
    %
    For example, it can be ``Models,'' ``Functions,'' etc.
    \item \xmlAttr{type}, \xmlDesc{required string attribute}, is the object
    identifier or sub-type.
    %
    For example, it can be ``ROM,'' ``External,'' etc.
    %
  \end{itemize}
  The \textbf{LimitSurface} post-processor requires or optionally accepts the
  following objects' types:
   \begin{itemize}
    \item \xmlNode{ROM}, \xmlDesc{string, optional field}, body of this xml
    node must contain the name of a ROM defined in the \xmlNode{Models} block
    (see section \ref{subsec:models_ROM}).
    \item \xmlNode{Function}, \xmlDesc{string, required field}, the body of
    this xml block needs to contain the name of an External Function defined
    within the \xmlNode{Functions} main block (see section \ref{sec:functions}).
    %
    This object represents the boolean function that defines the transition
    boundaries.
    %
    This function must implement a method called
    \textit{\_\_residuumSign(self)}, that returns either -1 or 1, depending on
    the system conditions (see section \ref{sec:functions}).
    %
    \end{itemize}
\end{itemize}

\textbf{Example:}
\begin{lstlisting}[style=XML,morekeywords={name,subType,debug,class,type}]
<Simulation>
 ...
 <Models>
  ...
    <PostProcessor name="computeLimitSurface" subType='LimitSurface' verbosity='debug'>
      <parameters>x0,y0</parameters>
      <ROM class='Models' type='ROM'>Acc</ROM>
      <!-- Here, you can add a ROM defined in Models block.
           If it is not Present, a nearest neighbor algorithm
           will be used.
       -->
      <Function class='Functions' type='External'>
        goalFunctionForLimitSurface
      </Function>
    </PostProcessor>
    ...
  </Models>
  ...
</Simulation>
\end{lstlisting}

%%%%% PP LimitSurfaceIntegral %%%%%%%

\subsubsection{LimitSurfaceIntegral}
\label{LimitSurfaceIntegral}
The \textbf{LimitSurfaceIntegral} post-processor is aimed to compute the likelihood (probability) of the event, whose boundaries are
represented by the Limit Surface (either from the LimitSurface post-processor or Adaptive sampling strategies).
The inputted Limit Surface needs to be, in the  \textbf{PostProcess} step, of type  \textbf{PointSet} and needs to contain
both boundary sides (-1.0, +1.0).
%\\ The \textbf{LimitSurfaceIntegral} post-processor accepts as outputs both files (CSV) and/or  \textbf{PointSet}s.
\\ The \textbf{LimitSurfaceIntegral} post-processor accepts as output  \textbf{PointSet}s only.

\ppType{LimitSurfaceIntegral}{LimitSurfaceIntegral}
\begin{itemize}
\item \variableDescription
 \variableChildIntro
 \begin{itemize}
     \item  \xmlNode{outputName}, \xmlDesc{string, required field}, specifies the name of the output variable where the probability is going to be stored.
               \nb This variable name must be listed in the \xmlNode{Output} field of the Output DataObject
    \item   \xmlNode{distribution}, \xmlDesc{string,
               optional field}, name of the distribution that is associated to this variable.
              Its name needs to be contained in the \xmlNode{Distributions} block explained
              in Section \ref{sec:distributions}. If this node is not present, the  \xmlNode{lowerBound}
              and  \xmlNode{upperBound} XML nodes must be inputted.
   \item   \xmlNode{lowerBound}, \xmlDesc{float,
               optional field}, lower limit of integration domain for this dimension (variable).
               If this node is not present, the  \xmlNode{distribution} XML node must be inputted.
   \item   \xmlNode{upperBound}, \xmlDesc{float,
               optional field}, upper limit of integration domain for this dimension (variable).
               If this node is not present, the  \xmlNode{distribution} XML node must be inputted.
  \end{itemize}

    \item  \xmlNode{tolerance}, \xmlDesc{float, optional field}, specifies the tolerance for
               numerical integration confidence.
                \default{1.0e-4}
     \item  \xmlNode{integralType}, \xmlDesc{string, optional field}, specifies the type of integrations that
                need to be used. Currently only MonteCarlo integration is available
                \default{MonteCarlo}
     \item  \xmlNode{seed}, \xmlDesc{integer, optional field}, specifies the random number generator seed.
                \default{20021986}
     \item  \xmlNode{target}, \xmlDesc{string, optional field}, specifies the target name that represents
                the $f\left ( \bar{x} \right )$ that needs to be integrated.
                \default{last output found in the inputted PointSet}
\end{itemize}

\textbf{Example:}
\begin{lstlisting}[style=XML,morekeywords={name,subType,debug,class,type}]
<Simulation>
 ...
 <Models>
  ...
    <PostProcessor name="LimitSurfaceIntegralDistributions" subType='LimitSurfaceIntegral'>
        <tolerance>0.0001</tolerance>
        <integralType>MonteCarlo</integralType>
        <seed>20021986</seed>
        <target>goalFunctionOutput</target>
        <outputName>EventProbability</outputName>
        <variable name='x0'>
          <distribution>x0_distrib</distribution>
        </variable>
        <variable name='y0'>
          <distribution>y0_distrib</distribution>
        </variable>
    </PostProcessor>
    <PostProcessor name="LimitSurfaceIntegralLowerUpperBounds" subType='LimitSurfaceIntegral'>
        <tolerance>0.0001</tolerance>
        <integralType>MonteCarlo</integralType>
        <seed>20021986</seed>
        <target>goalFunctionOutput</target>
        <outputName>EventProbability</outputName>
        <variable name='x0'>
          <lowerBound>-2.0</lowerBound>
          <upperBound>12.0</upperBound>
        </variable>
        <variable name='y0'>
            <lowerBound>-1.0</lowerBound>
            <upperBound>11.0</upperBound>
        </variable>
    </PostProcessor>
    ...
  </Models>
  ...
</Simulation>
\end{lstlisting}



%%%%% PP External %%%%%%%
\subsubsection{External}
\label{External}
The \textbf{External} post-processor will execute an arbitrary python function
defined externally using the \textit{Functions} interface (see
Section~\ref{sec:functions} for more details).
%

\ppType{External}{External}

\begin{itemize}
  \item \xmlNode{method}, \xmlDesc{comma separated string, required field},
  lists the method names of an external Function that will be computed (each
  returning a post-processing value). The name of the method represents a new
  variable that can be stored in a new \textbf{\textit{DataObjects}} entity.
  \item \xmlNode{Function}, \xmlDesc{xml node, required string field}, specifies
  the name of a Function where the \textit{methods} listed above are defined.
  %
  \nb This name should match one of the Functions defined in the
  \xmlNode{Functions} block of the input file.
  %
  The objects must be listed with a rigorous syntax that, except for the XML
  node tag, is common among all the objects.
  %
  Each of these sub-nodes must contain 2 attributes that are used to map them
  within the simulation framework:

   \begin{itemize}
     \item \xmlAttr{class}, \xmlDesc{required string attribute}, is the main
     ``class'' the listed object is from, the only acceptable class for
     this post-processor is \xmlString{Functions};
     \item \xmlAttr{type}, \xmlDesc{required string attribute}, is the object
     identifier or sub-type, the only acceptable type for this post-processor is
     \xmlString{External}.
  \end{itemize}
\end{itemize}

  This Post-Processor accepts as Input/Output both \xmlString{PointSet} and \xmlString{HistorySet}:
   \begin{itemize}
    \item If a \xmlString{PointSet}  is used as Input, the parameters are passed in the external  \xmlString{Function}
  as numpy arrays. The methods' return type must be either a new array or a scalar. In the following it is reported an example
  with two methods, one that returns a scalar and the other one that returns an array:
      \begin{lstlisting}[language=python]
import numpy as np
def sum(self):
  return np.sum(self.aParameterInPointSet)

def sumTwoArraysAndReturnAnotherone(self):
  return self.aParamInPointSet1+self.aParamInPointSet2
      \end{lstlisting}
    \item If a \xmlString{HistorySet}  is used as Input, the parameters are passed in the external  \xmlString{Function}
     as a list of numpy arrays. The methods' return type must be either a new list of arrays (if the Output is another
     \xmlString{HistorySet}), a scalar or a single array (if the  Output is  \xmlString{PointSet} . In the following it
     is reported an example
     with two methods, one that returns a new list of arrays (Output = HistorySet) and the other one that returns an array (Output =
     PointSet):
      \begin{lstlisting}[language=python]
import numpy as np
def newHistorySetParameter(self):
  x = []*len(self.time)
  for history in range(len(self.time)):
    for ts in range(len(self.time[history])):
      if self.time[history][ts] >= 0.001: break
    x[history] = self.x[history][ts:]
  return x

def aNewPointSetParameter(self):
  x = []*len(self.time)
  for history in range(len(self.time)):
    x[history] = self.x[history][-1]
  return x
      \end{lstlisting}
   \end{itemize}

\textbf{Example:}
\begin{lstlisting}[style=XML,morekeywords={subType,debug,name,class,type}]
<Simulation>
  ...
  <Models>
    ...
    <PostProcessor name="externalPP" subType='External' verbosity='debug'>
      <method>Delta,Sum</method>
      <Function class='Functions' type='External'>operators</Function>
        <!-- Here, you can add a Function defined in the
             Functions block. This should be present or
             else RAVEN will not know where to find the
             defined methods. -->
    </PostProcessor>
    ...
  </Models>
  ...
</Simulation>
\end{lstlisting}

\nb The calculation results from this post-processor are stored in the internal variables. These variables
are accessible by the users through RAVEN entities \textbf{DataObjects} and \textbf{OutStreams}. The names
of these variables are defined as: ``Function Name in this post-processor'' + ``\_`` + ``variable name in XML
node \xmlNode{method}''. For example, in previous case, variables ``operators\_Delta'' and ``operators\_Sum''
are defined by RAVEN to store the outputs of this post-processor.

%%%%% PP TopologicalDecomposition %%%%%%%
\subsubsection{TopologicalDecomposition}
\label{TopologicalDecomposition}
The \textbf{TopologicalDecomposition} post-processor will compute an
approximated hierarchical Morse-Smale complex which will add two columns to a
dataset, namely \texttt{minLabel} and \texttt{maxLabel} that can be used to
decompose a dataset.
%

The topological post-processor can also be run in `interactive' mode, that is
by passing the keyword \texttt{interactive} to the command line of RAVEN's
driver.
%
In this way, RAVEN will initiate an interactive UI that allows one to explore
the topological hierarchy in real-time and adjust the simplification setting
before adjusting a dataset. Use in interactive mode will replace the parameter
\xmlNode{simplification} described below with whatever setting is set in the UI
upon exiting it.

In order to use the \textbf{TopologicalDecomposition} post-processor, the user
needs to set the attribute \xmlAttr{subType}:
\xmlNode{PostProcessor \xmlAttr{subType}=\xmlString{TopologicalDecomposition}}.
The following is a list of acceptable sub-nodes:
\begin{itemize}
  \item \xmlNode{graph} \xmlDesc{, string, optional field}, specifies the type
  of neighborhood graph used in the algorithm, available options are:
  \begin{itemize}
    \item \texttt{beta skeleton}
    \item \texttt{relaxed beta skeleton}
    \item \texttt{approximate knn}
    %\item Delaunay \textit{(disabled)}
  \end{itemize}
  \default{\texttt{beta skeleton}}
  \item \xmlNode{gradient}, \xmlDesc{string, optional field}, specifies the
  method used for estimating the gradient, available options are:
  \begin{itemize}
    \item \texttt{steepest}
    %\item \xmlString{maxflow} \textit{(disabled)}
  \end{itemize}
  \default{\texttt{steepest}}
  \item \xmlNode{beta}, \xmlDesc{float in the range: (0,2], optional field}, is
  only used when the \xmlNode{graph} is set to \texttt{beta skeleton} or
  \texttt{relaxed beta skeleton}.
  \default{1.0}
  \item \xmlNode{knn}, \xmlDesc{integer, optional field}, is the number of
  neighbors when using the \xmlString{approximate knn} for the \xmlNode{graph}
  sub-node and used to speed up the computation of other graphs by using the
  approximate knn graph as a starting point for pruning. -1 means use a fully
  connected graph.
  \default{-1}
  \item \xmlNode{weighted}, \xmlDesc{boolean, optional}, a flag that specifies
  whether the regression models should be probability weighted.
  \default{False}
  \item \xmlNode{interactive}, if this node is present \emph{and} the user has
  specified the keyword \texttt{interactive} at the command line, then this will
  initiate a graphical interface for exploring the different simplification
  levels of the topological hierarchy. Upon exit of the graphical interface, the
  specified simplification level will be updated to use the last value of the
  graphical interface before writing any ``output'' results.
  \item \xmlNode{persistence}, \xmlDesc{string, optional field}, specifies how
  to define the hierarchical simplification by assigning a value to each local
  minimum and maximum according to the one of the strategy options below:
  \begin{itemize}
    \item \texttt{difference} - The function value difference between the
    extremum and its closest-valued neighboring saddle.
    \item \texttt{probability} - The probability integral computed as the
    sum of the probability of each point in a cluster divided by the count of
    the cluster.
    \item \texttt{count} - The count of points that flow to or from the
    extremum.
    % \item \xmlString{area} - The area enclosed by the manifold that flows to
    % or from the extremum.
  \end{itemize}
  \default{\texttt{difference}}
  \item \xmlNode{simplification}, \xmlDesc{float, optional field}, specifies the
  amount of noise reduction to apply before returning labels.
  \default{0}
  \item \xmlNode{parameters}, \xmlDesc{comma separated string, required field},
  lists the parameters defining the input space.
  \item \xmlNode{response}, \xmlDesc{string, required field}, is a single
  variable name defining the scalar output space.
\end{itemize}
\textbf{Example:}
\begin{lstlisting}[style=XML,morekeywords={subType}]
<Simulation>
  ...
  <Models>
    ...
    <PostProcessor name="***" subType='TopologicalDecomposition'>
      <graph>beta skeleton</graph>
      <gradient>steepest</gradient>
      <beta>1</beta>
      <knn>8</knn>
      <normalization>None</normalization>
      <parameters>X,Y</parameters>
      <response>Z</response>
      <weighted>true</weighted>
      <simplification>0.3</simplification>
      <persistence>difference</persistence>
    </PostProcessor>
    ...
  <Models>
  ...
<Simulation>
\end{lstlisting}

%%%%% PP DataMining %%%%%%%
\input{DataMining.tex}

%%%%% PP PrintCSV %%%%%%%
%\paragraph{PrintCSV}
%\label{PrintCSV}
%TO BE MOVED TO STEP ``IOSTEP''
%%%%% PP LoadCsvIntoInternalObject %%%%%%%
%\paragraph{LoadCsvIntoInternalObject}
%\label{LoadCsvIntoInternalObject}
%TO BE MOVED TO STEP ``IOSTEP''
%

%%%%% PP External %%%%%%%
\subsubsection{Interfaced}
\label{Interfaced}
The \textbf{Interfaced} post-processor is a Post-Processor that allows the user
to create its own Post-Processor. While the External Post-Processor (see
Section~\ref{External} allows the user to create case-dependent
Post-Processors, with this new class the user can create new general
purpose Post-Processors.
%

\ppType{Interfaced}{Interfaced}

\begin{itemize}
  \item \xmlNode{method}, \xmlDesc{comma separated string, required field},
  lists the method names of a method that will be computed (each
  returning a post-processing value). All available methods need to be included
  in the ``/raven/framework/PostProcessorFunctions/'' folder
\end{itemize}

\textbf{Example:}
\begin{lstlisting}[style=XML,morekeywords={subType,debug,name,class,type}]
<Simulation>
  ...
  <Models>
    ...
    <PostProcessor name="example" subType='InterfacedPostProcessor'verbosity='debug'>
       <method>testInterfacedPP</method>
       <!--Here, the xml nodes required by the chosen method have to be
       included.
        -->
    </PostProcessor>
    ...
  </Models>
  ...
</Simulation>
\end{lstlisting}

All the \textbf{Interfaced} post-processors need to be contained in the
``/raven/framework/PostProcessorFunctions/'' folder. In fact, once the
\textbf{Interfaced} post-processor is defined in the RAVEN input file, RAVEN
search that the method of the post-processor is located in such folder.

The class specified in the \textbf{Interfaced} post-processor has to inherit the
PostProcessorInterfaceBase class and the user must specify this set of
methods:
\begin{itemize}
  \item initialize: in this method, the internal parameters of the
  post-processor are initialized. Mandatory variables that needs to be
  specified are the following:
\begin{itemize}
  \item self.inputFormat: type of dataObject expected in input
  \item self.outputFormat: type of dataObject generated in output
\end{itemize}
  \item readMoreXML: this method is in charge of reading the PostProcessor xml
  node, parse it and fill the PostProcessor internal variables.
  \item run: this method performs the desired computation of the dataObject.
\end{itemize}

\begin{lstlisting}[language=python]
from PostProcessorInterfaceBaseClass import PostProcessorInterfaceBase
class testInterfacedPP(PostProcessorInterfaceBase):
  def initialize(self)
  def readMoreXML(self,xmlNode)
  def run(self,inputDic)
\end{lstlisting}

\paragraph{Data Format}
The user is not allowed to modify directly the DataObjects, however the
content of the DataObjects is available in the form of a python dictionary.
Both the dictionary give in input and the one generated in the output of the
PostProcessor are structured as follows:

\begin{lstlisting}[language=python]
inputDict = {'data':{}, 'metadata':{}}
\end{lstlisting}

where:

\begin{lstlisting}[language=python]
inputDict['data'] = {'input':{}, 'output':{}}
\end{lstlisting}

In the input dictonary, each input variable is listed as a dictionary that
contains a numpy array with its own values as shown below for a simplified
example

\begin{lstlisting}[language=python]
inputDict['data']['input'] = {'inputVar1': array([ 1.,2.,3.]),
                              'inputVar2': array([4.,5.,6.])}
\end{lstlisting}

Similarly, if the dataObject is a PointSet then the output dictionary is
structured as follows:

\begin{lstlisting}[language=python]
inputDict['data']['output'] = {'outputVar1': array([ .1,.2,.3]),
                               'outputVar2':array([.4,.5,.6])}
\end{lstlisting}

Howevers, if the dataObject is a HistorySet then the output dictionary is
structured as follows:

\begin{lstlisting}[language=python]
inputDict['data']['output'] = {'hist1': {}, 'hist2':{}}
\end{lstlisting}

where

\begin{lstlisting}[language=python]
inputDict['output']['data'][hist1] = {'time': array([ .1,.2,.3]),
                              'outputVar1':array([ .4,.5,.6])}
inputDict['output']['data'][hist2] = {'time': array([ .1,.2,.3]),
                              'outputVar1':array([ .14,.15,.16])}
\end{lstlisting}


\paragraph{Method: HStoPSOperator}

This Post-Processor performs the conversion from HistorySet to PointSet performing a projection of the output space.

In the \xmlNode{PostProcessor} input block, the following XML sub-nodes are available:

\begin{itemize}
   \item \xmlNode{pivotParameter}, \xmlDesc{string, optional field}, ID of the temporal variable. Default is ``time''. 
   \nb Used just in case the  \xmlNode{pivotValue}-based operation  is requested
    \item \xmlNode{operator}, \xmlDesc{string, optional field}, the operation to perform on the output space:
      \begin{itemize}
        \item \textbf{min}, compute the minimum of each variable along each single history
         \item \textbf{max}, compute the maximum of each variable along each single history
         \item \textbf{average}, compute the average of each variable along each single history
       \end{itemize}
        \nb This node can be inputted only if \xmlNode{pivotValue} and \xmlNode{row} are not present
     \item \xmlNode{pivotValue}, \xmlDesc{float, optional field}, the value of the pivotParameter with respect to the other outputs need to be extracted.
       \nb This node can be inputted only if \xmlNode{operator} and \xmlNode{row} are not present
     \item \xmlNode{pivotStrategy}, \xmlDesc{string, optional field}, The strategy to use for the pivotValue:
       \begin{itemize}
        \item \textbf{nearest}, find the value that is the nearest with respect the \xmlNode{pivotValue}
        \item \textbf{floor}, find the value that is the nearest with respect to the \xmlNode{pivotValue} but less then the  \xmlNode{pivotValue}
        \item \textbf{celing}, find the value that is the nearest with respect to the \xmlNode{pivotValue} but greater then the  \xmlNode{pivotValue}
        \item \textbf{interpolate}, if the exact  \xmlNode{pivotValue}  can not be found, interpolate using a linear approach
       \end{itemize}
       
       \nb Valid just in case \xmlNode{pivotValue} is present
     \item \xmlNode{row}, \xmlDesc{int, optional field}, the row index at which the outputs need to be extracted.
       \nb This node can be inputted only if \xmlNode{operator} and \xmlNode{pivotValue} are not present   
\end{itemize}

This example will show how the XML input block would look like:

\begin{lstlisting}[style=XML,morekeywords={subType,debug,name,class,type}]
<Simulation>
  ...
  <Models>
    ...
    <PostProcessor name="HStoPSperatorRows" subType="InterfacedPostProcessor">
      <method>HStoPSOperator</method>
      <row>-1</row>
    </PostProcessor>
    <PostProcessor name="HStoPSoperatorPivotValues" subType="InterfacedPostProcessor">
        <method>HStoPSOperator</method>
        <pivotParameter>time</pivotParameter>
        <pivotValue>0.3</pivotValue>
    </PostProcessor>
    <PostProcessor name="HStoPSoperatorOperatorMax" subType="InterfacedPostProcessor">
        <method>HStoPSOperator</method>
        <pivotParameter>time</pivotParameter>
        <operator>max</operator>
    </PostProcessor>
    <PostProcessor name="HStoPSoperatorOperatorMin" subType="InterfacedPostProcessor">
        <method>HStoPSOperator</method>
        <pivotParameter>time</pivotParameter>
        <operator>min</operator>
    </PostProcessor>
    <PostProcessor name="HStoPSoperatorOperatorAverage" subType="InterfacedPostProcessor">
        <method>HStoPSOperator</method>
        <pivotParameter>time</pivotParameter>
        <operator>average</operator>
    </PostProcessor>
    ...
  </Models>
  ...
</Simulation>
\end{lstlisting}

\paragraph{Method: HistorySetSampling}
This Post-Processor performs the conversion from HistorySet to HistorySet
The conversion is made so that each history H is re-sampled accordingly  to a
specific sampling strategy.
It can be used to reduce the amount of space required by the HistorySet.

In the \xmlNode{PostProcessor} input block, the following XML sub-nodes are required,
independent of the \xmlAttr{subType} specified:

\begin{itemize}
   \item \xmlNode{samplingType}, \xmlDesc{string, required field}, specifies the type of sampling method to be used (uniform, firstDerivative secondDerivative, filteredFirstDerivative or
   filteredSecondDerivative).
   \item \xmlNode{numberOfSamples}, \xmlDesc{integer, optional field}, number of samples (required only for the following sampling types: uniform, firstDerivative secondDerivative)
   \item \xmlNode{pivotParameter}, \xmlDesc{string, required field}, ID of the temporal variable
   \item \xmlNode{interpolation}, \xmlDesc{string, optional field}, type of interpolation to be employed for the history recostruction (required only for the following sampling types: uniform,
   firstDerivative secondDerivative). Valid types of interpolation to specified: linear, nearest, zero, slinear, quadratic, cubic, intervalAverage;
   \item \xmlNode{tolerance}, \xmlDesc{string, optional field}, tolerance level (required only for the following sampling types: filteredFirstDerivative or filteredSecondDerivative)
\end{itemize}

\paragraph{Method: HistorySetSync}
This Post-Processor performs the conversion from HistorySet to HistorySet
The conversion is made so that all histories are synchronized in time.
It can be used to allow the histories to be sampled at the same time instant.

There are two possible synchronization methods, specified through the \xmlNode{syncMethod} node.  If the
\xmlNode{syncMethod} is \xmlString{grid}, a \xmlNode{numberOfSamples} node is specified,
which yields an equally-spaced grid of time points. The output values for these points will be linearly derived
using nearest sampled time points, and the new HistorySet will contain only the new grid points.

The other methods are used by specifying \xmlNode{syncMethod} as \xmlString{all}, \xmlString{min}, or
\xmlString{max}.  For \xmlString{all}, the postprocessor will iterate through the
existing histories, collect all the time points used in any of them, and use these as the new grid on which to
establish histories, retaining all the exact original values and interpolating linearly where necessary.
In the event of \xmlString{min} or \xmlString{max}, the postprocessor will find the smallest or largest time
history, respectively, and use those time values as nodes to interpolate between.

In the \xmlNode{PostProcessor} input block, the following XML sub-nodes are required,
independent of the \xmlAttr{subType} specified:

\begin{itemize}
   \item \xmlNode{pivotParameter}, \xmlDesc{string, required field}, ID of the temporal variable
   \item \xmlNode{extension}, \xmlDesc{string, required field}, type of extension when the sync process goes outside the boundaries of the history (zeroed or extended)
   \item \xmlNode{syncMethod}, \xmlDesc{string, required field}, synchronization strategy to employ (see
     description above).  Options are \xmlString{grid}, \xmlString{all}, \xmlString{max}, \xmlString{min}.
   \item \xmlNode{numberOfSamples}, \xmlDesc{integer, optional field}, required if \xmlNode{syncMethod} is
     \xmlString{grid}, number of new time samples
\end{itemize}

\paragraph{Method: HistorySetSnapShot}
This Post-Processor performs the conversion from HistorySet to PointSet
The conversion is made so that each history H is converted to a single point P.
There are several methods that can be employed to choose the single point from the history:
\begin{itemize}
  \item min: Take a time slice when the \xmlNode{pivotVar} is at its smallest value,
  \item max: Take a time slice when the \xmlNode{pivotVar} is at its largest value,
  \item average: Take a time slice when the \xmlNode{pivotVar} is at its time-weighted average value,
  \item value: Take a time slice when the \xmlNode{pivotVar} \emph{first passes} its specified value,
  \item timeSlice: Take a time slice index from the sampled time instance space.
\end{itemize}
To demonstrate the timeSlice, assume that each history H is a dict of n output variables $x_1=[...],
x_n=[...]$, then the resulting point P is at time instant index t: $P=[x_1[t],...,x_n[t]]$.

Choosing one the these methods for the \xmlNode{type} node will take a time slice for all the variables in the
output space based on the provided parameters.  Alternatively, a \xmlString{mixed} type can be used, in which
each output variable can use a different time slice parameter.  In other words, you can take the max of one
variable while taking the minimum of another, etc.

In the \xmlNode{PostProcessor} input block, the following XML sub-nodes are required,
independent of the \xmlAttr{subType} specified:

\begin{itemize}
  \item \xmlNode{type}, \xmlDesc{string, required field}, type of operation: \xmlString{min}, \xmlString{max},
                        \xmlString{average}, \xmlString{value}, \xmlString{timeSlice}, or \xmlString{mixed}
   \item \xmlNode{extension}, \xmlDesc{string, required field}, type of extension when the sync process goes outside the boundaries of the history (zeroed or extended)
   \item \xmlNode{pivotParameter}, \xmlDesc{string, optional field}, name of the temporal variable.  Required for the
     \xmlString{average} and \xmlString{timeSlice} methods.
\end{itemize}

If a \xmlString{timeSlice} type is in use, the following nodes also are required:
\begin{itemize}
   \item \xmlNode{timeInstant}, \xmlDesc{integer, required field}, required and only used in the
     \xmlString{timeSlice} type.  Location of the time slice (integer index)
   \item \xmlNode{numberOfSamples}, \xmlDesc{integer, required field}, number of samples
\end{itemize}

If instead a \xmlString{min}, \xmlString{max}, \xmlString{average}, or \xmlString{value} is used, the following nodes
are also required:
\begin{itemize}
   \item \xmlNode{pivotVar}, \xmlDesc{string, required field},  Name of the chosen indexing variable (the
         variable whose min, max, average, or value is used to determine the time slice)
       \item \xmlNode{pivotVal}, \xmlDesc{float, optional field},  required for \xmlString{value} type, the value for the chosen variable
\end{itemize}

Lastly, if a \xmlString{mixed} approach is used, the following nodes apply:
\begin{itemize}
  \item \xmlNode{max}, \xmlDesc{string, optional field}, the names of variables whose output should be their
    own maximum value within the history.
  \item \xmlNode{min}, \xmlDesc{string, optional field}, the names of variables whose output should be their
    own minimum value within the history.
  \item \xmlNode{average}, \xmlDesc{string, optional field}, the names of variables whose output should be their
    own average value within the history. Note that a \xmlNode{pivotParameter} node is required to perform averages.
  \item \xmlNode{value}, \xmlDesc{string, optional field}, the names of variables whose output should be taken
    at a time slice determined by another variable.  As with the non-mixed \xmlString{value} type, the first
    time the \xmlAttr{pivotVar} crosses the specified \xmlAttr{pivotVal} will be the time slice taken.
    This node requires two attributes, if used:
    \begin{itemize}
      \item \xmlAttr{pivotVar}, \xmlDesc{string, required field}, the name of the variable on which the time
        slice will be performed.  That is, if we want the value of $y$ when $t=0.245$,
        this attribute would be \xmlString{t}.
      \item \xmlAttr{pivotVal}, \xmlDesc{float, required field}, the value of the \xmlAttr{pivotVar} on which the time
        slice will be performed.  That is, if we want the value of $y$ when $t=0.245$,
        this attribute would be \xmlString{0.245}.
    \end{itemize}
  Note that all the outputs of the \xmlNode{DataObject} output of this postprocessor must be listed under one
  of the \xmlString{mixed} node types in order for values to be returned.
\end{itemize}

\textbf{Example (mixed):}
This example will output the average value of $x$ for $x$, the value of $y$ at
time$=0.245$ for $y$, and the value of $z$ at $x=4.0$ for $z$.
\begin{lstlisting}[style=XML,morekeywords={subType,debug,name,class,type}]
<Simulation>
  ...
  <Models>
    ...
    <PostProcessor name="mampp2" subType="InterfacedPostProcessor">
      <method>HistorySetSnapShot</method>
      <type>mixed</type>
      <average>x</average>
      <value pivotVar="time" pivotVal="0.245">y</value>
      <value pivotVar="x" pivotVal="4.0">z</value>
      <pivotParameter>time</pivotParameter>
      <extension>zeroed</extension>
    </PostProcessor>
    ...
  </Models>
  ...
</Simulation>
\end{lstlisting}


\paragraph{Method: HSPS}

This Post-Processor performs the conversion from HistorySet to PointSet
The conversion is made so that each history H is converted to a single point P.
Assume that each history H is a dict of n output variables $x_1=[...],x_n=[...]$, then the resulting point P is as follows; $P=[x_1,...,x_n]$
Note: it is here assumed that all histories have been sync so that they have the same length, start point and end point. If you are not sure, do a pre-processing the the original history set.

In the \xmlNode{PostProcessor} input block, the following XML sub-nodes are required,
independent of the \xmlAttr{subType} specified (min, max, avg and value case):

\begin{itemize}
   \item \xmlNode{pivotParameter}, \xmlDesc{string, optional field}, ID of the temporal variable (only for avg)
\end{itemize}

\paragraph{Method: TypicalHistoryFromHistorySet}
This Post-Processor performs a simplified procedure of \cite{wilcox2008users} to form a ``typical'' time series from multiple time series. The input should be a HistorySet, with each history in the HistorySet synchronized. For HistorySet that is not synchronized, use Post-Processor method \textbf{HistorySetSync}  to synchronize the data before running this method.

Each history in input HistorySet is first converted to multiple histories each has maximum time specified in \xmlNode{outputLen} (see below). Each converted history $H_i$ is divided into a set of subsequences $\{H_i^j\}$, and the division is guided by the \xmlNode{subseqLen} node specified in the input XML. The value of \xmlNode{subseqLen} should be a list of positive numbers that specify the length of each subsequence. If the number of subsequence for each history is more than the number of values given in \xmlNode{subseqLen}, the values in \xmlNode{subseqLen} would be reused.

For each variable $x$, the method first computes the empirical CDF (cumulative density function) by using all the data values of $x$ in the HistorySet. This CDF is termed as long-term CDF for $x$. Then for each subsequence $H_i^j$, the method computes the empirical CDF by using all the data values of $x$ in $H_i^j$. This CDF is termed as subsequential CDF. For the first interval window (i.e., $j=1$), the method computes the Finkelstein-Schafer (FS) statistics \cite{finkelstein1971improved} between the long term CDF and the subsequential CDF of $H_i^1$ for each $i$. The FS statistics is defined as following.
\begin{align*}
FS & = \sum_x FS_x\\
FS_x &= \frac{1}{N}\sum_{n=1}^N\delta_n
\end{align*}
where $N$ is the number of value reading in the empirical CDF and $\delta_n$ is the absolute difference between the long term CDF and the subsequential CDF at value $x_n$. The subsequence $H_i^1$ with minimal FS statistics will be selected as the typical subsequence for the interval window $j=1$. Such process repeats for $j=2,3,\dots$ until all subsequences have been processed. Then all the typical subsequences will be concatenated to form a complete history.

In the \xmlNode{PostProcessor} input block, the following XML sub-nodes are required,
independent of the \xmlAttr{subType} specified:

\begin{itemize}
   \item \xmlNode{pivotParameter}, \xmlDesc{string, optional field}, ID of the temporal variable
   \default{Time}
   \item \xmlNode{subseqLen}, \xmlDesc{integers, required field}, length of the divided subsequence (see above)
   \item \xmlNode{outputLen}, \xmlDesc{integer, optional field}, maximum value of the temporal variable for the generated typical history
   \default{Maximum value of the variable with name of \xmlNode{pivotParameter}}
\end{itemize}

For example, consider history of data collected over three years in one-second increments,
where the user wants a single \emph{typical year} extracted from the data.
The user wants this data constructed by combining twelve equal \emph{typical month}
segments.  In this case, the parameter \xmlNode{outputLen} should be \texttt{31536000} (the number of seconds
in a year), while the parameter \xmlNode{subseqLen} should be \texttt{2592000} (the number of seconds in a
month).  Using a value for \xmlNode{subseqLen} that is either much, much smaller than \xmlNode{outputLen} or
of equal size to \xmlNode{outputLen} might have unexpected results.  In general, we recommend using a
\xmlNode{subseqLen} that is roughly an order of magnitude smaller than \xmlNode{outputLen}.

\paragraph{Method: dataObjectLabelFilter}
This Post-Processor allows to filter the portion of a dataObject, either PointSet or HistorySet, with a given clustering label.
A clustering algorithm associates a unique cluster label to each element of the dataObject (PointSet or HistorySet).
This cluster label is a natural number ranging from $0$ (or $1$ depending on the algorithm) to $N$ where $N$ is the number of obtained clusters.
Recall that some clustering algorithms (e.g., K-Means) receive $N$ as input while others (e.g., Mean-Shift) determine $N$ after clustering has been performed.
Thus, this Post-Processor is naturally employed after a data-mining clustering techniques has been performed on a dataObject so that each clusters
can be analyzed separately.

In the \xmlNode{PostProcessor} input block, the following XML sub-nodes are required,
independently of the \xmlAttr{subType} specified:

\begin{itemize}
   \item \xmlNode{label}, \xmlDesc{string, required field}, name of the clustering label
   \item \xmlNode{clusterIDs}, \xmlDesc{integers, required field}, ID of the selected clusters. Note that more than one ID can be provided as input
\end{itemize}

\paragraph{Method: HSPS}
The \xmlNode{HSPS} Post-Processor performs a filtering of the dataObject. This particular filtering is based on the labels generated by any clustering algorithm.
Given the selected label, this Post-Processor filters out all histories or points having a different label.
In the \xmlNode{PostProcessor} input block, the following XML sub-nodes are required:

\begin{itemize}
   \item \xmlNode{dataType}, \xmlDesc{string, required field}, type of dataObject (HistorySet or PointSet)
   \item \xmlNode{label}, \xmlDesc{string, required field}, varaiable which contains the cluster labels
   \item \xmlNode{clusterIDs}, \xmlDesc{int, required field}, cluster labels considered
\end{itemize}

\paragraph{Method: Discrete Risk Measures}
This Post-Processor calculates a series of risk importance measures from a PointSet. This calculation if performed for a set of input paramteres given an output target.

The user is required to provide the following information:
\begin{itemize}
   \item the set of input variables. For each variable the following need to be specified:
     \begin{itemize}
       \item the set of values that imply a reliability value equal to $1$ for the input variable
       \item the set of values that imply a reliability value equal to $0$ for the input variable
     \end{itemize}
   \item the output target variable. For this variable it is needed to specify the values of the output target variable that defines the desired outcome.
\end{itemize}

The following variables are first determined for each input variable $i$:
\begin{itemize}
   \item $R_0$ Probability of the outcome of the output target variable (nominal value)
   \item $R^{+}_i$ Probability of the outcome of the output target variable if reliability of the input variable is equal to $0$
   \item $R^{-}_i$ Probability of the outcome of the output target variable if reliability of the input variable is equal to $1$
\end{itemize}

Available measures are:
\begin{itemize}
   \item Risk Achievement Worth (RAW): $RAW = R^{+}_i / R_0 $
   \item Risk Achievement Worth (RRW): $RRW = R_0 / R^{-}_i$
   \item Fussell-Vesely (FV): $FV = (R_0 - R^{-}_i) / R_0$
   \item Birnbaum (B): $B = R^{+}_i - R^{-}_i$
\end{itemize}

In the \xmlNode{PostProcessor} input block, the following XML sub-nodes are required,
independent of the \xmlAttr{subType} specified:

\begin{itemize}
   \item \xmlNode{measures}, \xmlDesc{string, required field}, desired risk importance measures that have to be computed (RRW, RAW, FV, B)
   \item \xmlNode{variable}, \xmlDesc{string, required field}, ID of the input variable. This node is provided for each input variable. This nodes needs to contain also these attributes:
     \begin{itemize}
       \item \xmlAttr{R0values}, \xmlDesc{float, required field}, interval of values (comma separated values) that implies a reliability value equal to $0$ for the input variable
       \item \xmlAttr{R1values}, \xmlDesc{float, required field}, interval of values (comma separated values) that implies a reliability value equal to $1$ for the input variable
     \end{itemize}
   \item \xmlNode{target}, \xmlDesc{string, required field}, ID of the output variable. This nodes needs to contain also the attribute \xmlAttr{values}, \xmlDesc{string, required field}, interval of
                                                             values of the output target variable that defines the desired outcome
\end{itemize}

\textbf{Example:}
This example shows an example where it is desired to calculate all available risk importance measures for two input variables (i.e., pumpTime and valveTime)
given an output target variable (i.e., Tmax).
A value of the input variable pumpTime in the interval $[0,240]$ implies a reliability value of the input variable pumpTime equal to $0$.
A value of the input variable valveTime in the interval $[0,60]$ implies a reliability value of the input variable valveTime equal to $0$.
A value of the input variables valveTime and pumpTime in the interval $[1441,2880]$ implies a reliability value of the input variables equal to $1$.
The desired outcome of the output variable Tmax occurs in the interval $[2200,2500]$.
\begin{lstlisting}[style=XML,morekeywords={subType,debug,name,class,type}]
<Simulation>
  ...
  <Models>
    ...
    <PostProcessor name="riskMeasuresDiscrete" subType="InterfacedPostProcessor">
      <method>RiskMeasuresDiscrete</method>
      <measures>B,FV,RAW,RRW</measures>
      <variable R0values='0,240' R1values='1441,2880'>pumpTime</variable>
      <variable R0values='0,60'  R1values='1441,2880'>valveTime</variable>
      <target   values='2200,2500'                  >Tmax</target>
    </PostProcessor>
    ...
  </Models>
  ...
</Simulation>
\end{lstlisting}

This Post-Processor allows the user to consider also multiple datasets (a data set for each initiating event) and calculate the global risk importance measures.
This can be performed by:
\begin{itemize}
  \item Including all datasets in the step
\begin{lstlisting}[style=XML,morekeywords={subType,debug,name,class,type}]
<Simulation>
  ...
  </Steps>
    ...
    <PostProcess name="PP">
      <Input   class="DataObjects"  type="PointSet"        >outRun1</Input>
      <Input   class="DataObjects"  type="PointSet"        >outRun2</Input>
      <Model   class="Models"       type="PostProcessor"   >riskMeasuresDiscrete</Model>
      <Output  class="DataObjects"  type="PointSet"        >outPPS</Output>
      <Output  class="OutStreams"   type="Print"           >PrintPPS_dump</Output>
    </PostProcess>
  </Steps>
  ...
</Simulation>
\end{lstlisting}
  \item Adding in the Post-processor the frequency of the initiating event associated to each dataset
\begin{lstlisting}[style=XML,morekeywords={subType,debug,name,class,type}]
<Simulation>
  ...
  <Models>
    ...
    <PostProcessor name="riskMeasuresDiscrete" subType="InterfacedPostProcessor">
      <method>riskMeasuresDiscrete</method>
      <measures>FV,RAW</measures>
      <variable R1values='-0.1,0.1' R0values='0.9,1.1'>Astatus</variable>
      <variable R1values='-0.1,0.1' R0values='0.9,1.1'>Bstatus</variable>
      <variable R1values='-0.1,0.1' R0values='0.9,1.1'>Cstatus</variable>
      <variable R1values='-0.1,0.1' R0values='0.9,1.1'>Dstatus</variable>
      <target   values='0.9,1.1'>outcome</target>
      <data     freq='0.01'>outRun1</data>
      <data     freq='0.02'>outRun2</data>
    </PostProcessor>
    ...
  </Models>
  ...
</Simulation>
\end{lstlisting}

\end{itemize}

This post-processor can be made time dependendent if a single HistorySet is provided among the other data objects.
The HistorySet contains the temporal profiles of a subset of the input variables. This temporal profile can be only
boolean, i.e., 0 (component offline) or 1 (component online).
Note that the provided history set must contains a single History; multiple Histories are not allowed.
When this post-processor is in a dynamic configuration (i.e., time-dependent), the user is required to specify an xml
node \xmlNode{temporalID} that indicates the ID of the temporal variable.
For each time instant, this post-processor determines the temporal profiles of the desired risk importance measures.
Thus, in this case, an HistorySet must be chosen as an output data object.
An example is shown below:
\begin{lstlisting}[style=XML,morekeywords={subType,debug,name,class,type}]
<Simulation>
  ...
  <Models>
    ...
    <PostProcessor name="riskMeasuresDiscrete" subType="InterfacedPostProcessor">
      <method>riskMeasuresDiscrete</method>
      <measures>B,FV,RAW,RRW,R0</measures>
      <variable R1values='-0.1,0.1' R0values='0.9,1.1'>Astatus</variable>
      <variable R1values='-0.1,0.1' R0values='0.9,1.1'>Bstatus</variable>
      <variable R1values='-0.1,0.1' R0values='0.9,1.1'>Cstatus</variable>
      <target   values='0.9,1.1'>outcome</target>
      <data     freq='1.0'>outRun1</data>
      <temporalID>time</temporalID>
    </PostProcessor>
    ...
  </Models>
  ...
  <Steps>
    ...
    <PostProcess name="PP">
      <Input     class="DataObjects"  type="PointSet"        >outRun1</Input>
      <Input     class="DataObjects"  type="HistorySet"      >timeDepProfiles</Input>
      <Model     class="Models"       type="PostProcessor"   >riskMeasuresDiscrete</Model>
      <Output    class="DataObjects"  type="HistorySet"      >outHS</Output>
      <Output    class="OutStreams"   type="Print"           >PrintHS</Output>
    </PostProcess>
    ...
  </Steps>
  ...
</Simulation>
\end{lstlisting}

%
%%%%%%%%%%%%%%%%%%%%%%%%%%%%
%%%%%% RavenOutput PP   %%%%%%
%%%%%%%%%%%%%%%%%%%%%%%%%%%%
%

\subsubsection{RavenOutput}
\label{RavenOutput}
The \textbf{RavenOutput} post-processor is specifically used
to gather data from RAVEN output files and generate a PointSet suitable for plotting or other analysis.
It can do this in two modes: static and dynamic.  In static mode, the
PostProcessor reads from from several static XML output files produced by RAVEN.  In dynamic mode, the PostProcessor
reads from a single dynamic XML output file and builds a PointSet where the pivot parameter (e.g. time) is the
input and the requested values are returned for each of the pivot parameter values (e.g. points in time).  The
name for the pivot parameter will be taken directly from the XML structure.
%
Note: by default the PostProcessor operates in static mode; to read a dynamic file, the \xmlNode{dynamic} node must
be specified.
%
\ppType{RavenOutput}{RavenOutput}
%
\begin{itemize}
  \item \xmlNode{dynamic}, \xmlDesc{string, optional field}, if included will trigger reading a single dynamic
  file instead of multiple static files, unless the text of this field is \xmlString{false}, in which case it
  will return to the default (multiple static files).  \default(False)
  \item \xmlNode{File}, \xmlDesc{XML Node, required field}
  %
  For each file to be read by this postprocessor, an entry in the \xmlNode{Files} node must be added, and a
  \xmlNode{File} node must be added to the postprocessor input block.  The \xmlNode{File} requires two
  identifying attributes:
  \begin{itemize}
    \item \xmlAttr{name}, \xmlDesc{string, required field}, the RAVEN-assigned name of the file,
    \item \xmlAttr{ID}, \xmlDesc{float, optional field}, the floating point ID that will be unique to this
      file.  This will appear as an entry in the output \xmlNode{DataObject} and the corresponding column are
      the values extracted from this file.  If not specified, RAVEN will attempt to find a suitable integer ID
      to use, and a warning will be raised.

      When defining the \xmlNode{DataObject} that this postprocessor will write to, and when using the static
      (non-\xmlNode{dynamic}) form of the postprocessor, the \xmlNode{input} space should be given as
      \xmlString{ID}, and the output variables should be the outputs specified in the postprocessor. See the
      examples below.  In the data object, the variable values will be keyed on the \xmlString{ID} parameter.
  \end{itemize}
  Each value that needs to be extracted from the file needs to be specified by one of the following
  \xmlNode{output} nodes within the \xmlNode{File} node:
  \begin{itemize}
    \item \xmlNode{output}, \xmlDesc{|-separated string, required field},
           the specification of the output to extract from the file.
           RAVEN uses \texttt{xpath} as implemented in Python's \texttt{xml.etree} module to specify locations
           in XML.  For example, to search tags, use a path
           separated by forward slash characters (``/''), starting under the root; this means the root node should not
           be included in the path. See the example.  For more details on xpath options available, see
           \url{https://docs.python.org/2/library/xml.etree.elementtree.html#xpath-support}.
           %
           The \xmlNode{output} node requires the following attribute:
      \begin{itemize}
        \item \xmlAttr{name}, \xmlDesc{string, required field}, specifies the entry in the Data Object that
          this value should be stored under.
      \end{itemize}

  \end{itemize}
  %
\end{itemize}
\textbf{Example (Static):}
Using an example, let us have two input files, named \emph{in1.xml} and \emph{in2.xml}.  They appear as
follows.  Note that the name of the variables we want changes slightly between the XML; this is fine.

\textbf{\emph{in1.xml}}
\begin{lstlisting}[style=XML]
<BasicStatistics>
  <ans>
    <val1>6</val1>
    <val2>7</val2>
  </ans>
</BasicStatistics>
\end{lstlisting}
\textbf{\emph{in2.xml}}
\begin{lstlisting}[style=XML]
<ROM>
  <ans>
    <first>6.1</first>
    <second>7.1</second>
  </ans>
</BasicStatistics>
\end{lstlisting}

The RAVEN input to extract this information would appear as follows.
We include an example of defining the \xmlNode{DataObject} that this postprocessor will write out to, for
further clarity.

\begin{lstlisting}[style=XML]
<Simulation>
 ...
 <Files>
   <Input name='in1'>inp1.xml</Input>
   <Input name='in2'>inp2.xml</Input>
 </Files>
 ...
 <Models>
   ...
   <PostProcessor name='pp' subType='RavenOutput'>
     <File name='in1' ID='1'>
       <output name='first'>ans/val1</output>
       <output name='second'>ans/val2</output>
     </File>
     <File name='in2' ID='2'>
       <output name='first'>ans/first</output>
       <output name='second'>ans/second</output>
     </File>
   </PostProcessor>
   ...
 </Models>
 ...
 <DataObjects>
   ...
   <PointSet name='pointSetName'>
     <input>ID</input>
     <output>first,second</output>
   </PointSet>
   ...
 </DataObjects>
 ...
</Simulation>
\end{lstlisting}

\textbf{Example (Dynamic):}
For a dynamic example, consider this time-evolution of values example.  \emph{inFile.xml} is a RAVEN dynamic
XML output.

\textbf{\emph{in1.xml}}
\begin{lstlisting}[style=XML]
<BasicStatistics type='Dynamic'>
  <time value='0.0'>
    <ans>
      <val1>6</val1>
      <val2>7</val2>
    </ans>
  <\time>
  <time value='1.0'>
    <ans>
      <val1>9</val1>
      <val2>10</val2>
    </ans>
  <\time>
</BasicStatistics>
\end{lstlisting}
The RAVEN input to extract this information would appear as follows:
\begin{lstlisting}[style=XML]
<Simulation>
 ...
 <Files>
   <Input name='inFile'>inFile.xml</Input>
 </Files>
 ...
 <Models>
   ...
   <PostProcessor name='pp' subType='RavenOut'>
     <dynamic>true</dynamic>
     <File name='inFile'>
       <output name='first'>ans|val1</output>
     </File>
   </PostProcessor>
   ...
 </Models>
 ...
</Simulation>
\end{lstlisting}
The resulting PointSet has \emph{time} as an input and \emph{first} as an output.

%%%%%%%%%%%%%% ETImporter PP %%%%%%%%%%%%%%%%%%%

\subsubsection{ETImporter}
\label{ETImporterPP}
The \textbf{ETImporter} post-processor has been designed to import Event-Tree (ET) object into
RAVEN. This is performed by saving the structure of the ET (from file) as a \textbf{PointSet} (only \textbf{PointSet} are allowed). 
Since an ET is a static Boolean logic structure, the \textbf{PointSet} is structured as follows:
\begin{itemize}
  \item Input variables of the \textbf{PointSet} are the branching conditions of the ET. The value of each input variable can be:
  \begin{itemize}
    \item  0: event did occur (typically upper branch)
    \item  1: event did not occur (typically lower branch)
    \item -1: event is not queried (no branching occured)
  \end{itemize}   
  \item Output variables of the \textbf{PointSet} are the ID of each branch of the ET (i.e., positive integers greater than 0)
\end{itemize}
Since several ET file formats are available, as of now only the OpenPSA format (see https://open-psa.github.io/joomla1.5/index.php.html) is supported.
The ETImporter PP supports also:
\begin{itemize}
  \item links to sub-trees
  \item by-pass branches
  \item symbolic definition of outcomes: typically outcomes are defined as either 0 (upper branch) or 1 (lower branch). If instead the ET uses the
        success/failure labels, then they are converted into 0/1 labels
  \item symbolic/numerical definition of sequences: if the ET contains a symbolic sequence then a .xml file is generated.  This file contains
        the mapping between the sequences defined in the ET and the numerical IDs created by RAVEN. The file name is the concatenation of the ET name
        and "\_mapping". As an example the file "eventTree\_mapping.xml" generated by RAVEN:
        \begin{lstlisting}[style=XML]
            <map Tree="eventTree">
              <sequence ID="0">seq_1</sequence>
              <sequence ID="1">seq_2</sequence>
              <sequence ID="2">seq_3</sequence>
              <sequence ID="3">seq_4</sequence>
            </map>
        \end{lstlisting}
        contains the mapping of four sequences defined in the ET (seq\_1,seq\_2,seq\_3,seq\_4) with the IDs generated by RAVEN (0,1,2,3).
        Note that if the sequences defined in the ET are both numerical and symbolic then they are all mapped.
\end{itemize}
The \xmlNode{collect-formula} are not considered since this node is used to connect the Boolean formulae generated by the
Fault-Trees to the branch (i.e., fork) point.

%
\ppType{ETImporter}{ETImporter}
%
\begin{itemize}
  \item \xmlNode{fileFormat}, \xmlDesc{string, required field}, specifies the format of the file that contains the ET structure (supported format: OpenPSA).
\end{itemize}

\textbf{Example:}

\begin{lstlisting}[style=XML]
<Simulation>
 ...
  <Models>
    ...
    <PostProcessor name="ETImporter" subType="ETImporter">
      <fileFormat>OpenPSA</fileFormat>
    </PostProcessor>   
    ...
  </Models>
 ...
  <Steps>
    ...
    <PostProcess name="import">
      <Input   class="Files"        type=""                >eventTreeTest</Input>
      <Model   class="Models"       type="PostProcessor"   >ETImporter</Model>
      <Output  class="DataObjects"  type="PointSet"        >ET_PS</Output>
    </PostProcess>
    ...
</Simulation>
\end{lstlisting}


%%%%%%%%%%%%%% FTImporter PP %%%%%%%%%%%%%%%%%%%

\subsubsection{FTImporter}
\label{FTImporterPP}
The \textbf{FTImporter} post-processor has been designed to import Fault-Tree (FT) object into
RAVEN. This is performed by saving the structure of the FT (from file) as a \textbf{PointSet}
(only \textbf{PointSet} are allowed). 

Each Point in the PointSet represents a unique combination of the basic events.
The PointSet is structured as follows: input variables are the basic events, output variable is the top event of the FT.
The value for each input and output variable can have the following values:
\begin{itemize}
  \item  0: False
  \item  1: True
\end{itemize}

Since several FT file formats are available, as of now only the OpenPSA format
(see https://open-psa.github.io/joomla1.5/index.php.html) is supported.

%
\ppType{FTImporter}{FTImporter}
%
\begin{itemize}
  \item \xmlNode{fileFormat}, \xmlDesc{string, required field}, specifies the format of the file that contains the
    FT structure (supported format: OpenPSA).
  \item  \xmlNode{topEventID},\xmlDesc{string, required parameter}, the name of the top event of the FT
\end{itemize}

\textbf{Example:}

\begin{lstlisting}[style=XML,morekeywords={anAttribute},caption=FT Importer input example., label=lst:FT_PP_InputExample]
  <Files>
    <Input name="faultTreeTest" type="">FTimporter_not.xml</Input>
  </Files>
  
  <Models>
    ...
    <PostProcessor name="FTimporter" subType="FTImporter">
      <fileFormat>OpenPSA</fileFormat>
      <topEventID>TOP</topEventID>
    </PostProcessor> 
    ...  
  </Models>

  <Steps>
    ...
    <PostProcess name="import">
      <Input   class="Files"        type=""                >faultTreeTest</Input>
      <Model   class="Models"       type="PostProcessor"   >FTimporter</Model>
      <Output  class="DataObjects"  type="PointSet"        >FT_PS</Output>
    </PostProcess>
    ...
  </Steps>

\end{lstlisting}

%%%%%%%%%%%%%% Metric PP %%%%%%%%%%%%%%%%%%%

\subsubsection{Metric}
\label{MetricPP}
The \textbf{Metric} post-processor is specifically used to calculate the distance values among points from PointSets and histories from HistorySets,
while the \textbf{Metrics} block (See Chapter \ref{sec:Metrics}) allows the user to specify the similarity/dissimilarity metrics to be used in this
post-processor. It is important to notice that this post-processor currently can only accept \textbf{PointSet} data
object and does not accept \textbf{HistorySet} data.  If the name of the variable is unique, it can be used,
otherwise the variable can be specified with $DataObjectName|InputOrOutput|VariableName$ like other places in RAVEN.
Some of the Metrics also accept distributions to calculate the distance against.  These are specified by using the name of the distribution.
%
\ppType{Metric}{Metric}
%
\begin{itemize}
  \item \xmlNode{Features}, \xmlDesc{comma separated string, required field}, specifies the names of the features.
    This xml-node accepts the following attribute:
    \begin{itemize}
      \item \xmlAttr{type}, \xmlDesc{required string attribute}, the type of provided features. Currently only
        accept `variable'.
    \end{itemize}
  \item \xmlNode{Targets}, \xmlDesc{comma separated string, required field}, contains a comma separated list of
    the targets. \nb Each target is paired with a feature listed in xml node \xmlNode{Features}. In this case, the
    number of targets should be equal to the number of features.
    This xml-node accepts the following attribute:
    \begin{itemize}
      \item \xmlAttr{type}, \xmlDesc{required string attribute}, the type of provided features. Currently only
        accept `variable'.
    \end{itemize}
  \item \xmlNode{Metric}, \xmlDesc{string, required field}, specifies the \textbf{Metric} name that is defined via
    \textbf{Metrics} entity. In this xml-node, the following xml attributes need to be specified:
    \begin{itemize}
      \item \xmlAttr{class}, \xmlDesc{required string attribute}, the class of this metric (e.g. Metrics)
      \item \xmlAttr{type}, \xmlDesc{required string attribute}, the sub-type of this Metric (e.g. SKL, Minkowski)
    \end{itemize}
\end{itemize}

\textbf{Example:}

\begin{lstlisting}[style=XML]
<Simulation>
 ...
  <Models>
    ...
    <PostProcessor name="pp1" subType="Metric">
      <Features type="variable">ans</Features>
      <Targets type="variable">ans2</Targets>
      <Metric class="Metrics" type="SKL">euclidean</Metric>
      <Metric class="Metrics" type="SKL">rbf</Metric>
      <Metric class="Metrics" type="SKL">poly</Metric>
      <Metric class="Metrics" type="SKL">sigmoid</Metric>
      <Metric class="Metrics" type="SKL">polynomial</Metric>
      <Metric class="Metrics" type="SKL">linear</Metric>
      <Metric class="Metrics" type="SKL">cosine</Metric>
      <Metric class="Metrics" type="SKL">cityblock</Metric>
      <Metric class="Metrics" type="SKL">l1</Metric>
      <Metric class="Metrics" type="SKL">l2</Metric>
      <Metric class="Metrics" type="SKL">manhattan</Metric>
      <Metric class="Metrics" type="SKL">laplacian</Metric>
    </PostProcessor>
    ...
  </Models>
 ...
</Simulation>
\end{lstlisting}

In order to access the results from this post-processor, RAVEN will define the variables as ``MetricName'' +
``\_'' + ``TargetVariableName'' + ``\_'' + ``FeatureVariableName'' to store the calculation results, and these
variables are also accessible by the users through RAVEN entities \textbf{DataObjects} and \textbf{OutStreams}.
In previous example, variables such as \textit{euclidean\_ans2\_ans, rbf\_ans2\_ans, poly\_ans2\_ans} are accessible
by the users.

%%%%%%%%%%%%%% Cross Validation PP %%%%%%%%%%%%%%%%%%%

\subsubsection{CrossValidation}
\label{CVPP}
The \textbf{CrossValidation} post-processor is specifically used to evaluate estimator (i.e. ROMs) performance.
Cross-validation is a statistical method of evaluating and comparing learning algorithms by dividing data into
two portions: one used to `train' a surrogate model and the other used to validate the model, based on specific
scoring metrics. In typical cross-validation, the training and validation sets must crossover in successive
rounds such that each data point has a chance of being validated against the various sets. The basic form of
cross-validation is k-fold cross-validation. Other forms of cross-validation are special cases of k-fold or involve
repeated rounds of k-fold cross-validation. \nb It is important to notice that this post-processor currently can
only accept \textbf{PointSet} data object.
%
\ppType{CrossValidation}{CrossValidation}
%
\begin{itemize}
  \item \xmlNode{SciKitLearn}, \xmlDesc{string, required field}, the subnodes specifies the necessary information
    for the algorithm to be used in the post-processor. `SciKitLearn' is based on algorithms in SciKit-Learn
    library, and currently it performs cross-validation over \textbf{PointSet} only.
  \item \xmlNode{Metric}, \xmlDesc{string, required field}, specifies the \textbf{Metric} name that is defined via
    \textbf{Metrics} entity. In this xml-node, the following xml attributes need to be specified:
    \begin{itemize}
      \item \xmlAttr{class}, \xmlDesc{required string attribute}, the class of this metric (e.g. Metrics)
      \item \xmlAttr{type}, \xmlDesc{required string attribute}, the sub-type of this Metric (e.g. SKL, Minkowski)
    \end{itemize}
    \nb Currently, cross-validation post-processor only accepts \xmlNode{SKL} metrics with \xmlNode{metricType}
    \xmlString{mean\_absolute\_error}, \xmlString{explained\_variance\_score}, \xmlString{r2\_score},
    \xmlString{mean\_squared\_error}, and \xmlString{median\_absolute\_error}.
\end{itemize}

\textbf{Example:}

\begin{lstlisting}[style=XML]
<Simulation>
 ...
  <Files>
    <Input name="output_cv" type="">output_cv.xml</Input>
    <Input name="output_cv.csv" type="">output_cv.csv</Input>
  </Files>
  <Models>
    ...
    <ROM name="surrogate" subType="SciKitLearn">
      <SKLtype>linear_model|LinearRegression</SKLtype>
      <Features>x1,x2</Features>
      <Target>ans</Target>
      <fit_intercept>True</fit_intercept>
      <normalize>True</normalize>
    </ROM>
    <PostProcessor name="pp1" subType="CrossValidation">
        <SciKitLearn>
            <SKLtype>KFold</SKLtype>
            <n_splits>3</n_splits>
            <shuffle>False</shuffle>
            <random_state>None</random_state>
        </SciKitLearn>
        <Metric class="Metrics" type="SKL">m1</Metric>
    </PostProcessor>
    ...
  </Models>
  <Metrics>
    <SKL name="m1">
      <metricType>mean_absolute_error</metricType>
    </SKL>
  </Metrics>
  <Steps>
    <PostProcess name="PP1">
        <Input class="DataObjects" type="PointSet">outputDataMC</Input>
        <Input class="Models" type="ROM">surrogate</Input>
        <Model class="Models" type="PostProcessor">pp1</Model>
        <Output class="Files" type="">output_cv</Output>
        <Output class="Files" type="">output_cv.csv</Output>
    </PostProcess>
  </Steps>
 ...
</Simulation>
\end{lstlisting}

In order to access the results from this post-processor, RAVEN will define the variables as ``cv'' +
``\_'' + ``MetricName'' + ``\_'' + ``ROMTargetVariable'' to store the calculation results, and these
variables are also accessible by the users through RAVEN entities \textbf{DataObjects} and \textbf{OutStreams}.
In previous example, variable \textit{cv\_m1\_ans} are accessible by the users.

\paragraph{SciKitLearn}

The algorithm for cross-validation is chosen by the subnode \xmlNode{SKLtype} under the parent node \xmlNode{SciKitLearn}.
In addition, a special subnode \xmlNode{average} can be used to obtain the average cross validation results.

\begin{itemize}
  \item \xmlNode{SKLtype}, \xmlDesc{string, required field}, contains a string that
    represents the cross-validation algorithm to be used. As mentioned, its format is:

    \xmlNode{SKLtype}algorithm\xmlNode{/SKLtype}.
  \item \xmlNode{average}, \xmlDesc{boolean, optional field}, if `True`, dump the average cross validation results into the
    output files.
\end{itemize}


Based on the \xmlNode{SKLtype} several different algorithms are available. In the following paragraphs a brief
explanation and the input requirements are reported for each of them.

\paragraph{K-fold}
\textbf{KFold} divides all the samples in $k$ groups of samples, called folds (if $k=n$, this is equivalent to the
\textbf{Leave One Out} strategy), of equal sizes (if possible). The prediction function is learned using $k-1$ folds,
and fold left out is used for test.
In order to use this algorithm, the user needs to set the subnode:
\xmlNode{SKLtype}KFold\xmlNode{/SKLtype}.
In addition to this XML node, several others are available:
\begin{itemize}
  \item \xmlNode{n\_splits}, \xmlDesc{integer, optional field}, number of folds, must be at least 2. \default{3}
  \item \xmlNode{shuffle}, \xmlDesc{boolean, optional field}, whether to shuffle the data before splitting into
    batches.
  \item \xmlNode{random\_state}, \xmlDesc{None, integer or RandomState, optional field}, when shuffle=True,
    pseudo-random number generator state used for shuffling. If None, use default numpy RNG for shuffling.
\end{itemize}

\paragraph{Stratified k-fold}
\textbf{StratifiedKFold} is a variation of \textit{k-fold} which returns stratified folds: each set contains approximately
the same percentage of samples of each target class as the complete set.
In order to use this algorithm, the user needs to set the subnode:

\xmlNode{SKLtype}StratifiedKFold\xmlNode{/SKLtype}.

In addition to this XML node, several others are available:
\begin{itemize}
  \item \xmlNode{y}, \xmlDesc{array-like, [n\_samples], required field}, samples to split in K folds.
  \item \xmlNode{n\_splits}, \xmlDesc{integer, optional field}, number of folds, must be at least 2. \default{3}
  \item \xmlNode{shuffle}, \xmlDesc{boolean, optional field}, whether to shuffle the data before splitting into
    batches.
  \item \xmlNode{random\_state}, \xmlDesc{None, integer or RandomState, optional field}, when shuffle=True,
    pseudo-random number generator state used for shuffling. If None, use default numpy RNG for shuffling.
\end{itemize}

\paragraph{Label k-fold}
\textbf{LabelKFold} is a variation of \textit{k-fold} which ensures that the same label is not in both testing and
training sets. This is necessary for example if you obtained data from different subjects and you want to avoid
over-fitting (i.e., learning person specific features) by testing and training on different subjects.
In order to use this algorithm, the user needs to set the subnode:

\xmlNode{SKLtype}LabelKFold\xmlNode{/SKLtype}.

In addition to this XML node, several others are available:
\begin{itemize}
  \item \xmlNode{labels}, \xmlDesc{array-like with shape (n\_samples, ), required field}, contains a label for
    each sample. The folds are built so that the same label does not appear in two different folds.
  \item \xmlNode{n\_splits}, \xmlDesc{integer, optional field}, number of folds, must be at least 2. \default{3}
\end{itemize}

\paragraph{Leave-One-Out - LOO}
\textbf{LeaveOneOut} (or LOO) is a simple cross-validation. Each learning set is created by taking all the samples
except one, the test set being the sample left out. Thus, for $n$ samples, we have $n$ different training sets and
$n$ different tests set. This is cross-validation procedure does not waste much data as only one sample is removed from
the training set.
In order to use this algorithm, the user needs to set the subnode:

\xmlNode{SKLtype}LeaveOneOut\xmlNode{/SKLtype}.

\paragraph{Leave-P-Out - LPO}
\textbf{LeavePOut} is very similar to \textbf{LeaveOneOut} as it creates all the possible training/test sets by removing
$p$ samples from the complete set. For $n$ samples, this produces $(^n_p)$ train-test pairs. Unlike \textbf{LeaveOneOut}
and \textbf{KFold}, the test sets will overlap for $p > 1$.
In order to use this algorithm, the user needs to set the subnode:

\xmlNode{SKLtype}LeavePOut\xmlNode{/SKLtype}.

In addition to this XML node, several others are available:
\begin{itemize}
  \item \xmlNode{p}, \xmlDesc{integer, required field}, size of the test sets
\end{itemize}

\paragraph{Leave-One-Label-Out - LOLO}
\textbf{LeaveOneLabelOut} (LOLO) is a cross-validation scheme which holds out the samples according to a third-party
provided array of integer labels. This label information can be used to encode arbitrary domain specific pre-defined
cross-validation folds. Each training set is thus constituted by all samples except the ones related to a specific
label.
In order to use this algorithm, the user needs to set the subnode:

\xmlNode{SKLtype}LeaveOneLabelOut\xmlNode{/SKLtype}.

In addition to this XML node, several others are available:
\begin{itemize}
  \item \xmlNode{labels}, \xmlDesc{array-like of integer with shape (n\_samples,), required field}, arbitrary
    domain-specific stratificatioin of the data to be used to draw the splits.
\end{itemize}

\paragraph{Leave-P-Label-Out}
\textbf{LeavePLabelOut} is imilar as \textit{Leave-One-Label-Out}, but removes samples related to $P$ labels for
each training/test set.
In order to use this algorithm, the user needs to set the subnode:

\xmlNode{SKLtype}LeavePLabelOut\xmlNode{/SKLtype}.

In addition to this XML node, several others are available:
\begin{itemize}
  \item \xmlNode{labels}, \xmlDesc{array-like of integer with shape (n\_samples,), required field}, arbitrary
    domain-specific stratificatioin of the data to be used to draw the splits.
  \item \xmlNode{p}, \xmlDesc{integer, optional field}, number of samples to leave out in the test split.
\end{itemize}

\paragraph{ShuffleSplit}
\textbf{ShuffleSplit} iterator will generate a user defined number of independent train/test dataset splits. Samples
are first shuffled and then split into a pair of train and test sets. it is possible to control the randomness for
reproducibility of the results by explicitly seeding the \xmlNode{random\_state} pseudo random number generator.
In order to use this algorithm, the user needs to set the subnode:

\xmlNode{SKLtype}ShuffleSplit\xmlNode{/SKLtype}.

In addition to this XML node, several others are available:
\begin{itemize}
  \item \xmlNode{n\_iter}, \xmlDesc{integer, optional field}, number of re-shuffling and splitting iterations
    \default{10}.
  \item \xmlNode{test\_size}, \xmlDesc{float, integer or None}, if float, should be between 0.0 and 1.0 and
    represent the proportion of the dataset to include in the test split. \default{0.1}
    If integer, represents the absolute number of test samples. If None, the value is automatically set to
    the complement of the train size.
  \item \xmlNode{train\_size}, \xmlDesc{float, integer or None}, if float, should be between 0.0 and 1.0 and represent
    the proportion of the dataset to include in the train split. If integer, represents the absolute number of train
    samples. If None, the value is automatically set to the complement of the test size. \default{None}
  \item \xmlNode{random\_state}, \xmlDesc{None, integer or RandomState, optional field}, when shuffle=True,
    pseudo-random number generator state used for shuffling. If None, use default numpy RNG for shuffling.
\end{itemize}

\paragraph{Label-Shuffle-Split}
\textbf{LabelShuffleSplit} iterator behaves as a combination of \textbf{ShuffleSplit} and \textbf{LeavePLabelOut},
and generates a sequence of randomized partitions in which a subset of labels are held out for each split.
In order to use this algorithm, the user needs to set the subnode:

\xmlNode{SKLtype}LabelShuffleSplit\xmlNode{/SKLtype}.

In addition to this XML node, several others are available:
\begin{itemize}
  \item \xmlNode{labels}, \xmlDesc{array, [n\_samples]}, labels of samples.
  \item \xmlNode{n\_iter}, \xmlDesc{integer, optional field}, number of re-shuffling and splitting iterations
    \default{10}.
  \item \xmlNode{test\_size}, \xmlDesc{float, integer or None}, if float, should be between 0.0 and 1.0 and
    represent the proportion of the dataset to include in the test split. \default{0.1}
    If integer, represents the absolute number of test samples. If None, the value is automatically set to
    the complement of the train size.
  \item \xmlNode{train\_size}, \xmlDesc{float, integer or None}, if float, should be between 0.0 and 1.0 and represent
    the proportion of the dataset to include in the train split. If integer, represents the absolute number of train
    samples. If None, the value is automatically set to the complement of the test size. \default{None}
  \item \xmlNode{random\_state}, \xmlDesc{None, integer or RandomState, optional field}, when shuffle=True,
    pseudo-random number generator state used for shuffling. If None, use default numpy RNG for shuffling.
\end{itemize}


%%%%%%%%%%%%%% Data Classifier PP %%%%%%%%%%%%%%%%%%%

\subsubsection{DataClassifier}
\label{DataClassifierPP}
The \textbf{DataClassifier} post-processor is specifically used to classify the data stored in the DataObjects. It
accepts two DataObjects, one is used as the classifier which must be a \textbf{PointSet}, the other one is used
as the input DataObject to be classified. 
%
\ppType{DataClassifier}{DataClassifier}
%
\begin{itemize}
  \item \xmlNode{label}, \xmlDesc{string, required field}, the name of the label that are used for the classifier. This
    label must exist in the DataObject that is used as the classifer. This name will also be used as
    the label name for the DataObject that is classified.
  \item \xmlNode{variable}, \xmlDesc{required, xml node}. In this node, the following attribute should be specified: 
    \begin{itemize}
      \item \xmlAttr{name}, \xmlDesc{required, string attribute}, the variable name, which should be exist in
        the DataObject that is used as classifier.
    \end{itemize}
    and the following sub-node should also be specified:
    \begin{itemize}
      \item \xmlNode{Function}, \xmlDesc{string, required field}, this function creates the mapping from input DataObject
        to the Classifier.
        \begin{itemize}
          \item \xmlAttr{class}, \xmlDesc{string, required field}, the class of this function (e.g. Functions)
          \item \xmlAttr{type}, \xmlDesc{string, required field}, the type of this function (e.g. external)
        \end{itemize}
    \end{itemize}
\end{itemize}

\textbf{Example:}

\begin{lstlisting}[style=XML]
<Simulation>
 ...
  <Models>
    ...
    <PostProcessor name="ET_Classifier" subType="DataClassifier">
      <label>sequence</label>
      <variable name='ACC'>
        <Function class="Functions" type="External">func_ACC</Function>
      </variable>
      <variable name='LPI'>
        <Function class="Functions" type="External">func_LPI</Function>
      </variable>
      <variable name='LPR'>
        <Function class="Functions" type="External">func_LPR</Function>
      </variable>
    </PostProcessor>
    ...
  </Models>
 ...
</Simulation>
\end{lstlisting}



%
%%%%%%%%%%%%%%%%%%%%%%%%%%%%%%%%%%%%%
%%%%%%  EnsembleModel  Model   %%%%%%
%%%%%%%%%%%%%%%%%%%%%%%%%%%%%%%%%%%%%
%

\subsection{EnsembleModel}
\label{subsec:models_EnsembleModel}
As already mentioned, the \textbf{EnsembleModel} is able to combine \textbf{Code}(see ~\ref{subsec:models_code}),
\textbf{ExternalModel}(see ~\ref{subsec:models_externalModel}) and \textbf{ROM}(see ~\ref{subsec:models_externalModel}) Models.
\\It is aimed to create a chain of Models (whose execution order is determined by the Input/Output relationships among them).
  If the relationships among the models evolve in a non-linear system, a Picard's Iteration scheme is employed.
\\Currently this model is able to share information (i.e. data) using \textbf{PointSet},  \textbf{HistorySet} and \textbf{DataSet}

The specifications of a EnsembleModel must be defined within the XML block
\xmlNode{EnsembleModel}.
%
This XML node needs to contain the attributes:

\vspace{-5mm}
\begin{itemize}
  \itemsep0em
  \item \xmlAttr{name}, \xmlDesc{required string attribute}, user-defined name
  of this EnsembleModel.
  %
  \nb As with the other objects, this is the name that can be used to refer to
  this specific entity from other input blocks in the XML.
  \item \xmlAttr{subType}, \xmlDesc{required string attribute}, must be kept
  empty.
  %
\end{itemize}
\vspace{-5mm}

Within the \xmlNode{EnsembleModel} XML node, the multiple Models that constitute
this EnsembleModel needs to be inputted. Each Model is specified within a \xmlNode{Model} block (\nb each model
here specified need to be inputted in the\xmlNode{Models} main XML block) :
\begin{itemize}
  \item \xmlNode{Model}, \xmlDesc{XML node, required parameter}.
  %
  The text portion of this node needs to contain the name of the Model
  %
  \\This XML node needs to contain the attributes:

\vspace{-5mm}
\begin{itemize}
  \itemsep0em
  \item \xmlAttr{class}, \xmlDesc{required string attribute}, the class of this sub-model (e.g. Models)
  %
  \item \xmlAttr{type}, \xmlDesc{required string attribute}, the sub-type of this Model (e.g. ExternalModel, ROM, Code)
  %
\end{itemize}
\vspace{-5mm}

  %
  In addition the following XML sub-nodes need to be inputted (or optionally inputted):
  \begin{itemize}
     \item \xmlNode{TargetEvaluation}, \xmlDesc{string, required field},
        represents the container where the output of this Model are stored.
        %
        From a practical point of view, this XML node must contain the name of
        a data object defined in the \xmlNode{DataObjects} block (see
        Section~\ref{sec:DataObjects}).
        %
        Currently, the  \xmlNode{EnsembleModel} accept all \textbf{\textit{DataObjects'}}  types:
        \textbf{PointSet},  \textbf{HistorySet} and \textbf{DataSet}
        \nb The  \xmlNode{TargetEvaluation} is primary used for input-output identification. If the linked
        DataObject is not placed as additional output of the Step where the EnsembleModel is used, it will
        not be filled with the data coming from the calculation and it will be kept empty.
     \item \xmlNode{Input}, \xmlDesc{string, required field},
        represents the input entities that need to be passed to this sub-model
        %
        The user can specify as many \xmlNode{Input} as required by the sub-model.
        \nb All the inputs here specified need to be listed in the Steps where the EnsembleModel
        is used.
     \item \xmlNode{Output}, \xmlDesc{string, optional field},
        represents the output entities that need to be linked to this sub-model.  \nb The \xmlNode{Output}s here 
        specified are not part 
        of the determination of the EnsembleModel execution but represent an additional storage of results from the
        sub-models. For example, if the \xmlNode{TargetEvaluation} is of type PointSet (since just scalar data needs to 
        be transferred to other 
        models) and the sub-model is able to also output history-type data, this Output can be of type HistorySet.
        Note that the structure of each Output dataObject must include only variables (either input or output) that are 
        defined among the model.
        As an example, the Output dataObjects cannot contained variables that are defined at the Ensemble model 
        level.
        %
        The user can specify as many \xmlNode{Output} (s) as needed. The optional \xmlNode{Output}s  can be of 
        both classes ``DataObjects'' and ``Databases'' 
        (e.g. \textit{PointSet}, \textit{HistorySet}, \textit{DataSet}, \textit{HDF5})
        \nb \textbf{The \xmlNode{Output} (s) here specified MUST be listed in the Step in which the EnsembleModel is used.}
    \end{itemize}
  %
\end{itemize}


It is important to notice that when the EnsembleModel detects a chain of models that evolve in a non-linear system, a Picard's Iteration scheme is activated. In this case, an additional XML sub-node within the main \xmlNode{EnsembleModel} XML node needs to be specified:
\begin{itemize}
  \item \xmlNode{settings}, \xmlDesc{XML node, required parameter (if Picard's activated)}.
  %
  The body of this sub-node  contains the following XML sub-nodes:
  %
  \begin{itemize}
     \item \xmlNode{maxIterations}, \xmlDesc{integer, optional field},
        maximum number of Picard's iteration to be performed (in case the iteration scheme does
        not previously converge). \default{30};
     \item \xmlNode{tolerance}, \xmlDesc{float, optional field},
        convergence criterion. It represents the L2 norm residue below which the Picard's iterative scheme is
        considered converged. \default{0.001};
     \item \xmlNode{initialConditions}, \xmlDesc{XML node, required parameter  (if Picard's activated)},
        Within this sub-node, the initial conditions for the input variables (that are part of a loop)  need to
        be specified in sub-nodes named with the variable name (e.g. \xmlNode{varName}). The body of the
        \xmlNode{varName} contains the value of the initial conditions (scalar or arrays, depending of the
        type of variable). If an array needs to be inputted, the user can specify the attribute  \xmlAttr{repeat}
        and the code is going to repeat for  \xmlAttr{repeat}-times the value inputted in the body.
     \item \xmlNode{initialStartModels}, \xmlDesc{XML node, only required parameter when Picard's iteration is 
     activated},
        specifies the list of models that will be initially executed. \nb Do not input this node for non-Picard calculations,
        otherwise an error will be raised.
  \end{itemize}
\end{itemize}

\nb \textcolor{red} { \textbf{ It is crucial to understand that the choice of the \xmlNode{DataObject} used as
 \newline \xmlNode{TargetEvaluation} determines how the data are going to be transferred from a model to
  the other. If for example the chain of models is $A \rightarrow B$:}}
\begin{itemize}
  \item \textcolor{red} { \textbf{ If model $B$ expects as input scalars and outputs time-series, the \xmlNode{TargetEvaluation}  
  of the  model $B$ will be a \textit{HistorySet} and the  \xmlNode{TargetEvaluation} of the model $A$ will be either 
  a \textit{PointSet} or a \textit{DataSet} (where the output variables that need to be transferred to the model $A$ are scalars) }    }
   \item \textcolor{red} { \textbf{ If model $B$ expects as input scalars and time-series and outputs time-series or scalars or both, the \xmlNode{TargetEvaluation}  
  of the  model $B$ will be a \textit{DataSet} and the \newline  \xmlNode{TargetEvaluation} of the model $A$ will be either 
  a \textit{HistorySet} or a \textit{DataSet}  }    }
  \item \textcolor{red} { \textbf{ If both model $A$ and $B$ expect as input scalars and output scalars, the \xmlNode{TargetEvaluation}  
  of the  both models  $A$  and $B$ will be  \textit{PointSet}s  }  }
\end{itemize}

\textbf{Example (Linear System):}
\begin{lstlisting}[style=XML,morekeywords={subType,debug,name,class,type}]
<Simulation>
  ...
  <Models>
    ...
    <EnsembleModel name="heatTransferEnsembleModel" subType="">
      <Model class="Models" type="ExternalModel">
        thermalConductivityComputation
        <TargetEvaluation class="DataObjects" type="PointSet">
          thermalConductivityComputationContainer
        </TargetEvaluation>
        <Input class="DataObjects" type="PointSet">
          inputHolder
        </Input>
      </Model>
      <Model class="Models" type="ExternalModel" >
          heatTransfer
          <TargetEvaluation class="DataObjects" type="PointSet">
            heatTransferContainer
          </TargetEvaluation>
        <Input class="DataObjects" type="PointSet">
          inputHolder
        </Input>
        <Output class="DataObjects" type="HistorySet">
          thisModelLinkedOutput
        </Output>
        <Output class="Databases" type="HDF5">
          thisModelLinkedHDF5
        </Output>
      </Model>
    </EnsembleModel>
    ...
  </Models>
  ...
</Simulation>
\end{lstlisting}

\textbf{Example (Non-Linear System):}
\begin{lstlisting}[style=XML,morekeywords={subType,debug,repeat,name,class,type}]
<Simulation>
  ...
  <Models>
    ...
    <EnsembleModel name="heatTransferEnsembleModel" subType="">
      <settings>
        <maxIterations>8</maxIterations>
         <tolerance>0.01</tolerance>
         <initialConditions>
           <!-- the value 0.7 is going to be repeated 10 times in order to create  an array for var1 -->
           <var1 repeat="10">0.7</var1>
           <!-- an array for var2 has been inputted -->
           <var2> 0.5 0.3 0.4</var2>
           <!-- a scalar for var3 has been inputted -->
           <var3> 45.0</var3>
         </initialConditions>
      </settings>

      <Model class="Models" type="ExternalModel">
        thermalConductivityComputation
        <TargetEvaluation class="DataObjects" type="PointSet">
          thermalConductivityComputationContainer
        </TargetEvaluation>
        <Input class="DataObjects" type="PointSet">
          inputHolder
        </Input>
      </Model>
      <Model class="Models" type="ExternalModel" >
          heatTransfer
          <TargetEvaluation class="DataObjects" type="PointSet">
            heatTransferContainer
          </TargetEvaluation>
        <Input class="DataObjects" type="PointSet">
          inputHolder
        </Input>
      </Model>
    </EnsembleModel>
    ...
  </Models>
  ...
</Simulation>
\end{lstlisting}


%
%%%%%%%%%%%%%%%%%%%%%%%%%%%%%%%%%%%%%
%%%%%%  HybridModel  Model   %%%%%%
%%%%%%%%%%%%%%%%%%%%%%%%%%%%%%%%%%%%%
%

\subsection{HybridModel}
\label{subsec:models_HybridModel}
The \textbf{HybridModel} is a new \textit{Model} entity. This new Model is able to combine reduced order model
(ROMs) and any other high-fidelity Model (i.e. Code, ExternalModel). The ROMs will be trained based on the results
from the high-fidelity model. The accuracy of the ROMs will be evaluated based on the cross validation scores,
and the validity of the ROMs will be determined via some local validation metrics (\nb currently only one metric
is available, i.e. CrowdingDistance). After these ROMs are trained, the \textbf{HybridModel} can decide which of
the Model (i.e the ROMs or high-fidelity model) to be executed based on the accuracy and validity of the ROMs.

Currently this model is only able to share information (i.e. data) using \textbf{PointSet}.

The specifications of a HybridModel must be defined within the XML block
\xmlNode{HybridModel}.
%
This XML node needs to contain the attributes:

\vspace{-5mm}
\begin{itemize}
  \itemsep0em
  \item \xmlAttr{name}, \xmlDesc{required string attribute}, user-defined name
  of this HybridModel.
  %
  \nb As with the other objects, this is the name that can be used to refer to
  this specific entity from other input blocks in the XML.
  \item \xmlAttr{subType}, \xmlDesc{required string attribute}, must be kept
  empty.
  %
\end{itemize}
\vspace{-5mm}

Within the \xmlNode{HybridModel} XML node, the multiple entities that constitute
this HybridModel needs to be inputted. 

\begin{itemize}
  \item \xmlNode{Model}, \xmlDesc{XML node, required parameter}.
  %
  The text portion of this node needs to contain the name of the Model
  %
  \assemblerAttrDescription{Model}
  %
  \item \xmlNode{ROM}, \xmlDesc{XML node, required parameter}.
  %
  The text portion of this node needs to contain the name of the ROM
  The user can specify as many \xmlNode{ROM} as required by the \xmlNode{Model}.
  \nb The outputs of each ROM should be different, and the total set of ROMs' outputs
  should be the same as the set of \textit{Model's} outputs.
  %
  \assemblerAttrDescription{Model}
  %
  \item \xmlNode{CV}, \xmlDesc{XML node, required parameter}.
  %
    The text portion of this node needs to contain the name of the \xmlNode{PostProcessor} with \xmlAttr{subType}
    ``CrossValidation``.
  %
    \assemblerAttrDescription{Model}
  % 
  \item \xmlNode{TargetEvaluation}, \xmlDesc{XML node, required parameter}.
  %
    The text portion of this node needs to contain the name of a data object defined in the \xmlNode{DataObjects} block.
    \nb currently only accept data object with type ``PointSet``. The \xmlNode{TargetEvaluation} is primary used for
    training ROMs. \nb The linked DataObject should be placed as additional output of the Step where the
    \textbf{HybridModel} is used.
  %
    \assemblerAttrDescription{DataObjects}
  %
\end{itemize}

An additional XML sub-node within the main \xmlNode{HybridModel} XML node needs to be specified:
\begin{itemize}
  \item \xmlNode{settings}, \xmlDesc{XML node, optional parameter}.
  %
  The body of this sub-node  contains the following XML sub-nodes:
  %
  \begin{itemize}
     \item \xmlNode{minInitialTrainSize}, \xmlDesc{integer, optional field}, the minimum initial number of high-fidelity
       model runs before starting train the ROMs.
       \default{10};
     \item \xmlNode{tolerance}, \xmlDesc{float, optional field}, ROMs convergence criterion indicates the displacement
       from the optimum results of cross validation. In other words, small tolerance indicates tight convergence criterion
       of the ROMs, while large tolerance indicates loose convergence criterion of the ROMs.
       \nb Currently, this tolerance can be only used for cross validations with SKL Metrics: \textit{explained\_variance\_score},
       \textit{r2\_score}, \textit{median\_absolute\_error}, \textit{mean\_squared\_error} and \textit{mean\_absolute\_error}.
       \default{0.01};
     \item \xmlNode{maxTrainSize}, \xmlDesc{XML node, optional field}, the maximum size of training set of ROMs.
       \default{1.0E6}
  \end{itemize}
  \item \xmlNode{validationMethod}, \xmlDesc{XML node, optional parameter}.
  %
  The validity methods that are used to determine which model to run (i.e. ROMs or high-fidelity Model).
  This XML node needs to contain the attributes:
  %
  \begin{itemize}
    \itemsep0em
    \item \xmlAttr{name}, \xmlDesc{required string attribute}, user-defined name
      of this \xmlNode{validationMethod}.
      \nb Currently, only one method is available, ie. ``CrowdingDistance``.
  \end{itemize}
  %
  The body of this sub-node  contains the following XML sub-nodes:
  %
  \begin{itemize}
     \item \xmlNode{threshold}, \xmlDesc{XML node, required field}, the threshold that is used for ``CrowdingDistance`` method.
  \end{itemize}
\end{itemize}


\textbf{Example (ExternalModel):}
\begin{lstlisting}[style=XML,morekeywords={subType,debug,name,class,type}]
<Simulation>
  ...
  <Metrics>
    <SKL name="m1">
      <metricType>mean_absolute_error</metricType>
    </SKL>
  </Metrics>

  <Models>
    <ExternalModel ModuleToLoad="EM2linear" name="thermalConductivityComputation" subType="">
      <variables>leftTemperature,rightTemperature,k,averageTemperature</variables>
    </ExternalModel>
    <ROM name="knr" subType="SciKitLearn">
      <SKLtype>neighbors|KNeighborsRegressor</SKLtype>
      <Features>leftTemperature, rightTemperature</Features>
      <Target>k</Target>
      <n_neighbors>5</n_neighbors>
      <weights>uniform</weights>
      <algorithm>auto</algorithm>
      <leaf_size>30</leaf_size>
      <metric>minkowski</metric>
      <p>2</p>
    </ROM>
    <PostProcessor name="pp1" subType="CrossValidation">
        <SciKitLearn>
            <SKLtype>KFold</SKLtype>
            <n_splits>10</n_splits>
            <shuffle>False</shuffle>
            <random_state>None</random_state>
        </SciKitLearn>
        <Metric class="Metrics" type="SKL">m1</Metric>
    </PostProcessor>
    <HybridModel name="hybrid" subType="">
        <Model class="Models" type="ExternalModel">thermalConductivityComputation</Model>
        <ROM class="Models" type="ROM">knr</ROM>
        <TargetEvaluation class="DataObjects" type="PointSet">thermalConductivityComputationContainer</TargetEvaluation>
        <CV class="Models" type="PostProcessor">pp1</CV>
        <settings>
            <tolerance>0.01</tolerance>
            <trainStep>1</trainStep>
            <maxTrainSize>1000</maxTrainSize>
            <initialTrainSize>10</initialTrainSize>
        </settings>
        <validationMethod name="CrowdingDistance">
            <threshold>0.2</threshold>
        </validationMethod>
    </HybridModel>
  </Models>
  ...
</Simulation>

\end{lstlisting}

\textbf{Example (Code):}
\begin{lstlisting}[style=XML,morekeywords={subType,debug,repeat,name,class,type}]
<Simulation>
  ...
  <Metrics>
    <SKL name="m1">
      <metricType>mean_absolute_error</metricType>
    </SKL>
  </Metrics>

  <Models>
    <Code name="poly" subType="GenericCode">
      <executable>runCode/poly_inp_io.py</executable>
      <clargs arg="python" type="prepend"/>
      <clargs arg="-i" extension=".one" type="input"/>
      <fileargs arg="aux" extension=".two" type="input"/>
      <fileargs arg="output" type="output"/>
      <prepend>python</prepend>
    </Code>
    <ROM name="knr" subType="SciKitLearn">
      <SKLtype>neighbors|KNeighborsRegressor</SKLtype>
      <Features>x, y</Features>
      <Target>poly</Target>
      <n_neighbors>5</n_neighbors>
      <weights>uniform</weights>
      <algorithm>auto</algorithm>
      <leaf_size>30</leaf_size>
      <metric>minkowski</metric>
      <p>2</p>
    </ROM>
    <PostProcessor name="pp1" subType="CrossValidation">
        <SciKitLearn>
            <SKLtype>KFold</SKLtype>
            <n_splits>10</n_splits>
            <shuffle>False</shuffle>
            <random_state>None</random_state>
        </SciKitLearn>
        <Metric class="Metrics" type="SKL">m1</Metric>
    </PostProcessor>
    <HybridModel name="hybrid" subType="">
        <Model class="Models" type="Code">poly</Model>
        <ROM class="Models" type="ROM">knr</ROM>
        <TargetEvaluation class="DataObjects" type="PointSet">samples</TargetEvaluation>
        <CV class="Models" type="PostProcessor">pp1</CV>
        <settings>
            <tolerance>0.1</tolerance>
            <trainStep>1</trainStep>
            <maxTrainSize>1000</maxTrainSize>
            <initialTrainSize>10</initialTrainSize>
        </settings>
        <validationMethod name="CrowdingDistance">
            <threshold>0.2</threshold>
        </validationMethod>
    </HybridModel>
  </Models>
  ...
  <Steps>
    <MultiRun name="hybridModelCode">
      <Input class="Files" type="">gen.one</Input>
      <Input class="Files" type="">gen.two</Input>
      <Input class="DataObjects" type="PointSet">inputHolder</Input>
      <Model class="Models" type="HybridModel">hybrid</Model>
      <Sampler class="Samplers" type="Stratified">LHS</Sampler>
      <Output class="DataObjects" type="PointSet">samples</Output>
      <Output class="OutStreams" type="Print">samples</Output>
    </MultiRun>
  </Steps>
  ...
</Simulation>
\end{lstlisting}
%
\nb For this example, the user needs to provide all the inputs for the \textbf{HybridModel}, i.e. Files for the
\textbf{Code} and DataObject for the \textbf{ROM} defined in the \textbf{HybridModel}.

\input{functions.tex}
\input{metrics.tex}
\input{step.tex}
\section{Existing Interfaces}
\label{sec:existingInterface}
%%%%%%%%%%%%%%%%%%%%%%%%%%%
%%%%%% Generic  INTERFACE  %%%%%%
%%%%%%%%%%%%%%%%%%%%%%%%%%%
\subsection{Generic Interface}
\label{subsec:genericInterface}
The GenericCode interface is meant to handle a wide variety of generic codes
that take take straightforward input files and produce output CSV files.  There are
some limitations for this interface.
If a code: \vspace{-20pt}
\begin{itemize}
\item accepts a keyword-based input file with no cross-dependent inputs,
\item has no more than one filetype extension per command line flag,
\item and returns a CSV with the input parameters and output parameters,
\end{itemize}\vspace{-20pt}
the GenericCode interface should cover the code for RAVEN.

If a code contains cross-dependent data, the generic interface is not able to
edit the correct values.  For example, if a geometry-building script specifies
inner\_radius, outer\_radius, and thickness, the generic interface cannot
calculate the thickness given the outer and inner radius, or vice versa.
In this case, the \textit{function} method explained in the Samplers (see \ref{sec:Samplers})
and Optimizers (see \ref{sec:Optimizers}) sections can be used.

 An example of the code interface is shown here.  The input parameters are read
 from the input files \texttt{gen.one} and \texttt{gen.two} respectively.
 The code is run using \texttt{python}, so that is part of the \xmlNode{prepend} node.
 The command line entry to normally run the code is
\begin{lstlisting}[language=bash]
python poly_inp.py -i gen.one -a gen.two -o myOut
\end{lstlisting}
and produces the output \texttt{myOut.csv}.

Example:
\begin{lstlisting}[style=XML]
    <Code name="poly" subType="GenericCode">
      <executable>GenericInterface/poly_inp.py</executable>
      <inputExtentions>.one,.two</inputExtentions>
      <clargs type='prepend' arg='python'/>
      <clargs type='input'   arg='-i' extension='.one'/>
      <clargs type='input'   arg='-a' extension='.two'/>
      <clargs type='output'  arg='-o'/>
      <prepend>python</prepend>
    </Code>
\end{lstlisting}

If a code doesn't accept necessary Raven-editable auxiliary input files
or output filenames through the command line, the GenericCode interface
can also edit the input files and insert the filenames there.  For example,
in the previous example, say instead of \texttt{-a gen.two} and \texttt{-o myOut}
in the command line, \texttt{gen.one} has the following lines:
\begin{lstlisting}[language=python]
...
auxfile = gen.two
case = myOut
...
\end{lstlisting}
Then, our example XML for the code would be

Example:
\begin{lstlisting}[style=XML]
    <Code name="poly" subType="GenericCode">
      <executable>GenericInterface/poly_inp.py</executable>
      <inputExtentions>.one,.two</inputExtentions>
      <clargs   type='prepend' arg='python'/>
      <clargs   type='input'   arg='-i'  extension='.one'/>
      <fileargs type='input'   arg='two' extension='.two'/>
      <fileargs type='output'  arg='out'/>
      <prepend>python</prepend>
    </Code>
\end{lstlisting}
and the corresponding template input file lines would be changed to read
\begin{lstlisting}[language=python]
...
auxfile = $RAVEN-two$
case = $RAVEN-out$
...
\end{lstlisting}


%%%%
If a code has hard-coded output file names that are not changeable,
the GenericCode interface can be invoked using the \xmlNode{outputFile}
node in which the output file name (CSV only) must be specified.
For example, in the previous example, say instead of \texttt{-a gen.two} and \texttt{-o myOut}
in the command line, the code always produce a CSV file named ``fixed\_output.csv'';

Then, our example XML for the code would be

Example:
\begin{lstlisting}[style=XML]
    <Code name="poly" subType="GenericCode">
      <executable>GenericInterface/poly_inp.py</executable>
      <inputExtentions>.one,.two</inputExtentions>
      <clargs   type='prepend' arg='python'/>
      <clargs   type='input'   arg='-i'  extension='.one'/>
      <fileargs type='input'   arg='two' extension='.two'/>
      <outputFile>fixed_output.csv</outputFile>
      <prepend>python</prepend>
    </Code>
\end{lstlisting}

In addition, the ``wild-cards'' above can contain two special and optional symbols:
\begin{itemize}
  \item  \texttt{:}, that defines an eventual default value;
  \item  \texttt{|}, that defines the format of the value. The  Generic Interface currently supports the following formatting options (* in the examples means blank space):
    \begin{itemize}
       \item \textbf{plain integer}, in this case  the value that is going to be replaced by the Generic Interface, will be left-justified with a string length equal to the integer value specified here (e.g. ``\texttt{|}6'', the value is left-justified with a string length of 6);
      \item \textbf{d}, signed integer decimal, the value is going to be formatted as an integer (e.g.  if the value is 9 and the format ``\texttt{|}10d'', the replaced value will be formatted as follows: ``*********9'');
      \item \textbf{e}, floating point exponential format (lowercase), the value is going to be formatted as a float in scientific notation (e.g. if the value is 9.1234 and the format ``\texttt{|}10.3e'', the replaced value will be formatted as follows: ``*9.123e+00'' );
      \item \textbf{E}, floating point exponential format (uppercase), the value is going to be formatted as a float in scientific notation (e.g. if the value is 9.1234 and the format ``\texttt{|}10.3E'', the replaced value will be formatted as follows: ``*9.123E+00'' );
      \item \textbf{f or F}, floating point decimal format, the value is going to be formatted as a float in decimal notation (e.g. if the value is 9.1234 and the format ``\texttt{|}10.3f'', the replaced value will be formatted as follows: ``*****9.123'' );
      \item \textbf{g}, floating point format. Uses lowercase exponential format if exponent is less than -4 or not less than precision, decimal format otherwise (e.g. if the value is 9.1234 and the format ``\texttt{|}10.3g'', the replaced value will be formatted as follows: ``******9.12'' );
      \item \textbf{G}, floating point format. Uses uppercase exponential format if exponent is less than -4 or not less than precision, decimal format otherwise (e.g. if the value is 0.000009 and the format ``\texttt{|}10.3G'', the replaced value will be formatted as follows: ``*****9E-06'' ).
    \end{itemize}|
\end{itemize}
For example:
\begin{lstlisting}[language=python]
...
auxfile = $RAVEN-two:3$
case = $RAVEN-out:5|10$
...
\end{lstlisting}
Where,
\begin{itemize}
  \item  \texttt{:}, in case the variable ``two'' is not defined in the RAVEN XML input file, the Parser, will replace it with the value ``3''.;
  \item  \texttt{|}, the value that is going to be replaced by the Generic Interface, will be left- justified with a string length of ``10'';
\end{itemize}

%%%%%%%%%%%%%%%%%%%%%%%%%%%%%%%%%%%%%
%%%%%% RAVEN  INTERFACE  (RAVEN running RAVEN) %%%%%%
%%%%%%%%%%%%%%%%%%%%%%%%%%%%%%%%%%%%%
\subsection{RAVEN Interface}
\label{subsec:RAVENInterface}
The RAVEN interface is meant to provide the possibility to execute a RAVEN input file
driving a set of SLAVE RAVEN calculations. For example, if the user wants to optimize the parameters
of a surrogate model (e.g. minimizing the distance between the surrogate predictions and the real data), he
can achieve this task by setting up  a RAVEN input file (master) that performs an optimization on the feature
space characterized by the surrogate model parameters, whose training and validation assessment  is performed in the SLAVE
RAVEN runs.
\\ There are some limitations for this interface:
\begin{itemize}
\item only one  sub-level of RAVEN can be executed (i.e. if the SLAVE RAVEN input file contains the run of another RAVEN SLAVE, the MASTER RAVEN will error out)
\item only data from Outstreams of type Print can be collected by the MASTER RAVEN
\item only a maximum of two Outstreams can be collected (1 PointSet and 1 HistorySet)
\end{itemize}


Like for every other interface, most of the RAVEN workflow stays the same independently of which type of Model (i.e. Code) is used.
\\ Similarly to any other code interface, the user provides paths to executables and aliases for sampled variables within the
\xmlNode{Models} block.  The \xmlNode{Code} block will contain attributes \xmlAttr{name} and
\xmlAttr{subType}.  \xmlAttr{name} identifies that particular \xmlNode{Code} model within RAVEN, and
\xmlAttr{subType} specifies which code interface the model will use (In this case \xmlAttr{subType}=``RAVEN'').
The \xmlNode{executable}
block should contain the absolute or relative (with respect to the current working
directory) path to the RAVEN framework script (\textbf{raven\_framework}).
\\ In addition to the attributes and xml nodes reported above, the RAVEN accepts the following XML nodes (required and optional):
\begin{itemize}
  \item  \xmlNode{outputExportOutStreams}, \xmlDesc{comma separated list,
  required parameter} will specify the  \xmlNode{OutStreams} that will be loaded as outputs of the SLAVE RAVEN.
  Maximum two  \xmlNode{OutStreams} can be listed here (1 for PointSet and/or 1 for HistorySet).
  \item  \xmlNode{conversionModule}, \xmlDesc{string,
  optional parameter} will specify the path to a \textit{Python} module that can contain two methods:
      \begin{itemize}
         \item \textbf{\textit{manipulateScalarSampledVariables}}, a method that is aimed to manipulate sampled variables and to create more in case needed.
         Example:
          \begin{lstlisting}[language=python]
def manipulateScalarSampledVariables(sampledVariables):
  """
  This method is aimed to manipulate scalar variables.
  The user can create new variables based on the
  variables sampled by RAVEN
   @ In, sampledVariables, dict, dictionary of
       sampled variables ({"var1":value1,"var2":value2})
   @ Out, None, the new variables should be
                 added in the "sampledVariables" dictionary
  """
  newVariableValue =
    sampledVariables['Distributions|Uniform@name:a_dist|lowerBound']
    + 1.0
  sampledVariables['Distributions|Uniform@name:a_dist|upperBound'] =
    newVariableValue
  return
           \end{lstlisting}

         \item \textbf{\textit{convertNotScalarSampledVariables}}, a method that is aimed to convert not scalar variables (e.g. 1D arrays) into multiple scalar variables
         (e.g.  \xmlNode{constant}(s) in a sampling strategy).
          This method is going to be required in case not scalar variables are detected by the interface.
          Example:
          \begin{lstlisting}[language=python]
 def convertNotScalarSampledVariables(noScalarVariables):
  """
  This method is aimed to convert not scalar
  variables into multiple scalar variables. The user MUST
   create new variables based on the not Scalar Variables
    sampled (and passed in) by RAVEN
  @ In, noScalarVariables, dict, dictionary of sampled
       variables that are not scalar ({"var1":1Darray1,"var2":1Darray2})
  @ Out, newVars, dict,  the new variables that have
       been created based on the not scalar variables
       contained in "noScalarVariables" dictionary
  """
  oneDimensionalArray =
      noScalarVariables['temperatureHistory']
  newVars = {}
  for cnt, value in enumerate(oneDimensionalArray):
    newVars['Samplers|MonteCarlo@name:myMC|constant'+
               '@name=temperatureHistory'+str(cnt)] =
               oneDimensionalArray[cnt]
  return newVars
           \end{lstlisting}
      \end{itemize}
\end{itemize}

Code input example:
\begin{lstlisting}[style=XML]
    <Code name="RAVENrunningRAVEN" subType="RAVEN">
      <executable>../../../raven_framework</executable>
      <outputExportOutStreams>
         HistorySetOutStream,PointSetOutStream
      </outputExportOutStreams>
      <conversionModule>
        ~/Users/username/whateverConversionModule.py
      </conversionModule>
    </Code>
\end{lstlisting}

Like for every other interface,  the syntax of the variable names is important to make the parser understand how to perturb an input file.
\\ For the RAVEN interface, a syntax inspired by the XPath nomenclature is used.
\begin{lstlisting}[style=XML]
  <Samplers>
    <MonteCarlo name="MC_external">
       ...
      <variable name="Models|ROM@subType:SciKitLearn@name:ROM1|C">
        <distribution>C_distrib</distribution>
      </variable>
      <variable name="Models|ROM@subType:SciKitLearn@name:ROM1|tol">
        <distribution>toll_distrib</distribution>
      </variable>
      <variable name="Samplers|Grid@name:'+
            'GridName|variable@name:var1|grid@construction:equal@type:value@steps">
        <distribution>categorical_step_distrib</distribution>
      </variable>
      ...
    </MonteCarlo>
  </Samplers>
\end{lstlisting}
In the above example, it can be inferred that each XML node (subnode) needs to be separated by a ``|'' separator. In addition,
every time an XML node has attributes, the user can specify them using the ``@'' separator to specify a value for them.
The first variable above will be pointing to the following XML sub-node ( \xmlNode{C}):
\begin{lstlisting}[style=XML]
      <Models>
        <ROM name="ROM1" subType="SciKitLearn">
           ...
           <C>10.0</C>
          ...
       </ROM>
      </Models>
\end{lstlisting}
The second variable above will be pointing to the following XML sub-node ( \xmlNode{tol}):
\begin{lstlisting}[style=XML]
      <Models>
        <ROM name="ROM1" subType="SciKitLearn">
           ...
           <tol>0.0001</tol>
          ...
       </ROM>
      </Models>
\end{lstlisting}
The third variable above will be pointing to the following XML attribute ( \xmlAttr{steps}):
\begin{lstlisting}[style=XML]
  <Samplers>
    <Grid name="GridName">
       ...
      <variable name="var1">
         ...
        <grid construction="equal" type="value" steps="1">0 1</grid>
         ...
      </variable>

      ...
    </MonteCarlo>
  </Samplers>
\end{lstlisting}

The above nomenclature must be used for all the variables to be sampled and for the variables generated by the two methods contained, in case, in
the module that gets specified by the \xmlNode{conversionModule} in the \xmlNode{Code} section.
\\ Finally the SLAVE RAVEN input file (s) must be ``tagged'' with the attribute  \xmlAttr{type="raven"} in the Files section. For example,

\begin{lstlisting}[style=XML]
<Files>
    <Input name="slaveRavenInputFile" type="raven" >
      test_rom_trainer.xml
    </Input>
</Files>
\end{lstlisting}

\subsubsection{ExternalXML and RAVEN interface}
Care must be taken if the SLAVE RAVEN uses \xmlNode{ExternalXML} nodes.  In this case, each file containing
external XML nodes must be added in the \xmlNode{Step} as an \xmlNode{Input} class \xmlAttr{Files} to make sure it gets copied to
the individual run directory.  The type for these files can be anything, with the exception of type
\xmlString{raven}.

%%%%%%%%%%%%%%%%%%%%%%%%%%%%
%%%%%% RELAP5  INTERFACE  %%%%%%
%%%%%%%%%%%%%%%%%%%%%%%%%%%%
\subsection{RELAP5 Interface}
\label{subsec:RELAP5Interface}

\subsubsection{Sequence}
In the \xmlNode{Sequence} section, the names of the steps declared in the
\xmlNode{Steps} block should be specified.
%
As an example, if we called the first multirun ``Grid\_Sampler'' and the second
multirun ``MC\_Sampler'' in the sequence section we should see this:
\begin{lstlisting}[style=XML]
<Sequence>Grid_Sampler,MC_Sampler</Sequence>
\end{lstlisting}
%%%%%%%%%%%%%%%%%%%%%%%%%%%%%%%%%%%%%%%%%%%%%%%%%%%

\subsubsection{batchSize and mode}
For the \xmlNode{batchSize} and \xmlNode{mode} sections please refer to the
\xmlNode{RunInfo} block in the previous chapters.
%
%%%%%%%%%%%%%%%%%%%%%%%%%%%%%%%%%%%%%%%%%%%%%%%%%%%%
\subsubsection{RunInfo}
After all of these blocks are filled out, a standard example RunInfo block may
look like the example below:
\begin{lstlisting}[style=XML]
<RunInfo>
  <WorkingDir>~/workingDir</WorkingDir>
  <Sequence>Grid_Sampler,MC_Sampler</Sequence>
  <batchSize>1</batchSize>
  <mode>mpi</mode>
  <expectedTime>1:00:00</expectedTime>
  <ParallelProcNumb>1</ParallelProcNumb>
</RunInfo>
\end{lstlisting}
%%%%%%%%%%%%%%%%%%%%%%%%%%%%%%%%%%%%%%%%%%%%%%%%%%%%%%%%%%%
\subsubsection{Files}
In the \xmlNode{Files} section, as specified before, all of the files needed for
the code to run should be specified.
%
In the case of RELAP5, the files typically needed are:
\begin{itemize}
  \item RELAP5 Input file
  \item Table file or files that RELAP needs to run
\end{itemize}
Example:
\begin{lstlisting}[style=XML]
<Files>
  <Input name='tpfh2o' type=''>tpfh2o</Input>
  <Input name='inputrelap.i' type=''>X10.i</Input>
</Files>
\end{lstlisting}

It is a good practice to put inside the working directory all of these files and
also:
\begin{itemize}
  \item the RAVEN input file
  \item the license for the executable of RELAP5
\end{itemize}
\textcolor{red}{
\textbf{It is important to notice that the interface output collection relies on the MINOR EDITS. The user must specify the MINOR
EDITS block and those variables are the only one the INTERFACE will read and make available to RAVEN. In addition, it is important to notice that:}
\begin{itemize}
  \item \textbf{the simulation time is stored in a variable called \textit{``time''}};
  \item \textbf{all the variables specified in the MINOR EDIT block are going to be converted using underscores (e.g.  an edit such as
  $301 \:\:\: p \:\:\: 345010000$ will be named in the converted CSVs as $p\_345010000$).In addition, if a variable contains spaces, the trailing spaces
   are going to be removed and internal spaces are replaced with underscores (e.g. $HTTEMP 1131008 12$ will become $HTTEMP\_1131008\_12$}.
\end{itemize}
}

Remeber also that a RELAP5 simulation run is considered successful (i.e., the simulation did not crash) if it terminates with the following
message:
\textcolor{red}{Transient terminated by end of time step cards}
or
\textcolor{red}{Transient terminated by trip}

If the a RELAP5 simulation run stops with messages other than this one (e.g., `` Transient terminated by failure.'') than the simulation is considered as
crashed, i.e., it will not be saved.
Hence, it is strongly recommended to set up the RELAP5 input file so that the simulation exiting conditions are set through control logic trip variables
(e.g., simulation mission time and clad temperature equal to clad failure temperature).

%%%%%%%%%%%%%%%%%%%%%%%%%%%%%%%%%%%%%%%%%%%%%%%%%%%%
\subsubsection{Models}
For the \xmlNode{Models} block here is a standard example of how it would look
when using RELAP5 as the external model:
\begin{lstlisting}[style=XML]
<Models>
  <Code name='MyRELAP' subType='Relap5'>
    <executable>~/path_to_the_executable</executable>
  </Code>
</Models>
\end{lstlisting}
In case the \textbf{multi-deck} approach is used in RELAP5, the interface is going to load all the outputs in one CSV RAVEN is
going to read. This means that all the decks' outputs are going to be loaded in one of the Output of RAVEN. In case the user
wants to select the outputs coming from only one deck, the following XML node needs to be specified:
\begin{itemize}
   \item \xmlNode{outputDeckNumber}, \xmlDesc{integer, optional parameter}, the deck number from
   which the results needs to be retrieved. \default{all}.
\end{itemize}
In addition, if some command line parameters need to be passed to RELAP5 \\(e.g. ``-r
$\: restartFileWithCustomName.r$''), the user might use (optionally) the \xmlNode{clargs} XML nodes.
\begin{lstlisting}[style=XML]
<Models>
  <Code name='MyRELAP' subType='Relap5'>
    <executable>~/path_to_the_executable</executable>
    <outputDeckNumber>1</outputDeckNumber>
    <clargs type="text" arg="-r restartFileWithCustomName.r"/>
  </Code>
</Models>
\end{lstlisting}

%%%%%%%%%%%%%%%%%%%%%%%%%%%%%%%%%%%%%%%%%%%%%%%%%%%%%%%%%
\subsubsection{Distributions}
The \xmlNode{Distribution} block defines the distributions that are going
to be used for the sampling of the variables defined in the \xmlNode{Samplers}
block.
%
For all the possibile distributions and all their possible inputs please see the
chapter about Distributions (see~\ref{sec:distributions}).
%
Here we give a general example of three different distributions:
\begin{lstlisting}[style=XML,morekeywords={name,debug}]
<Distributions verbosity='debug'>
  <Triangular name='BPfailtime'>
    <apex>5.0</apex>
    <min>4.0</min>
    <max>6.0</max>
  </Triangular>
  <LogNormal name='BPrepairtime'>
    <mean>0.75</mean>
    <sigma>0.25</sigma>
  </LogNormal>
  <Uniform name='ScalFactPower'>
    <lowerBound>1.0</lowerBound>
    <upperBound>1.2</upperBound>
  </Uniform>
 </Distributions>
\end{lstlisting}

It is good practice to name the distribution something similar to what kind of
variable is going to be sampled, since there might be many variables with the
same kind of distributions but different input parameters.
%
%%%%%%%%%%%%%%%%%%%%%%%%%%%%%%%%%%%%%%%%%%%%%%%%%%%%%%%%%
\subsubsection{Samplers}
In the \xmlNode{Samplers} block we want to define the variables that are going
to be sampled.
%
\textbf{Example}:
We want to do the sampling of 3 variables:
\begin{itemize}
  \item Battery Fail Time
  \item Battery Repair Time
  \item Scaling Factor Power Rate
\end{itemize}

We are going to sample these 3 variables using two different sampling methods:
grid and MonteCarlo.

In RELAP5, the sampler reads the variable as, given the name, the first number
is the card number and the second number is the word number.
%
In this example we are sampling:
\begin{itemize}
  \item For card 0000588 (trip) the word 6 (battery failure time)
  \item For card 0000575 (trip) the word 6 (battery repair time)
  \item For card 20210000 (reactor power) the word 4 (reactor scaling factor)
\end{itemize}

We proceed to do so for both the Grid sampling and the MonteCarlo sampling.

\begin{lstlisting}[style=XML,morekeywords={name,type,construction,lowerBound,steps,limit,initialSeed}]
<Samplers verbosity='debug'>
  <Grid name='Grid_Sampler' >
    <variable name='0000588:6'>
      <distribution>BPfailtime</distribution>
      <grid type='value' construction='equal'  steps='10'>0.0 28800</grid>
    </variable>
    <variable name='0000575:6'>
      <distribution>BPrepairtime</distribution>
      <grid type='value' construction='equal' steps='10'>0.0 28800</grid>
    </variable>
    <variable name='20210000:4'>
      <distribution>ScalFactPower</distribution>
      <grid type='value' construction='equal' steps='10'>1.0 1.2</grid>
    </variable>
  </Grid>
  <MonteCarlo name='MC_Sampler'>
     <samplerInit>
       <limit>1000</limit>
     </samplerInit>
    <variable name='0000588:6'>
      <distribution>BPfailtime</distribution>
    </variable>
    <variable name='0000575:6'>
      <distribution>BPrepairtime</distribution>
    </variable>
    <variable name='20210000:4'>
      <distribution>ScalFactPower</distribution>
    </variable>
  </MonteCarlo>
</Samplers>
\end{lstlisting}

In case the RELAP5 input file is a multi-deck, the user can specify the deck to which each sampled variable
corresponds to. As an example, the following sampling strategy:

\begin{lstlisting}[style=XML,morekeywords={name,type,construction,lowerBound,steps,limit,initialSeed}]
<MonteCarlo name='MC_Sampler'>
   <samplerInit>
     <limit>1000</limit>
   </samplerInit>
  <variable name='1|0000588:6'>
    <distribution>BPfailtime</distribution>
  </variable>
  <variable name='2|0000575:6'>
    <distribution>BPrepairtime</distribution>
  </variable>
</MonteCarlo>
</Samplers>
\end{lstlisting}
performs:
\begin{itemize}
  \item the sampling of the distribution \\\xmlNode{BPfailtime} and it provides the sampled value
        to the 6th word of card 0000588 for the first deck
  \item the sampling of the distribution \\\xmlNode{BPrepairtime} and it provides the sampled value
        to the 6th word of card 0000575 for the second deck
\end{itemize}

It can be seen that each variable is connected with a proper distribution
defined in the \\\xmlNode{Distributions} block (from the previous example).
%
The following demonstrates how the input for the first variable is read.

We are sampling a a variable situated in word 6 of the card 0000588 using a Grid
sampling method.
%
The distribution that this variable is following is a Triangular distribution
(see section above).
%
We are sampling this variable beginning from 0.0 in 10 \textit{equal} steps of
2880.
%
%%%%%%%%%%%%%%%%%%%%%%%%%%%%%%%%%%%%%%%%%%%%%%%%%%%%%%%%%%%
\subsubsection{Steps}
For a RELAP interface, the \xmlNode{MultiRun} step type will most likely be
used.
%
First, the step needs to be named: this name will be one of the names used in
the \xmlNode{Sequence} block.
%
In our example, \texttt{Grid\_Sampler} and \texttt{MC\_Sampler}.
%
\begin{lstlisting}[style=XML,morekeywords={name,debug,re-seeding}]
     <MultiRun name='Grid_Sampler' verbosity='debug'>
\end{lstlisting}

With this step, we need to import all the files needed for the simulation:
\begin{itemize}
  \item RELAP input file
  \item element tables -- tpfh2o
\end{itemize}
\begin{lstlisting}[style=XML,morekeywords={name,class,type}]
    <Input   class='Files' type=''>inputrelap.i</Input>
    <Input   class='Files' type=''>tpfh2o</Input>
\end{lstlisting}
We then need to define which model will be used:
\begin{lstlisting}[style=XML]
    <Model  class='Models' type='Code'>MyRELAP</Model>
\end{lstlisting}
We then need to specify which Sampler is used, and this can be done as follows:
\begin{lstlisting}[style=XML]
    <Sampler class='Samplers' type='Grid'>Grid_Sampler</Sampler>
\end{lstlisting}
And lastly, we need to specify what kind of output the user wants.
%
For example the user might want to make a database (in RAVEN the database
created is an HDF5 file).
%
Here is a classical example:
\begin{lstlisting}[style=XML,morekeywords={class,type}]
    <Output  class='Databases' type='HDF5'>Grid_out</Output>
\end{lstlisting}
Following is the example of two MultiRun steps which use different sampling
methods (grid and Monte Carlo), and creating two different databases for each
one:
\begin{lstlisting}[style=XML]
<Steps verbosity='debug'>
  <MultiRun name='Grid_Sampler' verbosity='debug'>
    <Input   class='Files' type=''>inputrelap.i</Input>
    <Input   class='Files'     type=''    >tpfh2o</Input>
    <Model   class='Models'    type='Code'>MyRELAP</Model>
    <Sampler class='Samplers'  type='Grid'>Grid_Sampler</Sampler>
    <Output  class='Databases' type='HDF5'>Grid_out</Output>
  </MultiRun>
  <MultiRun name='MC_Sampler' verbosity='debug' re-seeding='210491'>
    <Input   class='Files' type=''>inputrelap.i</Input>
    <Input   class='Files'     type=''          >tpfh2o</Input>
    <Model   class='Models'    type='Code'      >MyRELAP</Model>
    <Sampler class='Samplers'  type='MonteCarlo'>MC_Sampler</Sampler>
    <Output  class='Databases' type='HDF5'      >MC_out</Output>
  </MultiRun>
</Steps>
\end{lstlisting}
%%%%%%%%%%%%%%%%%%%%%%%%%%%%%%%%%%%%%%%%%%%%%%%%%%%%%%
\subsubsection{Databases}
As shown in the \xmlNode{Steps} block, the code is creating two database objects
called \texttt{Grid\_out} and \texttt{MC\_out}.
%
So the user needs to input the following:
\begin{lstlisting}[style=XML]
<Databases>
  <HDF5 name="Grid_out" readMode="overwrite"/>
  <HDF5 name="MC_out" readMode="overwrite"/>
</Databases>
\end{lstlisting}
As listed before, this will create two databases.
%
The files will have names corresponding to their \xmlAttr{name} appended with
the .h5 extension (i.e. \texttt{Grid\_out.h5} and \texttt{MC\_out.h5}).

\subsubsection{Modified Version of the Institute of Nuclear Safety System Incorporated (Japan)}
The Institute of Nuclear Safety System Incorporated (Japan) has modified the \textbf{RELAP5}  source code
in order to be able to control some additional parameters from an auxiliary input file (\textbf{modelPar.inp}).
\\In order to use this interface, the user needs to input the $subType$ attribute\textbf{Relap5inssJp}:
\begin{lstlisting}[style=XML]
<Models>
  <Code name='MyRELAP' subType='Relap5'>
    <executable>~/path_to_the_executable</executable>
    <!-- here is taking the output from the first deck only -->
    <outputDeckNumber>1</outputDeckNumber>
  </Code>
</Models>
\end{lstlisting}
For perturbing such input file, the approach presented in section \ref{subsec:genericInterface} (Generic Interface)
has been employed. For the standard \textbf{RELAP5} input, the same approach previously in this section is used.
\\For example, in the following Sampler block, the card $9100101$ is perturbed with the same approach used in standard \textbf{RELAP5}; in addition, the variable $modelParTest$  is going to be perturbed in the \textbf{modelPar.inp} input file.
\begin{lstlisting}[style=XML]
    <MonteCarlo name="mc_loca">
      <samplerInit>
        <limit>1</limit>
      </samplerInit>
      <variable name="9100101:3">
        <distribution>break_size</distribution>
      </variable>
      <variable name="modelParTest">
          <distribution>break_size</distribution>
      </variable>
    </MonteCarlo>
\end{lstlisting}

%%%%%%%%%%%%%%%%%%%%%%%%%%%
%%%%%% RELAP7 INTERFACE  %%%%%%
%%%%%%%%%%%%%%%%%%%%%%%%%%%
\subsection{RELAP7 Interface}
This section covers the input specifications for running RELAP7 through RAVEN.
It is important to notice that this short explanation assumes that the reader already knows
how to use the control logic system in RELAP7.
Since the presence of the control logic system in RELAP7, this code interface is different with respect to the others
and uses some special keyword available in RAVEN (see the following).

\subsubsection{Files}
In the \xmlNode{Files} section, as specified before, all of the files needed for
the code to run should be specified.
%
In the case of RELAP7, the files typically needed are the following:
\begin{itemize}
  \item RELAP7 Input file
  \item Control Logic file
\end{itemize}
Example:
\begin{lstlisting}[style=XML]
<Files>
  <Input name='nat_circ.i' type=''>nat_circ.i</Input>
  <Input name='control_logic.py' type=''>control_logic.py</Input>
</Files>
\end{lstlisting}
The RAVEN/RELAP7 interface recognizes as RELAP7 inputs the files with the extensions  ``*.i'', ``*.inp'' and ``*.in''.

%%%%%%%%%%%%%%%%%%%%%%%%%%%%%%%%%%%%%%%%%%%%%%%%%%%%%%%%
\subsubsection{Models}
For the \xmlNode{Models} block RELAP7 uses the RAVEN executable, since through this executable the stochastic
environment gets activated (possibility to sample parameters directly in the control logic system)
%
Here is a standard example of what can be used to use RELAP7 as the model:
\begin{lstlisting}[style=XML]
<Models>
    <Code name='MyRAVEN' subType='RAVEN'><executable>~path/to/RAVEN-opt</executable></Code>
</Models>
\end{lstlisting}
%%%%%%%%%%%%%%%%%%%%%%%%%%%%%%%%%%%%%%%%%%%%%%%%%%%%%%%%
\subsubsection{Distributions}
As for all the other codes interfaces  the \xmlNode{Distributions} block needs to be specified in order to employ
as sampling strategy (e.g. MonteCarlo, Stratified, etc.). In this block, the user specifies the distributions that need to be used.
Once the user defines the distributions in this block, RAVEN activates the Distribution environment in the RAVEN/RELAP7 control logic
system. The sampling of the parameters is then performed directly in the control logic input file.

%
For example, let's consider the sampling of a normal distribution for the primary pressure in
RELAP7:
%
\begin{lstlisting}[style=XML]
<Distributions>
 <Normal name="Prim_Pres">
 <mean>1000000</mean>
 <sigma>100<sigma/>
 </Normal>
</Distributions>
\end{lstlisting}
In order to change a parameter (independently on the sampling strategy), the control logic input file should be modified as follows:
%\lstset{margin=1.5cm}
\begin{lstlisting}[language=Python]
def initial_function(monitored, controlled, auxiliary)
    print("monitored",monitored,"controlled",
    controlled,"auxiliary",auxiliary)

    controlled.pressureInPressurizer =
     distributions.Prim_Pres.getDistributionRandom()
    return
\end{lstlisting}

%%%%%%%%%%%%%%%%%%%%%%%%%%%%%%%%%%%%%%%%%%%%%%%
\subsubsection{Samplers}
In the \xmlNode{Samplers} block, all the variables that needs to be sampled must be specified.
In case some of these variables are directly sampled in the Control Logic system, the
\xmlNode{variable} needs to be replaced with \xmlNode{Distribution}. In this way, RAVEN is able
to understand which variables needs to be directly modified through input file (i.e. modifying the original
input file *.i)  and which variables are going to be ``sampled'' through the control logic system.
%
For the example, we are performing Grid Sampling.
%
The global initial pressure wasn't specified in the control logic so it is going to be specified
using the node \xmlNode{variable}. The ``pressureInPressurizer'' variable is instead sampled in the
control logic system; for this reason, it is going to be specified using the node  \xmlNode{Distribution}.
%
For example,
%
\begin{lstlisting}[style=XML]
<Samplers>
 <Grid name="MC_samp">
   <samplerInit> <limit>500</limit> </samplerInit>
   <variable name="GlobalParams|global_init_P">
      <distribution>Prim_Pres</distribution>
      <grid construction="equal" steps="10" type="CDF">0.0 1.0</grid>
   </variable>
   <Distribution name="pressureInPressurizer">
      <distribution>Prim_Pres</distribution>
      <grid construction="equal" steps="10" type="CDF">0.0 1.0</grid>
   </Distribution>
 </Grid>
</Samplers>
\end{lstlisting}


%%%%%%%%%%%%%%%%%%%%%%%%%%%%%%%%%%
%%%%%% MooseBasedApp INTERFACE  %%%%%%
%%%%%%%%%%%%%%%%%%%%%%%%%%%%%%%%%%
\subsection{MooseBasedApp Interface}
\subsubsection{Files}
In the \xmlNode{Files} section, as specified before, all of the files needed for
the code to run should be specified.
%
In the case of any MooseBasedApp, the files typically needed are the following:
\begin{itemize}
  \item MooseBasedApp YAML input file
  \item Restart Files (if the calculation is instantiated from a restart point)
\end{itemize}
Example:
\begin{lstlisting}[style=XML]
<Files>
  <Input name='mooseBasedApp.i' type=''>mooseBasedApp.i</Input>
  <Input name='0020_mesh.cpr' type=''>0020_mesh.cpr</Input>
  <Input name='0020.xdr.0000'>0020.xdr.0000</Input>
  <Input name='0020.rd-0'>0020.rd-0</Input>
</Files>
\end{lstlisting}
%%%%%%%%%%%%%%%%%%%%%%%%%%%%%%%%%%%%%%%%%%%%%%%%%%%%%%%%
\subsubsection{Models}
In the \xmlNode{Models} block particular MooseBasedApp executable needs to be specified.
%
Here is a standard example of what can be used to use with a typical MooseBasedApp (Bison) as the model:
\begin{lstlisting}[style=XML]
<Models>
    <Code name='MyMooseBasedApp' subType='MooseBasedApp'><executable>~path/to/Bison-opt</executable></Code>
</Models>
\end{lstlisting}
%%%%%%%%%%%%%%%%%%%%%%%%%%%%%%%%%%%%%%%%%%%%%%%%%%%%%%%%
%%%%%%%%%%%%%%%%%%%%%%%%%%%%%%%%%%%%%%%%%%%%%%%%%%%%%%%%%
\subsubsection{Distributions}
The \xmlNode{Distributions} block defines the distributions that are going
to be used for the sampling of the variables defined in the \xmlNode{Samplers}
block.
%
For all the possible distributions and all their possible inputs please see the
chapter about Distributions (see~\ref{sec:distributions}).
%
Here we give a general example of three different distributions:
\begin{lstlisting}[style=XML,morekeywords={name,debug}]
<Distributions>
    <Normal name='ThermalConductivity1'>
        <mean>1</mean>
        <sigma>0.001</sigma>
        <lowerBound>0.5</lowerBound>
        <upperBound>1.5</upperBound>
    </Normal>
    <Normal name='SpecificHeat'>
        <mean>1</mean>
        <sigma>0.4</sigma>
        <lowerBound>0.5</lowerBound>
        <upperBound>1.5</upperBound>
    </Normal>
    <Triangular name='ThermalConductivity2'>
        <apex>1</apex>
        <min>0.1</min>
        <max>4</max>
    </Triangular>
</Distributions>
\end{lstlisting}

It is good practice to name the distribution something similar to what kind of
variable is going to be sampled, since there might be many variables with the
same kind of distributions but different input parameters.
%
%%%%%%%%%%%%%%%%%%%%%%%%%%%%%%%%%%%%%%%%%%%%%%%%%%%%%%%%%
\subsubsection{Samplers}
In the \xmlNode{Samplers} block we want to define the variables that are going
to be sampled.
%
\textbf{Example}:
We want to do the sampling of 3 variables:
\begin{itemize}
  \item Thermal Conductivity of the Fuel;
  \item Specific Heat Transfer Ratio of the Cladding;
  \item Thermal Conductivity of the Cladding.
\end{itemize}

We are going to sample these 3 variables using two different sampling methods:
Grid and Monte-Carlo.

In order to perturb any MooseBasedApp, the user needs to specify the variables to be
sampled indicating the path to the value separated with the symbol ``$|$''. For example,
if the variable that we want to perturb is specified in the input as follows:
\begin{lstlisting}[style=XML]
[Materials]
  ...
  [./heatStructure]
     ...
     thermal_conductivity = 1.0
     ...
  [../]
  ...
[]
\end{lstlisting}
the variable name in the Sampler input block needs to be named as follows:
\begin{lstlisting}[style=XML]
...
<Samplers>
  <aSampler name='aUserDefinedName' >
    <variable name='Materials|heatStructure|thermal_conductivity'>
      ...
    </variable>
  </aSampler>
</Samplers>
...
\end{lstlisting}
%
In this example, we proceed to do so for both the Grid sampling and the Monte-Carlo sampling.

\begin{lstlisting}[style=XML,morekeywords={name,type,construction,lowerBound,steps,limit,initialSeed}]
<Samplers verbosity='debug'>
    <Grid name='myGrid'>
      <variable name='Materials|heatStructure1|thermal_conductivity' >
        <distribution>ThermalConductivity1</distribution>
        <grid         type='value' construction='custom' >0.6 0.7 0.8</grid>
      </variable>
      <variable name='Materials|heatStructure1|specific_heat' >
        <distribution >SpecificHeat</distribution>
        <grid         type='CDF'    construction='custom'>0.5 1.0 0.0</grid>
      </variable>
      <variable name='Materials|heatStructure2|thermal_conductivity'>
        <distribution  >ThermalConductivity2</distribution>
        <grid type='value' upperBound='4' construction='equal' steps='1'>0.5</grid>
      </variable>
    </Grid>
  <MonteCarlo name='MC_Sampler' limit='1000'>
      <variable name='Materials|heatStructure1|thermal_conductivity' >
        <distribution>ThermalConductivity1</distribution>
      </variable>
      <variable name='Materials|heatStructure1|specific_heat' >
        <distribution >SpecificHeat</distribution>
      </variable>
      <variable name='Materials|heatStructure2|thermal_conductivity'>
        <distribution  >ThermalConductivity2</distribution>
      </variable>
  </MonteCarlo>
</Samplers>
\end{lstlisting}
%%%%%%%%%%%%%%%%%%%%%%%%%%%%%%%%%%%%%%%%%%%%%%%%%%%%%%%%%%%
\subsubsection{Steps}
For a MooseBasedApp, the \xmlNode{MultiRun} step type will most likely be
used, as first step.
%
First, the step needs to be named: this name will be one of the names used in
the \xmlNode{Sequence} block.
%
In our example, \texttt{Grid\_Sampler} and \texttt{MC\_Sampler}.
%
\begin{lstlisting}[style=XML,morekeywords={name,debug,re-seeding}]
     <MultiRun name='Grid_Sampler' >
\end{lstlisting}

With this step, we need to import all the files needed for the simulation:
\begin{itemize}
  \item MooseBasedApp YAML input file;
  \item eventual restart files (optional);
  \item other auxiliary files (e.g., powerHistory tables, etc.).
\end{itemize}
\begin{lstlisting}[style=XML,morekeywords={name,class,type}]
    <Input   class='Files' type=''>mooseBasedApp.i</Input>
    <Input   class='Files' type=''>0020_mesh.cpr</Input>
    <Input   class='Files' type=''>0020.xdr.0000</Input>
    <Input   class='Files' type=''>0020.rd-0</Input>
\end{lstlisting}
We then need to define which model will be used:
\begin{lstlisting}[style=XML]
    <Model  class='Models' type='Code'>MyMooseBasedApp</Model>
\end{lstlisting}
We then need to specify which Sampler is used, and this can be done as follows:
\begin{lstlisting}[style=XML]
    <Sampler class='Samplers' type='Grid'>Grid_Sampler</Sampler>
\end{lstlisting}
And lastly, we need to specify what kind of output the user wants.
%
For example the user might want to make a database (in RAVEN the database
created is an HDF5 file) and a DataObject of type PointSet, to use in sub-sequential
post-processing.
%
Here is a classical example:
\begin{lstlisting}[style=XML,morekeywords={class,type}]
    <Output  class='Databases' type='HDF5'>MC_out</Output>
    <Output  class='DataObjects' type='PointSet'>MCOutData</Output>
\end{lstlisting}

Following is the example of two MultiRun steps which use different sampling
methods (grid and Monte Carlo), and creating two different databases for each
one:
\begin{lstlisting}[style=XML]
<Steps verbosity='debug'>
  <MultiRun name='Grid_Sampler' verbosity='debug'>
    <Input  class='Files' type=''>mooseBasedApp.i</Input>
    <Input  class='Files' type=''>0020_mesh.cpr</Input>
    <Input  class='Files' type='' >0020.xdr.0000</Input>
    <Input  class='Files' type=''>0020.rd-0</Input>
    <Model  class='Models'    type='Code'>MyMooseBasedApp</Model>
    <Sampler class='Samplers'  type='Grid'>Grid_Sampler</Sampler>
    <Output  class='Databases' type='HDF5'>Grid_out</Output>
    <Output  class='DataObjects' type='PointSet'>gridOutData</Output>
  </MultiRun>
  <MultiRun name='MC_Sampler' verbosity='debug' re-seeding='210491'>
    <Input  class='Files' type=''>mooseBasedApp.i</Input>
    <Input  class='Files' type=''>0020_mesh.cpr</Input>
    <Input  class='Files' type='' >0020.xdr.0000</Input>
    <Input  class='Files' type=''>0020.rd-0</Input>
    <Model  class='Models'    type='Code'>MyMooseBasedApp</Model>
    <Sampler class='Samplers'  type='MonteCarlo' >MC_Sampler</Sampler>
    <Output  class='Databases' type='HDF5' >MC_out</Output>
    <Output  class='DataObjects' type='PointSet'>MCOutData</Output>
  </MultiRun>
</Steps>
\end{lstlisting}
%%%%%%%%%%%%%%%%%%%%%%%%%%%%%%%%%%%%%%%%%%%%%%%%%%%%%%
\subsubsection{Databases}
As shown in the \xmlNode{Steps} block, the code is creating two database objects
called \texttt{Grid\_out} and \texttt{MC\_out}.
%
So the user needs to input the following:
\begin{lstlisting}[style=XML]
<Databases>
  <HDF5 name="Grid_out" readMode="overwrite"/>
  <HDF5 name="MC_out" readMode="overwrite"/>
</Databases>
\end{lstlisting}
As listed before, this will create two databases.
%
The files will have names corresponding to their \xmlAttr{name} appended with
the .h5 extension (i.e. \texttt{Grid\_out.h5} and \texttt{MC\_out.h5}).
%%%%%%%%%%%%%%%%%%%%%%%%%%%%%%%%%%%%%%%%%%%%%%%%%%%%%%
\subsubsection{DataObjects}
As shown in the \xmlNode{Steps} block, the code is creating two DataObjects of type PointSet
called \texttt{gridOutData} and \texttt{MCOutData}.
%
So the user needs to input the following:
\begin{lstlisting}[style=XML]
<DataObjects>
    <PointSet name='gridOutData'>
      <Input>
          Materials|heatStructure2|thermal_conductivity,
          Materials|heatStructure1|specific_heat,
          Materials|heatStructure2|thermal_conductivity
      </Input>
      <Output>aveTempLeft</Output>
    </PointSet>
    <PointSet name='MCOutData'>
      <Input>
          Materials|heatStructure2|thermal_conductivity,
          Materials|heatStructure1|specific_heat,
          Materials|heatStructure2|thermal_conductivity
      </Input>
      <Output>aveTempLeft</Output>
    </PointSet>
</DataObjects>
\end{lstlisting}
As listed before, this will create two DataObjects that can be used in sub-sequential post-processing.
%%%%%%%%%%%%%%%%%%%%%%%%%%%%%%%%%%%%%%%%%%%%%%%%%%%%%%
\subsubsection{OutStreams}
As fully explained in section~\ref{sec:outstream}, if the user want to print out or plot the content of a \textbf{DataObjects},
he needs to create an \textbf{OutStream} in the \xmlNode{OutStreams} XML block.
\\As it shown in the example below, for MooseBasedApp (and any other Code interface that might use the symbol $|$
for the Sampler's variable syntax), in the Plot \xmlNode{x} and \xmlNode{y} specification, the user needs to
utilize curly brackets.
\begin{lstlisting}[style=XML]
<OutStreams>
  <Print name='gridOutDataDumpCSV'>
    <type>csv</type>
    <source>gridOutData</source>
  </Print>
   <Plot verbosity='debug' name='test'   overwrite='False'>
    <plotSettings>
       <plot>
        <type>line</type>
        <x>MCOutData|Input|{Materials|heatStructure2|thermal_conductivity}</x>
        <y>MCOutData|Output|aveTempLeft</y>
        <kwargs><color>blue</color></kwargs>
      </plot>
    </plotSettings>
    <actions><how>screen,png</how></actions>
  </Plot>
</OutStreams>
\end{lstlisting}

%%%%%%%%%%%%%%%%%%%%%%%%%%%%%%%%%%
%%%%%% Moose VectorPostProcessor INTERFACE  %%%%%%
%%%%%%%%%%%%%%%%%%%%%%%%%%%%%%%%%%
\subsection{MooseVPP Interface}

The Moose Vector Post Processor is used mainly in the solid mechanics analysis.
This interface loads the values of the vector ouput processor to a \xmlNode{DataObjects} object.

To use this interface the [DomainIntegral] needs to be present in the MooseBasedApp's
 input file and the subnode \xmlNode{fileargs} should be defined in the subnode \xmlNode{Code} in
 the \xmlNode{Models} block of the RAVEN input file. The \xmlNode{fileargs} is required to have attributes with
the below specified values:

\begin{itemize}
    \item \xmlAttr{type}, \xmlDesc{string, required field}, must be "MooseVPP"
    \item \xmlAttr{arg}, \xmlDesc{string, required field}, the string value attached to the vector post processor action
         while creating the output files.
\end{itemize}

This interface is actually identical to the MooseBasedApp interface, however there is
 few constraints on defining the output values of the post processor.
The definition of these outputs in the \xmlNode{DataObjects} depends on the definition of
 the [DomainIntegral].

The location of the value outputted is defined as \textit{ID\#} and the value is as
\textit{value\#}. The ''\#'' defines
the number of the location. The example below contains 3 locations in the [DomainIntegral]
where the values are outputted.

%
Example:
\begin{lstlisting}[style=XML]
 ...
  <Models>
    <Code name="MOOSETestApp" subType="MooseBasedApp">
      <executable>%FRAMEWORK_DIR%/../../moose/
        modules/combined/modules-%METHOD%</executable>
      <fileargs type = "MooseVPP" arg = "_J_1_" />
      <alias variable = "poissonsRatio" >
        Materials|stiffStuff|poissons_ratio</alias>
      <alias variable = "youngModulus"  >
        Materials|stiffStuff|youngs_modulus</alias>
    </Code>
  </Models>
 ...
  <DataObjects>
    <PointSet name="collset">
      <Input>youngModulus,poissonsRatio</Input>
      <Output>ID1,ID2,ID3,value1,value2,value3</Output>
    </PointSet>
  </DataObjects>
 ...
\end{lstlisting}




%%%%%%%%%%%%%%%%%%%%%%%%%%%%%%%%%%%%%
%%%%%% OPENMODELICA INTERFACE  %%%%%%
%%%%%%%%%%%%%%%%%%%%%%%%%%%%%%%%%%%%%
\subsection{OpenModelica Interface}
OpenModelica (\url{http://www.openmodelica.org}) is an open souce implementation of the Modelica simulation language.  Modelica is "a non-proprietary,
object-oriented, equation based language to conveniently model complex physical systems containing, e.g., mechanical, electrical, electronic, hydraulic,
thermal, control, electric power or process-oriented subcomponents."\footnote{\url{http://www.modelica.org}}.  Modelica models are specified in text files
with a file extension of .mo.  A standard Modelica example called BouncingBall which simulates the trajectory of an object falling in one dimension from a
height is shown as an example:
\begin{lstlisting}
model BouncingBall
  parameter Real e=0.7 "coefficient of restitution";
  parameter Real g=9.81 "gravity acceleration";
  Real h(start=1) "height of ball";
  Real v "velocity of ball";
  Boolean flying(start=true) "true, if ball is flying";
  Boolean impact;
  Real v_new;
  Integer foo;

equation
  impact = h <= 0.0;
  foo = if impact then 1 else 2;
  der(v) = if flying then -g else 0;
  der(h) = v;

  when {h <= 0.0 and v <= 0.0,impact} then
    v_new = if edge(impact) then -e*pre(v) else 0;
    flying = v_new > 0;
    reinit(v, v_new);
  end when;

end BouncingBall;
\end{lstlisting}

\subsubsection{Files}
An OpenModelica installation specific to the operating system is used to create a stand-alone executable program that performs the model calculations.
A separate XML file containing model parameters and initial conditions is also generated as part of the build process.  The RAVEN OpenModelica interface
modifies input parameters by changing copies of this file.  Both the executable and XML parameter file names must be provided to RAVEN.  In the case of
the BouncingBall model previously mentioned on the Windows operating system, the \textless Files\textgreater  specification would look like:
\begin{lstlisting}[style=XML]
<Files>
  <Input name='BouncingBall_init.xml' type=''>BouncingBall_init.xml</Input>
  <Input name='BouncingBall.exe' type=''>BouncingBall.exe</Input>
</Files>
\end{lstlisting}
\subsubsection{Models}
OpenModelica models may provide simulation output in a number of formats.  The particular format used is specified during the model generation
process.  RAVEN works best with Comma-Separated Value (CSV) files, which is one of the possible output format options.  Models are generated
using the OpenModelica Shell (OMS) command-line interface, which is part of the OpenModelica installation.  To generate an executable that provides
CSV-formatted output, use OMSl commands as follows:
\lstset{
    frame=single,
    breaklines=true,
    postbreak=\raisebox{0ex}[0ex][0ex]{\ensuremath{\color{red}\hookrightarrow\space}}
}
 \begin{enumerate}
\item Change to the directory containing the .mo file to generate an executable for:
\begin{lstlisting}
>> cd("C:/MinGW/msys/1.0/home/bobk/projects/raven/framework/CodeInterfaces/OpenModelica")
"C:/MinGW/msys/1.0/home/bobk/projects/raven/framework/CodeInterfaces/OpenModelica"
\end{lstlisting}
\item Load the model file into memory:
\begin{lstlisting}
>> loadFile("BouncingBall.mo")
true
\end{lstlisting}
\item Create the model executable, specifying CSV output format:
\begin{lstlisting}
>> buildModel(BouncingBall, outputFormat="csv")
{"C:/MinGW/msys/1.0/home/bobk/projects/raven/framework/CodeInterfaces/OpenModelica/BouncingBall","BouncingBall_init.xml"}
Warning: The initial conditions are not fully specified. Use +d=initialization for more information.
\end{lstlisting}
At this point the model executable and XML initialization file should have been created in the same directory as the original model file.
\end{enumerate}
The model executable is specified to RAVEN using the \textless Models\textgreater  section of the input file as follows:
\begin{lstlisting}[style=XML]
<Simulation>
    ...
  <Models>
    <Code name="BouncingBall" subType = "OpenModelica">
      <executable>BouncingBall.exe</executable>
    </Code>
  </Models>
    ...
</Simulation>
\end{lstlisting}
\subsubsection{CSV Output}
The CSV files produced by OpenModelica model executables require adjustment before it may be read by RAVEN.
The first few lines of original CSV output from the
BouncingBall example is shown below:
\begin{lstlisting}
"time","h","v","der(h)","der(v)","v_new","foo","flying","impact",
0,1,0,0,-9.810000000000001,0,2,1,0,
  ...
\end{lstlisting}
RAVEN will not properly read this file as-generated for two reasons:
\begin{itemize}
  \item The variable names in the first line are each enclosed in double-quotes.
  \item Each line has a trailing comma.
\end{itemize}
 The OpenModelica inteface will automatically remove the double-quotes and trailing commas through its implementation of the
finalizeCodeOutput function.


%%%%%%%%%%%%%%%%%%%%%%%%%%%%%%%%%%%%%
%%%%%% DYMOLA INTERFACE  %%%%%%%%%%%%
%%%%%%%%%%%%%%%%%%%%%%%%%%%%%%%%%%%%%
\subsection{Dymola Interface}
Modelica is "a non-proprietary, object-oriented, equation-based language to conveniently model complex physical systems containing, e.g., mechanical, electrical, electronic, hydraulic,
thermal, control, electric power or process-oriented subcomponents."\footnote{\url{http://www.modelica.org}}.  Modelica models (with a file extension of .mo) are built, translated (compiled), and simulated in Dymola (http://www.modelon.com/p-
roducts/dymola/), which is a commercial modeling and simulation environment based on the Modelica modeling language.
A standard Modelica example called BouncingBall, which simulates the trajectory of an object falling in one dimension from a height, is shown as an example:
\begin{lstlisting}
model BouncingBall
  parameter Real e=0.7 "coefficient of restitution";
  parameter Real g=9.81 "gravity acceleration";
  parameter Real hstart = 10 "height of ball at time zero";
  parameter Real vstart = 0 "velocity of ball at time zero";
  Real h(start=hstart,fixed=true) "height of ball";
  Real v(start=vstart,fixed=true) "velocity of ball";
  Boolean flying(start=true) "true, if ball is flying";
  Boolean impact;
  Real v_new;
  Integer foo;

equation
  impact = h <= 0.0;
  foo = if impact then 1 else 2;
  der(v) = if flying then -g else 0;
  der(h) = v;

  when {h <= 0.0 and v <= 0.0,impact} then
    v_new = if edge(impact) then -e*pre(v) else 0;
    flying = v_new > 0;
    reinit(v, v_new);
  end when;

  annotation (uses(Modelica(version="3.2.1")),
    experiment(StopTime=10, Interval=0.1),
    __Dymola_experimentSetupOutput);

end BouncingBall;
\end{lstlisting}

\subsubsection{Files}
When a modelica model, e.g., BouncingBall model, is implemented in Dymola, the platform dependent C-code from a Modelica model and the corresponding executable code
(i.e., by default dymosim.exe on the Windows operating system) are generated for simulation.  After the executable is generated, it may be run multiple times (with Dymola license).
A separate TEXT file (by default dsin.txt) containing model parameters and initial conditions are also generated as part of the build process.  The RAVEN Dymola interface
modifies input parameters by changing copies of this file.  Both the executable and TEXT parameter file (or simulation initialization file) names must be provided to RAVEN. The TEXT parameter file must be of type 'DymolaInitialisation'.  In the case of
the BouncingBall model previously mentioned on the Windows operating system, the \textless Files\textgreater  specification would look like:
\begin{lstlisting}[style=XML]
<Files>
  <Input name='dsin.txt' type='DymolaInitialisation'>dsin.txt</Input>
</Files>
\end{lstlisting}

The Dymola interface can only pass scalar values into the TEXT parameter file. If the user wants to pass vector information to Dymola, he can do so by providing an optional TEXT vector file to Dymola. This file must have the type 'DymolaVectors'. This additional file can then be read by the Dymola model. If vecor data is passed from RAVEN to the Dymola interface and the TEXT vector file is not specified, the interface will display an error and stop the Dymola execution. If the TEXT vector file is specified (and vector data is passed to the interface), the interface will write the datd into the specified file, but also display a warning, saying that the Dymola interface found vector data to be passed and if this data is supposed to go into the simulation initialisation file of type 'DymolaInitialisation' the array must be split into scalars. The \textless Files\textgreater specification for the vector data look as follows:
\begin{lstlisting}[style=XML]
<Files>
  <Input name='timeSeriesData.txt' type='DymolaVectors'>timeSeriesData.txt</Input>
</Files>
\end{lstlisting}

\subsubsection{Models}
An executable (dymosim.exe) and a simulation initialization file (dsin.txt) can be generated after either translating or simulating the
Modelica model (BouncingBall.mo) using the Dymola Graphical User Interface (GUI) or Dymola Application Programming Interface (API)-routines.
To generate an executable and a simulation initialization file, use the Dymola API-routines (or Dymola GUI) to translate the model as follows:
\lstset{
    frame=single,
    breaklines=true,
    postbreak=\raisebox{0ex}[0ex][0ex]{\ensuremath{\color{red}\hookrightarrow\space}}
}
\begin{enumerate}
\item Change to the directory containing the .mo file to generate an executable.  In Dymola GUI, this corresponds to File/Change Directory in menus:
\begin{lstlisting}
>> cd("C:/msys64/home/KIMJ/projects/raven/framework/CodeInterfaces/Dymola");
C:/msys64/home/KIMJ/projects/raven/framework/CodeInterfaces/Dymola
 = true
\end{lstlisting}
\item Reads the specified file and displays its window.  In Dymola GUI, this corresponds to File/Open in the menus:
\begin{lstlisting}
>> openModel("BouncingBall.mo")
 = true
\end{lstlisting}
\item Compile the model (with current settings), and create the model executable and the corresponding simulation initialization file.  In Dymola GUI, this corresponds to Translate Model in the menus:
\begin{lstlisting}
>> translateModel("BouncingBall");
 = true
\end{lstlisting}
At this point the model executable and the simulation initialization file should have been created in the same directory as the original model file.
Additionally, they could be created by simulating the model.  The following command corresponds to Simulate in the menus in Dymola GUI:
\begin{lstlisting}
>> simulateModel("BouncingBall", stopTime=10, numberOfIntervals=0, outputInterval=0.1, method="dassl", resultFile="BouncingBall");
 = true
\end{lstlisting}
The file extension (.mat) is automatically added to a output file (resultFile), e.g., BouncingBall.mat.  If the generated executable code is triggered directly from a
command prompt, the output file is always named as "dsres.mat".
\end{enumerate}
The model executable is specified to RAVEN using the \textless Models\textgreater  section of the input file as follows:
\begin{lstlisting}[style=XML]
<Simulation>
    ...
  <Models>
    <Code name="BouncingBall" subType = "Dymola">
      <executable>dymosim.exe</executable>
    </Code>
  </Models>
    ...
</Simulation>
\end{lstlisting}
RAVEN works best with Comma-Separated Value (CSV) files.  Therefore, the default
.mat output type needs to be converted to .csv output.
The Dymola interface will automatically convert the .mat output to human-readable
forms, i.e., .csv output, through its implementation of the finalizeCodeOutput function.
\\In order to speed up the reading and conversion of the .mat file, the user can specify
the list of variables (in addition to the Time variable) that need to be imported and
converted into a csv file minimizing
the IO memory usage as much as possible. Within the \xmlNode{Code} the following
XML
node (in addition ot the \xmlNode{executable} one) can be inputted:

\begin{itemize}
   \item \xmlNode{outputVariablesToLoad}, \xmlDesc{space separated list, optional
   parameter}, a space separated list of variables that need be exported from the .mat
   file (in addition to the Time variable). \default{all the variables in the .mat file}.
\end{itemize}
For example:
\begin{lstlisting}[style=XML]
<Simulation>
    ...
  <Models>
    <Code name="BouncingBall" subType = "Dymola">
      <executable>dymosim.exe</executable>
      <outputVariablesToLoad>var1 var2 var3</outputVariablesToLoad>
    </Code>
  </Models>
    ...
</Simulation>
\end{lstlisting}


%%%%%%%%%%%%%%%%%%%%%%%%%%%%%%%%%%%%%%%%%%%%%%%%%
%%%%%% MESH GENERATION COUPLED INTERFACES %%%%%%%
%%%%%%%%%%%%%%%%%%%%%%%%%%%%%%%%%%%%%%%%%%%%%%%%%
\subsection{Mesh Generation Coupled Interfaces}
Some software requires a provided mesh that requires a separate code run to generate.
In these cases, we use sampled geometric
variables to generate a new mesh for each perturbation of the original problem, then run the input with
the remainder of the perturbed parameters and the perturbed mesh.  RAVEN currently provides two interfaces for
this type of calculation, listed below.

%%%%%%%%%% CUBIT MOOSE INTERFACE %%%%%%%%%%
\subsubsection{MooseBasedApp and Cubit Interface}
Many MOOSE-based applications use Cubit (\url{https://cubit.sandia.gov}) to generate Exodus II files as
geometry and meshing for calculations.  To use the developed interface, Cubit's
bin directory must be added to the user's PYTHONPATH.  Input parameters for Cubit can be listed in a journal
(\texttt{.jou}) file.  Parameter values
are typically hardcoded into the Cubit command syntax, but variables may be
predefined in a journal file through Aprepro syntax.  This is an example of a journal
file that generates a rectangle of given height and width, meshes it, defines its
volume and sidesets, lists its element type, and writes it as an Exodus file:

%Cubit (\url{https://cubit.sandia.gov}) is a toolkit developed at Sandia National
%Laboratory used to create two- and three-dimensional finite element meshes with
%various options for defining geometric properties as a part of the grid. It is
%capable of reading and writing a variety of standard mesh file types, including
%Genesis or Exodus II (*.e) files.  As MOOSE applications use Exodus II files for
%meshes and results, Cubit is commonly used to generate meshes for problems of
%interest.  Cubit commands are used to create the geometry, mesh the object, and
%identify volumes, sidesets, and nodesets for a mesh.  These commands may be
%placed in journal files (*.jou) to be used as input to Cubit.  Parameter values
%are typically hardcoded into the Cubit command syntax, but variables may be
%predefined in a journal file through Aprepro syntax.  An example of a journal
%file that generates a rectangle of given height and width, meshes it, defines its
%volume and sidesets, lists its element type, and writes it as an Exodus file is given:

\begin{lstlisting}
#{x = 3}
#{y = 3}
#{out_name = "'out_mesh.e'"}
create surface rectangle width {x} height {y} zplane
mesh surface 1
set duplicate block elements off
block 1 surface 1
Sideset 1 curve 3
Sideset 2 curve 4
Sideset 3 curve 1
Sideset 4 curve 2
Block all element type QUAD4
export genesis {out_name} overwrite
\end{lstlisting}

The first three lines are the Aprepro variable definitions that RAVEN requires to
insert sampled variables.  All variables that RAVEN samples
need to be defined as Aprepro variables in the journal
file.
%These are typically geometric parameters, though almost anything that Cubit
%quantifies in a command may be defined as a variable and sampled through RAVEN
%such as internal mesh refinement values.
One essential caveat to running
this interface is that an Aprepro variable MUST be defined with the name "out\_name".
In order to run this script without RAVEN inserting the correct syntax for the
output file name and properly generate the Exodus file for a mesh, the output file
name is REQUIRED to be in both single and double quotation marks with the file
extension appended to the end of the file base name (e.g. '"output\_file.e"').

%%%%%%%%%%%%%%%%%%%%%%%%%%%%%%%%%%%%%%%%%%%%%%%%%%
\paragraph{Files}
\xmlNode{Files} works the same as in other interfaces with name and type
attributes for each node entry.  The \xmlAttr{name} attribute is a user-chosen internal
name for the file contained in the node, and \xmlAttr{type} identifies which base-level
interface the file is used within.  \xmlNode{type} should only be specified for inputs
that RAVEN will perturb.  For Moose input files, \xmlNode{type} should be \xmlString{MooseInput} and for
Cubit journal files, the \xmlNode{type} should be \xmlString{CubitInput}.  The node should contain the
path to the file from the working directory.  The following is an example
of a typical \xmlNode{Files} block.

\begin{lstlisting}[style=XML]
<Files>
  <Input name='moose_test' type='MooseInput'>simple_diffusion.i</Input>
  <Input name='mesh_in'    type='CubitInput'>rectangle.jou</Input>
  <Input name='other_file' type=''          >some_file_moose_input_needs.ext</Input>
</Files>
\end{lstlisting}

%%%%%%%%%%%%%%%%%%%%%%%%%%%%%%%%%%%%%%%%%%%%%%%%%%
\paragraph{Models}
A user provides paths to executables and aliases for sampled variables within the
\xmlNode{Models} block.  The \xmlNode{Code} block will contain attributes name and
subType.  Name identifies that particular \xmlNode{Code} model within RAVEN, and
subType specifies which code interface the model will use. The \xmlNode{executable}
block should contain the absolute or relative (with respect to the current working
directory) path to the MooseBasedApp that RAVEN will use to run generated input
files.  The absolute or relative path to the Cubit executable is specified within
\xmlNode{preexec}.  If the \xmlNode{preexec} block is not needed, the
MooseBasedApp interface is probably preferable to the Cubit-Moose interface.

Aliases are defined by specifying the variable attribute in an \xmlNode{alias} node with
the internal RAVEN variable name chosen with the node containing the model
variable name.  The Cubit-Moose interface uses the same syntax as the
MooseBasedApp to refer to model variables, with pipes separating terms starting
with the highest YAML block going down to the individual parameter that RAVEN
will change.  To specify variables that are going to be used in the Cubit
journal file, the syntax is "Cubit|aprepro\_var".  The Cubit-Moose interface
will look for the Cubit tag in all variables passed to it and upon finding it,
send it to the Cubit interface.  If the model variable does not begin with \xmlString{Cubit},
the variable MUST be specified in the MooseBasedApp input file.  While the model
variable names are not required to have aliases defined (the \xmlNode{alias}
blocks are optional), it is highly suggested to do so not only to ensure brevity
throughout the RAVEN input, but to easily identify where variables are being sent
in the interface.

An example \xmlNode{Models} block follows.

\begin{lstlisting}[style=XML]
<Models>
  <Code name="moose-modules" subType="CubitMoose">
    <executable>%FRAMEWORK_DIR%/../../moose/modules/combined/...
      modules-%METHOD%</executable>
    <preexec>/hpc-common/apps/local/cubit/13.2/bin/cubit</preexec>
    <alias variable="length">Cubit@y</alias>
    <alias variable="bot_BC">BCs|bottom|value</alias>
  </Code>
</Models>
\end{lstlisting}

%%%%%%%%%%%%%%%%%%%%%%%%%%%%%%%%%%%%%%%%%%%%%%%%%%
\paragraph{Distributions}
The \xmlNode{Distributions} block defines all distributions used to
sample variables in the current RAVEN run.

For all the possible distributions and their possible inputs please
refer to the Distributions chapter (see~\ref{sec:distributions}).
%
It is good practice to name the distribution something similar to what kind of
variable is going to be sampled, since there might be many variables with the
same kind of distributions but different input parameters.

%%%%%%%%%%%%%%%%%%%%%%%%%%%%%%%%%%%%%%%%%%%%%%%%%%
\paragraph{Samplers}
The \xmlNode{Samplers} block defines the variables to be sampled.

After defining a sampling scheme, the variables to be sampled and
their distributions are identified in the \xmlNode{variable} blocks.
The name attribute in the \xmlNode{variable} block must either be the
full MooseBasedApp model variable name or the alias name specifed in
\xmlNode{Models}.  If the sampled variable is a geometric property
that will be used to generate a mesh with Cubit, remember the syntax for
variables being passed to journal files (Cubit|aprepro\_var).

For listings of available samplers
refer to the Samplers chapter (see~\ref{sec:Samplers}).

See the following for an example of a grid based sampler for
length and the bottom boundary condition (both of which have aliases
defined in \xmlNode{Models}).

\begin{lstlisting}[style=XML]
<Samplers>
  <Grid name="Grid_sampling">
    <variable name="length" >
      <distribution>length_dist</distribution>
      <grid type="value" construction="custom">1.0 2.0</grid>
    </variable>
    <variable name="bot_BC">
      <distribution>bot_BC_dist</distribution>
      <grid type="value" construction="custom">3.0 6.0</grid>
    </variable>
  </Grid>
</Samplers>
\end{lstlisting}

%%%%%%%%%%%%%%%%%%%%%%%%%%%%%%%%%%%%%%%%%%%%%%%%%%
\paragraph{Steps,OutStreams,DataObjects}
This interface's \xmlNode{Steps}, \xmlNode{OutStreams}, and
\xmlNode{DataObjects} blocks do not deviate significantly from
other interfaces' respective nodes.  Please refer to previous
entries for these blocks if needed.

%%%%%%%%%%%%%%%%%%%%%%%%%%%%%%%%%%%%%%%%%%%%%%%%%%
\paragraph{File Cleanup}
The Cubit-Moose interface automatically removes files that are commonly
unwanted after the RAVEN run reaches completion. Cubit has been described as
"talkative" due to additional journal files with execution information
being generated by the program after every completed journal file run.
The quantity of these files can quickly become unwieldly if the working
directory is not kept clean; thus these files are removed.  In addition, some users
may wish to remove Exodus files after the RAVEN run is complete as
the typical size of each file is quite large and it is assumed that any
output quantities of interest will be collected by appropriate postprocessors
and the OutStreams.  Exodus files are not automatically removed,
but by using the \xmlNode{deleteOutExtension} node in \xmlNode{RunInfo}, one
may specify the Exodus extension to save a fair amount of storage space
after RAVEN completes a sequence. For example:

\begin{lstlisting}[style=XML]
<RunInfo>
  ...
  <deleteOutExtension>e</deleteOutExtension>
  ...
</RunInfo>
\end{lstlisting}

%%%%%%%%%% BISON MESH SCRIPT MOOSE INTERFACE %%%%%%%%%%
\subsubsection{MooseBasedApp and Bison Mesh Script Interface}
For BISON users, a Python mesh generation script is included in
the \%BISON\_DIR\%/tools/UO2/ directory.  This script generates
3D or 2D (RZ) meshes for nuclear fuel rods using Cubit with
templated commands.  The BISON Mesh Script (BMS) is capable of
generating rods with discrete fuel pellets of various size in
assorted configurations.  To use this interface, Cubit's bin
directory must be added to the user's PYTHONPATH.

%%%%%%%%%%%%%%%%%%%%%%%%%%%%%%%%%%%%%%%%%%%%%%%%%%
\paragraph{Files}
Similar to the Cubit-Moose interface, the BisonAndMesh interface
requires users to specify all files required to run their input
so that these file may be copied into the respective sequence's
working directory.  The user will give each file an internal
RAVEN designation with the name attribute, and the MooseBasedApp
and BISON Mesh Script inputs must be assigned their respective types
in another attribute of the \xmlNode{Input} node.  An example follows.

\begin{lstlisting}[style=XML]
<Files>
  <Input name='bison_test' type='MooseInput'>simple_bison_test.i</Input>
  <Input name='mesh_in'    type='BisonMeshInput'>coarse_input.py</Input>
  <Input name='other_file' type=''>some_file_moose_input_needs.ext</Input>
</Files>
\end{lstlisting}

%%%%%%%%%%%%%%%%%%%%%%%%%%%%%%%%%%%%%%%%%%%%%%%%%%
\paragraph{Models}
A user provides paths to executables and aliases for sampled variables within the
\xmlNode{Models} block.  The \xmlNode{Code} block will contain attributes \xmlAttr{name} and
\xmlAttr{subType}.  \xmlAttr{name} identifies that particular \xmlNode{Code} model within RAVEN, and
\xmlAttr{subType} specifies which code interface the model will use. The \xmlNode{executable}
block should contain the absolute or relative (with respect to the current working
directory) path to the MooseBasedApp that RAVEN will use to run generated input
files.  The absolute or relative path to the mesh script python file is specified within
\xmlNode{preexec}.  If the \xmlNode{preexec} block is not needed, use the
MooseBasedApp interface.

Aliases are defined by specifying the variable attribute in an \xmlNode{alias} node with
the internal RAVEN variable name chosen with the node containing the model
variable name.  The BisonAndMesh interface uses the same syntax as the
MooseBasedApp to refer to model variables, with pipes separating terms starting
with the highest YAML block going down to the individual parameter that RAVEN
will change.  To specify variables that are going to be used in the BISON Mesh Script
python input, the syntax is "Cubit|dict\_name|var\_name".  The interface
will look for the Cubit tag in all variables passed to it and upon finding the tag,
send it to the BISON Mesh Script interface.  If the model variable does not begin with Cubit,
the variable MUST be specified in the MooseBasedApp input file.  While the model
variable names are not required to have aliases defined (the \xmlNode{alias}
blocks are optional), it is highly suggested to do so not only to ensure brevity
throughout the RAVEN input, but to easily identify where variables are being sent
in the interface.

An example \xmlNode{Models} block follows.

\begin{lstlisting}[style=XML]
<Models>
  <Code name="Bison-opt" subType="BisonAndMesh">
    <executable>%FRAMEWORK_DIR%/../../bison/bison-%METHOD%</executable>
    <preexec>%FRAMEWORK_DIR%/../../bison/tools/UO2/mesh_script.py</preexec>
    <alias variable="pellet_radius" >Cubit@Pellet1|outer_radius</alias>
    <alias variable="clad_thickness">Cubit@clad|clad_thickness</alias>
    <alias variable="fuel_k"        >Materials|fuel_thermal|thermal_conductivity</alias>
    <alias variable="clad_k"        >Materials|clad_thermal|thermal_conductivity</alias>
  </Code>
</Models>
\end{lstlisting}

%%%%%%%%%%%%%%%%%%%%%%%%%%%%%%%%%%%%%%%%%%%%%%%%%%
\paragraph{Distributions}
The \xmlNode{Distributions} block defines all distributions used to
sample variables in the current RAVEN run.

For all the possible distributions and their possible inputs please
refer to the Distributions chapter (see~\ref{sec:distributions}).
%
It is good practice to name the distribution something similar to what kind of
variable is going to be sampled, since there might be many variables with the
same kind of distributions but different input parameters.

%%%%%%%%%%%%%%%%%%%%%%%%%%%%%%%%%%%%%%%%%%%%%%%%%%
\paragraph{Samplers}
The \xmlNode{Samplers} block defines the variables to be sampled.

After defining a sampling scheme, the variables to be sampled and
their distributions are identified in the \xmlNode{variable} blocks.
The name attribute in the \xmlNode{variable} block must either be the
full MooseBasedApp model variable name or the alias name specified in
\xmlNode{Models}.  If the sampled variable is a geometric property
that will be used to generate a mesh with Cubit, remember the syntax for
variables being passed to journal files (Cubit|aprepro\_var).

For listings of available samplers
refer to the Samplers chapter (see~\ref{sec:Samplers}).

See the following for an example of a grid based sampler for
length and the bottom boundary condition (both of which have aliases
defined in \xmlNode{Models}).

\begin{lstlisting}[style=XML]
<Samplers>
  <Grid name="Grid_sampling">
    <variable name="length" >
      <distribution>length_dist</distribution>
      <grid type="value" construction="custom">1.0 2.0</grid>
    </variable>
    <variable name="bot_BC">
      <distribution>bot_BC_dist</distribution>
      <grid type="value" construction="custom">3.0 6.0</grid>
    </variable>
  </Grid>
</Samplers>
\end{lstlisting}

%%%%%%%%%%%%%%%%%%%%%%%%%%%%%%%%%%%%%%%%%%%%%%%%%%
\paragraph{Steps,OutStreams,DataObjects}
This interface's \xmlNode{Steps}, \xmlNode{OutStreams}, and
\xmlNode{DataObjects} blocks do not deviate significantly from
other interfaces' respective nodes.  Please refer to previous
entries for these blocks if needed.

%%%%%%%%%%%%%%%%%%%%%%%%%%%%%%%%%%%%%%%%%%%%%%%%%%
\paragraph{File Cleanup}
The BisonAndMesh interface automatically removes files that are commonly
unwanted after the RAVEN run reaches completion. Cubit has been described as
"talkative" due to additional journal files with execution information
being generated by the program after every completed journal file run.
The BISON Mesh Script creates a journal file to run with cubit after reading input parameters;
so Cubit will generate its "redundant" journal files, and .pyc files will
litter the working directory as artifacts of the python mesh script
reading from the .py input files.  The quantity of these files can quickly
become unwieldly if the working directory is not kept clean, thus these
files are removed.  Some users
may wish to remove Exodus files after the RAVEN run is complete as
the typical size of each file is quite large and it is assumed that any
output quantities of interest will be collected by appropriate postprocessors
and the OutStreams.  Exodus files are not automatically removed,
but by using the \xmlNode{deleteOutExtension} node in \xmlNode{RunInfo}, one
may specify the Exodus extension (*.e) to save a fair amount of storage space
after RAVEN completes a sequence. For example:

\begin{lstlisting}[style=XML]
<RunInfo>
  ...
  <deleteOutExtension>e</deleteOutExtension>
  ...
</RunInfo>
\end{lstlisting}

%%%%%%%%%%%%%%%%%%%%%%%%%%%%%%%%%%%%%%%%%%%%%%%%%
%%%%%%%%%%%%% RATTLESNAKE INTERFACE %%%%%%%%%%%%%
%%%%%%%%%%%%%%%%%%%%%%%%%%%%%%%%%%%%%%%%%%%%%%%%%
\subsection{Rattlesnake Interfaces} \label{RattlesnakeInterfaces}
%
This section covers the input specification for running Rattlesnake through RAVEN. It is important
to notice that this short explanation assumes that the reader already knows how to use Rattlesnake.
The interface can be used to perturb the Rattlesnake MOOSE-based input file as well as the Yak
cross section libraries XML input files (e.g. multigroup cross section libraries) and Instant format
cross section libraries.
%
%%%%%%%%%%%%%%%%%%%%%%%%%%%%%%%%%%%%%%%%%%%%%%%%%%
\subsubsection{Files}
\xmlNode{Files} works the same as in other interfaces with name and type
attributes for each node entry.  The \xmlAttr{name} attribute is a user-chosen internal
name for the file contained in the node, and \xmlAttr{type} identifies which base-level
interface the file is used within.  \xmlAttr{type} should only be specified for inputs
that RAVEN will perturb. Take Rattlesnake input files for example, \xmlAttr{type} should
be \xmlString{RattlesnakeInput}.

\paragraph{Perturb Yak Multigroup Cross Section Libraries}
If the user would like to perturb the Yak multigroup cross section libraries, the user need to use the
\xmlString{YakXSInput} for the \xmlAttr{type} of the libaries. In addition, the \xmlAttr{type} of the
alias files that are used to perturb the Yak multigroup cross section libraries should be
\xmlString{YakXSAliasInput}. The following is an example of a typical \xmlNode{Files} block.
%
\begin{lstlisting}[style=XML]
<Files>
  <Input name='rattlesnakeInput' type='RattlesnakeInput'>simple_diffusion.i</Input>
  <Input name='crossSection'    type='YakXSInput'>xs.xml</Input>
  <Input name='alias' type='YakXSAliasInput'>alias.xml</Input>
</Files>
\end{lstlisting}
%
The alias files are employed to define the variables that will be used to perturb Yak multigroup cross section
libraries. The following is an example of a typical alias file:
%
\begin{lstlisting}[style=XML]
<Multigroup_Cross_Section_Libraries Name="twigl" NGroup="2" Type="rel">
    <Multigroup_Cross_Section_Library ID="1">
        <Fission gridIndex="1" mat="pseudo-seed1" gIndex="1">f11</Fission>
        <Capture gridIndex="1" mat="pseudo-seed1" gIndex="1">c11</Capture>
        <TotalScattering gridIndex="1" mat="pseudo-seed1" gIndex="1">t11</TotalScattering>
        <Nu gridIndex="1" mat="pseudo-seed1" gIndex="1">n11</Nu>
        <Fission gridIndex="1" mat="pseudo-seed2" gIndex="2">f22</Fission>
        <Capture gridIndex="1" mat="pseudo-seed2" gIndex="2">c22</Capture>
        <TotalScattering gridIndex="1" mat="pseudo-seed2" gIndex="2">t22</TotalScattering>
        <Nu gridIndex="1" mat="pseudo-seed2" gIndex="2">n22</Nu>
        <Fission gridIndex="1" mat="pseudo-seed1-dup" gIndex="1">f11</Fission>
        <Capture gridIndex="1" mat="pseudo-seed1-dup" gIndex="1">c11</Capture>
        <TotalScattering gridIndex="1" mat="pseudo-seed1-dup" gIndex="1">t11</TotalScattering>
        <Nu gridIndex="1" mat="pseudo-seed1-dup" gIndex="1">n11</Nu>
        <Transport gridIndex="1" mat="pseudo-seed1-dup" gIndex="1">d11</Transport>
    </Multigroup_Cross_Section_Library>
</Multigroup_Cross_Section_Libraries>
\end{lstlisting}
%
In the above alias file, the \xmlAttr{Name} of \xmlNode{Multigroup\_Cross\_Section\_Libraries} are used to indicate
which Yak multigroup cross section library input file will be perturbed.
The \xmlAttr{NGroup}, \xmlAttr{ID}, and \xmlNode{Multigroup\_Cross\_Section\_Library}
should be consistent with the Yak multigroup cross section library input files.
The \xmlNode{Fission}, \xmlNode{Capture}, \xmlNode{TotalScattering}, \xmlNode{Nu}, \xmlAttr{gridIndex},
\xmlAttr{mat}, and \xmlAttr{gIndex} are used to find the corresponding cross sections in the Yak multigroup cross
section library input files. For example:
%
\begin{lstlisting}[style=XML]
<Fission gridIndex="1" mat="pseudo-seed1" gIndex="1">f11</Fission>
\end{lstlisting}
%
This node defines an alias with name \xmlString{f11} used to represent the fission cross section at energy group \xmlString{1}
for material with name 'pseudo-seed1' at grid index \xmlString{1} in the Yak multigroup cross section library input files.

\nb The attribute \xmlAttr{Type="rel"} indicates that the cross sections will be perturbed relatively (i.e. perturbed by
percents). In this case, the user also needs to specify a relative covariance matrix for \xmlNode{covaraince \xmlAttr{type="rel"}} in
\xmlNode{MultivariateNormal} distribution, and the values for \xmlNode{mu} should be `ones'. In the other case, if
the user choose \xmlAttr{Type="abs"}, the cross sections will be perturbed absolutely (i.e. perturbed by values), and
the user needs to provide an absolute covariance matrix and specify `zeros' for \xmlNode{mu} in \xmlNode{MultivariateNormal}
distribution.

\nb Currently, only the following cross sections can be perturbed by the user: Fission, Capture, Nu, TotalScattering,
and Transport.

\paragraph{Perturb Instant format Cross Section Libraries}
If the user would like to perturb the Instant cross section libraries, the user need to use the
\xmlString{InstantXSInput} for the \xmlAttr{type} of the libaries. In addition, the \xmlAttr{type} of the
alias files that are used to perturb the Instant format cross section libraries should be
\xmlString{InstantXSAliasInput}. The following is an example of a typical \xmlNode{Files} block.
%
\begin{lstlisting}[style=XML]
<Files>
  <Input name='rattlesnakeInput' type='RattlesnakeInput'>iaea2d_ls_sn.i</Input>
  <Input name='crossSection'    type='InstantXSInput'>iaea2d_materials.xml</Input>
  <Input name='alias' type='InstantXSAliasInput'>alias.xml</Input>
</Files>
\end{lstlisting}
%
The alias files are employed to define the variables that will be used to perturb Instant format cross section
libraries. The following is an example of a typical alias file:
%
\begin{lstlisting}[style=XML]
<Materials>
  <Macros NG="2" Type="rel">
    <material ID="1">
      <FissionXS gIndex="1">f11</FissionXS>
      <CaptureXS gIndex="1">c11</CaptureXS>
      <TotalScatteringXS gIndex="1">t11<TotalScatteringXS>
      <Nu gIndex="1">n11</Nu>
      <DiffusionCoefficient gIndex="1">d11</DiffusionCoefficient>
    </material>
  </Macros>
</Materials>
\end{lstlisting}

%
In the above alias file, the \xmlAttr{NG} and \xmlAttr{ID} should be consistent with the Instant format cross
section library input files. The \xmlNode{FissionXS}, \xmlNode{CaptureXS}, \xmlNode{TotalScatteringXS}, \xmlNode{Nu}, \xmlAttr{gIndex},
are used to find the corresponding cross sections in the Instant format cross
section library input files. For example, the variable \xmlString{f11} used to represent the fission cross section at energy group \xmlString{1}
for material with \xmlString{ID} equal \xmlString{1} in the given cross section library.

\nb The attribute \xmlAttr{Type="rel"} indicates that the cross sections will be perturbed relatively (i.e. perturbed by
percents). In this case, the user also needs to specify a relative covariance matrix for \xmlNode{covaraince \xmlAttr{type="rel"}} in
\xmlNode{MultivariateNormal} distribution, and the values for \xmlNode{mu} should be `ones'. In the other case, if
the user choose \xmlAttr{Type="abs"}, the cross sections will be perturbed absolutely (i.e. perturbed by values), and
the user needs to provide an absolute covariance matrix and specify `zeros' for \xmlNode{mu} in \xmlNode{MultivariateNormal}
distribution.

\nb Currently, only the following cross sections can be perturbed by the user: FissionXS, CaptureXS, Nu, TotalScatteringXS,
and DiffusionCoefficient.

%%%%%%%%%%%%%%%%%%%%%%%%%%%%%%%%%%%%%%%%%%%%%%%%%%
\subsubsection{Models}
A user provides paths to executables and aliases for sampled variables within the
\xmlNode{Models} block.  The \xmlNode{Code} block will contain attributes \xmlNode{name} and
\xmlNode{subType}. The \xmlNode{name} identifies that particular \xmlNode{Code} model within RAVEN, and
\xmlNode{subType} specifies which code interface the model will use. The \xmlNode{executable}
block should contain the absolute or relative (with respect to the current working
directory) path to Rattlesnake that RAVEN will use to run generated input
files.

An example \xmlNode{Models} block follows.

\begin{lstlisting}[style=XML]
<Models>
  <Code name="Rattlesnake" subType="Rattlesnake">
    <executable>%FRAMEWORK_DIR%/../../rattlesnake/
     rattlesnake-%METHOD%</executable>
  </Code>
</Models>
\end{lstlisting}

%%%%%%%%%%%%%%%%%%%%%%%%%%%%%%%%%%%%%%%%%%%%%%%%%%
\subsubsection{Distributions}
The \xmlNode{Distributions} block defines all distributions used to
sample variables in the current RAVEN run.

For all the possible distributions and their possible inputs please
refer to the Distributions chapter (see~\ref{sec:distributions}).
%
It is good practice to name the distribution something similar to what kind of
variable is going to be sampled, since there might be many variables with the
same kind of distributions but different input parameters.

%%%%%%%%%%%%%%%%%%%%%%%%%%%%%%%%%%%%%%%%%%%%%%%%%%
\paragraph{Samplers}
The \xmlNode{Samplers} block defines the variables to be sampled.
After defining a sampling scheme, the variables to be sampled and
their distributions are identified in the \xmlNode{variable} blocks.
The name attribute in the \xmlNode{variable} block must either be the
full MooseBasedApp (Rattlesnake) model variable name, the alias name specifed in
\xmlNode{Models}, or the variable name specified in the provided alias files.

For listings of available samplers, please refer to the Samplers chapter (see~\ref{sec:Samplers}).
See the following for an example of a grid based sampler for
the first energy group fission and capture cross sections  (both of which have
defined in alias files provided in \xmlNode{Files}).

\begin{lstlisting}[style=XML]
<Samplers>
  <Grid name="Grid_sampling">
    <variable name="fission_group_1" >
      <distribution>fission_dist</distribution>
      <grid type="value" construction="custom">1.0 2.0</grid>
    </variable>
    <variable name="capture_group_1">
      <distribution>capture_dist</distribution>
      <grid type="value" construction="custom">3.0 6.0</grid>
    </variable>
  </Grid>
</Samplers>
\end{lstlisting}

%%%%%%%%%%%%%%%%%%%%%%%%%%%%%%%%%%%%%%%%%%%%%%%%%%
\subsubsection{Steps}
For a Rattlesnake interface, the \xmlNode{MultiRun} step type will most likely be used. First, the step needs
to be named: this name will be one of the names used in the \xmlNode{Sequence} block. In our example, \xmlString{Grid\_Rattlesnake}.
%
\begin{lstlisting}[style=XML]
<MultiRun name='Grid_Rattlesnake' verbosity='debug'>
    <Input   class='Files' type=''>RattlesnakeInput.i</Input>
    <Input   class='Files' type=''>xs.xml</Input>
    <Input   class='Files' type=''>alias.xml</Input>
    <Model   class='Models' type='Code'>Rattlesnake</Model>
    <Sampler class='Samplers' type='Grid'>Grid_Samplering</Sampler>
    <Output  class='DataObjects' type='PointSet'>solns</Ouput>
\end{lstlisting}
%
With this step, we need to import all the files needed for the simulation:
%
\begin{itemize}
  \item Rattlesnake MOOSE-based input file;
  \item Yak multigroup cross section libraries input files (XML);
  \item Yak alias files used to define the perturbed variables (XML).
\end{itemize}
We then need to define \xmlNode{Model}, \xmlNode{Sampler} and \xmlNode{Output}. The \xmlNode{Output} can be
\xmlNode{DataObjects} or \xmlNode{OutStreams}.

%%%%%%%%%%%%%%%%%%%%%%%%%%%%%%%%%%%%%%%%%%%%%%%%%%%%%%%%%%%%%%%%%%%%%%%%%%%%%%%%%%
%%%%%%%%%%%%%%%%%%%%%%%%%%%%%%%% MAAP5  INTERFACE %%%%%%%%%%%%%%%%%%%%%%%%%%%%%%%%
%%%%%%%%%%%%%%%%%%%%%%%%%%%%%%%%%%%%%%%%%%%%%%%%%%%%%%%%%%%%%%%%%%%%%%%%%%%%%%%%%%
\subsection{MAAP5 Interface}
This section presents the main aspects of the interface coupling RAVEN with MAAP5,
the consequent RAVEN input adjustments and the modifications of the MAAP5
files required to run the two coupled codes.
The interface works both for forward sampling and the DET,
however there are some differences depending on the selected sampling strategy.
%%%%%%%%%%%%%%%%%%%%%%%%%%%%%%%%%%%%%%%%%%%%%%%%%%%%%%%%%%%%%%%%%%%%%%%%%%%%%%%%%%
\subsubsection{RAVEN Input file}
%%%%%%%%%%%%%%%%%%%%%%%%%%%%%%%%%%%%%%%%%%%%%%%%%%%%%%%%%%%%%%%%%%%%%%%%%%%%%%%%%%
\paragraph{Files}
MAAP5 requires more than one file to run a simulation.
This means that, since the \xmlNode{Files} section has to contain all the files required by
the external model (MAAP5) to be run, all these files need to be included within this node.
This involves not only the input file (.inp) but also the include file, the parameter file, all the
files defining the different ``PLOTFILs'', if any, and the other files which could
result useful for the MAAP5 simulation run.

Example:
\begin{lstlisting}[style=XML]
<Files>
  <Input name="test.inp" type="">test.inp</Input>
  <Input name="include" type="">include</Input>
  <Input name="plot.txt" type="">plot.txt</Input>
  <Input name="plant.par" type="">plant.par</Input>
</Files>
\end{lstlisting}
The files mentioned in this section
 need, then, to be put into the working directory specified
by the \xmlNode{workingDir} node into the \xmlNode{RunInfo} block.
%%%%%%%%%%%%%%%%%%%%%%%%%%%%%%%%%%%%%%%%%%%%%%%%%%%%%%%%%%%%%%%%%%%%%%%%%%%%%%%%%%
\paragraph{Models}
The \xmlNode{Models} block contains the name of the executable file of MAAP5
(with the path, if necessary),
and the name of the interface (e.g. MAAP5\_GenericV7).
The block has also some required nodes:
\begin{itemize}
  \item \xmlNode{boolMaapOutputVariables}: containing the number of the MAAP5 IEVNT corresponding to the boolean events of interest;
  \item \xmlNode{contMaapOutputVariables}: containing the list of all the continuous variables we are interested at,
  and that we want to monitor;
  \item \xmlNode{stopSimulation}: this node is required only in case of DET sampling strategy.
The user needs to specify
  if the MAAP5 simulation run stops due to the reached END TIME, specifying ''mission\_time'',
or due to the occurrence of a specific event by
  inserting the number of the corresponding MAAP5 IEVNT (e.g IEVNT(691) for core uncovery)
 \item \xmlNode{includeForTimer}: also this node is required only in case of DET sampling
 strategy and it contains the name of the
 MAAP5 include file where the TIMERS for the different variables are defined
(see paragraph ''MAAP5 include file below'' for more information about timers).
\end{itemize}

A \xmlNode{Models} block is shown as an example below:
\begin{lstlisting}[style=XML]
<Models>
  <Code name="MyMAAP" subType="MAAP5\_GenericV7">
    <executable>MAAP5.exe</executable>
    <clargs type='input' extension='.inp'/>
    <boolMaapOutputVariables>691</boolMaapOutputVariables>
    <contMaapOutputVariables>PPS,PSGGEN(1),ZWDC2SG(1)
    </contMaapOutputVariables>
    <stopSimulation>mission_time</stopSimulation>
    <includeForTimer>include</includeForTimer>
  </Code>
</Models>
\end{lstlisting}
%%%%%%%%%%%%%%%%%%%%%%%%%%%%%%%%%%%%%%%%%%%%%%%%%%%%%%%%%%%%%%%%%%%%%%%%%%%%%%%%%%
\paragraph{Other blocks}
All the other blocks (e.g. \xmlNode{Distributions}, \xmlNode{Samplers}, \xmlNode{Steps},
 \xmlNode{Databases}, \xmlNode{OutStream}, etc.)
do not require any particular arrangements than already provided by a RAVEN input.
User can, therefore, refer to the corresponding sections of the User's Manual.
This is valid for both forward sampling and DET.
%%%%%%%%%%%%%%%%%%%%%%%%%%%%%%%%%%%%%%%%%%%%%%%%%%%%%%%%%%%%%%%%%%%%%%%%%%%%%%%%%%
\subsubsection{MAAP5 Input files}
%%%%%%%%%%%%%%%%%%%%%%%%%%%%%%%%%%%%%%%%%%%%%%%%%%%%%%%%%%%%%%%%%%%%%%%%%%%%%%%%%%
The coupling of RAVEN and MAAP5 requires modifications to some
MAAP5 files in order to work. This is particularly true when a DET analysis is performed.
The MAAP5 input files that need to be modified are:
\begin{itemize}
  \item MAAP5 include file
  \item MAAP5 input file (.inp)
  \item PLOTFIL blocks
\end{itemize}
%%%%%%%%%%%%%%%%%%%%%%%%%%%%%%%%%%%%%%%%%%%%%%%%%%%%%%%%%%%%%%%%%%%%%%%%%%%%%%%%%%
\paragraph{MAAP5 include file}
Usually MAAP5 simulation provides the presence of some include files, for example,
containing the user-defined variables, timers, definition of the plotfil, etc.
The adjustments explained in this section are required only in case of a DET analysis.
The user needs to modify the include file containing the set of the
timers used into the run, by adding the definition of the different timers,
one for each variable that causes a branching.
The include file to be modified should correspond to that one defined in the \xmlNode{includeForTimer}
block of the RAVEN xml input.

User is supposed to check that the numbers used for the different timers definition
are not already used in any of the other MAAP5 files.
These timers should be preceeded by a line reporting ''C Branching + name of the variable
sampled by RAVEN causing the branching''.

For example, we assume that DIESEL is the name of the variable corresponding to the failure time
of the Diesel generators (user defined). User has to firstly ensure that, for example,
''TIMER 100'' is not already used into the model, then the following lines
need to be added into the selected include file for the set of the timer corresponding
to the Diesel generators failure:
\begin{lstlisting}[style=XML]
C Branching DIESEL
WHEN (TIM>DIESEL)
   SET TIMER 100
END
\end{lstlisting}
It is worth mentioning that at this step a TIMER should be defined
also for the event IEVNT specified into the \xmlNode{stopSimulation},
 if this is the stop condition for the MAAP5 run:
\begin{lstlisting}[style=XML]
WHEN IEVNT(691) == 1.0
  SET TIMER 10
END
\end{lstlisting}
The interface will check that one timer is defined for each variable
of the DET. If not, an error arises suggesting to user the name of the variable having no
timer defined.
%%%%%%%%%%%%%%%%%%%%%%%%%%%%%%%%%%%%%%%%%%%%%%%%%%%%%%%%%%%%%%%%%%%%%%%%%%%%%%%%%%
\paragraph{MAAP5 input file}
In the ''parameter change'' section of the MAAP5 input file, the user should declare
the name of the variables sampled by RAVEN according to the following statement:
\begin{lstlisting}[style=XML]
 variable = $ RAVEN-variable:default$
\end{lstlisting}
where the dafault value is optional.

For example:
\begin{lstlisting}[style=XML]
DIESEL = $RAVEN-DIESEL:-1$
\end{lstlisting}
This is valid for both forward and DET sampled variables.
In particular, in case of DET analysis, the variables causing the occurrence of the branch should be
assigned within a block identified by the comment ''C DET Sampled variables'':
\begin{lstlisting}[style=XML]
C DET Sampled Variables
DIESEL = $RAVEN-DIESEL:-1$
C End DET Sampled Variables
\end{lstlisting}
If the sampled variables are user-defined, then the user shall ensure that they are initialized
(to the default value) and set within the user-defined variables section of one of the include
file.
As usual, a distribution and a sampling strategy should then correspond to each of these variables
into the RAVEN xml input file.

Only for the DET analysis, then, the occurence of a branch will be identified by a comment before. This comment is
 ''C BRANCHING + name of the variable determining the branch'' and acts as a sort of branching marker.
Looking for these markers, indeed, the interface (in case of DET sampler) verifies that at least
one branching exists, and furthermore, that one branching is defined for each of the
variables contained into ''DET sampled variables''.

Within the block, the occurrence of the branching leads the value of a variable (user-defined)
called ''TIM+number of the corresponding timer set into
the include file'' to switch to 1.0. The code, in fact, detects if a branch has occurred by monitoring
the value of these kind of variables. SInce these variables are user-defined, they need to be
initialized to a value (different from 1.0), into the ''user-defined variables'' section of one of the include
files.

Therefore following the previous example, if we want that, when the diesels failure occurs it leads
to the event ''Loss of AC Power'' (IEVNT(205) of MAAP5), we will have:
\begin{lstlisting}[style=XML]
C Branching TIMELOCA
WHEN TIM > DIESEL
 TIM100=1.0
 IEVNT(205)=1.0
END
\end{lstlisting}
It is worth noticing that no comments should be contained within the line of assignment
(i.e. IEVNT(205)=1.0 //LOSS OF AC POWER is not allowed).

Finally, only in case of DET analysis, a stop simulation condition (provided by the comment
''C Stop Simulation condition'') needs to be put into the input.
The original input should have all the timers (linked with the branching) separated by an OR
condition, even including that one of the event that stops the simulation (e.g. IEVNT(691)),
if any.
\begin{lstlisting}[style=XML]
C Stop Simulation condition
IF (TIMER 10 > 0) OR (TIMER 100 > 0) OR ... (TIMER N > 0)
 TILAST=TIM
END
\end{lstlisting}
This allows the simulation run to stop when a branch condition occurs, creating the restart file that will
be used by the two following branches.

For each branch, then, the interface will automatically update the name of the RESTART FILE to be used and
of the RESTART TIME that will be equal to the difference between the END TIME of the ''parent'' simulation
and the PRINT INTERVAL (which specifies the interval at which the restart output is written).
%%%%%%%%%%%%%%%%%%%%%%%%%%%%%%%%%%%%%%%%%%%%%%%%%%%%%%%%%%%%%%%%%%%%%%%%%%%%%%%%%%
\paragraph{MAAP5 PLOTFIL blocks}
This section refers to the ''PLOTFIL blocks'' used to modify the plot file (.csv) defined into the parameter file.
These blocks need to be modified in order to include some variables.
It is important, indeed, that the MAAP5 csv PLOTFIL files contain the evolution of:
\begin{itemize}
  \item RAVEN sampled variables (e.g. DIESEL) (both for Forward and DET sampling)
  \item the variables whose value is modified by the occurrence of one of the branches, either continuous or boolean (e.g. IEVNT(225))
  \item the variables of interest defined within \xmlNode{boolMaapOutputVariables}
  and \\ \xmlNode{contMaapOutputVariables} blocks (both for Forward and DET sampling)
\end{itemize}
If one of these variables is not contained into one of csv files, RAVEN will give an error.

%%%%%%%%%%%%%%%%%%%%%%%%%%%%%%%%%%%%%%%%%%%%%%%%%
%%%%%%%%%%%%% MAMMOTH INTERFACE %%%%%%%%%%%%%
%%%%%%%%%%%%%%%%%%%%%%%%%%%%%%%%%%%%%%%%%%%%%%%%%
\subsection{MAMMOTH Interface}
%
This section covers the input specification for running MAMMOTH through RAVEN.
It is important to notice that this short explanation assumes that the reader already knows how to use MAMMOTH.
The interface can be used to perturb Bison, Rattlesnake, RELAP-7, and general MOOSE input files that utilize
MOOSE's standard YAML input structure as well as Yak multigroup cross section library XML input files.
%

%%%%%%%%%%%%%%%%%%%%%%%%%%%%%%%%%%%%%%%%%%%%%%%%%%
\subsubsection{Files}
\xmlNode{Files} works the same as in other interfaces with name and type
attributes for each node entry.  The \xmlAttr{name} attribute is a user-chosen internal
name for the file contained in the node, and \xmlAttr{type} identifies which base-level
interface the file is used within.  \xmlAttr{type} should be specified for all inputs
used in RAVEN's MultiRun for MAMMOTH (including files not perturbed by RAVEN).
The MAMMOTH input file's \xmlAttr{type} should have \xmlString{MAMMOTHInput} prepended
to the driver app's input specification (e.g. \xmlString{MAMMOTHInput|appNameInput}).
Any other app's input file needs a \xmlAttr{type} with the app's name prepended to \xmlString{Input}
(e.g. \xmlString{BisonInput}, \xmlString{Relap7Input}, etc.).  In addition, the \xmlAttr{type} for any mesh
input is the app in which that mesh is utilized prepended to \xmlString{|Mesh}; so a Bison mesh would have
a \xmlAttr{type} of \xmlString{Bison|Mesh} and similarly a mesh for Rattlesnake would have \xmlString{Rattlesnake|Mesh}
as its \xmlAttr{type}. In cases where a file needs to be copied to each perturbed run
directory (to be used as function input, control logic, etc.), one can use the \xmlAttr{type}
\xmlString{AncillaryInput} to make it clear in the RAVEN input file that this is file
is required for the simulation to run but contains no perturbed parameters.
For Yak multigroup cross section libraries,
the \xmlAttr{type} should be \xmlString{YakXSInput}, and for the Yak
alias files that are used to perturb the Yak multigroup cross section libraries, the \xmlAttr{type} should be
\xmlString{YakXSAliasInput}.

The node should contain the path to the file from the working directory.
The following is an example of a typical \xmlNode{Files} block.
%
\begin{lstlisting}[style=XML]
<Files>
  <Input name='mammothInput' type='MAMMOTHInput|RattlesnakeInput'>test_mammoth.i</Input>
  <Input name='crossSection'    type='YakXSInput'>xs.xml</Input>
  <Input name='alias' type='YakXSAliasInput'>alias.xml</Input>
  <Input name='bisonInput'    type='BisonInput'>test_bison.xml</Input>
  <Input name='bisonMesh'    type='Bison|Mesh'>bisonMesh.e</Input>
  <Input name='fuelCTEfunct' type='AncillaryInput'>uo2_CTE.csv</Input>
  <Input name='rattlesnakeMesh'    type='Rattlesnake|Mesh'>rattlesnakeMesh.e</Input>
</Files>
\end{lstlisting}
%
The alias files are employed to define the variables that will be used to perturb Yak multigroup cross section
libraries. Please see the section \ref{RattlesnakeInterfaces} for the example.
%
%%%%%%%%%%%%%%%%%%%%%%%%%%%%%%%%%%%%%%%%%%%%%%%%%%
\subsubsection{Models}
A user provides paths to executables and aliases for sampled variables within the
\xmlNode{Models} block.  The \xmlNode{Code} block will contain \xmlAttr{name} and
\xmlAttr{subType}.  The attribute \xmlAttr{name} identifies that particular \xmlNode{Code} model within RAVEN, and
\xmlAttr{subType} specifies which code interface the model will use. The \xmlNode{executable}
block should contain the absolute or relative (with respect to the current working
directory) path to MAMMOTH that RAVEN will use to run generated input
files.

An example \xmlNode{Models} block follows.

\begin{lstlisting}[style=XML]
<Models>
  <Code name="Mammoth" subType="MAMMOTH">
    <executable>\%FRAMEWORK_DIR\%/../../mammoth/
     mammoth-%METHOD%</executable>
  </Code>
</Models>
\end{lstlisting}

%%%%%%%%%%%%%%%%%%%%%%%%%%%%%%%%%%%%%%%%%%%%%%%%%%
\subsubsection{Distributions}
The \xmlNode{Distributions} block defines all distributions used to
sample variables in the current RAVEN run.

For all the possible distributions and their possible inputs please
refer to the Distributions chapter (see~\ref{sec:distributions}).
%
It is good practice to name the distribution something similar to what kind of
variable is going to be sampled, since there might be many variables with the
same kind of distributions but different input parameters.

%%%%%%%%%%%%%%%%%%%%%%%%%%%%%%%%%%%%%%%%%%%%%%%%%%
\paragraph{Samplers}
The \xmlNode{Samplers} block defines the variables to be sampled.
After defining a sampling scheme, the variables to be sampled and
their distributions are identified in the \xmlNode{variable} blocks.
The \xmlAttr{name} attribute in the \xmlNode{variable} block must either be the
app's name prepended to the full MooseBasedApp model variable name, the alias name specifed in
\xmlNode{Models}, or the variable name specified in the provided alias files.

For listings of available samplers, please refer to the Samplers chapter (see~\ref{sec:Samplers}).
See the following for an example of a grid based sampler used to generate the samples for
the first energy group fission and capture cross sections (both of which have
defined in alias files provided in \xmlNode{Files}), the initial condition temperature defined in Rattlesnake
input file and the poissons ratio, clad thickness, and gap width defined in Bison input files with clad
and gap parameters calculated using an external function with sampled clad inner and outer diameters
as inputs.
%
\begin{lstlisting}[style=XML]
<Samplers>
  <Grid name="Grid_sampling">
    <variable name="Rattlesnake@fission_group_1" >
      <distribution>fission_dist</distribution>
      <grid type="value" construction="custom">1.0 2.0</grid>
    </variable>
    <variable name="Rattlesnake@capture_group_1">
      <distribution>capture_dist</distribution>
      <grid type="value" construction="custom">3.0 6.0</grid>
    </variable>
    <variable name="Rattlesnake@AuxVariables|Temp|initial_condition">
      <distribution>uniform</distribution>
      <grid type="value" construction="custom">3.0 6.0</grid>
    </variable>
    <variable name="Bison@Materials|fuel_solid_mechanics_elastic|poissons_ratio">
      <distribution>normal</distribution>
      <grid type="value" construction="custom">3.0 6.0</grid>
    </variable>
    <variable name='clad_outer_diam'>
      <distribution>clad_outer_diam_dist</distribution>
      <grid construction='equal' steps='144' type='CDF'>0.02275 0.97725</grid>
    </variable>
    <variable name='clad_inner_diam'>
      <distribution>clad_inner_diam_dist</distribution>
      <grid construction='equal' steps='144' type='CDF'>0.02275 0.97725</grid>
    </variable>
    <variable name='Bison@Mesh|clad_thickness'>
      <function>clad_thickness_calc</function>
    </variable>
    <variable name='Bison@Mesh|clad_gap_width'>
      <function>clad_gap_width_calc</function>
    </variable>
  </Grid>
</Samplers>
\end{lstlisting}
%
In order to make the input variables of one application distinct from input variables of another,
an app's name followed by the '@' symbol is prepended to the variable name (e.g. \xmlString{appName@varName}).
Each variable to be used in an app's input file and sampled in the MAMMOTH interface is required
to have a destination app specified. All variables utilizing Rattlesnake's executable (whether
they are in the Rattlesnake input file or not) are listed as Rattlesnake variables
as that application's interface will sort input file and cross section
variables itself.  Notice that the clad inner and outer diameter sampled parameters have no app
name specified.  These parameters are utilized to sample values used as inputs
for the clad thickness and gap width variables in BISON, so by not specifying a destination
app, these are passed through the interface having only been used in an external function
to calculate parameters usable in an app's input.
%%%%%%%%%%%%%%%%%%%%%%%%%%%%%%%%%%%%%%%%%%%%%%%%%%
\subsubsection{Steps}
For a MAMMOTH interface run, the \xmlNode{MultiRun} step type will most likely be used. First, the step needs
to be named: this name will be one of the names used in the \xmlNode{Sequence} block. In our example, \xmlString{Grid\_Mammoth}.
%
\begin{lstlisting}[style=XML]
<MultiRun name='Grid_Mammoth' verbosity='debug'>
    <Input   class='Files' type=''>mammothInput</Input>
    <Input   class='Files' type=''>crossSection</Input>
    <Input   class='Files' type=''>alias</Input>
    <Input   class='Files' type=''>bisonInput</Input>
    <Input   class='Files' type=''>bisonMesh</Input>
    <Input   class='Files' type=''>fuelCTEfunct</Input>
    <Input   class='Files' type=''>rattlesnakeMesh</Input>
    <Model   class='Models' type='Code'>Mammoth</Model>
    <Sampler class='Samplers' type='Grid'>Grid_Samplering</Sampler>
    <Output  class='DataObjects' type='PointSet'>solns</Output>
</MultiRun>
\end{lstlisting}
%
With this step, we need to import all the files needed for the simulation:
%
\begin{itemize}
  \item MAMMOTH|Rattlesnake YAML input file;
  \item Yak multigroup cross section libraries input files (XML);
  \item Yak alias files used to define the perturbed variables (XML);
  \item Bison YAML input file;
  \item Bison mesh file;
  \item Bison function file for the fuel's coefficient of thermal expansion as a function of temperature;
  \item Rattlesnake mesh file.
\end{itemize}
As well as \xmlNode{Model}, \xmlNode{Sampler} and outputs, such as \xmlNode{OutStreams} and \xmlNode{DataObjects}.

%%%%%%%%%%%%%%%%%%%%%%%%%%%%
%%%%%% MELCOR  INTERFACE  %%%%%%
%%%%%%%%%%%%%%%%%%%%%%%%%%%%
\subsection{MELCOR Interface}
\label{subsec:MELCORInterface}

The current implementation of MELCOR interface is valid for MELCOR 2.1/2.2; its validity for MELCOR
1.8 is \textbf{not been tested}.

\subsubsection{Sequence}
In the \xmlNode{Sequence} section, the names of the steps declared in the
\xmlNode{Steps} block should be specified.
%
As an example, if we called the first multirun ``Grid\_Sampler'' and the second
multirun ``MC\_Sampler'' in the sequence section we should see this:
\begin{lstlisting}[style=XML]
<Sequence>Grid_Sampler,MC_Sampler</Sequence>
\end{lstlisting}
%%%%%%%%%%%%%%%%%%%%%%%%%%%%%%%%%%%%%%%%%%%%%%%%%%%

\subsubsection{batchSize and mode}
For the \xmlNode{batchSize} and \xmlNode{mode} sections please refer to the
\xmlNode{RunInfo} block in the previous chapters.
%
%%%%%%%%%%%%%%%%%%%%%%%%%%%%%%%%%%%%%%%%%%%%%%%%%%%%
\subsubsection{RunInfo}
After all of these blocks are filled out, a standard example RunInfo block may
look like the example below:
\begin{lstlisting}[style=XML]
<RunInfo>
  <WorkingDir>~/workingDir</WorkingDir>
  <Sequence>Grid_Sampler,MC_Sampler</Sequence>
  <batchSize>8</batchSize>
</RunInfo>
\end{lstlisting}
In this example, the \xmlNode{batchSize} is set to $8$; this means that 8 simultaneous (parallel) instances
of MELCOR are going to be executed when a sampling strategy is employed.
%%%%%%%%%%%%%%%%%%%%%%%%%%%%%%%%%%%%%%%%%%%%%%%%%%%%%%%%%%%
\subsubsection{Files}
In the \xmlNode{Files} section, as specified before, all of the files needed for
the code to run should be specified.
%
In the case of MELCOR, the files typically needed are:
\begin{itemize}
  \item MELCOR Input file (file extension ``.i'' or ``.inp'')
  \item Restart file (if present)
\end{itemize}
Example:
\begin{lstlisting}[style=XML]
<Files>
  <Input name='melcorInputFile' type=''>inputFileMelcor.i</Input>
  <Input name='aRestart' type=''>restartFile</Input>
</Files>
\end{lstlisting}

It is a good practice to put inside the working directory (\xmlNode{WorkingDir}) all of these files.

\textcolor{red}{
\textbf{It is important to notice that the interface output collection  (i.e. the parser of the MELCOR output)
currently is able to extract \textit{CONTROL VOLUME HYDRODYNAMICS EDIT AND CONTROL FUNCTION EDIT} data only. Only those
variables are going to be exported and make available to RAVEN.
In addition, it is important to notice that:}
\begin{itemize}
  \item \textbf{the simulation time is stored in a variable called \textit{``time''}};
  \item \textbf{all the variables specified in the \textit{CONTROL VOLUME HYDRODYNAMICS EDIT}
   block are going to be converted using underscores. For example, the following EDITs:}
    \begin{table}[h]
    \centering
    \begin{tabular}{ccccc}
        VOLUME & PRESSURE & TLIQ   & TVAP   & MASS     \\
                & PA       & K      & K      & KG       \\
             1      & 1.00E+07 & 584.23 & 584.23 & 1.66E+03
     \end{tabular}
    \end{table}
    \\\textbf{will be converted in the following way (CSV):}
    \begin{table}[h]
    \centering
    \begin{tabular}{ccccc}
         $time$ & $volume\_1\_PRESSURE$& $volume\_1\_TLIQ$ & $volume\_1\_TVAP$   & $volume\_1\_MASS$     \\
             1.0   & 1.00E+07 & 584.23 & 584.23 & 1.66E+03
     \end{tabular}
    \end{table}
\end{itemize}
}

CONTROL FUNCTION EDIT data will not be converted in this manner. All data will be labeled using a label identical to what was entered in the MELCOR input file, with no changes.

Remember also that a MELCOR simulation run is considered successful (i.e., the simulation did not crash) if it terminates with the
following message:

\textcolor{red}{Normal termination}

If the a MELCOR simulation run stops with messages other than this one than the simulation is considered as
crashed, i.e., it will not be saved.
Hence, it is strongly recommended to set up the MELCOR input file so that the simulation exiting conditions are set through control logic
trip variables.

%%%%%%%%%%%%%%%%%%%%%%%%%%%%%%%%%%%%%%%%%%%%%%%%%%%%
\subsubsection{Models}
For the \xmlNode{Models} block here is a standard example of how it would look
when using MELCOR 2.1/2.2 as the external code:
\begin{lstlisting}[style=XML]
<Models>
  <Code name='MyMELCOR' subType='Melcor'>
    <executable>~/path_to_the_executable_of_melcor</executable>
    <preexec>~/path_to_the_executable_of_melgen</preexec>
  </Code>
</Models>
\end{lstlisting}
As it can be seen above, the \xmlNode{preexec} node must be specified, since MELCOR 2.1/2.2 must run the MELGEN utility
code before executing. Once the \xmlNode{preexec} node is inputted, the execution of MELGEN is performed automatically by the Interface.
\\In addition, if some command line parameters need to be passed to MELCOR, the user might use (optionally) the \xmlNode{clargs} XML nodes.
\begin{lstlisting}[style=XML]
<Models>
  <Code name='MyMELCOR' subType='Melcor'>
    <executable>~/path_to_the_executable_of_melcor</executable>
    <preexec>~/path_to_the_executable_of_melgen</preexec>
    <clargs type="text" arg="-r whatever command line instruction"/>
  </Code>
</Models>
\end{lstlisting}

%%%%%%%%%%%%%%%%%%%%%%%%%%%%%%%%%%%%%%%%%%%%%%%%%%%%%%%%%
\subsubsection{Distributions}
The \xmlNode{Distribution} block defines the distributions that are going
to be used for the sampling of the variables defined in the \xmlNode{Samplers}
block.
%
For all the possible distributions and all their possible inputs please see the
chapter about Distributions (see~\ref{sec:distributions}).
%
Here we report an example of a Normal distribution:
\begin{lstlisting}[style=XML,morekeywords={name,debug}]
<Distributions verbosity='debug'>
    <Normal name="temper">
      <mean>1.E+7</mean>
      <sigma>1.5</sigma>
      <upperBound>9.E+6</upperBound>
      <lowerBound>1.1E+7</lowerBound>
    </Normal>
 </Distributions>
\end{lstlisting}

It is good practice to name the distribution something similar to what kind of
variable is going to be sampled, since there might be many variables with the
same kind of distributions but different input parameters.
%
%%%%%%%%%%%%%%%%%%%%%%%%%%%%%%%%%%%%%%%%%%%%%%%%%%%%%%%%%
\subsubsection{Samplers}
In the \xmlNode{Samplers} block we want to define the variables that are going
to be sampled.
%
\textbf{Example}:
We want to do the sampling of 1 single variable:
\begin{itemize}
  \item The in pressure ($P\_in$) of a control volume regulated by a Tabular Function $TF\_TAB$
\end{itemize}

We are going to sample this variable using two different sampling methods:
Grid and MonteCarlo.

The interface of MELCOR uses the \textbf{\textit{GenericCode}} (see section \ref{subsec:genericInterface})
interface for the input perturbation; this means that the original input file (listed in the \xmlNode{Files} XML block)
needs to implement wild-cards.
%
In this example we are sampling the variable:
\begin{itemize}
  \item \textit{PRE}, which acts on the Tabular Function $TF\_TAB$ whose $TF\_ID $ is $P\_in$.
\end{itemize}

We proceed to do so for both the Grid sampling and the MonteCarlo sampling.

\begin{lstlisting}[style=XML,morekeywords={name,type,construction,lowerBound,steps,limit,initialSeed}]
<Samplers verbosity='debug'>
  <Grid name='Grid_Sampler' >
    <variable name='PRE'>
      <distribution>temper</distribution>
      <grid type='CDF' construction='equal'  steps='10'>0.001 0.999</grid>
    </variable>
  </Grid>
  <MonteCarlo name='MC_Sampler'>
     <samplerInit>
       <limit>1000</limit>
     </samplerInit>
    <variable name='PRE'>
      <distribution>temper</distribution>
  </MonteCarlo>
</Samplers>
\end{lstlisting}

It can be seen that each variable is connected with a proper distribution
defined in the \\\xmlNode{Distributions} block (from the previous example).
%
The following demonstrates how the input for the variable is read.

We are sampling a variable whose wild-card in the original input file is named $\$RAVEN-PRE\$$
using a Grid sampling method.
%
The distribution that this variable is following is a Normal distribution
(see section above).
%
We are sampling this variable beginning from 0.001 (CDF) in 10 \textit{equal} steps of
0.0998 (CDF).
%
%%%%%%%%%%%%%%%%%%%%%%%%%%%%%%%%%%%%%%%%%%%%%%%%%%%%%%%%%%%
\subsubsection{Steps}
For a MELCOR interface, the \xmlNode{MultiRun} step type will most likely be
used.
%
First, the step needs to be named: this name will be one of the names used in
the \xmlNode{sequence} block.
%
In our example, \texttt{Grid\_Sampler} and \texttt{MC\_Sampler}.
%
\begin{lstlisting}[style=XML,morekeywords={name,debug,re-seeding}]
     <MultiRun name='Grid_Sampler' verbosity='debug'>
\end{lstlisting}

With this step, we need to import all the files needed for the simulation:
\begin{itemize}
  \item MELCOR input file
  \item any other file needed by the calculation (e.g. restart file)
\end{itemize}
\begin{lstlisting}[style=XML,morekeywords={name,class,type}]
    <Input   class='Files' type=''>inputFileMelcor.i</Input>
    <Input   class='Files' type=''>restartFile</Input>
\end{lstlisting}
We then need to define which model will be used:
\begin{lstlisting}[style=XML]
    <Model  class='Models' type='Code'>MyMELCOR</Model>
\end{lstlisting}
We then need to specify which Sampler is used, and this can be done as follows:
\begin{lstlisting}[style=XML]
    <Sampler class='Samplers' type='Grid'>Grid_Sampler</Sampler>
\end{lstlisting}
And lastly, we need to specify what kind of output the user wants.
%
For example the user might want to make a database (in RAVEN the database
created is an HDF5 file).
%
Here is a classical example:
\begin{lstlisting}[style=XML,morekeywords={class,type}]
    <Output  class='Databases' type='HDF5'>Grid_out</Output>
\end{lstlisting}
Following is the example of two MultiRun steps which use different sampling
methods (Grid and Monte Carlo), and creating two different databases for each
one:
\begin{lstlisting}[style=XML]
<Steps verbosity='debug'>
  <MultiRun name='Grid_Sampler' verbosity='debug'>
    <Input   class='Files' type=''>inputFileMelcor.i</Input>
    <Input   class='Files' type=''>restartFile</Input>
    <Model   class='Models'    type='Code'>MyMELCOR</Model>
    <Sampler class='Samplers'  type='Grid'>Grid_Sampler</Sampler>
    <Output  class='Databases' type='HDF5'>Grid_out</Output>
    <Output  class='DataObjects' type='PointSet'   >GridMelcorPointSet</Output>
    <Output  class='DataObjects' type='HistorySet'>GridMelcorHistorySet</Output>
  </MultiRun>
  <MultiRun name='MC_Sampler' verbosity='debug' re-seeding='210491'>
    <Input   class='Files' type=''>inputFileMelcor.i</Input>
    <Input   class='Files' type=''>restartFile</Input>
    <Model   class='Models'    type='Code'>MyMELCOR</Model>
    <Sampler class='Samplers'  type='MonteCarlo'>MC_Sampler</Sampler>
    <Output  class='Databases' type='HDF5'      >MC_out</Output>
    <Output  class='DataObjects' type='PointSet'   >MonteCarloMelcorPointSet</Output>
    <Output  class='DataObjects' type='HistorySet'>MonteCarloMelcorHistorySet</Output>
  </MultiRun>
</Steps>
\end{lstlisting}
%%%%%%%%%%%%%%%%%%%%%%%%%%%%%%%%%%%%%%%%%%%%%%%%%%%%%%
\subsubsection{Databases}
As shown in the \xmlNode{Steps} block, the code is creating two database objects
called \texttt{Grid\_out} and \texttt{MC\_out}.
%
So the user needs to input the following:
\begin{lstlisting}[style=XML]
<Databases>
  <HDF5 name="Grid_out" readMode="overwrite"/>
  <HDF5 name="MC_out" readMode="overwrite"/>
</Databases>
\end{lstlisting}
As listed before, this will create two databases.
%
The files will have names corresponding to their \xmlAttr{name} appended with
the .h5 extension (i.e. \texttt{Grid\_out.h5} and \texttt{MC\_out.h5}).
%%%%%%%%%%%%%%%%%%%%%%%%%%%%%%%%%%%%%%%%%%%%%%%%%%%%%%
\subsubsection{DataObjects}
As shown in the \xmlNode{Steps} block, the code is creating $4$ data objects ($2$ HistorySet and $2$ PointSet)
called \texttt{GridMelcorPointSet} \texttt{GridMelcorHistorySet} \texttt{MonteCarloMelcorPointSet} and
 \texttt{MonteCarloMelcorHistorySet}.
%
So the user needs to input the following block as well, where the Input and Output variables are listed:
\begin{lstlisting}[style=XML]
  <DataObjects>
    <PointSet name="GridMelcorPointSet">
      <Input>PRE</Input>
      <Output>
        time,volume_1_PRESSURE,volume_1_TLIQ,
        volume_1_TVAP,volume_1_MASS
      </Output>
    </PointSet>
    <HistorySet name="GridMelcorHistorySet">
      <Input>PRE</Input>
      <Output>
        time,volume_1_PRESSURE,volume_1_TLIQ,
        volume_1_TVAP,volume_1_MASS
      </Output>
    </HistorySet>
    <PointSet name="MonteCarloMelcorPointSet">
      <Input>PRE</Input>
      <Output>
        time,volume_1_PRESSURE,volume_1_TLIQ,
        volume_1_TVAP,volume_1_MASS
      </Output>
    </PointSet>
    <HistorySet name="MonteCarloMelcorHistorySet">
      <Input>PRE</Input>
      <Output>
        time,volume_1_PRESSURE,volume_1_TLIQ,
        volume_1_TVAP,volume_1_MASS
      </Output>
    </HistorySet>
  </DataObjects>
\end{lstlisting}
As mentioned before, this will create $4$ DataObjects.
%
%%%%%%%%%%%%%%%%%%%%%%%
%%%%%% SCALE  INTERFACE %%%%%%
%%%%%%%%%%%%%%%%%%%%%%%
\subsection{SCALE Interface}
This section presents the main aspects of the interface between RAVEN and SCALE system,
the consequent RAVEN input adjustments and the modifications of the SCALE
files required to run the two coupled codes.
\\ \textcolor{red}{
\textbf{\textit{\nb Considering the large amount of SCALE sequences, this interface is
currently limited in driving the following SCALE calculation codes:}}
\begin{itemize}
  \item \textbf{\textit{ORIGEN}}
  \item \textbf{\textit{TRITON (using NEWT as transport solver)}}
\end{itemize}
}

In the following sections a short explanation on how to use RAVEN coupled with SCALE is reported.

%%%%%%%%%%%%%%%%%%%%%%%%%%%%%%%%%%%%%%%%%%%%%%%%%%%%%%%%%

\subsubsection{Models}
As for any other Code, in order to activate the SCALE interface, a  \xmlNode{Code} XML node needs to be inputted, within the
main XML node \xmlNode{Models}.
\\The  \xmlNode{Code} XML node contains the
information needed to execute the specific External Code.

\attrsIntro
%
\vspace{-5mm}
\begin{itemize}
  \itemsep0em
  \item \nameDescription
  \item \xmlAttr{subType}, \xmlDesc{required string attribute}, specifies the
  code that needs to be associated to this Model.
  %
  \nb See Section~\ref{sec:existingInterface} for a list of currently supported
  codes.
  %
\end{itemize}
\vspace{-5mm}

\subnodesIntro
%
\begin{itemize}
  \item \xmlNode{executable} \xmlDesc{string, required field} specifies the path
  of the executable to be used.
  %
  \nb Either an absolute or relative path can be used.
  \item \aliasSystemDescription{Code}
  %
\end{itemize}

In addition (and specifc for the SCALE interface), the  \xmlNode{Code} can contain the following optional nodes:

\begin{itemize}
  \item \xmlNode{sequence}, optional, comma separated list. In this node the user can specify a list of sequences that need to be
  executed in sequence. For example, if a TRITON calculation needs to be followed by an ORIGEN decay heat calculation the user
  would input here the sequence ``\textit{triton,origen}''. \default{triton}.
  \\\nb Currently only the following entries are supported:
    \begin{itemize}
     \item  ``\textit{triton}''
     \item  ``\textit{origen}''
     \item  ``\textit{triton,origen}''
    \end{itemize}
  \item \xmlNode{timeUOM}, optional, string. In this node the user can specify  the \textit{units} for the independent variable ``time''.
   If the outputs are exported by SCALE in a different unit (e.g days, years, etc.), the SCALE interface will convert all the different
   time scales into the unit here specified (in order to have a consistent  (and unique) time scale). Available are:
    \begin{itemize}
     \item ``\textit{s}'', seconds
     \item ``\textit{m}'', minutes
     \item ``\textit{h}'', hours
     \item ``\textit{d}'', days
     \item ``\textit{y}'', years
    \end{itemize}
    \default{s}
\end{itemize}

An example  is shown  below:
\begin{lstlisting}[style=XML]
<Models>
    <Code name="MyScale" subType="Scale">
      <executable>path/to/scalerte</executable>
      <sequence>triton,origen</sequence>
      <timeUOM>d</timeUOM>
    </Code>
</Models>
\end{lstlisting}

%%%%%%%%%%%%%%%%%%%%%%%%%%%%%%%%%%%%%%%%%%%%%%%%%%%%%%%%%%%%%%%%%%%%%%%%%%%%%%%%%%
\subsubsection{Files}
%%%%%%%%%%%%%%%%%%%%%%%%%%%%%%%%%%%%%%%%%%%%%%%%%%%%%%%%%%%%%%%%%%%%%%%%%%%%%%%%%%
The \xmlNode{Files} XML node has to contain all the files required by the particular
sequence (s) of the external code  (SCALE) to be run.
This involves not only the input file(s) (.inp) but also the auxiliary files that might be needed (e.g. binary initial compositions, etc.).
As mentioned, the current SCALE interface only supports TRITON and ORIGEN sequences. For this reason, depending on the
type of sequence (see previous section) to be run, the relative input files need to be marked with the sequence they are associated
with. This means that the type of the input file must be either ``triton'' or ``origen''. The auxiliary files that might be needed by
a particular sequence (e.g. binary initial compositions, etc.) should not be marked with any specific type (i.e. \textit{type=``''}).
Example:
\begin{lstlisting}[style=XML]
<Files>
  <Input name="triton_input" type="triton">pwr_depletion.inp</Input>
  <Input name="origen_input" type="origen">decay_calc.inp</Input>
  <Input name="binary_comp" type="">pwr_depletion.f71</Input>
</Files>
\end{lstlisting}
The files mentioned in this section
 need, then, to be placed into the working directory specified
by the \xmlNode{workingDir} node in the \xmlNode{RunInfo} XML node.

\paragraph{Output Files conversion}
Since RAVEN expects to receive a CSV file containing the outputs of the simulation, the results in the SCALE output
files need to be converted by the code interface.
\\As mentioned, the current interface \textcolor{red}{ is able to collect data from TRITON and ORIGEN sequences only}.
%% TRITON
\\The following information is collected from TRITON output:
\begin{itemize}
  \item \textit{\textbf{k-eff and k-inf time-dep information}}
  \begin{lstlisting}[basicstyle=\tiny]
  Outer   Eigenvalue Eigenvalue Max Flux   Max Flux     Max Fuel   Max Fuel     Wall   Elapsed   Iteration  CPU   Inners
 Iter. #              Delta      Delta   Location(r,g)   Delta   Location(r,g) Clock   CPU Time   CPU Time Usage Converged
 - - - - - - - - - - - - - - - - - - - - - - - - - - - - - - - - - - - - - - - - - - - - - - - - - - - - - - - - - - - -
     1    1.00000   0.000E+00 6.480E+09 (    4,252)   1.000E+00 (  614,  0) 14:16:42   89.9 s    89.9 s  92.7%    F
     2    0.35701   1.801E+00 4.149E+01 (  319,  4)   2.673E+00 ( 7035,  0) 14:18:16  182.8 s    92.9 s  98.8%    F
 k-eff =       0.94724509     Time=      0.00d Nominal conditions

   Four-Factor Estimate of k-infinity.  Fast/Thermal boundary:   0.6250 eV
      Fiss. neutrons/thermal abs. (eta):          1.279827
      Thermal utilization (f):                    0.960903
      Resonance Escape probability (p):           0.706209
      Fast-fission factor (epsilon):              1.091716
                                            --------------
      Infinite neutron multiplication             0.948143

\end{lstlisting}
   that will be converted in the following way (CSV):
   \begin{table}[h]
    \centering
    \caption{CSV transport info}
    \label{CSVkeff}
    \tabcolsep=0.11cm
    \tiny
    \begin{tabular}{|c|c|c|c|c|c|c|c|c|c|}
     time & keff       & iter\_number & keff\_delta & max\_flux\_delta & kinf     & kinf\_epsilon & kinf\_p  & kinf\_f  & kinf\_eta \\
     0.00 & 0.94724509 & 2            & 1.801E+00   & 4.149e+01        & 0.948143 & 1.091716      & 0.706209 & 0.960903 & 1.279827
    \end{tabular}
   \end{table}

  \item \textit{\textbf{material powers}}
  \begin{lstlisting}[basicstyle=\tiny]
  --- Material powers for depletion pass no.   1 (MW/MITHM) ---
       Time =     0.00 days (   0.000 y), Burnup =    0.000     GWd/MTIHM, Transport k=  0.9473

                    Total    Fractional  Mixture     Mixture       Mixture
         Mixture    Power      Power      Power    Thermal Flux  Total Flux
          Number (MW/MTIHM)    (---)   (MW/MTIHM)  n/(cm^2*sec)  n/(cm^2*sec)
            13      32.985    0.99054     32.985    5.3666e+13    1.2574e+14
             6       0.252    0.00757     N/A       2.7587e+13    9.1781e+13
         Total      33.300    1.00000
\end{lstlisting}
   that will be converted in the following way (CSV):
   \begin{table}[h]
     \centering
     \caption{CSV material powers}
     \label{CSVmatPowers}
     \tabcolsep=0.11cm
     \tiny
     \begin{tabular}{|c|c|c|c|c|c|c|c|c|c|l}
     \cline{1-10}
     time    & bu  & tot\_power\_mix\_13 & fract\_power\_mix\_13 & th\_flux\_mix\_13 & tot\_flux\_mix\_13 & tot\_power\_mix\_6 & fract\_power\_mix\_6 & th\_flux\_mix\_6 & tot\_flux\_mix\_6 &  \\ \cline{1-10}
     1.0E-06 & 0.0 & 32.985              & 0.99054               & 5.3666e+13        & 1.2574e+14         & 0.252              & 0.00757              & 2.7587e+13       & 9.1781e+13        &  \\ \cline{1-10}
     \end{tabular}
   \end{table}


 \item \textit{\textbf{nuclide/element tables}}
  \begin{lstlisting}[basicstyle=\tiny]
            | nuclide concentrations
            | time: days
      grams |    0.00e+00d
------------+--------------------
       u235 |   2.9619e+04
       u238 |   9.6993e+05
   subtotal |   1.0010e+06
      total |   1.1858e+06
\end{lstlisting}
   that will be converted in the following way (CSV):
   \begin{table}[h]
    \centering
    \caption{CSV Nuclide/element Tables}
    \label{CSVnuclideTables}
    \tabcolsep=0.11cm
    \tiny
    \begin{tabular}{|c|c|c|}
     time & u235\_conc       & u238\_conc   \\
     0.00 & 2.9619e+04  & 9.6993e+05
    \end{tabular}
   \end{table}
\end{itemize}
%% ORIGEN
The following information is collected from ORIGEN output:
\begin{itemize}
  \item \textit{\textbf{history overview}}
  \begin{lstlisting}[basicstyle=\tiny]
=========================================================================================================================
=   History overview for case 'decay' (#1/1)                                                                            =
-------------------------------------------------------------------------------------------------------------------------
   step          t0          t1          dt           t        flux     fluence       power      energy
    (-)       (sec)       (sec)         (s)         (s)   (n/cm2-s)     (n/cm2)        (MW)       (MWd)
      1  0.0000E+00  1.0000E-06  1.0000E-06  1.0000E-06  0.0000E+00  0.0000E+00  0.0000E+00  0.0000E+00
\end{lstlisting}
   that will be converted in the following way (CSV):
    \begin{table}[h]
    \centering
    \caption{CSV History Overview}
    \label{CSVhistoryOverview}
    \tabcolsep=0.11cm
    \tiny
    \begin{tabular}{|c|c|c|c|c|c|c|c|}
    \hline
     time    & t0  & t1      & dt      & flux & fluence & power & energy \\
     1.0E-06 & 0.0 & 1.0E-06 & 1.0E-06 & 0.0  & 0.0     & 0.0   & 0.0
    \end{tabular}
   \end{table}

   \item \textit{\textbf{concentration tables}}
  \begin{lstlisting}[basicstyle=\tiny]
=========================================================================================================================
=   Nuclide concentrations in watts, actinides for case 'decay' (#1/1)                                                  =
-------------------------------------------------------------------------------------------------------------------------
  (relative cutoff; integral of concentrations over time >   1.00E-04 % of integral of all concentrations over time)
.
                0.0E+00sec  1.0E-06sec
  th231       8.6167E-08  8.6167E-08
  th234       7.7763E-09  7.7763E-09
------------
  totals       4.6831E+03  4.6831E+03
=========================================================================================================================
.
.
=========================================================================================================================
=   Nuclide concentrations in watts, fission products for case 'decay' (#1/1)                                           =
-------------------------------------------------------------------------------------------------------------------------
  (relative cutoff; integral of concentrations over time >   1.00E-04 % of integral of all concentrations over time)
.
                0.0E+00sec  1.0E-06sec
  ga74        2.4264E-01  2.4264E-01
  ga75        1.8106E+00  1.8106E+00
------------
  totals       1.2266E+06  1.2266E+06
  \end{lstlisting}
  that will be converted in the following way (CSV):
   \begin{table}[h]
    \centering
    \caption{CSV Concentration Tables}
    \label{CSVconcentrationTables}
    \tabcolsep=0.11cm
    \tiny
    \begin{tabular}{|c|c|c|c|c|c|c|c|}
    \hline
     time    & ga74\_watts  & ga75\_watts      & subtotals\_fission\_products      & th231\_watts & th234\_watts & subtotals\_actinides & totals\_watts \\ \hline
     0.0E+00 & 2.4264E-01 & 1.8106E+00 & 1.2266E+06 & 8.6167E-08  & 7.7763E-09     & 4.6831E+03   & 1.2313E+06    \\
     1.0E-06 & 2.4264E-01 & 1.8106E+00 & 1.2266E+06 & 8.6167E-08  & 7.7763E-09     & 4.6831E+03  & 1.2313E+06
    \end{tabular}
   \end{table}
\end{itemize}

\textbf{Remember also that a SCALE simulation run is considered successful (i.e., the simulation did not crash) if it does not contain, in
the last 20 lines, the following message:}

\textcolor{red}{terminated due to errors}

\textbf{If the a SCALE simulation terminates with this message, the simulation is considered ``failed'', i.e., it will not be saved.}

%%%%%%%%%%%%%%%%%%%%%%%%%%%%%%%%%%%%%%%%%%%%%%%%%%%%%%%%%%%%%%%%%%%%%%%%%%%%%%%%%%
\subsubsection{Samplers or Optimizers}
In the \xmlNode{Samplers} or  \xmlNode{Optimizers} block we want to define the variables that are going
to be sampled or optimized.
%
\\The perturbation or optimization of the input of any SCALE sequence is performed using the approach detailed in the \textit{Generic Interface} section (see \ref{subsec:genericInterface}). Briefly, this approach uses
 ``wild-cards'' (placed in the original input files) for injecting the perturbed values.
 For example, if the original input file (that needs to be perturbed) is the following:
\begin{lstlisting}[language=python]
=origen
case(actual_mass){
  lib{ file="end7dec" }
  mat{ iso=[zr-95=1.0] units="moles" }
  time=[1.0] %1 day
}
end
\end{lstlisting}
and  the initial moles of ``zr-95'' need to be perturbed, a RAVEN ``wild-card'' will be defined:
\begin{lstlisting}[language=python]
=origen
case(actual_mass){
  lib{ file="end7dec" }
  mat{ iso=[zr-95=$RAVEN-zrMoles$] units="moles" }
  time=[1.0] %1 day
}
end
\end{lstlisting}

Finally, the variable \textbf{\textit{zrMoles}} needs to be specified in the specific Sampler or Optimizer that will be used:

\begin{lstlisting}[style=XML]
...
<Samplers>
  <aSampler name='aUserDefinedName' >
    <variable name='zrMoles'>
      ...
    </variable>
  </aSampler>
</Samplers>
...
<Optimizers>
  <anOptimizer name='aUserDefinedName' >
    <variable name='zrMoles'>
      ...
    </variable>
  </anOptimizer>
</Samplers>
...
\end{lstlisting}
%
%%%%%%%%%%%%%%%%%%%%%%%
%%%%%% COBRA-TF  INTERFACE %%%%%%
%%%%%%%%%%%%%%%%%%%%%%%
\subsection{COBRA-TF Interface}
This section presents the main aspects of the interface between RAVEN and Cobra-TF system,
the consequent RAVEN input adjustments and the modifications of the Cobra-TF
files required to run the two coupled codes.
\\ \textcolor{red}{
\textbf{\textit{\nb This interface is currently working only with the specific type of Cobra-TF output file (.ctf.out) }}
}

In the following sections a short explanation on how to use RAVEN coupled with Cobra-TF is reported.
%%%%%%%%%%%%%%%%%%%%%%%%%%%%%%%%%%%%%%%%%%%%%%%%%%%
\subsubsection{Sequence}
%%%%%%%%%%%%%%%%%%%%%%%%%%%%%%%%%%%%%%%%%%%%%%%%%%%
In the \xmlNode{Sequence} section, the names of the steps declared in the
\xmlNode{Steps} block should be specified.
%
As an example, if we called the first MultiRun ``Grid\_Sampler'' and the second
MultiRun ``MC\_Sampler'' in the sequence section we should see this:
\begin{lstlisting}[style=XML]
<Sequence>Grid_Sampler, MC_Sampler</Sequence>
\end{lstlisting}

%%%%%%%%%%%%%%%%%%%%%%%%%%%%%%%%%%%%%%%%%%%%%%%%%%%
\subsubsection{batchSize and mode}
%%%%%%%%%%%%%%%%%%%%%%%%%%%%%%%%%%%%%%%%%%%%%%%%%%%
For the \xmlNode{batchSize} and \xmlNode{mode} sections please refer to the
\xmlNode{RunInfo} block in the previous chapters.

%%%%%%%%%%%%%%%%%%%%%%%%%%%%%%%%%%%%%%%%%%%%%%%%%%%%
\subsubsection{RunInfo}
%%%%%%%%%%%%%%%%%%%%%%%%%%%%%%%%%%%%%%%%%%%%%%%%%%%%
After all of these blocks are filled out, a standard example RunInfo block may
look like the example below:
\begin{lstlisting}[style=XML]
<RunInfo>
  <WorkingDir>~/workingDir</WorkingDir>
  <Sequence>Grid_Sampler,MC_Sampler</Sequence>
  <batchSize>8</batchSize>
</RunInfo>
\end{lstlisting}
In this example, the \xmlNode{batchSize} is set to $8$; this means that 8 simulatenous (parallel) instances
of COBRA-TF are going to be executed when a sampling strategy is employed.

%%%%%%%%%%%%%%%%%%%%%%%%%%%%%%%%%%%%%%%%%%%%%%%%%%%%%%%%%
\subsubsection{Models}
%%%%%%%%%%%%%%%%%%%%%%%%%%%%%%%%%%%%%%%%%%%%%%%%%%%%%%%%%
As any other Code, in order to activate the Cobra-TF interface, a \xmlNode{Code} XML node needs to be inputted, within the
main XML node \xmlNode{Models}.
\\The  \xmlNode{Code} XML node contains the
information needed to execute the specific External Code.

\attrsIntro
%
\vspace{-5mm}
\begin{itemize}
  \itemsep0em
  \item \nameDescription
  \item \xmlAttr{subType}, \xmlDesc{required string attribute}, specifies the
  code that needs to be associated to this Model.
  %
  \nb See Section~\ref{sec:existingInterface} for a list of currently supported
  codes.
  %
\end{itemize}
\vspace{-5mm}

\subnodesIntro
%
\begin{itemize}
  \item \xmlNode{executable} \xmlDesc{string, required field} specifies the path
  of the executable to be used.
  %
  \nb Either an absolute or relative path can be used.
  \item \aliasSystemDescription{Code}
  %
\end{itemize}

An example  is shown  below:
\begin{lstlisting}[style=XML]
<Models>
    <Code name="MyCobraTF" subType="CobraTF">
      <executable>path/to/cobratf</executable>
    </Code>
</Models>
\end{lstlisting}

%%%%%%%%%%%%%%%%%%%%%%%%%%%%%%%%%%%%%%%%%%%%%%%%%%%%%%%%%%%%%%%%%%%%%%%%%%%%%%%%%%
\subsubsection{Files}
%%%%%%%%%%%%%%%%%%%%%%%%%%%%%%%%%%%%%%%%%%%%%%%%%%%%%%%%%%%%%%%%%%%%%%%%%%%%%%%%%%
The \xmlNode{Files} XML node has to contain all the files required to run the external code  (Cobra-TF).
For RAVEN coupled with Cobra-TF, only input file(s) (.inp) are needed.

Example:
\begin{lstlisting}[style=XML]
<Files>
  <Input name="cobraTF_input" type="">case1.inp</Input>
</Files>
\end{lstlisting}
The files mentioned in this section
 need, then, to be placed into the working directory specified
by the \xmlNode{workingDir} node in the \xmlNode{RunInfo} XML node.

\paragraph{Output Files conversion}
Since RAVEN expects to receive a CSV file containing the outputs of the simulation, the results in the Cobra-TF output
files (.ctf.out) need to be converted by the code interface.

\textcolor{red}{
\textbf{It is important to note that the interface output collection (i.e., the parser of the Cobra-TF output) is currently able to extract
major edit data (.ctf.out) only. Only those variables printed in the "major edit" output files are exported and made available to RAVEN.} }
\\The following information is collected from Cobra-TF output file (.ctf.out):
\begin{itemize}
  \item \textit{\textbf{average properties for channels}}
  \begin{lstlisting}[basicstyle=\tiny]


 ************************************************************************************************************************
          simulation time =      1.03030  seconds           aver. properties for channels
 node  dist.  quality    void fraction            mass flow               enthalpy incr.    enthalpy    heat added
  no.   (ft.)                                     (lbm/s)                    (btu/s)         (btu/s)      (btu/s)
                      liq.  vapor  entr.  liquid vapor entr.  integr.  liquid vapor integr.  mixture  liquid vapor integr.

  50   12.00  -.119   1.000 0.000 0.000   16.39  0.00  0.00  16.39     32.41  0.00  32.41   10683.98  32.37  0.00  32.37

\end{lstlisting}

   that will be converted in the following way (CSV):
   \begin{table}[h]
    \centering
    \caption{CSV transport info (average properties for channels)}
    \label{CSVaverageProperties}
    \tabcolsep=0.11cm
    \tiny
    \begin{tabular}{|c|c|c|c|c|c|c|c|c|c|c|}
     time & AVG\_ch\_ax50\_quality  & AVG\_ch\_ax50\_voidFractionLiquid & AVG\_ch\_ax50\_voidFractionVapor & AVG\_ch\_ax50\_volumeEntrainFraction & ...\\
     1.03030 & -.119 & 1.000  & 0.000   & 0.000  & ...
    \end{tabular}
   \end{table}

  \item \textit{\textbf{fluid properties for each sub-channel}}
  \begin{lstlisting}[basicstyle=\tiny]
              simulation time =      0.00000  seconds           fluid properties for channel   19
 node  dist. pressure  velocity             void fraction           flow rate          flow    heat added         gama
  no.  (ft.) (psi)     (ft/sec)                                      (lbm/s)           reg.     (btu/s)          (lbm/s)
                     liquid vapor entr. liquid  vapor  entr.   liquid  vapor   entr.         liquid    vapor


 155 0.00  1251.687  2.66   2.66  0.01  1.0000 0.0000 0.0000  0.12456  0.0000  0.00000  0   0.595E-01  0.000E+00   0.00

  \end{lstlisting}
   that will be converted in the following way (CSV):
   \begin{table}[h]
    \centering
    \caption{CSV transport info (fluid properties for channels)}
    \label{CSVfluidProperties}
    \tabcolsep=0.11cm
    \tiny
    \begin{tabular}{|c|c|c|c|c|c|c|c|c|c|c|}
     time & ch19\_ax155\_pressure  & ch19\_ax155\_velocityLiquid & ch19\_ax155\_velocityVapor & ch19\_ax155\_velocityEntrain & ch19\_ax155\_voidFractionLiquid & ...\\
     0.00 & 1251.687 & 2.66           & 2.66   & 0.01        & 1.00 & ...
    \end{tabular}
   \end{table}

  \item \textit{\textbf{nuclear fuel rod}}
  \begin{lstlisting}[basicstyle=\tiny]
          nuclear fuel rod no.  1                         simulation time =    0.00 seconds
             surface no.  1 of  1
          -----------------------        conducts heat to channels  1  0  0  0  0  0               geometry type =  1
                                         and azimuthally to surfaces   1 and   1                   no. of radial nodes = 13

 **********************************************************************************************

   rod    axial    fluid temperatures  surface      heat       -clad temperatures-     gap        -fuel temperatures-
   node  location      (deg-f)         heat flux   transfer          (deg-f)        conductance         (deg-f)
   no.    (in.)    liquid  vapor       (b/h-ft2)    mode       outside  inside      (b/h-ft2-f)    surface   center
   ----  --------  ------  -----      ---------    --------    -------  ------      -----------    -------   ------

    10   22.80     464.1   467.1     0.5929E+04     spl        466.08   592.98       1594.2        859.58     2946.22
\end{lstlisting}
   that will be converted in the following way (CSV):
   \begin{table}[h]
     \centering
     \caption{CSV transport info (nuclear fuel rod)}
     \label{CSVfuelRod}
     \tabcolsep=0.11cm
     \tiny
     \begin{tabular}{|c|c|c|c|c|c|c|c|c|c|}
     time    & fuel\_rod10\_fluidTemperatureLiquid  & fuel\_rod10\_fluidTemperatureVapor & fuel\_rod10\_surfaceHeatflux & fuel\_rod10\_heatTransferMode & fuel\_rod10\_caldOutTemperature & ...  \\
     0.00 & 464.1 & 467.1  & 0.5929E+04               & 0        & 466.08         & ...
     \end{tabular}
   \end{table}

 \item \textit{\textbf{Cobra-TF's Output Variables and Corresponding Names in CSV file}}

   In CSV file, the output results obtained from the Cobra-TF output file (.ctf.out) will be saved with the names as described in Table 12.
   \begin{table}[h]
    \centering
    \caption[caption]{Variables Name List in CSV File \\\hspace{\textwidth} \textcolor{red}{(NN: Axial Node Number; CN: Channel Number)}}
    \label{CSVvariableNames}
    \tabcolsep=0.11cm
    \scriptsize
    \begin{tabular}{|l|l|}
     \hline
     \textbf{Output Variable} & \textbf{Name in CSV file} \\  \hline
     simulation time  & time   \\
     channels' average quality  & AVG\_ch\_ax\textcolor{red}{NN}\_quality \\
     channels' average void fraction (liquid)  & AVG\_ch\_ax\textcolor{red}{NN}\_voidFractionLiquid \\
     channels' average void fraction (vapor)  & AVG\_ch\_ax\textcolor{red}{NN}\_voidFractionVapor \\
     channels' average entrainment (volumetric) fraction  & AVG\_ch\_ax\textcolor{red}{NN}\_volumeEntrainFraction \\
     channels' average mass flow rate (liquid)  & AVG\_ch\_ax\textcolor{red}{NN}\_massFlowRateLiquid \\
     channels' average mass flow rate (vapor)  & AVG\_ch\_ax\textcolor{red}{NN}\_massFlowRateVapor \\
     channels' average entrainment rate (mass flow rate)  & AVG\_ch\_ax\textcolor{red}{NN}\_massFlowRateEntrain \\
     channels' average mass flow rate (integrated)  & AVG\_ch\_ax\textcolor{red}{NN}\_massFlowRateIntegrated \\
     channels' average enthalpy increase (liquid)  & AVG\_ch\_ax\textcolor{red}{NN}\_enthalpyIncreaseLiquid \\
     channels' average enthalpy increase (vapor)  & AVG\_ch\_ax\textcolor{red}{NN}\_enthalpyIncreaseVapor \\
     channels' average enthalpy increase (integrated)  & AVG\_ch\_ax\textcolor{red}{NN}\_enthalpyIncreaseIntegrated \\
     channels' average mixture enthalpy  & AVG\_ch\_ax\textcolor{red}{NN}\_enthalpyMixture \\
     channels' average heat added to liquid & AVG\_ch\_ax\textcolor{red}{NN}\_heatAddedToLiquid \\
     channels' average heat added to vapor & AVG\_ch\_ax\textcolor{red}{NN}\_heatAddedToVapor \\
     channels' average heat added (integrated)  & AVG\_ch\_ax\textcolor{red}{NN}\_heatAddedIntegrated' \\

     channel pressure & ch\textcolor{red}{CN}\_ax\textcolor{red}{NN}\_pressure \\
     channel liquid velocity  & ch\textcolor{red}{CN}\_ax\textcolor{red}{NN}\_velocityLiquid \\
     channel vapor velocity  & ch\textcolor{red}{CN}\_ax\textcolor{red}{NN}\_velocityVapor \\
     channel entrainment rate (velocity) & ch\textcolor{red}{CN}\_ax\textcolor{red}{NN}\_velocityEntrain \\
     channel void fraction (liquid) & ch\textcolor{red}{CN}\_ax\textcolor{red}{NN}\_voidFractionLiquid \\
     channel void fraction (vapor)  & ch\textcolor{red}{CN}\_ax\textcolor{red}{NN}\_voidFractionVapor \\
     channel volume fraction of entrainment liquid  & ch\textcolor{red}{CN}\_ax\textcolor{red}{NN}\_volumeEntrainFraction \\
     channel mass flow rate (liquid) & ch\textcolor{red}{CN}\_ax\textcolor{red}{NN}\_massFlowRateLiquid \\
     channel mass flow rate (vapor)  & ch\textcolor{red}{CN}\_ax\textcolor{red}{NN}\_massFlowRateVapor \\
     channel entrainment rate (mass flow rate) & ch\textcolor{red}{CN}\_ax\textcolor{red}{NN}\_massFlowRateEntrain \\
     channel flow regime ID & ch\textcolor{red}{CN}\_ax\textcolor{red}{NN}\_flowRegimeID \\
     channel heat added to liquid & ch\textcolor{red}{CN}\_ax\textcolor{red}{NN}\_heatAddedToLiquid \\
     channel heat added to vapor  & ch\textcolor{red}{CN}\_ax\textcolor{red}{NN}\_heatAddedToVapor \\
     channel evaporation rate  & ch\textcolor{red}{CN}\_ax\textcolor{red}{NN}\_evaporationRate \\

     channel enthalpy of vapor  & ch\textcolor{red}{CN}\_ax\textcolor{red}{NN}\_enthalpyVapor \\
     channel enthalpy of saturated vapor & ch\textcolor{red}{CN}\_ax\textcolor{red}{NN}\_enthalpySaturatedVapor \\
     channel enthalpy difference between vapor and saturated vapor & ch\textcolor{red}{CN}\_ax\textcolor{red}{NN}\_enthalpyVapor-SaturatedVapor \\
     channel enthalpy of liquid & ch\textcolor{red}{CN}\_ax\textcolor{red}{NN}\_enthalpyLiquid \\
     channel enthalpy of saturated liquid & ch\textcolor{red}{CN}\_ax\textcolor{red}{NN}\_enthalpySaturatedLiquid \\
     channel enthalpy difference between liquid and saturated liquid & ch\textcolor{red}{CN}\_ax\textcolor{red}{NN}\_enthalpyLiquid-SaturatedLiquid \\
     channel enthalpy of mixture & ch\textcolor{red}{CN}\_ax\textcolor{red}{NN}\_enthalpyMixture \\
     channel density of liquid  & ch\textcolor{red}{CN}\_ax\textcolor{red}{NN}\_densityLiquid \\
     channel density of vapor  & ch\textcolor{red}{CN}\_ax\textcolor{red}{NN}\_densityVapor \\
     channel density of mixture & ch\textcolor{red}{CN}\_ax\textcolor{red}{NN}\_densityMixture \\
     channel net entrainment rate (difference between entrainment rate and de-entrainment rate)  & ch\textcolor{red}{CN}\_ax\textcolor{red}{NN}\_netEntrainRate \\
     % gas volumetric analysis
     channel enthalpy of the mixture of non-condensable gases & ch\textcolor{red}{CN}\_ax\textcolor{red}{NN}\_enthalpyNonCondensableMixture \\
     channel density of the mixture of non-condensable gases & ch\textcolor{red}{CN}\_ax\textcolor{red}{NN}\_densityNonCondensableMixture \\
     channel steam volume fraction [0-100] & ch\textcolor{red}{CN}\_ax\textcolor{red}{NN}\_volumeFractionSteam \\
     channel air volume fraction [0-100] & ch\textcolor{red}{CN}\_ax\textcolor{red}{NN}\_volumeFractionAir \\
     channel total equivalent diameter of the liquid droplets (all droplets as a single big one) (diam-ld) & ch\textcolor{red}{CN}\_ax\textcolor{red}{NN}\_equiDiameterLiquidDroplet \\
     channel averaged diameter of liquid droplets field (diam-sd) & ch\textcolor{red}{CN}\_ax\textcolor{red}{NN}\_avgDiameterLiquidDroplet \\
     channel averaged flow rate of liquid droplets field (flow-sd) & ch\textcolor{red}{CN}\_ax\textcolor{red}{NN}\_avgFlowRateLiquidDroplet \\
     channel averaged velocity of liquid droplets field (veloc-sd) & ch\textcolor{red}{CN}\_ax\textcolor{red}{NN}\_avgVelocityLiquidDroplet \\
     channel evaporation rate of liquid droplets field (gamsd) & ch\textcolor{red}{CN}\_ax\textcolor{red}{NN}\_evaporationRateLiquidDroplet \\
     % fuel rod
     fuel rod fluid temperatures (liquid) & fuel\_rod\textcolor{red}{NN}\_fluidTemperatureLiquid \\
     fuel rod fluid temperatures (vapor)  & fuel\_rod\textcolor{red}{NN}\_fluidTemperatureVapor \\
     fuel rod surface heat flux & fuel\_rod\textcolor{red}{NN}\_surfaceHeatflux \\
     %fuel rod surface heat transfer mode [-] & fuel\_rod\textcolor{red}{(Rod Node Number)}\_heatTransferMode \\
     clad outer surface temperature  & fuel\_rod\textcolor{red}{NN}\_caldOutTemperature \\
     clad inner surface temperature  & fuel\_rod\textcolor{red}{NN}\_cladInTemperature \\
     gap conductance  & fuel\_rod\textcolor{red}{NN}\_gapConductance \\
     fuel outer suface temperature  & fuel\_rod\textcolor{red}{NN}\_fuelTemperatureSurface \\
     fuel center temperature  & fuel\_rod\textcolor{red}{NN}\_fuelTemperatureCenter \\

    \end{tabular}

   \end{table}
\end{itemize}

%%%%%%%%%%%%%%%%%%%%%%%%%%%%%%%%%%%%%%%%%%%%%%%%%%%%%%%%%
\subsubsection{Distributions}
%%%%%%%%%%%%%%%%%%%%%%%%%%%%%%%%%%%%%%%%%%%%%%%%%%%%%%%%%
The \xmlNode{Distribution} block defines the distributions that are going
to be used for the sampling of the variables defined in the \xmlNode{Samplers} block.
%
For all the possibile distributions and all their possible inputs please see the
chapter about Distributions (see~\ref{sec:distributions}).
%
Here we report an example of a Normal distribution:
\begin{lstlisting}[style=XML,morekeywords={name,debug}]
<Distributions verbosity='debug'>
    <Normal name="GridLossCoeff">
      <mean>0.7</mean>
      <sigma>0.1</sigma>
      <upperBound>0.9</upperBound>
      <lowerBound>0.6</lowerBound>
    </Normal>
 </Distributions>
\end{lstlisting}

It is good practice to name the distribution something similar to what kind of
variable is going to be sampled, since there might be many variables with the
same kind of distributions but different input parameters.

%%%%%%%%%%%%%%%%%%%%%%%%%%%%%%%%%%%%%%%%%%%%%%%%%%%%%%%%%
\subsubsection{Samplers}
%%%%%%%%%%%%%%%%%%%%%%%%%%%%%%%%%%%%%%%%%%%%%%%%%%%%%%%%%
In the \xmlNode{Samplers} block we want to define the variables that are going to be sampled.
%
\\The perturbation is performed by assigning the 'line number' and 'position (word number)' of the values that need to be changed in the original Cobra-TF input to the RAVEN input.
Specifically, the variable name in the form of \textbf{\textit{Line Number|Position}} needs to be defined in the specific Sampler.

\textbf{Example}:
We want to do the sampling of 1 single variable:
\begin{itemize}
  \item The Grid Loss Coefficient Data used in the line number 343 and word number 1 (position).
\end{itemize}

We are going to sample this variable using two different sampling methods: Grid and MonteCarlo.
The RAVEN input is then written as follows:

\begin{lstlisting}[style=XML,morekeywords={name,type,construction,lowerBound,steps,limit,initialSeed}]
<Samplers verbosity='debug'>
  <Grid name='Grid_Sampler' >
    <variable name='343|1'>
      <distribution>GridLossCoeff</distribution>
      <grid type='CDF' construction='equal'  steps='10'>0.001 0.999</grid>
    </variable>
  </Grid>
  <MonteCarlo name='MC_Sampler'>
     <samplerInit>
       <limit>1000</limit>
     </samplerInit>
    <variable name='343|1'>
      <distribution>GridLossCoeff</distribution>
  </MonteCarlo>
</Samplers>
\end{lstlisting}

It can be seen that each variable is connected with a proper distribution
defined in the \\\xmlNode{Distributions} block (from the previous example).

%%%%%%%%%%%%%%%%%%%%%%%%%%%%%%%%%%%%%%%%%%%%%%%%%%%%%%%%%%%
\subsubsection{Steps}
For a Cobra-TF interface, the \xmlNode{MultiRun} step type will most likely be
used.
%
First, the step needs to be named: this name will be one of the names used in
the \xmlNode{sequence} block.
%
In our example, \texttt{Grid\_Sampler} and \texttt{MC\_Sampler}.
%
\begin{lstlisting}[style=XML,morekeywords={name,debug,re-seeding}]
     <MultiRun name='Grid_Sampler' verbosity='debug'>
\end{lstlisting}

With this step, we need to import all the files needed for the simulation:
\begin{itemize}
  \item Cobra-TF input file
\end{itemize}
\begin{lstlisting}[style=XML,morekeywords={name,class,type}]
    <Input   class='Files' type=''>inputFileCobra</Input>
\end{lstlisting}
We then need to define which model will be used:
\begin{lstlisting}[style=XML]
    <Model  class='Models' type='Code'>MyCobraTF</Model>
\end{lstlisting}
We then need to specify which Sampler is used, and this can be done as follows:
\begin{lstlisting}[style=XML]
    <Sampler class='Samplers' type='Grid'>Grid_Sampler</Sampler>
\end{lstlisting}
And lastly, we need to specify what kind of output the user wants.
%
For example the user might want to make a database (in RAVEN the database
created is an HDF5 file).
%
Here is a classical example:
\begin{lstlisting}[style=XML,morekeywords={class,type}]
    <Output  class='Databases' type='HDF5'>Grid_out</Output>
\end{lstlisting}
Following is the example of two MultiRun steps which use different sampling
methods (Grid and Monte Carlo), and creating two different databases for each
one:
\begin{lstlisting}[style=XML]
<Steps verbosity='debug'>
  <MultiRun name='Grid_Sampler' verbosity='debug'>
    <Input   class='Files' type=''>inputFileCobra.inp</Input>
    <Model   class='Models'    type='Code'>MyCobraTF</Model>
    <Sampler class='Samplers'  type='Grid'>Grid_Sampler</Sampler>
    <Output  class='Databases' type='HDF5'>Grid_out</Output>
    <Output  class='DataObjects' type='PointSet'   >GridCTFPointSet</Output>
    <Output  class='DataObjects' type='HistorySet'>GridCTFHistorySet</Output>
  </MultiRun>
  <MultiRun name='MC_Sampler' verbosity='debug' re-seeding='210491'>
    <Input   class='Files' type=''>inputFileCobra.inp</Input>
    <Model   class='Models'    type='Code'>MyCobraTF</Model>
    <Sampler class='Samplers'  type='MonteCarlo'>MC_Sampler</Sampler>
    <Output  class='Databases' type='HDF5'      >MC_out</Output>
    <Output  class='DataObjects' type='PointSet'   >MonteCarloCobraPointSet</Output>
    <Output  class='DataObjects' type='HistorySet'>MonteCarloCobraHistorySet</Output>
  </MultiRun>
</Steps>
\end{lstlisting}
%%%%%%%%%%%%%%%%%%%%%%%%%%%%%%%%%%%%%%%%%%%%%%%%%%%%%%

\input{couplingAcode.tex}
\input{advanced_users_plugins.tex}
\input{examplesPrimer.tex}
\section*{Document Version Information}
\input{../version.tex}


    % ---------------------------------------------------------------------- %
    % References
    %
    \clearpage
    % If hyperref is included, then \phantomsection is already defined.
    % If not, we need to define it.
    \providecommand*{\phantomsection}{}
    \phantomsection
    \addcontentsline{toc}{section}{References}
    \bibliographystyle{ieeetr}
    \bibliography{raven_user_manual}


    % ---------------------------------------------------------------------- %
    %

    % \printindex

    %\include{distribution}

\end{document}
