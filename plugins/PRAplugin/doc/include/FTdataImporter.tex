\section{FT Data Importer}
\label{sec:FTdataImporter}

This Post-Processor is designed to import a FT as a PointSet in RAVEN.
The FT must be specified in a specific format: the OpenPSA format (\href{<url>}{https://github.com/open-psa}). 
As an example, the FT of Fig.~\ref{fig:FT} is translated in the OpenPSA as shown in Listing~\ref{lst:FTModel}.
The work-flow is shown below:

\begin{lstlisting}[style=XML,morekeywords={anAttribute},caption=FT Importer input example., label=lst:FT_PP_InputExample]
  <Files>
    <Input name="faultTreeTest" type="">FTimporter_not.xml</Input>
  </Files>
  
  <Models>
    ...
    <PostProcessor name="FTimporter" subType="FTImporter">
      <fileFormat>OpenPSA</fileFormat>
      <topEventID>TOP</topEventID>
    </PostProcessor> 
    ...  
  </Models>

  <Steps>
    ...
    <PostProcess name="import">
      <Input   class="Files"        type=""                >faultTreeTest</Input>
      <Model   class="Models"       type="PostProcessor"   >FTimporter</Model>
      <Output  class="DataObjects"  type="PointSet"        >FT_PS</Output>
    </PostProcess>
    ...
  </Steps>

  <DataObjects>
    ...
    <PointSet name="ET_PS">
      <Input>BE1,BE2,BE3,BE4</Input>
      <Output>TOP</Output>
    </PointSet>
    ...
  </DataObjects>
\end{lstlisting}

All the specifications of the FT importer are given in the 
\xmlNode{PostProcess} block. 
Inside the \xmlNode{PostProcess} block, the XML
nodes that belong to this models are:
\begin{itemize}
  \item  \xmlNode{fileFormat}, \xmlDesc{string, required parameter}, type of file format (e.g., OpenPSA)
  \item  \xmlNode{topEventID},\xmlDesc{string, required parameter}, the name of the top event of the FT
\end{itemize}

Each Point in the PointSet represents a unique combination of the basic events.
The PointSet is structured as follows: input variables are the basic events, output variable is the top event of the FT.
The value for each input and output variable can have the following values:
\begin{itemize}
  \item  0: False
  \item  1: True
\end{itemize}

Provided this definition, the FT model of Fig.~\ref{fig:FT} and described in Listing~\ref{lst:FTModel}, 
the resulting model in RAVEN is characterized by these variables:
\begin{itemize}
	\item Input variables: BE1, BE2, BE3, BE4
	\item Output variable: out
\end{itemize}
and it is structured is as follows:

\begin{table}
    \centering
    \caption{PointSet generated by RAVEN by employing the FT Importer Post-Processor for the FT of Fig.~\ref{fig:FT}.}
	\begin{tabular}{c | c | c | c | c} 
		\hline 
		BE1 & BE2 & BE3 & BE4 & TOP \\ 
		\hline 
		 0. &  0. &  0. &  0. &  0. \\
		 0. &  0. &  0. &  1. &  0. \\
		 0. &  0. &  1. &  0. &  0. \\
		 0. &  0. &  1. &  1. &  0. \\
		 0. &  1. &  0. &  0. &  0. \\
		 0. &  1. &  0. &  1. &  0. \\
		 0. &  1. &  1. &  0. &  0. \\
		 0. &  1. &  1. &  1. &  0. \\
		 1. &  0. &  0. &  0. &  0. \\
		 1. &  0. &  0. &  1. &  1. \\
		 1. &  0. &  1. &  0. &  1. \\
		 1. &  0. &  1. &  1. &  1. \\
		 1. &  1. &  0. &  0. &  1. \\
		 1. &  1. &  0. &  1. &  1. \\
		 1. &  1. &  1. &  0. &  1. \\
		 1. &  1. &  1. &  1. &  1. \\
		\hline 
	\end{tabular}
\end{table}

Important notes and capabilities:
\begin{itemize}
	\item If the FT is split in two or more FTs (and thus one file for each FT), then it is only required to list 
	      all files in the Step. RAVEN automatically detect links among FTs and merge all of them into a single PointSet.
	\item Allowed gates: AND, OR, NOT, ATLEAST, CARDINALITY, IFF, imply, NAND, NOR, XOR
	\item If an house-event is defined in the FT:
\begin{lstlisting}[style=XML,morekeywords={anAttribute},caption=FT Importer input example: house-event., label=lst:FT_house event]
<opsa-mef>
    <define-fault-tree name="FT">
        <define-gate name="TOP">
            <or>
                <basic-event name="BE1"/>
                <basic-event name="BE2"/>
                <house-event name="HE1"/>
            </or>
        </define-gate>
        <define-house-event name="HE1">
        	<constant value="true"/>
        </define-house-event>
    </define-fault-tree>
</opsa-mef>
\end{lstlisting}
           then the HE1 is not part of the PointSet (value is fixed)
\end{itemize}

\subsection{FT Importer reference tests}
\begin{itemize}
	\item test\_FTimporter\_and\_withNOT\_embedded.xml
	\item test\_FTimporter\_and\_withNOT\_withNOT\_embedded.xml
	\item test\_FTimporter\_and\_withNOT.xml
	\item test\_FTimporter\_and.xml
	\item test\_FTimporter\_atleast.xml
	\item test\_FTimporter\_cardinality.xml
	\item test\_FTimporter\_component.xml
	\item test\_FTimporter\_doubleNot.xml
	\item test\_FTimporter\_iff.xml
	\item test\_FTimporter\_imply.xml
	\item test\_FTimporter\_multipleFTs.xml
	\item test\_FTimporter\_nand.xml
	\item test\_FTimporter\_nor.xml
	\item test\_FTimporter\_not.xml
	\item test\_FTimporter\_or\_houseEvent.xml
	\item test\_FTimporter\_or.xml
	\item test\_FTimporter\_xor.xml
\end{itemize}
